\documentclass[a4paper, 12pt]{report}

%%%%%%%%%%%%
% Packages %
%%%%%%%%%%%%

\usepackage[spanish, mexico-com]{babel}
\usepackage{packages/sleek}
\usepackage{packages/sleek-title}
\usepackage[spanish]{packages/sleek-theorems}

%%%%%%%%%%%%%%%%
% Mis paquetes %
%%%%%%%%%%%%%%%%

\usepackage{tikz}
\usetikzlibrary{calc, patterns, arrows, arrows.meta}
\usepackage{pgfplots}
\pgfplotsset{compat=1.8}
\usepackage{standalone}
\usepackage{ifthen}
\usepackage{venndiagram}
\usepackage{makecell}

% Paquete para el glosario
%\usepackage[acronym, nonumberlist]{glossaries}
%\usepackage[acronym,symbols,nomain,nonumberlist]{glossaries-extra}

\usepackage[automake, toc, nonumberlist, symbols]{glossaries}

\makeglossaries


\usepackage{pifont}% http://ctan.org/pkg/pifont
\newcommand{\cmark}{\ding{51}}%
\newcommand{\xmark}{\ding{55}}%


%%%%%%%%%%%%%%
% Title-page %
%%%%%%%%%%%%%%

\logo{./resources/img/logo-fiuna}
\institute{Universidad Nacional de Asunción}
\faculty{Facultad de Ingeniería}
%\department{Cátedra de Álgebra Moderna}
\title{Apuntes de Álgebra Moderna\\{\large En proceso de elaboración}}
\subtitle{Notas de teoría}
\author{Fredy Gabriel \textsc{Ramírez Villanueva}}
%\supervisor{Linus \textsc{Torvalds}}
\context{versión 0.1.0-alpha}
\date{\today}

%%%%%%%%%%%%%%%%
% Bibliography %
%%%%%%%%%%%%%%%%

\addbibresource{resources/bib/references.bib}

%%%%%%%%%%%%%%%%
% Mis comandos %
%%%%%%%%%%%%%%%%

\newcommand{\N}{\mathbb{N}}
\newcommand{\Z}{\mathbb{Z}}
\newcommand{\Q}{\mathbb{Q}}
\newcommand{\R}{\mathbb{R}}
\newcommand{\C}{\mathbb{C}}
\newcommand{\K}{\mathbb{K}}
\newcommand{\bdot}{\boldsymbol{\cdot}}

% Definición del entorno custommatrix
\newenvironment{custommatrix}[1]
{
	\left[ \kern -4pt
	\begin{array}{#1}
	}
	{
	\end{array} \kern -4pt
	\right]
}

% Para mostrar QR en bibliografía
\DeclareFieldFormat[misc]{qrcode}{
	\begin{tabular}{c}
		\includegraphics[width=1cm, color=green]{#1}
	\end{tabular}
}

% Para generar QR de la bibliografía
\newcommand{\myqrcode}[1]{
	\begin{tabular}{c}
		\includegraphics[width=1cm]{#1}
	\end{tabular}
}


%%%%%%%%%%%%
% Document %
%%%%%%%%%%%%

% Cargar el glosario desde el archivo externo
\loadglsentries{resources/bib/glossary}

\begin{document}
    \maketitle
%    \romantableofcontents
    \phantomsection
    \tableofcontents

	% Capí\tableofcontentstulos
	\chapter{Conjuntos y funciones}

\section*{Introducción}

Este capítulo introduce los conceptos fundamentales de \glspl{conjunto} y \glspl{funcion}, herramientas matemáticas esenciales en diversas áreas de la ingeniería. Se aborda la definición de conjunto, sus diferentes representaciones (extensión y comprensión), y las operaciones básicas entre conjuntos como unión, intersección, diferencia y complemento. Se estudian las relaciones de pertenencia e inclusión entre conjuntos, y el concepto de conjunto vacío y conjunto universal.

Además, se introduce el concepto de función, tipos de funciones (inyectiva, sobreyectiva, biyectiva) y su representación cartesiana. Se estudian operaciones entre funciones como la composición y la función inversa, y se analizan ejemplos de funciones especiales relevantes en matemáticas.


\subsection*{Relevancia en ingeniería}

\begin{itemize}
	\item \textit{Modelado de sistemas}: Los conjuntos se utilizan para representar colecciones de objetos o elementos en un sistema, como por ejemplo, un conjunto de usuarios de una red de telecomunicaciones, un conjunto de componentes en un circuito eléctrico o para analizar las propiedades del suelo de un terreno;
	\item \textit{Análisis de datos}: Las funciones son esenciales para modelar relaciones entre variables en el análisis de datos, como la resistencia de un material en función del tiempo o modelación de fenómenos espaciales, como la altitud de un terreno.
	\item \textit{Programación}: Los conceptos de conjuntos y funciones son fundamentales en la programación, donde se utilizan para definir estructuras de datos y algoritmos.
	\item \textit{Optimización}: Las funciones se utilizan en problemas de optimización para representar la función objetivo que se busca maximizar o minimizar.
\end{itemize}

\section{Conjuntos} \label{sec:conjuntos}

\index{conjunto}

\subsection{Definición intuitiva de conjunto}
\vspace{1em}
\begin{fmd-definition}[Conjunto]
	Un \gls{conjunto} es una colección o agrupación de objetos, llamados \glspl{elemento} que comparten una característica común.
\end{fmd-definition}

Estos objetos pueden ser números, letras, personas, animales, o cualquier otra cosa que deseemos considerar juntos. En otras palabras, un conjunto es como una ``caja'' que contiene elementos relacionados.

\begin{fmd-example}[Conjuntos]
	\begin{enumerate}
		\item Conjunto de números naturales ($\N$):
		
		El conjunto de todos los números naturales es un ejemplo fundamental. Se denota como $\N$ y contiene los números $\{ 1, 2, 3, 4, 5, \ldots \}$.
		
		\item Conjunto de vocales: $\{a, e, i, o, u\}$.
		
		\item Conjunto de días de la semana:
		
		$\{ \text{lunes}, \text{martes}, \text{miércoles}, \text{jueves}, \text{viernes}, \text{sábado}, \text{domingo} \}$
		
		\item Conjunto de números pares: $\{2, 4, 6, 8, 10, \ldots \}$
		
		\item Conjunto de colores primarios: $\{\text{rojo}, \text{azul}, \text{amarillo}\}$.
	\end{enumerate}
\end{fmd-example}

\begin{lgnote}
	En un conjunto, los elementos no se repiten, y el orden no importa. Los elementos de un conjunto se escriben entre llaves $\{ \}$.
\end{lgnote}

\subsubsection{Notación}
Para denotar conjuntos utilizaremos generalmente letras mayúsculas $(A, B, C, \ldots)$, y para especificar elementos se usarán letras minúsculas \( (a, b, c, \ldots) \), a menos que dichos elementos sean, a su vez, conjuntos.

\subsubsection{Conjuntos numéricos}

Las notaciones usuales para caracterizar conjuntos numéricos son las siguientes:

\begin{itemize}
	\item $\N$: conjunto de números naturales;
	\item $\Z$: conjunto de números enteros;
	\item $\Q$: conjunto de números racionales;
	\item $\R$: conjunto de números reales;
	\item $\C$: conjunto de números complejos;
\end{itemize}

\subsubsection{Conjuntos especiales}

\begin{itemize}
	\item El \gls{conjuntouniversal}, \index{conjunto!universal} denotado como \glsentrysymbol{conjuntouniversal}, es el que contiene todos los elementos posibles que estamos considerando en un contexto particular. Es el conjunto más grande en ese contexto y actúa como un marco de referencia para otros conjuntos más pequeños.
	
	Por ejemplo, si estamos trabajando con números naturales, el conjunto universal sería \(\mathbb{N}\), que incluye todos los números positivos enteros: \(\{1, 2, 3, 4, \ldots\} \).
	
	\item El \gls{conjuntovacio}, \index{conjunto!vacío}  denotado como \glsentrysymbol{conjuntovacio}, es aquel que no contiene ningún elemento. El conjunto vacío es único, en otras palabras, todos los conjuntos vacíos son iguales.
	
	Por ejemplo, el conjunto vacío puede representar el conjunto de números reales que son mayores que 10 y menores que 5. Dado que no hay números que cumplan esta condición, el conjunto resultante es vacío: \glsentrysymbol{conjuntovacio}.
	
	\item Un \gls{conjuntounitario} está formado por un único elemento. Ejemplo $A = \{ a \}$.
\end{itemize}

\subsection{Relaciones de pertenencia e inclusión}

\subsubsection{Relación de pertenencia}

Para indicar la \gls{pertenencia} \index{pertenencia} de un elemento a un conjunto será utilizado el símbolo \glsentrysymbol{pertenencia}.

La proposición $a \in A$ se lee ``$a$ \textit{pertenece} a $A$'', o bien ``el elemento $a$ \textit{pertenece} al conjunto $A$''. Su negación es $a \not \in A$, que se lee ``$a$ \textit{no pertenece} a $A$''

\subsubsection{Notación por extensión}

La \gls{notacionextension} \index{conjunto!extensión} lista cada elemento del conjunto de forma individual.

\begin{fmd-example}[Notación por extensión]
	\begin{enumerate}
		\item Conjunto de números pares menores que 10: $E = \{2, 4, 6, 8\}$
		\item Conjunto de vocales: \(V = \{a, e, i, o, u\}\)
		\item Conjunto de meses del año: \(M = \{\text{enero}, \text{febrero}, \ldots, \text{diciembre}\}\)
	\end{enumerate}
\end{fmd-example}

\subsubsection{Notación por comprensión}
\index{conjunto!comprensión}
En la \gls{notacioncompresion}, describimos las propiedades o características que deben cumplir los elementos del conjunto, luego, utilizamos una condición lógica para definir el conjunto.

En general su estructura tiene la forma:
\[ A = \{ x \in U \mid P(x) \}\]
el conjunto cuyos elementos verifican la propiedad $P(x)$, o más brevemente, si $U$ está sobreentendido:
\[ A = \{ x \mid P(x) \}\]
se lee: ``$A$ es el conjunto formado por los elementos $x$, tales que $P(x)$'', en donde $P(x)$ es un función proposicional.

Un objeto $a$ del universal pertenece al conjunto $A$ \gls{siysolosi} verifica la propiedad $P(x)$, en consecuencia, el elemento $a$ no pertenece al conjunto $A$ si no cumple la propiedad $P(x)$.

\begin{fmd-example}[Notación por comprensión]
	\begin{enumerate}
		\item Conjunto de números pares: \(P = \{x \in \mathbb{N} \mid x = 2 k \mbox{ con } k \in \mathbb{N} \}\)
		
		``Son todos los números naturales $x$ tal que cada $x$ es igual al doble de un número natural $k$''.
		
		\item Conjunto de números primos menores que 20: \[N = \{x \in \mathbb{N} \mid (x \mbox{ es un número primo}) \mbox{ y } (x < 20) \}\]
		``Son todos los números naturales que son a la vez primos y menores que 20''
		
		\item Conjunto de letras del alfabeto: \(L = \{x \mid x \mbox{ es una letra del alfabeto}\}\)
		
		``El conjunto de elementos $x$ tal que $x$ es una letra del alfabeto''
		
		\item El \gls{conjuntovacio} puede definirse simbólicamente como: $\emptyset = \{ x \mid x \ne x \}$
		
		``Los elementos $x$ tal que cada $x$ es distinto de sí mismo'' (no existen tales elementos).
		
		En este caso la propiedad relativa a $x$ es $P(x): x \ne x$, la cual resulta falsa cualquiera sea $x$.
		
		\item Si $A$ es un conjunto unitario cuyo único elemento es $a$, escribiremos:
		\[ A = \{ a\} = \{ x \mid x = a \} \]
	\end{enumerate}
\end{fmd-example}

\subsubsection{Inclusión}
La \gls{inclusion} tiene la siguiente definición: \index{inclusión}

\begin{fmd-definition}[Inclusión]
	Sean $A$ y $B$ dos conjuntos, si ocurre que todo elemento de $A$ pertenece a $B$, diremos que $A$ está incluido en $B$, o que $A$ es parte de $B$, o que $A$ es un subconjunto de $B$.
	
	Notación:  \[A \subset B\]
\end{fmd-definition}

\begin{fmd-example}[Inclusión]
	\begin{enumerate}
		\item Consideremos los conjuntos: \(A = \{1, 2\}\) y \(B = \{1, 2, 3\}\).
		
		En este caso, \(A \subset B\), ya que todos los elementos de \(A\) (1 y 2) también pertenecen a \(B\).
		
		\item Consideremos los conjuntos:
		
		\begin{itemize}
			\item \(C = \{x \mid x \mbox{ es un número par}\}\)
			\item \(D = \{x \mid x \mbox{ es un número entero}\}\)
			
			Aquí, \(C \subset D\), ya que todo número par es también entero.
		\end{itemize}
	\end{enumerate}
\end{fmd-example}


\begin{itemize}
	\item Propiedades de la inclusión
	\begin{enumerate}[label=\roman*)]
		\item \textbf{Reflexibidad}: Todo conjunto es parte de sí mismo.
		\[ A \subset A \]
		\item \textbf{Transitividad}: Si un conjunto es parte de otro y este es parte de un tercero, el primero está incluido en el tercero:
		\[\mbox{Si } A \subset B \mbox{ y } B \subset C \mbox{ entonces } A \subset C \]
		\item \textbf{Antisimetría}: Si un conjunto es parte de otro y éste es parte del primero, entonces son iguales\footnote{Este hecho se usa para establecer la igualdad de conjuntos dada en la definición \ref{def:igualdad_conjuntos}}.
		\[\mbox{Si } A \subset B \mbox{ y } B \subset A \mbox{ entonces } A = B \]
	\end{enumerate}
	
	\item Observaciones:
	
	\begin{enumerate}
		\item En repetidas ocasiones se necesitará demostrar que un conjunto es parte de otro; entonces, de acuerdo con la definición, será suficiente demostrar que cualquier elemento del primero pertenece al segundo, esto significa que en la inclusión no puede darse que haya un elemento de $A$ que no pertenezca a $B$.
		\[A \subset B \mbox{ significa que, si } x \in A \mbox{ también } x \in B \]
		
		\item El conjunto vacío es subconjunto de cualquier conjunto.
		\[ \emptyset \subset A \]
	\end{enumerate}
\end{itemize}

\subsubsection{Igualdad de conjuntos}
\vspace{1em}
\index{conjuntos!igualdad}
\begin{fmd-definition}[Igualdad de conjuntos] \label{def:igualdad_conjuntos}
	Dos conjuntos $A$ y $B$ son iguales, es decir, se cumple la \gls{igualdad_conjuntos} $A = B$, si y solo si todo elemento de $A$ pertenece a $B$ y todo elemento de $B$ pertenece a $A$, en otras palabras, dos conjuntos son iguales si se contienen mutuamente.
	\[A = B \mbox{ si y sólo si } A \subset B \mbox{ y } B \subset A \]
\end{fmd-definition}

\subsubsection{Diagrama de Venn}
\index{diagrama Venn}

El \gls{venn} es una representación visual de los conjuntos utilizando diagramas llamados de Venn en homenaje a su autor\footnote{John Venn (1834-1923), matemático y lógico británico.}. En este sentido, el conjunto universal suele representarse por un rectángulo y los conjuntos por recintos cerrados. Es claro que todo elemento de $A$ pertenece a $U$, o sea, $A \subset U$. $A$, $B$ y $C$ subconjuntos de $U$, como indica el diagrama de la fig. \ref{fig:Venn}.

\begin{figure}[H]
	\centering
	\begin{tikzpicture}
		% Conjunto Universal U
		\draw[rounded corners] (-3,-2) rectangle (4,2) node[pos=.1] {$U$};
		
		% Conjunto B
		\draw[fill=blue!30, opacity=.5] (0,0) circle (1.5cm) node[opacity=1, above right] {$B$};
		
		% Conjunto C
		\draw[fill=green!50, opacity=.5] (2,0) circle (1cm) node[opacity=1, above right] {$C$};
		
		% Conjunto A
		\draw[fill=gray!50] (-.75, 0) circle (0.5cm) node {$A$};
	\end{tikzpicture}
	\caption{Diagrama de Venn}
	\label{fig:Venn}
\end{figure}

\subsection{Operaciones con conjuntos}

\subsubsection{Unión \glsentrysymbol{union}}
\vspace{1em} \index{unión}
\begin{fmd-definition}[Unión]
	La \gls{union} de dos conjuntos \(A\) y \(B\) consiste en todos los elementos que pertenecen a \(A\), a \(B\), o a ambos conjuntos.
	
	Notación: \(A \cup B\)
	
	Formalmente:
	\[ A \cup B = \left\{ x \mid x \in A \mbox{ o } x \in B \right\} \]
\end{fmd-definition}

\begin{itemize}
	\item Diagrama de Venn
	\begin{figure}[H]
		\centering
		\begin{venndiagram2sets}
			\fillA \fillB
		\end{venndiagram2sets}
		\caption*{$A \cup B$}
	\end{figure}
	
	\item Propiedades:
	\begin{enumerate}[label=\roman*)]
		\item Idempotencia\footnote{La idempotencia se define en la pág. \index{idempotencia} \pageref{def:idempotencia}, definición \ref{def:idempotencia}.}: \( A \cup A = A \)
		\item Asociatividad: \( \left( A \cup B \right) \cup C = A \cup \left( B \cup C \right) \)
		\item Conmutatividad: \( A \cup B = B \cup A\)
	\end{enumerate}
\end{itemize}

\begin{example}[Unión]
	Si \(A = \{1, 2, 3\}\) y \(B = \{3, 4, 5\}\), entonces \(A \cup B = \{1, 2, 3, 4, 5\}\).
\end{example}

\subsubsection{Intersección \glsentrysymbol{interseccion}}
\vspace{1em} \index{intersección}
\begin{fmd-definition}[Intersección]
	La \gls{interseccion} de dos conjuntos \(A\) y \(B\) contiene los elementos que pertenecen tanto a \(A\) como a \(B\).
	
	Notación: \(A \cap B\)
	
	Formalmente:
	\[ A \cap B = \left\{ x \mid x \in A \mbox{ y } x \in B \right\} \]
\end{fmd-definition}

\begin{itemize}
	\item Diagrama de Venn
	\begin{figure}[H]
		\centering
		\begin{venndiagram2sets}
			\fillACapB
		\end{venndiagram2sets}
		\caption*{\(A \cap B \)}
	\end{figure}
	
	\item Propiedades:
	\begin{enumerate}[label=\roman*)]
		\item Idempotencia: \( A \cap A = A \) \index{idempotencia}
		\item Asociatividad: \( \left( A \cap B \right) \cap C = A \cap \left( B \cap C \right) \) \index{asociatividad}
		\item Conmutatividad: \( A \cap B = B \cap A\) \index{conmutatividad}
	\end{enumerate}
\end{itemize}

\begin{example}[Intersección]
	Si \(A = \{1, 2, 3\}\) y \(B = \{3, 4, 5\}\), entonces \(A \cap B = \{3\}\).
\end{example}

\subsubsection{Complemento $(A^c)$}
\vspace{1em} \index{complemento}

\begin{fmd-definition}[Complemento]
	El \gls{complemento} de un conjunto \(A\) con respecto a un conjunto universal \(U\) contiene todos los elementos que no están en \(A\).
	Notación: \(A^c\) o \(\overline{A}\) o \(A'\)
	
	Formalmente:
	\[ A^c = \left\{ x \mid x \in U \mbox{ y } x \not \in A \right\} \]
\end{fmd-definition}

\begin{itemize}
	\item Diagrama de Venn
	
	\begin{figure}[H]
		\centering
		\begin{tikzpicture}[scale=1]
			\filldraw[fill=lightgray] (0, 0) rectangle (5,3.2); % Rectángulo del universo
			\filldraw[fill=white] (2, 1.6) circle (1.2); % Dibuja y rellena el conjunto A
			
			\node at (2, 2.2) {$A$}; % Etiqueta el conjunto A
			\node at (4, 1.5) {$A^c$}; % Etiqueta el complemento de A
		\end{tikzpicture}
		\caption*{\(A^c\)}
	\end{figure}
	\item Propiedades:
	\begin{enumerate}
		\item Involución $(A^c)^c = A$.
		
		De donde: Si \( A^c = B \) entonces \( B^c = A \)
		\item $A \subset B \implies B^c \subset A^c$
		\item El complemento del vacío es el universal: \( \emptyset^c = U \)
		\item El complemento del universal es el vacío: \( U^c = \emptyset \)
	\end{enumerate}
\end{itemize}

\begin{lgnote}
	El símbolo $\implies$ es la implicación o condicional ``\textit{si ..., entonces}''. Se explica con rigor en \ref{sec:condicional}.
\end{lgnote}

\begin{example}[Complemento]
	Si \(U\) es el conjunto de números naturales y \(A\) el conjunto de números pares,  \(A = \{2, 4, 6, \ldots \}\), entonces \(A^c\) es el conjunto de números impares, \(A^c = \{1, 3, 5, 7, \ldots\}\).
\end{example}

\subsubsection{Diferencia $(-, \setminus)$}

\vspace{1em}
\index{conjunto!diferencia}
\begin{fmd-definition}[Diferencia de conjuntos]
	La diferencia entre dos conjuntos \(A\) y \(B\) contiene los elementos que están en \(A\) pero no en \(B\).
	
	Notación: \(A - B\) o \( A \setminus B \)
	
	Formalmente:
	\[ A \setminus B = \left\{ x \mid x \in A \mbox{ y } x \not \in B \right\} \]
\end{fmd-definition}

\begin{itemize}
	\item Diagrama de Venn
	\begin{figure}[H]
		\centering
		\begin{venndiagram2sets}
			\fillOnlyA
		\end{venndiagram2sets}
		\caption*{$A \setminus B$}
	\end{figure}
	\item Propiedades:
	\begin{enumerate}[label=\roman*)]
		\item La diferencia entre dos conjuntos es igual a la intersección del primero con el complemento del segundo: \( A \setminus B = A \cap B^c \)
	\end{enumerate}
\end{itemize}

\begin{example}[Diferencia de conjuntos]
	Si \(A = \{1, 2, 3\}\) y \(B = \{3, 4, 5\}\), entonces \(A - B = \{1, 2\}\).
\end{example}

\subsubsection{Diferencia simétrica $\triangle$}
\vspace{1em}
\index{conjunto!dif. sim.}
\begin{fmd-definition}[Diferencia simétrica]
	La \gls{difsimetrica} de dos conjuntos \(A\) y \(B\), es el conjunto que contiene los elementos que pertenecen a \(A\) o a \(B\), pero no a ambos.
	
	Notación: \(A \triangle B\)
	
	Formalmente:
	$$A \triangle B = (A \setminus B) \cup (B \setminus A)$$
	
	La diferencia simétrica es el conjunto de elementos que están en uno de los conjuntos, pero no en ambos. Es como si ``excluyéramos'' la intersección de los dos conjuntos, por lo tanto también podemos escribir:
	\[ A \triangle B = \left( A \cup B \right) \setminus \left( A \cap B \right) \]
\end{fmd-definition}

\begin{itemize}
	\item Diagrama de Venn
		\begin{figure}[H]
			\centering
			\begin{venndiagram2sets}
				\fillOnlyA \fillOnlyB
			\end{venndiagram2sets}
			\caption*{$A \triangle B$}
		\end{figure}
	
	\item Propiedades:
	\begin{enumerate}[label=\roman*)]
		\item Conmutatividad: \(A \triangle B = B \triangle A\)
		\item Asociatividad: \((A \triangle B) \triangle C = A \triangle (B \triangle C)\)
		\item Existencia del neutro\footnote{El elemento neutro se define formalmente en \ref{sec:axiomas-grupo}, pág. \pageref{sec:axiomas-grupo}}: \( A \triangle \emptyset = \emptyset \triangle A = A \)
		\item Existencia de inversas\footnote{Ibid.}: \( A \triangle A = \emptyset \)
	\end{enumerate}
\end{itemize}

\begin{example}[Diferencia simétrica]
	\
	\begin{enumerate}
		\item Consideremos los conjuntos: \(A = \{1, 2, 3, 4\}\) y \(B = \{3, 4, 5, 6\}\). Entonces, la diferencia simétrica es:
		
		$$A \triangle B = \{1, 2, 5, 6\}$$
		
		Los elementos 1 y 2 están solo en \(A\), 5 y 6 están solo en \(B\). Los elementos 3 y 4 se excluyen porque están en ambos conjuntos.
		
		\item Consideremos los conjuntos:
		\begin{itemize}
			\item \(C = \{x \mid x \mbox{ es un número par}\}\)
			\item \(D = \{x \mid x \mbox{ es un número primo}\}\)
		\end{itemize}
		
		La diferencia simétrica es:
		$$C \triangle D = \{x \mid x \text{ es par y no es 2, o } x \text{ es primo y no es 2}\}$$
		El número 2 se excluye porque es el único par que también es primo.
	\end{enumerate}
\end{example}


\subsubsection{Conjunto potencia o conjunto de partes $\mathcal{P}(A)$}
\vspace{1em}
\index{conjunto!potencia} \index{conjunto!de partes}
\begin{fmd-definition}[Conjunto potencia]
	Dado un conjunto \(A\), el \gls{conjuntopotencia} o conjunto de partes de \(A\), denotado por \glsentrysymbol{conjuntopotencia} o \(2^A\), es el conjunto de todos los subconjuntos de \(A\). 
	
	Formalmente:
	$$\mathcal{P}(A) = \{B \mid B \subseteq A\}$$
	
	El conjunto potencia de un conjunto \(A\) es una colección que contiene a todos los posibles subconjuntos de \(A\), incluyendo el conjunto vacío (\(\emptyset\)) y el propio conjunto \(A\).
\end{fmd-definition}


\begin{example}[Conjunto potencia]
	\
	\begin{enumerate}
		\item Consideremos el conjunto: \(A = \{1, 2\}\)
		
		Entonces, el conjunto potencia es:
		$$\mathcal{P}(A) = \{\emptyset, \{1\}, \{2\}, \{1, 2\}\}$$
		
		Observe que hay \(2^2 = 4\) elementos en el conjunto potencia, ya que cada elemento de \(A\) tiene dos opciones: estar o no estar en un subconjunto particular.
		
		\item Consideremos el conjunto: \(B = \{a, b, c\}\)
		
		El conjunto potencia es:
		$$\mathcal{P}(B) = \{\emptyset, \{a\}, \{b\}, \{c\}, \{a, b\}, \{a, c\}, \{b, c\}, \{a, b, c\}\}$$
		En este caso, hay \(2^3 = 8\) elementos en el conjunto potencia.
	\end{enumerate}
\end{example}

\begin{lgnote}
	En general, si un conjunto \(A\) tiene \(n\) elementos, entonces su conjunto potencia \(\mathcal{P}(A)\) tiene \(2^n\) elementos.
\end{lgnote}

\subsubsection{Producto cartesiano}
\vspace{1em}
\index{producto!cartesiano}
\begin{fmd-definition}[Producto cartesiano]
	El \gls{productocartesiano} de dos conjuntos \(A\) y \(B\), es el conjunto de todos los pares ordenados \((a, b)\) donde el primer elemento \(a\) pertenece al conjunto \(A\) y el segundo elemento \(b\) pertenece al conjunto \(B\).

	Notación: \(A \times B\)
	
	Formalmente:
	$$A \times B = \{(a, b) \mid a \in A \text{ y } b \in B\}$$
	
	Esto es, el producto cartesiano combina cada elemento de un conjunto con cada elemento del otro conjunto, formando pares ordenados donde el orden de los elementos es importante.
\end{fmd-definition}


\begin{example}[Producto cartesiano]
	\
	\begin{enumerate}
		\item Consideremos los conjuntos: \(A = \{1, 2\}\) y \(B = \{x, y\}\), entonces, el producto cartesiano es:
		$$A \times B = \{(1, x), (1, y), (2, x), (2, y)\}$$
		
		\item Consideremos los conjuntos: \(C = \{a, b\}\) y \(D = \{1, 2, 3\}\), el producto cartesiano es:
		$$C \times D = \{(a, 1), (a, 2), (a, 3), (b, 1), (b, 2), (b, 3)\}$$
	\end{enumerate}
\end{example}

\begin{itemize}
	\item Observaciones:
	
	\begin{enumerate}
		\item El producto cartesiano no es conmutativo, en general, \(A \times B \neq B \times A\).
		\item Si \(A\) tiene \(m\) elementos y \(B\) tiene \(n\) elementos, entonces \(A \times B\) tiene \(m \cdot n\) elementos.
		\item El producto cartesiano se puede extender a más de dos conjuntos. Por ejemplo, el producto cartesiano de tres conjuntos \(A\), \(B\) y \(C\) es:
		$$A \times B \times C = \{(a, b, c) \mid a \in A, b \in B, c \in C\}$$
		\item Si los elementos son del mismo conjunto $A$:
		\[ A \times A \times A = A^3 = \{(a, b, c) \mid a, b, c \in A\} \]
	\end{enumerate}
	
	\item Propiedades:
	\begin{enumerate}[label=\roman*)]
		\item El producto cartesiano es \textbf{distributivo} respecto de la unión:
		\[ \left( A \cup B \right) \times C = \left( A \times C \right) \cup \left( B \times C \right) \]
	\end{enumerate}
	
\end{itemize}


\subsection{Álgebra de conjuntos} \label{sec:algebra_conjuntos}
\index{álgebra!conjuntos}
\subsubsection{Propiedades fundamentales}

\begin{itemize}
	\item \textbf{Idempotencia}:
	\begin{itemize}[itemsep=0pt]
		\item \( A \cup A = A \)
		\item \( A \cap A = A \)
	\end{itemize}
	\item \textbf{Conmutatividad}:
	\begin{itemize}[itemsep=0pt]
		\item \(A \cup B = B \cup A\)
		\item \(A \cap B = B \cap A\)
	\end{itemize}
	
	\item \textbf{Asociatividad}:
	\begin{itemize}[itemsep=0pt]
		\item \((A \cup B) \cup C = A \cup (B \cup C)\)
		\item \((A \cap B) \cap C = A \cap (B \cap C)\)
	\end{itemize}
	
	\item \textbf{Distributividad}:
	\begin{itemize}[itemsep=0pt]
		\item \(A \cup (B \cap C) = (A \cup B) \cap (A \cup C)\)
		\item \(A \cap (B \cup C) = (A \cap B) \cup (A \cap C)\)
	\end{itemize}
		
	\item \textbf{Leyes de De Morgan}:
	
	\begin{enumerate}[label={\textbf{\arabic*})}]
		\item El complemento de la unión de dos conjuntos es igual a la intersección de los complementos de dichos conjuntos.
		\[ (A \cup B)^c = A^c \cap B^c \]
		
		\item El complemento de la intersección de dos conjuntos es igual a la unión de los complementos de dichos conjuntos.
		\[ (A \cap B)^c = A^c \cup B^c \]
	\end{enumerate}
	
	\item\textbf{Elemento neutro}\footnote{El concepto de elemento neutro se presenta en detalle en la sección \ref{sec:axiomas-grupo}, pág. \pageref{sec:axiomas-grupo}}:
	\begin{itemize}
		\item De la unión es el conjunto vacío: \(A \cup \emptyset = A\)
		\item De la intersección es el universal: \(A \cap U = A\)
		\item De la diferencia simétrica es el conjunto vacío: \( A \triangle \emptyset = \emptyset \triangle A = A \)
	\end{itemize}
	
	\item \textbf{Existencia de inversas\footnote{ibid.}}:
	\begin{itemize}
		\item Para la diferencia simétrica, $A$ es su propia inversa: \( A \triangle A = \emptyset \)
	\end{itemize}
	
	\item \textbf{Otras propiedades}:
	\begin{itemize}
		\item \(A \cup A^c = U\)
		\item \(A \cap A^c = \emptyset\)
	\end{itemize}
	
\end{itemize}

\subsubsection{Operaciones generalizadas}

\begin{itemize}
	\item \textbf{Unión generalizada} \index{unión!generalizada}
	
	Dada una colección finita de conjuntos \( \{ A_1, A_2, \dots, A_n \} \) la unión generalizada de estos conjuntos se denota por:
	\[ A_1 \cup A_2 \cup \dots \cup A_n = \bigcup_{i=1}^n A_i  \]
	Se define como el conjunto que contiene todos los elementos que pertenecen a \textit{al menos uno} de los conjuntos \(A_i\).
	
	Formalmente:
	
	Si definimos el conjunto de índices \( I_n = \{ 1, 2, \dots, n \} \) (conjunto de los $n$ primeros números naturales):
	\[ \bigcup_{i=1}^n A_i = \left\{ x \ | \ x \in A_i \mbox{ para algún } i \in I_n \right\} \]
	
	Si el conjunto $I_n$ se identifica con $\N$ (conjunto de números naturales):
	\[ \bigcup_{i=1}^\infty A_i = \left\{ x \ | \ x \in A_i \mbox{ para algún } i \in \N \right\} \]
	
	\item \textbf{Intersección generalizada} \index{intersección!generalizada}
	
	Dada una colección finita de conjuntos \( \{ A_1, A_2, \dots, A_n \} \) la intersección generalizada de estos conjuntos se denota por:
	\[ A_1 \cap A_2 \cap \dots \cap A_n = \bigcap_{i=1}^n A_i  \]
	Se define como el conjunto que contiene todos los elementos que pertenecen a \textit{todos} los conjuntos \(A_i\).
	
	Formalmente:
	\[ \bigcap_{i=1}^n A_i = \left\{ x \ | \ x \in A_i \ \forall i \in I_n \right\} \]
	
	Si el conjunto $I_n$ se identifica con $\N$:
	\[ \bigcap_{i=1}^\infty A_i = \left\{ x \ | \ x \in A_i \ \forall i \in \N \right\} \]
	
	\item \textbf{Leyes de De Morgan generalizadas}
	
	\begin{enumerate}
		\item El complemento de la unión es igual a la intersección de los complementos. 
		\[ \left( \bigcup_{i \in I} A_i \right)^c = \bigcap_{i\in I} A_i^c \]
		\item El complemento de la intersección es igual a la unión de los complementos.
		\[ \left( \bigcap_{i \in I} A_i \right)^c = \bigcup_{i\in I} A_i^c \]
	\end{enumerate}
\end{itemize}

\begin{fmd-example}[Unión e intersección generalizadas]
	Consideremos los conjuntos:
	\begin{itemize}[itemsep=0pt]
		\item \(A_1 = \{1, 2, 3\}\)
		\item \(A_2 = \{2, 3, 4\}\)
		\item \(A_3 = \{3, 4, 5\}\)
	\end{itemize}
	Entonces:
	\[\bigcup_{i=1}^{3} A_i = \{1, 2, 3, 4, 5\}\]
	\[\bigcap_{i=1}^{3} A_i = \{3\}\]
\end{fmd-example}

\subsubsection{Uniones disjuntas}
\vspace{1em}
\index{unión!disjunta}
\begin{fmd-definition}[Uniones disjuntas]
	Se dice que una colección de conjuntos tiene \glspl{uniondisjunta} si los conjuntos son mutuamente excluyentes, es decir, si la intersección de cualquier par de conjuntos de la colección es el conjunto vacío.
	
	Formalmente, una colección de conjuntos \(\{A_i\}_{i \in I}\) es una colección de uniones disjuntas si:
	\[A_i \cap A_j = \emptyset \quad \forall i, j \in I \mbox{ con } i \neq j\]
	
	Los conjuntos en una unión disjunta no comparten ningún elemento\footnote{La noción de uniones disjuntas se puede generalizar a un número infinito de conjuntos.}.
	
	Notación:
	
	Para dos conjuntos disjuntos $A$ y $B$, es usual escribir:
	\[ A + B \]
	en lugar de $A \cup B$ para el caso $A \cap B = \emptyset$.
\end{fmd-definition}

Algunas aplicaciones:

\begin{enumerate}
	\item \textit{Partición de un conjunto}\footnote{Las particiones se estudian en la sección \ref{sec:particiones}, pág. \pageref{sec:particiones}}: Una partición de un conjunto \(A\) es una colección de subconjuntos no vacíos de \(A\) que son disjuntos dos a dos y cuya unión es igual a \(A\).  Las particiones son útiles en diversas áreas, como la teoría de probabilidades y estadística.
	
	\item \textit{Clasificación}: En muchas situaciones, es necesario clasificar objetos en diferentes categorías. Estas pueden ser representadas como conjuntos disjuntos, donde cada objeto pertenece a una y solo una categoría. Por ejemplo, en biología, los organismos se clasifican en diferentes especies, que son conjuntos disjuntos.
\end{enumerate}

\begin{fmd-example}[Uniones disjuntas]
	Consideremos los siguientes conjuntos: \(A_1 = \{1, 2, 3\}\), \(A_2 = \{4, 5\}\), \(A_3 = \{6\}\)
	
	La colección \(\{A_1, A_2, A_3\}\) es una colección de uniones disjuntas, ya que:
	\begin{itemize}[itemsep=0pt]
		\item \(A_1 \cap A_2 = \emptyset\)
		\item \(A_1 \cap A_3 = \emptyset\)
		\item \(A_2 \cap A_3 = \emptyset\)
	\end{itemize}
\end{fmd-example}

\rule{\textwidth}{.5pt}

\subsection{Ejercicios}

\begin{enumerate}
	
	\item Sean $A$, $B$ y $C$ conjuntos. Demuestre que las siguientes afirmaciones son equivalentes: $A = B$; $A \cap B = A$; $A \cup B = B$; $A \setminus B = \emptyset$; $B^c = A^c$.
	
	\item Sean $A$ y $B$ conjuntos. Simplifique la siguiente expresión:
	\( (A \cap B^c) \cup (A^c \cap B) \cup (A \cap B) \)
	\item Sean $A$, $B$ y $C$ conjuntos. Demuestre que: \( A \triangle (B \triangle C) = (A \triangle B) \triangle C \)
	
	\item Sea $A$ un conjunto finito. Demuestre que $|\mathcal{P}(A)| = 2^{|A|}$.
	
	\item Sean $A=\{1,2,3\}$, $B=\{a,b\}$ y $C=\{x,y,z\}$. Encuentre $(A \times B) \cap (A \times C)$
	
	\item Sea $A=\{\emptyset, \{\emptyset\}\}$. Encuentre $P(P(A))$
	
	\item Sean $A$ y $B$ conjuntos. Demuestre que $A$ y $B$ son disjuntos si y solo si $A \setminus B=A$.
	
	\item Sean $A$ y $B$ conjuntos finitos. Si $|A \cup B|=10$, $|A \cap B|=3$ y $|A|=7$, ?`cuál es la cardinalidad de $B$?
	
	\item En un sistema de comunicación, se transmiten mensajes utilizando un alfabeto de 5 símbolos. Cada mensaje tiene una longitud de 3 símbolos.
	\begin{enumerate}[itemsep=-3pt]
		\item ?`Cuántos mensajes diferentes se pueden transmitir?
		\item Si se introduce un nuevo símbolo en el alfabeto, ?`cuántos mensajes adicionales se pueden transmitir?
	\end{enumerate}
	
	\item Defina un conjunto bien ordenado y dé un ejemplo de un conjunto que no esté bien ordenado.
	
	\item En un sistema de control, se tienen tres sensores que miden diferentes variables. El sensor $A$ detecta eventos en el conjunto $\{1,2,3,4\}$, el sensor $B$ en el conjunto $\{3,4,5,6\}$ y el sensor $C$ en el conjunto $\{1,3,5,7\}$.
	\begin{enumerate}[itemsep=-3pt]
		\item ?`Qué eventos son detectados por al menos dos sensores?
		\item ?`Qué eventos son detectados por los tres sensores?
		\item Si se considera un evento crítico aquel que es detectado por al menos dos sensores, ?`cuáles son los eventos críticos?
	\end{enumerate}

	\item Una imagen digital binaria (en blanco y negro) se puede representar como un subconjunto del plano $\mathbb{R}^2$. Cada punto de la imagen corresponde a un píxel, que puede ser blanco o negro.
	\begin{enumerate}[itemsep=-3pt]
		\item Defina matemáticamente una imagen digital en blanco y negro.
		\item ?`Cómo se representaría la operación de ``invertir los colores'' de una imagen utilizando la teoría de conjuntos?
	\end{enumerate}
	
	\item En un sistema de aprendizaje automático, se tienen dos conjuntos de datos: un conjunto de entrenamiento $A$ y un conjunto de prueba $B$. El conjunto de entrenamiento se utiliza para ajustar los parámetros del modelo, y el conjunto de prueba se utiliza para evaluar su rendimiento.
	\begin{enumerate}[itemsep=-3pt]
		\item ?`Por qué es importante que $A$ y $B$ sean disjuntos?
		\item ?`Qué problemas podrían surgir si $A$ y $B$ no fueran disjuntos?
	\end{enumerate}
	
	\item Investigue y presente un problema abierto en la teoría de conjuntos que tenga relevancia en la ingeniería o en las ciencias de la computación.
	
\end{enumerate}

\rule{\textwidth}{.5pt}
\section{Relaciones y particiones}
\index{relaciones!particiones}
	Una \gls{relacion} es un vínculo o una correspondencia. Se trata de la correspondencia que existe entre dos conjuntos: a cada elemento del primer conjunto le corresponde al menos un elemento del segundo conjunto.
	
	Cuando a cada elemento de un conjunto le corresponde solo uno del otro, se habla de \gls{funcion}. Esto quiere decir que las funciones siempre son, a su vez, relaciones, pero que las relaciones no siempre son funciones.

\begin{fmd-definition}[Relaciones] \index{relaciones}
	Dados dos conjuntos $A$ y $B$, una \gls{relacion} entre ellos es un subconjunto $\mathcal{R} \subset A \times B$, en el que el par ordenado $(a, b) \in \mathcal{R}$ con $a \in A$ y $b \in B$, así decimos que $a$ está relacionado con $b$ y se denota:
	\[ a \mathcal{R} b \]
\end{fmd-definition}

\subsection{Relaciones de equivalencia}
\vspace{3mm} \index{relaciones!equivalencia}
\begin{fmd-definition}[Relación de equivalencia]
	$\mathcal{R} \subset A^2$ es una \gls{relequivalencia} en $A$ si y sólo si es reflexiva, simétrica y transitiva
\end{fmd-definition}

Se suele utilizar el símbolo ``\glsentrysymbol{relequivalencia}'' o ``$\sim$''. La notación $a \sim b$ o $a \equiv b$ se lee ``$a$ es equivalente a $b$''.

Conforme a la definición, las relaciones de equivalencia satisfacen:

\begin{enumerate}[label=\roman*)]
	\item \textbf{Reflexividad}: Todo elemento en $A$ es equivalente a sí mismo.
	\[ \forall x \in A: \implies x \equiv x \]
	
	\item \textbf{Simetría}: Si un elemento es equivalente a otro, entonces este es equivalente al primero.
	\[ \forall x, y \in A: x \equiv y \implies y \equiv x \]
	
	\item \textbf{Transitividad}: Si un elemento es equivalente a otro y éste es equivalente a un tercero, entonces el primero es equivalente al tercero.
	\[ \forall x, y, z \in A: x \equiv y \text{ y } y \equiv z \implies x \equiv z \]
\end{enumerate}

\subsubsection{Clases de equivalencia y conjunto cociente}
\vspace{3mm} \index{clases!equivalencia}
\begin{fmd-definition}[Clase de equivalencia]
	Sea $A$ un conjunto y $\mathcal{R}$ una relación de equivalencia en $A$. Para cada elemento $a \in A$, la \gls{clasequiv} de $a$, que se denota $[a]$ o $\bar{a}$, es el conjunto de todos los elementos $x \in A$ tales que $x$ está relacionado con $a$ en $\mathcal{R}$.
	\[ [a] = \bar{a} = \{ x \in A / x \mathcal{R} a \}\]
\end{fmd-definition}

\begin{fmd-example}[Clases de equivalencia]
	Sea $A = \{ z_1, z_2, z_3, z_4, z_5, z_6 \}$ y la relación:
	\[ \begin{split}
		\mathcal{R} = \{ & (z_1, z_1), (z_1, z_3), (z_1, z_5), (z_2, z_2), (z_2, z_4), (z_3, z_1), (z_3, z_3),\\
		& (z_3, z_5), (z_4, z_2), (z_4, z_4), (z_5, z_1), (z_5, z_3), (z_5, z_5), (z_6, z_6) \}
	\end{split} \]
	
\begin{minipage}{.45\textwidth}
	Las clases de equivalencia son:
	\[ \begin{split}
		[z_1] =& \{ z_1, z_3, z_5 \}\\
		[z_2] = & \{ z_2, z_4 \}\\
		[z_3] = & \{ z_1, z_3, z_5 \} = [z_1]\\
		[z_4] = & \{ z_2, z_4 \} = [z_2]\\
		[z_5] = & \{ z_1, z_3, z_5 \} = [z_1]\\
		[z_6] = & \{ z_6\}
	\end{split} \]
\end{minipage}
\begin{minipage}{.45\textwidth}
	\begin{figure}[H]
		\centering
		\includestandalone[scale=1.1]{resources/ltx/clases1}
		\caption*{}
		\label{fig:clases1}
	\end{figure}
\end{minipage}

En la fig. se muestra, en un diagrama, las relaciones.
\label{ex:clases1}
\end{fmd-example}

\begin{fmd-definition}[Conjunto cociente] \index{conjunto!cociente}
	Sea $A$ un conjunto y $\mathcal{R}$ (o $\sim$)una relación de equivalencia en $A$. El \gls{conjuntocociente} se define por:
	\[ A/\mathcal{R} = \frac{A}{\sim} = \{ [a] / a \in A \}\]
\end{fmd-definition}

\begin{fmd-example}[Conjunto cociente]
	Del ejemplo \ref{ex:clases1}, el conjunto cociente es:
	\[ A/ \mathcal{R} = \frac{A}{\sim} = \{ [z_1], [z_2], [z_6] \} = \left\{ \{ z_1, z_3, z_5 \}, \{ z_2, z_4 \}, \{ z_6 \} \right\} \] 
\end{fmd-example}

\subsection{Particiones} \label{sec:particiones}
\vspace{3mm}

\begin{fmd-definition}[Partición] \index{partición}
	Dado un conjunto no vacío $A$, una \gls{particion} de $A$ es una colección finita o infinita de conjuntos de $A$ que cumplen:
	\[ A_i \cap A_j = \emptyset \; \forall \, i \ne j \quad \text{ y } \quad \cup_{i=1}^n A_i = A \]
\end{fmd-definition}

En la fig. \ref{fig:particion} se muestra en un diagrama, una partición de $A$.

\begin{minipage}{.45\textwidth}
	\begin{figure}[H]
		\centering
		\includestandalone[]{resources/ltx/particion}
		\caption{}
		\label{fig:particion}
	\end{figure}
\end{minipage}
\begin{minipage}{.45\textwidth}
	$A_i \cap A_j = \emptyset$, siempre que $i \ne j$
	
	$A_1 \cup A_2 \cup \cdots \cup A_5 = A$
\end{minipage}

\subsubsection{Teorema fundamental de las relaciones de equivalencia}
Toda relación de equivalencia definida en un conjunto no vacío determina una partición de éste en clases de equivalencia.

\begin{fmd-theorem}[Fundamental de las relaciones de equivalencia]
	Si $\sim$ es una relación de equivalencia definida en el conjunto $A \ne \emptyset$, entonces existe un subconjunto $I \subset \N$, tal que cualquiera sea $i \in I$, existe $A_i \subset A$, de modo que se verifican las siguientes proposiciones:
	\begin{enumerate}[label=\roman*)]
		\item $i \in I \implies A_i \ne \emptyset$;
		\item $a \sim b \iff a$ y $b$ pertenecen al mismo $A_i$;
		\item $A_i \cap A_j \ne \emptyset \implies A_i = A_j$;
		\item $i \ne j \implies A_i \cap A_j = \emptyset$;
		\item $\forall a \in A, \exists i \in I / a \in A_i$
	\end{enumerate}
\end{fmd-theorem}

\begin{lgnote}
	El símbolo $\iff$ se lee ``si y sólo si'', es la doble implicación que se explica con rigor en \ref{sec:bicondicional}.
\end{lgnote}

Las clases de equivalencia conforman una partición del conjunto $A$.

\begin{fmd-example}[Particiones]
	Del ejemplo \ref{ex:clases1}:
	
	\begin{minipage}{.45\textwidth}
		\begin{figure}[H]
			\centering
			\includestandalone[]{resources/ltx/clases2}
		\end{figure}
	\end{minipage}
	\begin{minipage}{.45\textwidth}
		\centering
		$A/\mathcal{R} = \{ [z_1], [z_2], [z_3] \}$
		\vspace{2mm}
		
		$[z_1] \cap [z_2] = \emptyset$, $[z_1] \cap [z_6] = \emptyset$, $[z_2] \cap [z_6] = \emptyset$
		\vspace{2mm}
		
		$[z_1] \cup [z_2] \cup [z_6] = A$
	\end{minipage}
\end{fmd-example}

\rule{\textwidth}{.5pt}

\subsection{Ejercicios}

\begin{enumerate}
	
	\item Sea $A$ un conjunto y sea $\{B_i\}_{i \in I}$ una partición de $A$. Demuestre que para cualquier conjunto $C$, se cumple que:
	\[ C \cap A = \bigcup_{i \in I} (C \cap B_i) \]
\end{enumerate}

\section{Funciones}
Esta sección está basada en \cite{rojoAlgebra8vaEd} cap. 4. \index{funciones}
\subsection{Relaciones funcionales} \label{sec:relaciones}

Sean $A$ y $B$ dos conjuntos no vacíos, que llamaremos \gls{dominio} y \gls{codominio} respectivamente. \par

Una \gls{funcion} de $A$ en $B$ asigna a cada elemento de $A$ un único elemento de $B$.

Para denotar que $f$ (o $g$, $h$, etc.) es una función de $A$ en $B$, se escribe:
\[f: A \rightarrow B\]
se lee: \emph{$f$ es una función o aplicación de $A$ en $B$}, o bien \emph{$f$ es una función con dominio $A$ y codominio $B$}.
\index{función}

\begin{fmd-example}[Función] \label{ex:funcion} \index{función}
	En particular, si $A = \{ -1, 0, 1, 2 \}$, $B = \{ 0, 1, 2, 3, 4 \}$ y $f$ es la relación
	\[ (x, y) \in f \iff y = x^2 \]
	se tiene (cada 2da. componente es el cuadrado de la 1ra.):
	\[ f = \{ (-1,1), (0,0), (1,1), (2,4) \} \]
	El diagrama de Venn correspondiente es:
	\begin{figure}[H]
		\centering
		\begin{tikzpicture}[scale=1]
			% Coordenadas de puntos
			\coordinate (A0) at (0.5,1.5);
			\coordinate (Am1) at (.5,.5);
			\coordinate (A1) at (.5,-.5);
			\coordinate (A2) at (.5,-1.5);
			
			\coordinate (B0) at (5.5,1.5);
			\coordinate (B1) at (5,.5);
			\coordinate (B2) at (6,0);
			\coordinate (B3) at (6.5,-.75);
			\coordinate (B4) at (6,-1.5);
			
			% Circunferencias
			\node at (-1.5, 2) {$A$};
			\node at (7.5, 2) {$B$};
			\draw[very thick] (0,0) circle (2);
			\draw[very thick] (6,0) circle (2);
			\node at (0, -2.5) {Dominio};
			\node at (6, -2.5) {Codominio};
			
			% Puntos
			\draw[fill] (A0) circle (1mm) node[left] {0};
			\draw[fill] (Am1) circle (1mm) node[left] {-1};
			\draw[fill] (A1) circle (1mm) node[left] {1};
			\draw[fill] (A2) circle (1mm) node[left] {2};
			
			\draw[fill] (B0) circle (1mm) node[right] {0};
			\draw[fill] (B1) circle (1mm) node[right] {1};
			\draw[fill] (B2) circle (1mm) node[right] {2};
			\draw[fill] (B3) circle (1mm) node[right] {3};
			\draw[fill] (B4) circle (1mm) node[right] {4};
			
			% flechas
			\draw[red, thick, -latex] (A0) to [bend left = 25] (B0);
			\draw[blue, thick, -latex] (Am1) to [bend left = 25] (B1);
			\draw[blue, thick, -latex] (A1) to [bend left = -25] (B1);
			\draw[red, thick, -latex] (A2) to [bend left = -25] (B4);
		\end{tikzpicture}
	\end{figure}
\end{fmd-example}


\begin{fmd-definition}[Función] \index{función}
	$f$ es una función o aplicación de $A$ en $B$ si y sólo si $f$ es una relación
	entre $A$ y $B$ (en consecuencia, $f \subset A \times B$), tal que todo elemento de $A$ tiene un único correspondiente en $B$.
\end{fmd-definition}

\textbf{Observaciones}:
\begin{itemize}
	\item Si $(a, b) \in f$ decimos que $b$ es el correspondiente o \textit{imagen} de $a$,
	por $f$, y suele escribirse $b = f(a)$, en otras palabras, $b$ es el transformado de $a$
	por la función $f$.
	\item Si $f$ es como arriba, una aplicación o un \textit{mapeo} de $A$ a $B$ a menudo se escribe $x \mapsto f(x)$ para denotar la imagen de $x$ por $f$. Por ejemplo: Si $A = \R$ y  $B = \R^{+}$, sea $f: \R \rightarrow \R^{+}$ la aplicación $f(x) = x^2$, significa el mapeo cuyo valor en $x$ es $x^2$. Podemos también decir: $f$ es la aplicación tal que $x \mapsto x^2$ ($x$ se mapea a $x^2$ o $x$ se transforma en $x^2$ o $x$ se aplica a $x^2$). En este caso la imagen de $f$ es el conjunto de números reales no negativos.
	\item Una función queda especificada si se da el dominio $A$, el codominio $B$,
	y además la relación $f \subset A \times B$, que satisface las condiciones de la definición.
	\item Por ser un conjunto, $f$ puede estar dado por \textit{extensión}, como conjunto de pares ordenados, o bien por \textit{comprensión}, mediante una fórmula o ley de correspondencia que permita asignar a cada objeto del dominio su imagen en el codominio.
\end{itemize}%

\begin{fmd-example}[Funciones]
	Determinamos si las siguientes relaciones son funciones.
	\begin{enumerate}
		\item Sean $A =\{a,b,c,d\}, B=\{1,2,3\}$ y la relación:
		\[ f = \{(a, 1), (b, 2), (c,3), (d,1)\} \]
		Se cumplen las condiciones de la definición, y resulta $f$ una función tal que:
		\[ f(a) = 1, \quad f(b) = 2, \quad f(c)=2, \quad f(d)=1 \]
		
		\item Con los mismos $A$ y $B$ la relación
		\[ \{(a,1),(a,2),(b,2),(c,1)\} \]
		\textit{No es una función}.
		\begin{itemize}
			\item No todo elemento en $A$ ($d$) tiene imagen en $B$;
			\item Un mismo elemento en $A$ ($a$) tiene dos imágenes en $B$ (1 y 2).
		\end{itemize}
		El diagrama de la relación es:
		\begin{figure}[H]
			\centering
			\begin{tikzpicture}[scale=.8]
				% Coordenadas de puntos
				\coordinate (a) at (0,1.5);
				\coordinate (b) at (0,.5);
				\coordinate (c) at (0,-.5);
				\coordinate (d) at (0,-1.5);
				
				\coordinate (B1) at (6,1);
				\coordinate (B2) at (6,0);
				\coordinate (B3) at (6,-1);
				
				% Circunferencias
				\node at (-1.5, 2) {$A$};
				\node at (7.5, 2) {$B$};
				\draw[very thick] (0,0) circle (2);
				\draw[very thick] (6,0) circle (2);
				
				% Puntos
				\draw[fill] (a) circle (1mm) node[left] {a};
				\draw[fill] (b) circle (1mm) node[left] {b};
				\draw[fill] (c) circle (1mm) node[left] {c};
				\draw[fill] (d) circle (1mm) node[left] {d};
				
				\draw[fill] (B1) circle (1mm) node[right] {1};
				\draw[fill] (B2) circle (1mm) node[right] {2};
				\draw[fill] (B3) circle (1mm) node[right] {3};
				
				% flechas
				\draw[red, thick, -latex] (a) to [bend left = 25] (B1);
				\draw[red, thick, -latex] (a) to [bend left = -25] (B2);
				\draw[blue, thick, -latex] (b) to [bend left = -35] (B2);
				\draw[blue, thick, -latex] (c) to [bend left = -25] (B1);
			\end{tikzpicture}
		\end{figure}
		
		\item Si $A$ es el conjunto de las personas y $f$ es la relación en $A$ definida por:
		\[(x, y) \in f \iff x \mbox{ es hijo de } y\]
		entonces $f$ es una función de $A$ en $A$, ya que toda persona tiene padre y
		este es único.
		
		En cambio la relación definida en el mismo $A$ mediante:
		\[ (x,y) \in f \iff x \mbox{ es padre de } y \]
		no es una función de $A$ en $A$, ya que existen en $A$ personas que no son padres,
		es decir, elementos del dominio que carecen de imagen en el codominio, por otra
		parte, tampoco se verifica la unicidad pues existen personas que son padres de más
		de un hijo.
	\end{enumerate}
\end{fmd-example}

\begin{lgnote}
	Si una relación es una función, la relación inversa no lo es necesariamente.
\end{lgnote}

\subsection{Representación cartesiana de funciones} \label{sec:cartesiano}

Las funciones pueden representarse mediante un sistema de coordenadas cartesianas
en el plano o en el espacio, según el dominio sea unidimensional o bidimensional
respectivamente. En el caso del plano, el dominio es un subconjunto del eje horizontal,
y el codominio, del eje vertical.

\begin{fmd-example} \label{frame:ejm2}
	Representación cartesiana de la función del ejemplo \ref{ex:funcion}.
	\[A = \{ -1, 0, 1, 2 \} \quad B = \{ 0, 1, 2, 3, 4 \}\]
	\[ f(x) = y = x^2 \]
	
	Si $g: \mathbb{R} \rightarrow \mathbb{R}^{+}_0 / g(x) = y = x^2$, resulta la parábola de la figura.
	
	\begin{figure}[H]
		\centering
		\begin{tikzpicture}[scale=1]
			\begin{axis}[domain=-2:2,
				y domain=0:4,
				ymin=-.9,
				xmax = 2.7,
				ymax = 4.5,
				axis lines = middle,
				xlabel = {$A, x$},
				ylabel = {$B, y$}]
				\addplot[blue] {x^2};
				\addplot [only marks] coordinates {(-1,1) (0,0) (1,1) (2,4)};
				\node [left] at (axis cs:-1,1) {(-1,1)};
				\node [below] at (axis cs:0,0) {(0,0)};
				\node [right] at (axis cs:1,1) {(1,1)};
				\node [right] at (axis cs:2,4) {(2,4)};
				\addplot[dashed] coordinates {(-1,0) (-1,1) (1,1) (1,0)};
				\addplot[dashed] coordinates {(0,4) (2,4) (2,0)};
			\end{axis}
		\end{tikzpicture}
		\label{fig:parabola}
	\end{figure}
\end{fmd-example}

\begin{fmd-example}
	Sea $f: \mathbb{Z} \rightarrow \mathbb{Z}$ tal que la imagen de cada entero es su
	opuesto aumentado en 1, es decir: $f(x) = -x + 1$.
	
	\begin{figure}[H]
		\centering
		\begin{tikzpicture}[scale=1]
			\begin{axis}[domain=-2:3,
				y domain=-2:3,
				xmin=-2.5,
				ymin=-2.5,
				xmax = 3.5,
				ymax = 3.5,
				axis lines = middle,
				xlabel = {$Z, R, x$},
				ylabel = {$Z, f, R, g$},
				xtick=false, ytick=false]
				\addplot [only marks] coordinates {(-2,3) (-1,2) (0,1) (1,0) (2,-1)
					(3, -2)};
				\node [above] at (axis cs:-2,3) {(-2,3)};
				\node [above right] at (axis cs:-1,2) {(-1,2)};
				\node [right] at (axis cs:0,1) {(0,1)};
				\node [above right] at (axis cs:1,0) {(1,0)};
				\node [below left] at (axis cs:2,-1) {(2,-1)};
				\node [below] at (axis cs:3,-2) {(3,-2)};
				\addplot[dashed] coordinates {(-2,0) (-2,3) (0,3)};
				\addplot[dashed] coordinates {(-1,0) (-1,2) (0,2)};
				\addplot[dashed] coordinates {(2,0) (2,-1) (0,-1)};
				\addplot[dashed] coordinates {(3,0) (3,-2) (0,-2)};
				
				\addplot[very thick, blue] {-x+1} node[pos=.1, below, black] {$g$};
			\end{axis}
		\end{tikzpicture}
		\label{fig:recta}
	\end{figure}
	
	Si $g: \mathbb{R} \rightarrow \mathbb{R}$ es tal que $g(x) = -x + 1$. Su representación 
	es un conjunto continuo de $\mathbb{R}^2$, consistente en una recta del plano.
	
	Notar que $f \ne g$ aunque $f \subset g$.
\end{fmd-example}

\begin{example}[Función de dos variables]
	\
	
	Consideremos $A = \{1, 2\}$, $B = \{1,2,3,4\}$ y la función:
	\[ f: A^2 \rightarrow B \]
	que asigna a cada elemento del dominio $A^2$, la suma de sus componentes:
	\[ f(x, y) = x + y \]
	\begin{enumerate}
		\item Representación en una tabla de simple entrada.
		\begin{table}[H]
			\centering
			\begin{tabular}{c|c}
				$(x, y)$ & $f(x, y) = x + y$\\ \hline
				$(1,1)$ & 2\\
				$(1,2)$ & 3\\
				$(2,1)$ & 3\\
				$(2,2)$ & 4
			\end{tabular}
		\end{table}
		El elemento 1 de $B$ carece de antecedente o \gls{preimagen}\footnote{Ver definición \ref{def:preimagen}, pág. \pageref{def:preimagen}.}  en $A$.
	\end{enumerate}
	
	\begin{enumerate}
		\setcounter{enumi}{1}
		\item Representación en una tabla de doble entrada.
		\begin{table}[H]
			\centering
			\begin{tabular}{c|cc}
				$f$ & 1 & 2\\ \hline
				1 & 2 & 3\\
				2 & 3 & 4
			\end{tabular}
		\end{table}
		\item El diagrama de Venn es:
		\begin{figure}[H]
			\centering
			\begin{tikzpicture}[scale=1]
				% Coordenadas de puntos
				\coordinate (a) at (0,1.5);
				\coordinate (b) at (0,.5);
				\coordinate (c) at (0,-.5);
				\coordinate (d) at (0,-1.5);
				
				\coordinate (B1) at (6,1.5);
				\coordinate (B2) at (6,.5);
				\coordinate (B3) at (6,-.5);
				\coordinate (B4) at (6,-1.5);
				
				% Circunferencias
				\node at (-1.5, 2) {$A^2$};
				\node at (7.5, 2) {$B$};
				\draw[very thick] (0,0) circle (2);
				\draw[very thick] (6,0) circle (2);
				\draw[fill=black!20] (B3) ellipse (.6 and 1.5);
				\node at (0, -2.5) {Dominio};
				\node at (6, -2.5) {Codominio};
				\node at (8.8, -1.2) {Rango};
				\draw[latex-] ($(B3) + (.6,0)$) to [bend left = 20] (8,-1.1);
				
				% Puntos
				\draw[fill] (a) circle (1mm) node[left] {$(1,1)$};
				\draw[fill] (b) circle (1mm) node[left] {$(1,2)$};
				\draw[fill] (c) circle (1mm) node[left] {$(2,1)$};
				\draw[fill] (d) circle (1mm) node[left] {$(2,2)$};
				
				\draw[fill] (B1) circle (1mm) node[right] {1};
				\draw[fill] (B2) circle (1mm) node[right] {2};
				\draw[fill] (B3) circle (1mm) node[right] {3};
				\draw[fill] (B4) circle (1mm) node[right] {4};
				
				% flechas
				\draw[red, thick, -latex] (a) to [bend left = 25] (B2);
				\draw[red, thick, -latex] (b) to [bend left = 15] (B3);
				\draw[blue, thick, -latex] (c) to [bend left = -35] (B3);
				\draw[blue, thick, -latex] (d) to [bend left = -25] (B4);
			\end{tikzpicture}
		\end{figure}
	\end{enumerate}
	
	\begin{enumerate}
		\setcounter{enumi}{3}
		\item Representación cartesiana en el espacio.
		\begin{figure}[H]
			\centering
			\begin{tikzpicture}[scale=1]
				\begin{axis}[axis lines = middle,
					xmin=0, xmax=2.5,
					ymin=0, ymax=2.5, zmax=5,
					xlabel={$x$}, ylabel={$y$}, zlabel={$f(x,y)$}]
					\addplot3[only marks] coordinates {(1,1,2) (1,2,3) (2,1,3) (2,2,4)};
					\addplot3[dashed] coordinates {(1,0,0) (1,1,0) (0,1,0)};
					\addplot3[dashed] coordinates {(1,1,0) (1,1,2)} node[above] {$(1,1,2)$};
					\addplot3[dashed] coordinates {(1,0,0) (1,2,0) (0,2,0)};
					\addplot3[dashed] coordinates {(1,2,0) (1,2,3)} node[above] {$(1,2,3)$};
					\addplot3[dashed] coordinates {(2,0,0) (2,1,0) (0,1,0)};
					\addplot3[dashed] coordinates {(2,1,0) (2,1,3)} node[above] {$(2,1,3)$};
					\addplot3[dashed] coordinates {(2,0,0) (2,2,0) (0,2,0)};
					\addplot3[dashed] coordinates {(2,2,0) (2,2,4)} node[above] {$(2,2,4)$};
				\end{axis}
			\end{tikzpicture}
		\end{figure}
	\end{enumerate}
	
	\begin{enumerate}
		\setcounter{enumi}{4}
		\item La misma función puede representarse de la siguiente manera, desconectando
		el dominio del codominio.
		\begin{figure}[H]
			\centering
			\begin{tikzpicture}[scale=1]
				% Puntos
				\coordinate (a) at (1,1);
				\coordinate (b) at (2,1);
				\coordinate (c) at (1,2);
				\coordinate (d) at (2,2);
				
				\coordinate (r) at (4,.2);
				\coordinate (B1) at ($(r) + (1,1)$);
				\coordinate (B2) at ($(r) + (2,2)$);
				\coordinate (B3) at ($(r) + (3,3)$);
				\coordinate (B4) at ($(r) + (4,4)$);
				
				% Ejes
				\draw[-latex] (-1,0) -- (3,0) node[below]{$x$};
				\node[below] at (1,0) {1};
				\node[below] at (2,0) {2};
				\node[left] at (0,1) {1};
				\node[left] at (0,2) {2};
				\draw[-latex] (0,-1) -- (0,3) node[left]{$y$};
				
				% Puntos
				\draw[fill] (a) circle (2pt);
				\draw[fill] (b) circle (2pt);
				\draw[fill] (c) circle (2pt);
				\draw[fill] (d) circle (2pt);
				
				% Dashes
				\draw[dashed] (1, 0) -- (1, 3); 
				\draw[dashed] (2, 0) -- (2, 3); 
				\draw[dashed] (0, 1) -- (3, 1); 
				\draw[dashed] (0, 2) -- (3, 2);
				
				% Circunferencia
				\draw[blue] (1.5,1.5) circle (1);
				\node[right] at (.3,2.5) {$A^2$};
				
				% Recta
				\draw[red, very thick, -latex] (4, .2) -- (8.5, 4.7) node[left, pos=.98]
				{$B$};
				\draw[fill] (B1) circle (2pt) node[right]{1};
				\draw[fill] (B2) circle (2pt) node[right]{2};
				\draw[fill] (B3) circle (2pt) node[right]{3};
				\draw[fill] (B4) circle (2pt) node[right]{4};
				
				% flechas
				\draw[-latex] (a) to [bend left = 25] (B2);
				\draw[-latex] (b) to [bend left = 15] (B3);
				\draw[-latex] (c) to [bend left = 25] (B3);
				\draw[-latex] (d) to [bend left = 25] (B4);
			\end{tikzpicture}
		\end{figure}
	\end{enumerate}
\end{example}

\subsection{Clasificación de funciones} \label{sec:clasif}
Sea una función $f: A \rightarrow B$
\begin{itemize}
	\item Si ocurre que elementos distintos del dominio tienen imágenes distintas 
	en el codominio, entonces $f$ se llama \gls{funcioninyectiva} o uno a uno.
	\index{función!inyectiva}
	\item Por otra parte, si todo elemento del codominio es imagen de algún elemento 
	del dominio, se llama \gls{funcionsobreyectiva}. \index{función!sobreyectiva}
	\item Cuando se presentan ambas situaciones simultáneamente, se llama \gls{funcionbiyectiva} o correspondencia biunívoca. \index{función!biyectiva}
\end{itemize}

\subsubsection{Función inyectiva}
\vspace{1em} \index{función!inyectiva}
\begin{fmd-definition}[Función inyectiva]
	Una función $f: A \rightarrow B$ es inyectiva si, y solo si, para cualquier par de elementos distintos $x'$ y $x''$ en el conjunto $A$, sus imágenes bajo la función $f$ también son distintas.
\end{fmd-definition}
\vspace{1mm}

O, de forma equivalente:

Una función $f: A \rightarrow B$ es inyectiva si, y solo si, cuando dos elementos cualesquiera $x', x'' \in A$ tienen la misma imagen, $f(x') = f(x'')$ entonces esos dos elementos deben ser el mismo elemento ($x' = x''$).

\begin{itemize}
	\item En la inyectividad no puede darse que elementos distintos del dominio den la misma
	imagen, fig. \ref{fig:inyectiva_discreta}.
	\item En el diagrama de Venn no puede presentarse ninguna bifurcación de elementos del 
	dominio hacia el codominio.
	\item En la representación plana cartesiana no puede ocurrir que una ordenada
	corresponda a más de una abscisa, fig. \ref{fig:inyectiva_continua}.
\end{itemize}

% TODO: pasar cada gráfica a standalone para llamarlas y referenciarlas independientemente.
\begin{figure}[H]
	\centering
	\begin{minipage}{.3\textwidth}
		\centering
		\begin{tikzpicture}[scale=.7]
			% Coordenadas
			\coordinate (A) at (0,0);
			\coordinate (A1) at (0,1.5);
			\coordinate (A2) at (0,.5);
			\coordinate (A3) at (0,-.5);
			\coordinate (A4) at (0,-1.5);
			
			\coordinate (B) at (3,0);
			\coordinate (B1) at (3,1.5);
			\coordinate (B2) at (3,.5);
			\coordinate (B3) at (3,-.5);
			\coordinate (B4) at (3,-1.5);
			
			% Elipses
			\draw[] (A) ellipse (1 and 2);
			\draw[] (B) ellipse (1 and 2);
			
			% Elementos
			\fill[black] (A1) circle (3pt);
			\fill[black] (A2) circle (3pt);
			\fill[black] (A3) circle (3pt);
			\fill[black] (A4) circle (3pt);
			
			\fill[black] (B1) circle (3pt);
			\fill[black] (B2) circle (3pt);
			\fill[black] (B3) circle (3pt);
			\fill[black] (B4) circle (3pt);
			
			% Flechas
			\draw[-latex, red] (A1) to [bend left = 10] (B1);
			\draw[-latex, red] (A2) to [bend left = 10] (B2);
			\draw[-latex, red] (A3) to [bend left = 10] (B3);
			\draw[-latex, red] (A4) to [bend left = 10] (B4);
			
			% Caption
			\node at (1.5,-2.7) {Sí es inyectiva};
		\end{tikzpicture}
	\end{minipage}
	\begin{minipage}{.3\textwidth}
		\centering
		\begin{tikzpicture}[scale=.7]
			% Coordenadas
			\coordinate (A) at (0,0);
			\coordinate (A1) at (0,1.5);
			\coordinate (A2) at (0,.5);
			\coordinate (A3) at (0,-.5);
			\coordinate (A4) at (0,-1.5);
			
			\coordinate (B) at (3,0);
			\coordinate (B1) at (3,1.5);
			\coordinate (B2) at (3,.5);
			\coordinate (B3) at (3,-.5);
			\coordinate (B4) at (3,-1.5);
			
			% Elipses
			\draw[] (A) ellipse (1 and 2);
			\draw[] (B) ellipse (1 and 2);
			
			% Elementos
			\fill[black] (A1) circle (3pt);
			\fill[black] (A2) circle (3pt);
			\fill[black] (A3) circle (3pt);
			\fill[black] (A4) circle (3pt);
			
			\fill[black] (B1) circle (3pt);
			\fill[black] (B2) circle (3pt);
			\fill[black] (B3) circle (3pt);
			\fill[black] (B4) circle (3pt);
			
			% Flechas
			\draw[-latex, red] (A1) to [bend left = 10] (B1);
			\draw[-latex, red] (A2) to [bend left = 10] (B1);
			\draw[-latex, red] (A3) to [bend left = 10] (B3);
			\draw[-latex, red] (A4) to [bend left = 10] (B4);
			
			% Caption
			\node at (1.5,-2.7) {No es inyectiva};
		\end{tikzpicture}
	\end{minipage}
	\begin{minipage}{.3\textwidth}
		\centering
		\begin{tikzpicture}[scale=.7]
			% Coordenadas
			\coordinate (A) at (0,0);
			\coordinate (A1) at (0,1.5);
			\coordinate (A2) at (0,.5);
			\coordinate (A3) at (0,-.5);
			
			\coordinate (B) at (3,0);
			\coordinate (B1) at (3,1.5);
			\coordinate (B2) at (3,.5);
			\coordinate (B3) at (3,-.5);
			\coordinate (B4) at (3,-1.5);
			
			% Elipses
			\draw[] (A) ellipse (1 and 2);
			\draw[] (B) ellipse (1 and 2);
			
			% Elementos
			\fill[black] (A1) circle (3pt);
			\fill[black] (A2) circle (3pt);
			\fill[black] (A3) circle (3pt);
			
			\fill[black] (B1) circle (3pt);
			\fill[black] (B2) circle (3pt);
			\fill[black] (B3) circle (3pt);
			\fill[black] (B4) circle (3pt);
			
			% Flechas
			\draw[-latex, red] (A1) to [bend left = 10] (B1);
			\draw[-latex, red] (A2) to [bend left = 10] (B2);
			\draw[-latex, red] (A3) to [bend left = 10] (B3);
			
			% Caption
			\node at (1.5,-2.7) {Sí es inyectiva};
		\end{tikzpicture}
	\end{minipage}
	\caption{Identificación de funciones inyectivas discretas}
	\label{fig:inyectiva_discreta}
\end{figure}

\begin{figure}[H]
	\centering
	\begin{minipage}{.45\textwidth}
		\centering
		\begin{figure}[H]
			\centering
			\begin{tikzpicture}[scale=.5]
				\begin{axis}[axis lines=middle, xticklabels={}, yticklabels={}]
					\addplot[blue, very thick] {x^3};
				\end{axis}
			\end{tikzpicture}
			\caption*{$f:\mathbb{R} \rightarrow \mathbb{R} / f(x) = x^3$\\Sí es inyectiva}
		\end{figure}
	\end{minipage}
	\begin{minipage}{.45\textwidth}
		\centering
		\begin{figure}[H]
			\centering
			\begin{tikzpicture}[scale=.5]
				\begin{axis}[axis lines=middle, xticklabels={}, yticklabels={}]
					\addplot[blue, very thick] {x^2};
				\end{axis}
			\end{tikzpicture}
			\caption*{$f:\mathbb{R} \rightarrow \mathbb{R}_0^{+} / f(x) = x^2$\\No es inyectiva}
		\end{figure}
	\end{minipage}
	\caption{Identificación de funciones inyectivas continuas.}
	\label{fig:inyectiva_continua}
\end{figure}

\subsubsection{Función sobreyectiva} \index{función!sobreyectiva}
\vspace{1em}
\begin{fmd-definition}[Función sobreyectiva]
	Una función $f: A \rightarrow B$ es \textit{sobreyectiva} si, y solo si, para cada elemento $y \in B$, existe al menos un elemento $x \in A$ tal que $y = f(x)$.
\end{fmd-definition}
\begin{itemize}
	\item El conjunto de las imágenes (\gls{rango} o \textit{recorrido}) \index{rango} se identifica con el codominio de la función. No hay elementos del codominio que no sea la imagen de \textit{al menos un} elemento del dominio, en otras palabras, el rango y el codominio coinciden, figs. \ref{fig:sobreyectiva_discreta} y \ref{fig:sobreyectiva_continua}.
	\item Es usual nombrar a las funciones sobreyectivas como ``sobre'' o ``suryectiva''
\end{itemize}

\begin{figure}[H]
	\begin{minipage}{.32\textwidth}
		\centering
		\begin{tikzpicture}[scale=.7]
			% Coordenadas
			\coordinate (A) at (0,0);
			\coordinate (A1) at (0,1.5);
			\coordinate (A2) at (0,.5);
			\coordinate (A3) at (0,-.5);
			\coordinate (A4) at (0,-1.5);
			
			\coordinate (B) at (3,0);
			\coordinate (B1) at (3,1.5);
			\coordinate (B2) at (3,.5);
			\coordinate (B3) at (3,-.5);
			\coordinate (B4) at (3,-1.5);
			
			% Elipses
			\draw[] (A) ellipse (1 and 2);
			\draw[] (B) ellipse (1 and 2);
			
			% Elementos
			\fill[black] (A1) circle (3pt);
			\fill[black] (A2) circle (3pt);
			\fill[black] (A3) circle (3pt);
			\fill[black] (A4) circle (3pt);
			
			\fill[black] (B1) circle (3pt);
			\fill[black] (B2) circle (3pt);
			\fill[black] (B3) circle (3pt);
			\fill[black] (B4) circle (3pt);
			
			% Flechas
			\draw[-latex, red] (A1) to [bend left = 10] (B1);
			\draw[-latex, red] (A2) to [bend left = 10] (B2);
			\draw[-latex, red] (A3) to [bend left = 10] (B3);
			\draw[-latex, red] (A4) to [bend left = 10] (B4);
			
			% Caption
			\node at (1.5,-2.7) {Sí es sobreyectiva};
		\end{tikzpicture}
	\end{minipage}
	\begin{minipage}{.32\textwidth}
		\centering
		\begin{tikzpicture}[scale=.7]
			% Coordenadas
			\coordinate (A) at (0,0);
			\coordinate (A1) at (0,1.5);
			\coordinate (A2) at (0,.5);
			\coordinate (A3) at (0,-.5);
			\coordinate (A4) at (0,-1.5);
			
			\coordinate (B) at (3,0);
			\coordinate (B1) at (3,1.5);
			\coordinate (B2) at (3,.5);
			\coordinate (B3) at (3,-.5);
			\coordinate (B4) at (3,-1.5);
			
			% Elipses
			\draw[] (A) ellipse (1 and 2);
			\draw[] (B) ellipse (1 and 2);
			
			% Elementos
			\fill[black] (A1) circle (3pt);
			\fill[black] (A2) circle (3pt);
			\fill[black] (A3) circle (3pt);
			\fill[black] (A4) circle (3pt);
			
			\fill[black] (B1) circle (3pt);
			%\fill[black] (B2) circle (3pt);
			\fill[black] (B3) circle (3pt);
			\fill[black] (B4) circle (3pt);
			
			% Flechas
			\draw[-latex, red] (A1) to [bend left = 10] (B1);
			\draw[-latex, red] (A2) to [bend left = 10] (B1);
			\draw[-latex, red] (A3) to [bend left = 10] (B3);
			\draw[-latex, red] (A4) to [bend left = 10] (B4);
			
			% Caption
			\node at (1.5,-2.7) {Sí es sobreyectiva};
		\end{tikzpicture}
	\end{minipage}
	\begin{minipage}{.32\textwidth}
		\centering
		\begin{tikzpicture}[scale=.7]
			% Coordenadas
			\coordinate (A) at (0,0);
			\coordinate (A1) at (0,1.5);
			\coordinate (A2) at (0,.5);
			\coordinate (A3) at (0,-.5);
			
			\coordinate (B) at (3,0);
			\coordinate (B1) at (3,1.5);
			\coordinate (B2) at (3,.5);
			\coordinate (B3) at (3,-.5);
			\coordinate (B4) at (3,-1.5);
			
			% Elipses
			\draw[] (A) ellipse (1 and 2);
			\draw[] (B) ellipse (1 and 2);
			
			% Elementos
			\fill[black] (A1) circle (3pt);
			\fill[black] (A2) circle (3pt);
			\fill[black] (A3) circle (3pt);
			
			\fill[black] (B1) circle (3pt);
			\fill[black] (B2) circle (3pt);
			\fill[black] (B3) circle (3pt);
			\fill[black] (B4) circle (3pt);
			
			% Flechas
			\draw[-latex, red] (A1) to [bend left = 10] (B1);
			\draw[-latex, red] (A2) to [bend left = 10] (B2);
			\draw[-latex, red] (A3) to [bend left = 10] (B3);
			
			% Caption
			\node at (1.5,-2.7) {No es sobreyectiva};
		\end{tikzpicture}
	\end{minipage}
	\caption{Función sobreyectiva, caso discreto.}
	\label{fig:sobreyectiva_discreta}
\end{figure}

\begin{figure}[H]
	\centering
	\begin{minipage}{.45\textwidth}
		\centering
		\begin{figure}[H]
			\centering
			\begin{tikzpicture}[scale=.5]
				\begin{axis}[axis lines=middle, xticklabels={}, yticklabels={}]
					\addplot[blue, very thick] {x^3};
				\end{axis}
			\end{tikzpicture}
			\caption*{$f:\mathbb{R} \rightarrow \mathbb{R} / f(x) = x^3$\\Sí es sobreyectiva}
		\end{figure}
	\end{minipage}
	\begin{minipage}{.45\textwidth}
		\centering
		\begin{figure}[H]
			\centering
			\begin{tikzpicture}[scale=.5]
				\begin{axis}[axis lines=middle, xticklabels={}, yticklabels={}]
					\addplot[blue, very thick] {x^2};
				\end{axis}
			\end{tikzpicture}
			\caption*{$f:\mathbb{R} \rightarrow \mathbb{R}_0^{+} / f(x) = x^2$\\Sí es sobreyectiva}
		\end{figure}
	\end{minipage}
	\caption{Ejemplos de funciones sobreyectivas continuas.}
	\label{fig:sobreyectiva_continua}
\end{figure}

\subsubsection{Función biyectiva}
\vspace{1em} \index{función!biyectiva}
\begin{fmd-definition}[Función biyectiva]
	$f:A \rightarrow B \mbox{ es biyectiva si } f \mbox{ es inyectiva y sobreyectiva}$.
\end{fmd-definition}

\begin{lgnote}
	Si $f: A \rightarrow B$ no es biyectiva, entonces no es inyectiva o no es sobreyectiva, figuras \ref{fig:biyectiva_discreta} y \ref{fig:biyectiva_continua}.
\end{lgnote}

\begin{figure}[H]
	\centering
	\begin{minipage}{.32\textwidth}
		\centering
		\begin{tikzpicture}[scale=.7]
			% Coordenadas
			\coordinate (A) at (0,0);
			\coordinate (A1) at (0,1.5);
			\coordinate (A2) at (0,.5);
			\coordinate (A3) at (0,-.5);
			\coordinate (A4) at (0,-1.5);
			
			\coordinate (B) at (3,0);
			\coordinate (B1) at (3,1.5);
			\coordinate (B2) at (3,.5);
			\coordinate (B3) at (3,-.5);
			\coordinate (B4) at (3,-1.5);
			
			% Elipses
			\draw[] (A) ellipse (1 and 2);
			\draw[] (B) ellipse (1 and 2);
			
			% Elementos
			\fill[black] (A1) circle (3pt);
			\fill[black] (A2) circle (3pt);
			\fill[black] (A3) circle (3pt);
			\fill[black] (A4) circle (3pt);
			
			\fill[black] (B1) circle (3pt);
			\fill[black] (B2) circle (3pt);
			\fill[black] (B3) circle (3pt);
			\fill[black] (B4) circle (3pt);
			
			% Flechas
			\draw[-latex, red] (A1) to [bend left = 10] (B1);
			\draw[-latex, red] (A2) to [bend left = 10] (B2);
			\draw[-latex, red] (A3) to [bend left = 10] (B3);
			\draw[-latex, red] (A4) to [bend left = 10] (B4);
			
			% Caption
			\node at (1.5,-2.7) {Sí es biyectiva};
		\end{tikzpicture}
	\end{minipage}
	\begin{minipage}{.32\textwidth}
		\centering
		\begin{tikzpicture}[scale=.7]
			% Coordenadas
			\coordinate (A) at (0,0);
			\coordinate (A1) at (0,1.5);
			\coordinate (A2) at (0,.5);
			\coordinate (A3) at (0,-.5);
			\coordinate (A4) at (0,-1.5);
			
			\coordinate (B) at (3,0);
			\coordinate (B1) at (3,1.5);
			\coordinate (B2) at (3,.5);
			\coordinate (B3) at (3,-.5);
			\coordinate (B4) at (3,-1.5);
			
			% Elipses
			\draw[] (A) ellipse (1 and 2);
			\draw[] (B) ellipse (1 and 2);
			
			% Elementos
			\fill[black] (A1) circle (3pt);
			\fill[black] (A2) circle (3pt);
			\fill[black] (A3) circle (3pt);
			\fill[black] (A4) circle (3pt);
			
			\fill[black] (B1) circle (3pt);
			\fill[black] (B2) circle (3pt);
			\fill[black] (B3) circle (3pt);
			\fill[black] (B4) circle (3pt);
			
			% Flechas
			\draw[-latex, red] (A1) to [bend left = 10] (B1);
			\draw[-latex, red] (A2) to [bend left = 10] (B1);
			\draw[-latex, red] (A3) to [bend left = 10] (B3);
			\draw[-latex, red] (A4) to [bend left = 10] (B4);
			
			% Caption
			\node at (1.5,-2.7) {No es inyectiva};
		\end{tikzpicture}
	\end{minipage}
	\begin{minipage}{.32\textwidth}
		\centering
		\begin{tikzpicture}[scale=.7]
			% Coordenadas
			\coordinate (A) at (0,0);
			\coordinate (A1) at (0,1.5);
			\coordinate (A2) at (0,.5);
			\coordinate (A3) at (0,-.5);
			
			\coordinate (B) at (3,0);
			\coordinate (B1) at (3,1.5);
			\coordinate (B2) at (3,.5);
			\coordinate (B3) at (3,-.5);
			\coordinate (B4) at (3,-1.5);
			
			% Elipses
			\draw[] (A) ellipse (1 and 2);
			\draw[] (B) ellipse (1 and 2);
			
			% Elementos
			\fill[black] (A1) circle (3pt);
			\fill[black] (A2) circle (3pt);
			\fill[black] (A3) circle (3pt);
			
			\fill[black] (B1) circle (3pt);
			\fill[black] (B2) circle (3pt);
			\fill[black] (B3) circle (3pt);
			\fill[black] (B4) circle (3pt);
			
			% Flechas
			\draw[-latex, red] (A1) to [bend left = 10] (B1);
			\draw[-latex, red] (A2) to [bend left = 10] (B2);
			\draw[-latex, red] (A3) to [bend left = 10] (B3);
			
			% Caption
			\node at (1.5,-2.7) {No es sobreyectiva};
		\end{tikzpicture}
	\end{minipage}
	\caption{Identificación de función biyectiva discreta.}
	\label{fig:biyectiva_discreta}
\end{figure}

\begin{figure}[H]
	\centering
	\begin{minipage}{.45\textwidth}
		\centering
		\begin{figure}[H]
			\centering
			\begin{tikzpicture}[scale=.5]
				\begin{axis}[axis lines=middle, xticklabels={}, yticklabels={}]
					\addplot[blue, very thick] {x^3};
				\end{axis}
			\end{tikzpicture}
			\caption*{$f:\mathbb{R} \rightarrow \mathbb{R} / f(x) = x^3$\\Sí es biyectiva}
		\end{figure}
	\end{minipage}
	\begin{minipage}{.45\textwidth}
		\centering
		\begin{figure}[H]
			\centering
			\begin{tikzpicture}[scale=.5]
				\begin{axis}[axis lines=middle, xticklabels={}, yticklabels={}]
					\addplot[blue, very thick] {x^2};
				\end{axis}
			\end{tikzpicture}
			\caption*{$f:\mathbb{R} \rightarrow \mathbb{R}_0^{+} / f(x) = x^2$\\No es biyectiva}
		\end{figure}
	\end{minipage}
	\caption{Función biyectiva continua.}
	\label{fig:biyectiva_continua}
\end{figure}

\begin{fmd-example}
	Probar la inyectividad de $f$, siendo $ f: \mathbb{N} \rightarrow \mathbb{N} \mbox{ tal que } f(x) = 2x$
	\begin{figure}[H]
		\centering
		\begin{tikzpicture}[scale=1]
			\begin{axis}[axis lines=middle, xmin=0, xmax=4.5,
				ymin=0, ymax=4.5, xlabel={$\mathbb{N}$}, ylabel={$\mathbb{N}$}]
				
				% Verticales
				\addplot[dashed] coordinates {(1,0) (1,4)};
				\addplot[dashed] coordinates {(2,0) (2,4)};
				\addplot[dashed] coordinates {(3,0) (3,4)};
				\addplot[dashed] coordinates {(4,0) (4,4)};
				
				% Horizontales
				\addplot[dashed] coordinates {(0,1) (4,1)};
				\addplot[dashed] coordinates {(0,2) (4,2)};
				\addplot[dashed] coordinates {(0,3) (4,3)};
				\addplot[dashed] coordinates {(0,4) (4,4)};
				
				% Puntos
				\addplot[only marks, mark size=4pt] coordinates {(1,2) (2,4)};
			\end{axis}
		\end{tikzpicture}
	\end{figure}
	\begin{itemize}
		\item Sean $x'$ y $x''$ en $\mathbb{N}$ tales que $f(x') = f(x'')$, entonces
		$2x' = 2 x''$, en consecuencia $x'= x''$. De modo que $f$ es inyectiva uno a uno.
		\item $f$ no es sobreyectiva, pues los elementos impares del codominio, carecen
		de antecedente. Resulta que $f$ no es biyectiva.
	\end{itemize}
	Si se utiliza el conjunto $P$ de los números naturales pares y
	$ f: \mathbb{N} \rightarrow \mathbb{P} \mbox{ tal que } f(x) = 2x$, $f$ resulta
	biyectiva.
\end{fmd-example}

\begin{fmd-example}
	Sean $A = \{ 1,2,3 \}$ y $B = \{ 1,2 \}$.
	
	Definimos $f: P(A) \rightarrow P(B)$ mediante $f(X) = X \cap B$, la imagen
	de todo subconjunto de $A$ es su intersección con $B$.
	
	El diagrama muestra que $f$ es sobreyectiva pero no inyectiva.
	
	\begin{figure}[H]
		\centering
		\begin{tikzpicture}[scale=.6]
			% Coordenadas de puntos
			\coordinate (a) at (0,3);
			\coordinate (b) at (0,2);
			\coordinate (c) at (0,1);
			\coordinate (d) at (0,0);
			\coordinate (e) at (0,-1);
			\coordinate (f) at (0,-2);
			\coordinate (g) at (0,-3);
			\coordinate (h) at (1,-2.5);
			
			\coordinate (B1) at (8,1.5);
			\coordinate (B2) at (8,.5);
			\coordinate (B3) at (8,-.5);
			\coordinate (B4) at (8,-1.5);
			
			% Circunferencias
			\node at (-4.5, 2) {$P(A)$};
			\node at (10, 2) {$P(B)$};
			\draw[very thick] (0,0) circle (4);
			\draw[very thick] (8,0) circle (2);
			
			% Puntos
			\draw[fill] (a) circle (1mm) node[left] {$\varnothing$};
			\draw[fill] (b) circle (1mm) node[left] {$\{1\}$};
			\draw[fill] (c) circle (1mm) node[left] {$\{2\}$};
			\draw[fill] (d) circle (1mm) node[left] {$\{3\}$};
			\draw[fill] (e) circle (1mm) node[left] {$\{1,2\}$};
			\draw[fill] (f) circle (1mm) node[left] {$\{1,3\}$};
			\draw[fill] (g) circle (1mm) node[left] {$\{2,3\}$};
			\draw[fill] (h) circle (1mm) node[left] {$A$};
			
			\draw[fill] (B1) circle (1mm) node[right] {$\varnothing$};
			\draw[fill] (B2) circle (1mm) node[right] {$\{1\}$};
			\draw[fill] (B3) circle (1mm) node[right] {$\{2\}$};
			\draw[fill] (B4) circle (1mm) node[right] {$B$};
			
			% flechas
			\draw[blue, thick, -latex] (a) to [bend left = 25] (B1);
			\draw[blue, thick, -latex] (b) to [bend left = 15] (B2);
			\draw[blue, thick, -latex] (c) to [bend left = 20] (B3);
			\draw[blue, thick, -latex] (d) to [bend left = -25] (B1);
			\draw[blue, thick, -latex] (e) to [bend left = 10] (B4);
			\draw[blue, thick, -latex] (f) to [bend left = -25] (B2);
			\draw[blue, thick, -latex] (g) to [bend left = -25] (B3);
			\draw[blue, thick, -latex] (h) to [bend left = -25] (B4);
		\end{tikzpicture}
	\end{figure}
\end{fmd-example}

\begin{fmd-example}[Función biyectiva]
	Sea $f: \mathbb{R} \rightarrow \mathbb{R}$ definida por $f(x) = x^3$.
	\begin{enumerate}
		\item $f$ es 1-1. En efecto, sean $x_1$ y $x_2$ en $\mathbb{R}$ tales que
		$f(x_1) = f(x_2)$, esto significa: $x_1^3 = x_2^3$, o $x_1^3 - x_2^3 = 0$,
		factorizando: \[(x_1 - x_2)(x_1^2 + x_1x_2 + x_2^2) = 0\]
		\[ x_1 - x_2 = 0 \implies x_1 = x_2 \]
		de $x_1^2 + x_1x_2 + x_2^2=0$ se tiene:
		\[ x_1 = \frac{-x_2 \pm \sqrt{x_2^2 - 4 x_2} }{2} =
		\frac{-x_2 \pm \sqrt{-3x_2^2}}{2} \]
		De donde:
		\begin{equation} \label{eq:x1}
			x_1 = \left( -\frac{1}{2} \pm i \frac{\sqrt{3}}{2} \right) x_2
		\end{equation}
		Si $x_2=0$ entonces $x_1 = 0$ y resulta $x_1 = x_2$, los cuales son los únicos
		valores reales que satisfacen \eqref{eq:x1}, en consecuencia $f$ \textbf{es inyectiva}.
		
		\item $f$ \textbf{es sobreyectiva}, pues
		\[ \forall y \in \mathbb{R}, \exists \ x = \sqrt[3]{y} \mbox{ tal que }
		f(x) = f(\sqrt[3]{y}) = \left( \sqrt[3]{y} \right)^3 = y\]
		Ocurre entonces que $f$ \textbf{es biyectiva}.
	\end{enumerate}
\end{fmd-example}

\subsection{Funciones especiales} \label{sec:especiales}
\index{funciones!especiales}
\begin{itemize}
	\item \textbf{Función constante} \index{función!constante}
	
	La función $f: A \rightarrow B$, que asigna a todos los elementos del dominio el
	elemento $b \in B$, se llama constante.
	
	Está definida por $f(x) = b$ para todo $x \in A$, se tiene:
	\[ f = \{ (x, b) / x \in A \} \]
	
	A menos que $A$ sea unitario, la función constante \textbf{no es inyectiva}, y
	\textbf{es sobreyectiva} si $B$ se reduce a un único elemento, fig. \ref{fig:funcion_constante}.
\end{itemize}
\begin{figure}[H]
	\centering
	\begin{tikzpicture}
		\begin{axis}[axis lines=middle, xlabel={$A$}, ylabel={$B$},
			xticklabels={}, yticklabels={}, ymin=-.5, ymax=2.5,
			scale=.6, xmin=-2.5, xmax=2.5]
			\addplot[thick] coordinates {(-2, 1) (2,1)};
			\node[above] at (40, 150) {$b$};
		\end{axis}
	\end{tikzpicture}
	\caption{Función constante}
	\label{fig:funcion_constante}
\end{figure}

\begin{itemize}
	\item \textbf{Función identidad} \index{función!identidad}
	
	Identidad en $A$ es la aplicación que asigna a cada elemento de $A$ el mismo elemento.
	$ i_A: A \rightarrow A \mbox{ tal que } i_A(x) = x$.
	
	A veces se utiliza el símbolo $\mathbf{1}_A$.
	
	Se tiene: $i_A = \{ (x, x) / x \in A \}$
	
	La identidad en $A$ es la diagonal de $A^2$. Como relación es reflexiva,
	simétrica y transitiva, o sea, de equivalencia en $A$; además es antisimétrica.\vspace{1mm}
	
	La función $i_A$ es obviamente biyectiva, fig. \ref{fig:funcion_identidad}.
		\begin{figure}[H]
			\centering
			\begin{tikzpicture}
				\begin{axis}[axis lines=middle, xlabel={$A$}, ylabel={$A$},
					xticklabels={}, yticklabels={}, ymin=-2.5, ymax=2.5,
					scale=.6, xmin=-2.5, xmax=2.5, domain=-2:2]
					\addplot[thick] {x};
				\end{axis}
			\end{tikzpicture}
			\caption{Función identidad}
			\label{fig:funcion_identidad}
		\end{figure}
\end{itemize}

\begin{itemize}
	\item \textbf{Función proyección} \index{función!proyección}
	
	Consideremos $A \times B$, y las funciones $P_1: A \times B \rightarrow A$;
	$P_2: A \times B \rightarrow B$ definidas por $ P_1(a, b) = a \quad \mbox{y}
	\quad P_2(a, b) = b$\vspace{1mm}
	
	Tales funciones se llaman \textit{primera y segunda proyección del producto cartesiano}, y asignan a cada par ordenado la primera y la segunda componentes, respectivamente, fig. \ref{fig:funcion_proyeccion}.
	
	\begin{figure}[H]
		\centering
		\begin{tikzpicture}[scale=.8]
			% Ejes
			\draw[-latex] (-1, 0) -- (5,0) node[below]{$A$};			
			\draw[-latex] (0, -.5) -- (0,4) node[left]{$B$};
			
			% Rectángulo
			\draw[fill=black!20] (1,1) rectangle (4, 3) node[above]{$A \times B$};
			
			% Puntos
			\draw[fill] (2.5,2) circle (2pt) node[above]{$(a,b)$};
			\draw[fill] (2.5,0) circle (2pt) node[below]{$a = P_1(a,b)$};
			\draw[fill] (0,2) circle (2pt) node[left]{$b = P_2(a,b)$};
			\draw[dashed] (1,1) -- (1,0);
			\draw[dashed] (1,1) -- (0,1);
			\draw[dashed] (0, 2) -- (2.5,2) -- (2.5,0);
			\draw[dashed] (4,1) -- (4,0);
			\draw[dashed] (1,3) -- (0,3);
		\end{tikzpicture}		
		\caption{Función proyección}
		\label{fig:funcion_proyeccion}
	\end{figure}
\end{itemize}

\begin{itemize}
	\item \textbf{Función canónica} \index{función!canónica}
	
	Sea $\sim$ o $\mathcal{R}$ una relación de equivalencia definida en el conjunto \textit{no vacío}
	$A$. Por el teorema fundamental de las relaciones de equivalencia, queda determinado
	el conjunto cociente $\dfrac{A}{\sim} = A/ \mathcal{R}$, cuyos elementos son las clases de equivalencia.
	
	\begin{fmd-definition}[Función canónica]
		Aplicación canónica es la función:
		\[ \varphi: A \rightarrow A / \mathcal{R} \]
		que asigna a cada elemento de $A$, su clase de equivalencia, es decir, tal que 
		\[ \varphi(x) = K_x = [x] \]
	\end{fmd-definition}
	
	Dos elementos equivalentes pertenecen a la misma clase y en consecuencia admiten la misma imagen, significa que la aplicación canónica \textbf{no es inyectiva}, salvo en caso de clases unitarias.
	
	Como cada clase es no vacía, ocurre que siempre \textbf{es sobreyectiva}:
	\[ \forall \, [u] \in A / \mathcal{R}, \exists \ x \in A / \varphi(x) = [u]  \]
	Vale la siguiente proposición:
	\[ a \equiv b \iff \varphi(a) = \varphi(b)\]
	En el caso de congruencia módulo 3, definida en $\mathbb{Z}$ es
	$\varphi: \mathbb{Z} \rightarrow \mathbb{Z}_3$ tal que $\varphi(x) = [u]$, siendo 
	$u$ el resto de la división de $x$ por 3.
\end{itemize}

\begin{fmd-example}
	Sea en $\R^2$ dos pares ordenados de reales que están relacionados si y sólo si tienen la misma primera
	componente. \[ \mathcal{R}: \, (a,b) \equiv (a', b') \iff a = a'\]
	La relación es de equivalencia, y el propósito es caracterizar
	la aplicación canónica.\vspace{1mm}
	
	Las clases de equivalencia son del tipo: $ K_{(a,b)} = \{ (x, y) / x = a \}$
	(rectas paralelas al eje de ordenadas) \vspace{1mm}
	
	Definir el conjunto cociente requiere un conjunto de índices, y al elegir un
	único elemento en cada clase, lo tomamos sobre el eje de abscisas, de modo que:
	\[ \R^2 / \mathcal{R} = \{ K_{(u,0)} / u \in \mathbb{R}\} \]
	Así la función canónica es $\phi: \mathbb{R}^2 \rightarrow \R^2 / \mathcal{R}$
	tal que $ \phi (a, b) = K_{(u, 0)} \mbox{ si } u = a $
	
	\begin{figure}[H]
		\centering
		\begin{tikzpicture}[scale=1.3]
			% Ejes x e y
			\draw[->] (-3,0) -- (4,0) node[below] {$x$};
			\draw[->] (0,-2.5) -- (0,2.5) node[right] {$y$};
			
			% Recta vertical en x = a
			\def\a{1.5}
			\draw[dashed, very thick, blue] (\a,0) -- (\a,2) node[above] {$K_{(a, 0)}$};
			\draw[dashed, very thick, blue] (\a,-2) -- (\a,-.5) node[above] {$x = a$};
			\draw[] (\a, 0) circle (1pt);
			
			% Serie de rectas verticales equidistantes
			\foreach \i in {-2,-1.5,...,3.5} {
				\draw[opacity=0.2] (\i,-2) -- (\i,2);
			}
		\end{tikzpicture}
		\caption*{La recta vertical por $x=a$ representa a la clase de equivalencia $K_{(a, 0)}$, de todos los puntos del plano con la misma primera componente $x=a$.}
	\end{figure}
	
\end{fmd-example}

\subsection{Composición de funciones} \label{sec:compos}
\index{composición!funciones} \index{funciones!composición}
Sean $f: A \rightarrow B$ y $g: B \rightarrow C$
\begin{figure}[H]
	\centering
	\begin{tikzpicture}[scale=.8]
		% Puntos
		\coordinate (a) at (0,0);
		\coordinate (b) at (6,0);
		\coordinate (c) at (12,0);
		
		% Circunferencias
		\draw[] (a) circle (2);
		\draw[] (b) circle (2);
		\draw[] (c) circle (2);
		\draw[dashed] (b) circle (3);
		\node at ($(a) + (-2, 1.5)$) {$A$};
		\node at ($(b) + (-2, 1.5)$) {$B$};
		\node at ($(b) + (-2, 3)$) {$B'$};
		\node at ($(c) + (-2, 1.5)$) {$C$};
		
		% Centros
		\draw[fill] (a) circle (2pt) node[below]{$x$};
		\draw[fill] (b) circle (2pt) node[below]{$f(x)$};
		\draw[fill] (c) circle (2pt) node[above, yshift=8] {$(g \circ f)(x)$};
		
		% Flechas
		\draw[-latex, blue, thick] (a) to [bend left = 15] (b);
		\draw[-latex, blue, thick] (b) to [bend left = 15] (c);
		\draw[-latex, blue, thick] (a) to [bend right = 25] (c);
	\end{tikzpicture}
	\caption{Composición de funciones}
	\label{fig:compos1}
\end{figure}
El codominio de $f$ es dominio de $g$, pero es suficiente que el codominio de la
primera sea parte del dominio de la segunda: $B \subset B'$. En la fig. \ref{fig:compos1} se ve gráficamente esta afirmación.

\begin{fmd-definition}[Composición de funciones]
	La \gls{composicion} entre $f: A \rightarrow B$ y $g:B \rightarrow C$ es la
	función $g \circ f : A \rightarrow C$, tal que:
	$ (g \circ f)(x) = g\left[ f(x) \right] \ \forall x \in A$
\end{fmd-definition}

\begin{fmd-example}[Composición de funciones discretas]
	Sean $A = \{ 1, 2, 3 \}$, $B = \{ a, b, c, d \}$, $C = \{ 5, 6 \}$ y las funciones
	$f: A \rightarrow B$ y $g: B \rightarrow C$ definidas así:
	\[ f: \{(1, a), (2, b), (3, d)\}; \quad g = \{ (a, 5), (b, 5), (c, 5), (d, 6) \} \]
	Resulta: $g \circ f = \{(1,5), (2, 5), (3, 6) \}$. Ver figura.
	
	\begin{figure}[H]
		\centering
		\begin{tikzpicture}[scale=.8]
			% Centros
			\coordinate (a) at (0,0);
			\coordinate (b) at (6,0);
			\coordinate (c) at (12,0);
			
			% Circunferencias
			\draw[] (a) circle (2);
			\draw[] (b) circle (2);
			\draw[] (c) circle (2);
			
			\node at ($(a) + (-1.5,2)$) {$A$};
			\node at ($(b) + (-1.5,2)$) {$B$};
			\node at ($(c) + (-1.5,2)$) {$C$};
			
			% Elementos
			\coordinate (A1) at ($(a) + (-.5, 1)$);
			\coordinate (A2) at ($(a) + (-.5, 0)$);
			\coordinate (A3) at ($(a) + (-.5, -1)$);
			
			\draw[fill] (A1) circle (2pt) node[left]{$1$};
			\draw[fill] (A2) circle (2pt) node[left]{$2$};
			\draw[fill] (A3) circle (2pt) node[left]{$3$};
			
			\coordinate (B1) at ($(b) + (0, 1.5)$);
			\coordinate (B2) at ($(b) + (0, .5)$);
			\coordinate (B3) at ($(b) + (0, -.5)$);
			\coordinate (B4) at ($(b) + (0, -1.5)$);
			
			\draw[fill] (B1) circle (2pt) node[above]{$a$};
			\draw[fill] (B2) circle (2pt) node[above]{$b$};
			\draw[fill] (B3) circle (2pt) node[above]{$c$};
			\draw[fill] (B4) circle (2pt) node[above]{$d$};
			
			\coordinate (C1) at ($(c) + (0, 1)$);
			\coordinate (C2) at ($(c) + (0, -1)$);
			
			\draw[fill] (C1) circle (2pt) node[right] {$5$};
			\draw[fill] (C2) circle (2pt) node[right] {$6$};
			
			% Flechas
			\draw[-latex, blue, thick] (A1) to [bend left = 15] (B1);
			\draw[-latex, blue, thick] (A2) to [bend left = 15] (B2);
			\draw[-latex, blue, thick] (A3) to [bend right = 15] (B4);
			
			\draw[-latex, blue, thick] (B1) to [bend left = 15] (C1);
			\draw[-latex, blue, thick] (B2) to [bend right = 15] (C1);
			\draw[-latex, blue, thick] (B3) to [bend right = 15] (C1);
			\draw[-latex, blue, thick] (B4) to [bend right = 15] (C2);
		\end{tikzpicture}
		\label{fig:compos2}
	\end{figure}
	Nótese que no coexisten $g \circ f$ y $f \circ g$ ya que, en este caso, el codominio
	de $g$ es $C$ y el dominio de $f$ es $A$. Ambas composiciones existen si $C \subset A$.
\end{fmd-example}

\begin{fmd-example}[Composición de funciones continuas] \label{ex:compos}
	Sean: \[ \begin{cases}
		f: \mathbb{R} \rightarrow \mathbb{R} & \mbox{tal que } f(x) = 2x \\
		g: \mathbb{R} \rightarrow \mathbb{R} & \mbox{tal que } g(x) = x^2
	\end{cases}
	\]
	\begin{enumerate}
		\item $g \circ f: \mathbb{R} \rightarrow \mathbb{R}$ es:
		$ (g \circ f)(x) = g\left[ f(x) \right] = g(2x) = (2x)^2 = 4x^2$
		\item $f \circ g: \mathbb{R} \rightarrow \mathbb{R}$ es:
		$ \left( f \circ g \right)(x) = f\left[g(x)\right] = f(x^2) = 2 x^2 $
	\end{enumerate}
	Ambas funciones compuestas, a pesar de tener el mismo dominio y codominio, son
	distintas, por diferir en la ley de asignación.
\end{fmd-example}

\begin{definition}[Funciones iguales] \index{funciones!iguales}
	Dos funciones $f:A \rightarrow B$ y $g: A \rightarrow B$ son iguales si y sólo si
	para todo $x$ de $A$ se verifica $f(x) = g(x)$
\end{definition}\vspace{2mm}
Con relación al ejemplo \ref{ex:compos} $g \circ f \ne f \circ g$.

\subsubsection{Asociatividad de la composición} \label{sec:asocom}
Si $f: A \rightarrow B$, $g: B \rightarrow C$ y $h: C \rightarrow D$, entonces:
$\left( h \circ g \right) \circ f = h \circ \left( g \circ f \right)$
\begin{figure}[H]
	\centering
	\begin{tikzpicture}[scale=.8]
		% Puntos
		\coordinate (a) at (0,0);
		\coordinate (b) at (5,0);
		\coordinate (c) at (10,0);
		\coordinate (d) at (15,0);
		
		% Circunferencias
		\draw[] (a) circle (2);
		\draw[] (b) circle (2);
		\draw[] (c) circle (2);
		\draw[] (d) circle (2);
		
		\node at ($(a) + (-2, 1.5)$) {$A$};
		\node at ($(b) + (-2, 1.5)$) {$B$};
		\node at ($(c) + (-1.5, 2)$) {$C$};
		\node at ($(d) + (-2, 1.5)$) {$D$};
		
		% Centros
		\draw[fill] (a) circle (2pt) node[below]{$x$};
		\draw[fill] (b) circle (2pt) node[below left]{$f(x)$};
		\draw[fill] (c) circle (2pt) node[below] {$g[f(x)]$};
		\draw[fill] (d) circle (2pt) node[above] {$h\{g[f(x)]\}$};
		
		% Flechas
		\draw[-latex, blue, thick] (a) to [bend left = 15] (b);
		\draw[-latex, blue, thick] (b) to [bend left = 15] (c);
		\draw[-latex, blue, thick] (c) to [bend left = 10] (d);
		\draw[-latex, blue, thick] (a) to [bend left = 50] (c);
		\draw[-latex, blue, thick] (b) to [bend right = 50] (d);
		
		\node at (4, 2.8) {$g \circ f$};
		\node at (10, -2.8) {$h \circ g$};
	\end{tikzpicture}
	\caption{Asociatividad de la composición}
	\label{fig:compos3}
\end{figure}

En efecto, $\forall x \in A$:
\[\begin{cases}
	\left( (h \circ g) \circ f \right)(x) &= (h \circ g)\left(f(x)\right) = h \{ g[f(x)]\}\\
	\left( h \circ (g \circ f) \right)(x) &= h \left( (g \circ f)(x)\right) = h \{ g[f(x)]\}
\end{cases} \]
De donde resulta la igualdad buscada. Ver fig. \ref{fig:compos3}.


\subsubsection{Composición de funciones inyectivas} \label{sec:cominy}
\vspace{1em}
\begin{fmd-proposition}
	Si $f: A \rightarrow B$ y $g: B \rightarrow C$ son inyectivas, entonces $g \circ f: 
	A \rightarrow C$ es inyectiva.
\end{fmd-proposition}

\begin{fmd-proof}
	Debemos probar: H) $\forall \, x', x'' \in A$ con $(g \circ f)(x') = (g \circ f)(x'')$, 
	entonces T) $x' = x''$
	
	Por hipótesis y por definición de composición:
	\[ g[f(x')] = g[f(x'')] \]
	por ser $g$ inyectiva:
	\[ f(x') = f(x'')\]
	por ser $f$ inyectiva:
	\[x' = x'' \]
	\emph{La composición de funciones inyectivas es inyectiva}.
\end{fmd-proof}


\subsubsection{Composición de funciones sobreyectivas} \label{sec:comsob}
\vspace{1em}
\begin{fmd-proposition}
	Si $f: A \rightarrow B$ y $g: B \rightarrow C$ son sobreyectivas, entonces $g \circ f: 
	A \rightarrow C$ es sobreyectiva.
\end{fmd-proposition}
\begin{fmd-proof}
	Hay que probar que $\forall \, z \in C \, \exists \, x \in A$ tal que
	$(g \circ f)(x) = z$.\vspace{2mm}
	
	Por ser $g$ sobreyectiva:
	$\forall \, z \in C, \, \exists \, y \in B / g(y) = z$
	
	Ahora bien, dado que $y \in B$, por ser $f$ sobreyectiva
	$\exists \, x \in A / f(x) = y$
	
	De aquí se deduce que:
	\[g[f(x)] = g(y) = z\]
	
	\emph{La composición de funciones sobreyectivas es sobreyectiva}.
\end{fmd-proof}

Se sigue de los casos anteriores que, si $f: A \rightarrow B$ y $g: B \rightarrow C$ son biyectivas, entonces $g \circ f: 
A \rightarrow C$ es biyectiva.

\begin{fmd-example}
	Demostrar que si $f: A \rightarrow B$ y $g: B \rightarrow C$ son tales que
	$g \circ f: A \rightarrow C$ es inyectiva, entonces $f$ es inyectiva.\vspace{-2mm}
	
	\begin{proof}
		Sean $x', x'' \in A$ tales que $f(x') = f(x'')$, la imagen de este elemento de $B$
		por $g$, es:
		\[ g[f(x')] = g[f(x'')] \]
		ya que cada elemento del dominio $B$ tiene imagen única en $C$, por definición de
		función. Por definición de composición
		\[ (g \circ f)(x') = (g \circ f)(x'') \]
		y por ser $g \circ f$ inyectiva, resulta $x'=x''$, en consecuencia $f$ es
		inyectiva.
	\end{proof}
	
	Análogamente se demuestra que si la composición de dos aplicaciones es sobreyectiva,
	la segunda es sobreyectiva.
\end{fmd-example}

\subsection{Funciones inversas} \label{sec:finversas} \index{función!inversa}
Cabe preguntarse si, para la función $f: A \rightarrow B$, la relación inversa
es una función. En general la respuesta es negativa, como se ve a través del ejemplo 2,
diapositiva \ref{frame:ejm2}, donde $A = \{-1,0,1,2\}$, $B=\{0,1,2,3,4\}$ y $f(x) =x^2$
\[ f = \{ (-1,1), (0,0), (1,1), (2,4) \} \]
la inversa de esta relación es el subconjunto $B \times A$
\[ \{ (1,-1), (0,0), (1,1), (4,2) \} \]
se ve que esta relación no es una función de $B$ en $A$, pues los elementos 2
y 3 del eventual dominio carecen de imágenes en $A$ y además no se cumple la
condición de unicidad, ya que 1 tiene dos correspondientes en $A$.

Sea, en cambio, el siguiente caso $A = \{1,2,3\}$, $B=\{a,b,c\}$ y 
$f(x) =\{(1, a), (2, c), (3,b) \}$. La relación inversa es:
\[ g = \{ (a,1), (b,3), (c,2) \} \]
es claramente una función de $B$ en $A$, llamada \gls{funcioninversa} de $f$. La composición
\[ g \circ f = \{ (1,1),(2,2),(3,3) \} = i_A \]
en donde $g$ es la \textit{inversa izquierda} de $f$
y \[ f \circ g = \{ (a, a), (b,b), (c,c) \} = i_B\]
En este caso $g$ es la \textit{inversa derecha} de $f$ \vspace{2mm}

\begin{mdframed}[backgroundcolor=gray!20, linecolor=black, linewidth=1pt]
	La función $f: A \rightarrow B$ admite inversa si y sólo si existe
	$g: B \rightarrow A$ tal que $g \circ f = i_A$ y $f \circ g = i_B$
\end{mdframed}

\begin{fmd-example}[Función inversa]
	La función $f: \mathbb{R} \rightarrow \mathbb{R}$ definida por $f(x) = x + 2$ admite
	inversa $g: \mathbb{R} \rightarrow \mathbb{R}$ tal que $g(x) = x - 2$, pues
	\[
	\begin{cases}
		(g \circ f)(x) &= g[f(x)] = g(x+2) = x + 2 -2 = x = i_\mathbf{R}(x)\\
		(f \circ g)(x) &= f[g(x)] = f(x-2) = x - 2 + 2 = x = i_\mathbf{R}(x)\\
	\end{cases}	
	\]
	La representación cartesiana de dos funciones inversas conduce a gráficos simétricos
	respecto de la recta a $45\unit{\degree}$.
	
	\begin{figure}[H]
		\centering
		\begin{tikzpicture}[scale=1.2]
			\begin{axis}[axis lines=middle, xmin=-4, xmax=4,
				ymin=-4, ymax=4, xticklabels={}, yticklabels={},
				xlabel={$x$}, ylabel={$y$}, scale=.6]
				\addplot[dashed] {x};
				\addplot[blue, thick] {x - 2} node[below]{$g$};
				\addplot[blue, thick] {x + 2};
				\node at (axis cs:-1.5,1.5) {$f$};
				\node at (axis cs:2.2,-1.5) {$g=f^{-1}$};
			\end{axis}
		\end{tikzpicture}
	\end{figure}
\end{fmd-example}

\begin{fmd-theorem}
	Una función admite inversa si y sólo si es biyectiva.
\end{fmd-theorem}

\begin{enumerate}
	\item Si una función admite inversa, entonces es biyectiva.
	\begin{itemize}
		\item[H)] $f: A\rightarrow B$ es tal que $\exists \, g: B \rightarrow A$
		siendo $g \circ f = i_A$ y $f \circ g = i_B$
		\item[T)] $f$ es biyectiva.
	\end{itemize}
	\begin{proof}
		En dos partes.
		\begin{enumerate}[label=\alph*)]
			\item Inyectividad de $f$
			
			Sean $x', x'' \in A / f(x') = f(x'') \in B$.
			
			La imagen por $g$ es $g[f(x')] = g[f(x'')] $, o,
			$(g \circ f)(x') = (g \circ f)(x'')$
			
			siendo $g \circ f = i_A$, se tiene $i_A(x') = i_A(x'')$ o, lo que es lo mismo:
			$x' = x''$.
			
			Por tanto $f$ es 1-1.

		\item $f$ es sobreyectiva
		
		Según definición, hay que probar $\forall \, y \in B, \exists \, x \in A /
		f(x) = y$\vspace{2mm}
		
		Sea $y \in B$, entonces $y = i_B(y)$, y como $i_B = f \circ g$, se tiene:
		\[ y = (f \circ f)(y), \mbox{ es decir, } y= f[g(x)] \]
		por tanto, a expensas de $y \in B$, hemos determinado $x = g(y)$ en $A$,
		tal que $f(x) = y$
	\end{enumerate}
\end{proof}
Siendo $f$ inyectiva y sobreyectiva resulta biyectiva.


	\item Si una función es biyectiva, entonces admite inversa.
	
	\begin{itemize}
		\item[H)] $f: A \rightarrow B$ es biyectiva.
		\item[T)] $\exists \, g: B \rightarrow A$ tal que $g \circ f = i_A$ y
		$f \circ g = i_B$
	\end{itemize}
	
	\begin{proof}
		Necesitamos proceder en tres etapas.
		
		\begin{enumerate}[label=\textbf{\alph*)}]
			\item Definimos
			$g: B \rightarrow A \mbox{ mediante } g(y)= x \mbox{ si } f(x) = y
			\quad (1)$
			
			que satisface la definición de función, pues:
			\begin{enumerate}[label=\roman*), ref=\roman*]
				\item Todo $y \in B$ proviene de algún $x \in A$
				\item La $x$ asociada a $y$ es única, por ser $f$ inyectiva.
				
				En efecto, si $x$ y $x'$ fueran antecedentes distintos de $y$ por $f$
				se tendría $x \ne x'$ y $f(x) = f(x') = y$, lo que es absurdo por la
				inyectividad de $f$.
			\end{enumerate}
		\item Hay que probar que $g \circ f = i_A$.
		
		$\forall x \in A$, se tiene, por definición de composición, por (1), y
		por definición de identidad en $A$:
		\[ (g \circ  f)(x) = g[f(x)] = g(y) = x = i_A(x)\]
		Entonces, por definición de funciones iguales:
		\[ g \circ f = i_A \]

		\item Finalmente, demostramos que $f \circ g = i_B$
		
		Como $f \circ g: B \rightarrow B \ \forall \, y \in B$, tenemos, por definición
		de composición, por (1) y por identidad en $B$.
		
		\[ (f \circ g)(y) = f[g(y)] = f(x) = y = i_B(y) \]
		Con lo cual:
		\[ f \circ g = i_B \]
	\end{enumerate}
\end{proof}
\end{enumerate}

\begin{mdframed}[backgroundcolor=gray!20, linecolor=black, linewidth=1pt,
	frametitle={Consecuencia}]
	Si $f: A \rightarrow B$ es biyectiva, entonces la función $g:B \rightarrow A$ a 
	que se refiere el teorema anterior, es única y, además, biyectiva.
\end{mdframed}
\begin{proof}
	Si existieran dos funciones $g$ y $g'$ se tendría:
	\[ g' = g' \circ i_B = g' \circ (f \circ g) = (g' \circ f) \circ g = 
	i_A \circ g = g \]
	
	Por otra parte, como una función que admite inversa es biyectiva, se tiene 
	que $g: B \rightarrow A$ es tal que $\exists \, f: A \rightarrow B$, siendo
	$f \circ g = i_B$ y $g \circ f = i_A$. En consecuencia, $g$ es biyectiva
\end{proof}

La función $g$ se llama inversa y se denota por $f^{-1}$.

\begin{fmd-example}
	Probar que $f: \mathbb{R} \rightarrow (-1, 1)$ definida por 
	$f(x) = \dfrac{x}{1 + |x|}$ admite inversa.
	
	\begin{enumerate}[label=\textbf{\alph*)}]
		\item $f$ es inyectiva
		
		Sean $x', x'' \in A / f(x') = f(x'')$
		\[ \frac{x'}{1 + |x'|} = \frac{x''}{1 + |x''|} \implies x' + x'|x''| = 
		x'' + x'' |x'| \implies x' = x'' \]
		O sea, $f$ es 1-1.
		\item $f$ es sobreyectiva
		
		Sea $y \in (-1, 1)$. Si $\exists \, x \in \mathbb{R} / f(x) = y$, entonces 
		$\dfrac{x}{1 + |x|} = y$, notar que $x$ y $y$ tienen signos iguales, $\mathrm{sg}(x) = \mathrm{sg}(y)$.
		
		Operando (siendo $|y| < 1$):
		\[ x = y + y|x| \rightarrow x = y + y \, x \, \mathrm{sg}(x) \rightarrow
		x = y + x \, y \, \mathrm{sg}(y) \rightarrow x - x \, y \, \mathrm{sg}(y) = y \]
		\[ \rightarrow x(1 - |y|) = y \rightarrow x = \frac{y}{1 - |y|}\]
	
		de donde: $\forall \, y \in (-1, 1), \exists \, x = \dfrac{y}{1 - |y|}$ tal que:
		\[ f(x) = f \left( \frac{y}{1 - |y|} \right) = \frac{\dfrac{y}{1-|y|}}{1 - 
			\left| \dfrac{y}{1 - |y|} \right|} = \frac{\dfrac{y}{1 - |y|}}{1 - \dfrac{|y|}
			{1 - |y|}} = y \]
		lo que prueba que $f$ es sobreyectiva.
\end{enumerate}
	
	Por a) y b) resulta que $f$ es biyectiva y en consecuencia admite inversa. La 
	inversa es: \[ f^{-1}: (-1,1) \rightarrow \mathbb{R} / f^{-1}(x) = \dfrac{x}{1 - |x|}\]
	
	Se puede verificar que $g \circ f = i_\mathbb{R}$ y que $f \circ g = i_{(-1,1)}$.
\end{fmd-example}

\begin{fmd-example}
	La función $f: A \rightarrow B$ es inyectiva si y sólo si existe $g: B \rightarrow A$
	tal que $g \circ f = i_A$.
	
	\begin{itemize}
		\item[H)] $f: A \rightarrow B$ es 1-1;
		\item[T)] $\exists \, g: B \rightarrow A / g \circ f = i_A$
	\end{itemize}\vspace{3mm}
	
	La función $f$ no es necesariamente sobreyectiva. Nos apoyamos en el diagrama de la figura siguiente:
	\begin{figure}[H]
		\centering
		\begin{tikzpicture}[scale=1]
			% Coordenadas de puntos
			\coordinate (x1) at (0,1);
			\coordinate (x2) at (0,-1);
			
			\coordinate (y1) at (6,1);
			\coordinate (y2) at (8,0);
			
			% Circunferencias
			\node at (-1.5, 2) {$A$};
			\node at (6.5, 2) {$B$};
			\draw[very thick] (0,0) circle (2);
			\draw[very thick] (6,0) circle (1.5);
			\draw[dashed] (6,0) circle (2.5);
			
			% Puntos
			\draw[fill] (x1) circle (1mm) node[left] {$x$};
			\draw[fill] (x2) circle (1mm) node[left] {$x'$};
			
			\draw[fill] (y1) circle (1mm) node[right] {$y$};
			\draw[fill] (y2) circle (1mm) node[right] {$y$};
			
			% flechas
			\draw[red, thick, -latex] (x1) to [bend left = 25] (y1);
			\draw[red, thick, -latex] (x2) to [bend right = 15] (y2);
			\draw[red, thick, -latex] (y1) to [bend left = 20] (x1);
		\end{tikzpicture}
	\end{figure}
	
	\begin{proof}
		Definimos una función $g: B \rightarrow A$:
		\[ g(y) \begin{cases}
			x & \mbox{si } f(x) = y\\
			x' & \mbox{(cualquier elemento fijo de $A$) si } \nexists \, x \in A 
			/ f(x) = y
		\end{cases} \]
		De este modo, todo elemento de $B$ tiene su correspondiente en $A$, y es único por
		ser $f$ inyectiva. Ahora bien:
		
		\[ (g \circ f)(x) = g[f(x)] = g(y) = x = i_A(x) \]
		luego: \[ g \circ f = i_A \]
	\end{proof}
	
	\begin{enumerate}
		\setcounter{enumi}{1}
		\item \begin{itemize}
			\item[H)] $f: A \rightarrow B$ es tal que existe $g:B \rightarrow A$ de modo
			que $g \circ f = i_A$;
			\item[T)] $f$ es inyectiva.
		\end{itemize}\vspace{3mm}
		\begin{proof}
			Sean $x', x'' \in A / f(x') = f(x'')$, entonces
			\[ g[f(x')] = g[f(x'')] \mbox{o, lo que es lo mismo } (g \circ f)(x') = (g \circ f)(x'') \]
			por hipótesis: \[\mbox{Si } i_A(x') = i_A(x'') \mbox{ entonces } x' = x''\]
			
			en consecuencia $f$ es 1-1\
		\end{proof}
	\end{enumerate}
\end{fmd-example}

\textbf{Resumen de funciones inversas}
\begin{fmd-definition}[Función inversa]
	Sea $f$ una función biyectiva con dominio $X$ y rango $Y$. La \textbf{inversa}
	de $f$ es la función $f^{-1}$ cuyo dominio es $Y$ y rango es $X$, para los cuales
	
	\[ f \circ f^{-1} = f[f^{-1}(x)] = x \quad \forall x \in Y \]
	y \[ f^{-1} \circ f = f^{-1}[f(x)] = x  \quad \forall x \in X\]
\end{fmd-definition}

\underline{\textbf{Propiedades}}:

		\begin{enumerate}[label=\roman*)]
			\item Dominio $f^{-1}$ = rango $f$;
			\item Rango $f^{-1}$ = dominio $f$;
			\item $y = f(x)$ equivale a $x = f^{-1}(y)$;

			\setcounter{enumi}{3}
			\item $f^{-1}$ es biyectiva;
			\item $(f^{-1})^{-1} = f$;
			\item La inversa de $f$ es única.
		\end{enumerate}


\subsection{Imágenes de subconjuntos del dominio} \label{sec:imasub}
Sean $f : X \rightarrow Y$ y $A \subset X$ \index{imagen}
\begin{fmd-definition}[Imagen de subconjuntos]
	Imagen del subconjunto $A \subset X$ es el conjunto cuyos elementos son las
	imágenes de los elementos de $A$.
\end{fmd-definition}

\begin{figure}[H]
	\centering
	\begin{tikzpicture}[scale=1]
		% Centros
		\coordinate (a) at (0,0);
		\coordinate (b) at (5,0);
		
		% Circunferencias
		\draw[] (a) circle (2);
		\draw[] (b) circle (2);
		
		\node at ($(a) + (-1.5,2)$) {$X$};
		\node at ($(b) + (-1.5,2)$) {$Y$};
		
		% Elipses
		\draw[fill=black!20] (a) ellipse (1 and .6);
		\draw[fill=black!20] (b) ellipse (1 and .6);
		
		\node at ($(a) + (-1, .7)$) {$A$};
		\node[above right] at ($(b) + (3, -.5)$) {$f(A)$: ``imagen de $A$''};
		
		% Flechas
		\draw[blue, thick, -latex] (a) to [bend left = 15] ($(b) + (-1, .2)$);
		\draw[blue, thick, -latex] ($(b) + (.5,-.1)$) to [bend right = 10]
		($(b) + (3, -.5)$);
	\end{tikzpicture}
	\caption{Imagen de subconjuntos}
	\label{fig:imagenes}
\end{figure}
\[ f(A) = \{ f(x) / x \in A \} \mbox{ o bien } f(A) = \{y \in Y / \mbox{ existe } \, 
x \in A \mbox{ y } f(x) = y \} \]
De acuerdo con la definición: $y \in f(A) \mbox{ si y solo si existe } x \in A / y = f(x)$

Si $A = X$, entonces $f(X)$ es la imagen del dominio por $f$. Además
$f(\emptyset) = \emptyset$. $f$ es sobreyectiva si y sólo si $f(X) = Y$.

\subsubsection{Propiedades de la imagen}
Sean $f: X \rightarrow Y$ y $A$ y $B$ subconjuntos del dominio.
\begin{enumerate}[label=\textbf{\alph*)}]
	\item Si un subconjunto del dominio es parte de otro, entonces la misma relación
	vale para sus imágenes.
	\[ f: X \rightarrow Y, A \subset B, B \subset X \mbox{ y } A \subset B \mbox{ entonces }
	f(A) \subset f(B)\]
	En efecto, sea
	\[ z \in f(A) \rightarrow \mbox{ hay algún } \, x \in A / f(x) = z \rightarrow \mbox{ existe } \, x
	\in B / f(x) = z \rightarrow z \in f(B) \]
\end{enumerate}


\begin{enumerate}[label=\textbf{\alph*)}]
	\setcounter{enumi}{1}
	\item La imagen de la unión de dos subconjuntos del dominio, es igual a la 
	unión de sus imágenes.
	\[ f: X \rightarrow Y , A \subset X \land B \subset X \implies f\left( A 
	\cup B \right) = f(A) \cup f(B) \]
	\begin{figure}[H]
		\centering
		\begin{tikzpicture}[scale=1]
			% Centros
			\coordinate (x) at (0,0);
			\coordinate (a) at (-.5,.5);
			\coordinate (b) at (.5,-.5);
			
			\coordinate (y) at (5, 0);
			\coordinate (fa) at (4.5,.5);
			\coordinate (fb) at (5.5,-.5);
			
			% Circunferencias
			\draw[thick] (x) circle (2);
			\filldraw[pattern=north east lines] (a) circle (1);
			\filldraw[pattern=north west lines] (b) circle (1);
			\draw[thick] (y) circle (2);
			\filldraw[pattern=north east lines] (fa) circle (1);
			\filldraw[pattern=north west lines] (fb) circle (1);
			
			\node at ($(x) + (-2,1.5)$) {$X$};
			\node at ($(y) + (-2,1.5)$) {$Y$};
			\node at ($(a) + (-1,-1)$) {$A$};
			\node at ($(b) + (-1,-1)$) {$B$};
			\node at ($(fa) + (1,1)$) {$f(A)$};
			\node at ($(fb) + (-1,-1)$) {$f(B)$};
			
			% Flechas
			\draw[-latex, blue, thick] (a) to [bend left = 10] (fa);
			\draw[-latex, blue, thick] (b) to [bend right = 15] (fb);
		\end{tikzpicture}
		\caption{Imagen de la unión de dos subconjuntos.}
		\label{fig:imagenes2}
	\end{figure}
\end{enumerate}

\begin{enumerate}[label=\textbf{\alph*)}]
	\setcounter{enumi}{2}
	\item La imagen de la intersección de dos subconjuntos del dominio está incluida
	en la intersección de sus imágenes.
	\[ \mbox{Si } f: X \rightarrow Y, A \subset X \mbox{ y } B \subset X \mbox{ entonces } f\left( 
	A \cap B \right) \subset f(A) \cap f(B) \]
	
	El siguiente ejemplo prueba que no es válida la inclusión en el otro sentido.
	
	Sean $f: \mathbb{Z} \rightarrow \mathbb{N} / f(x) = x^2$ y los subconjuntos de 
	$\mathbb{Z}$
	\[ A = \{ -2, -3, 4 \} \mbox{ y } B = \{ 2, 3, 4, 5 \} \]
	Se tiene $A \cap B = \{ 4 \}$, $f\left( A \cap B \right) = \{16\}$, $f(A) \cap f(B)
	= \{ 4, 9, 16\}$
	
	Resulta:
	\[ f\left( A \cap B \right) \not \subset f(A) \cap f(B) \]
\end{enumerate}

\subsection{Imágenes inversas de subconjuntos del codominio} \label{sec:invcon}
\index{imagen!inversa}
Sean $f: X \rightarrow Y$ y $A \subset Y$.

\begin{fmd-definition}[Preimagen] \label{def:preimagen} \index{preimagen}
	Imagen inversa o preimagen del subconjunto $A \subset Y$, es el conjunto de los
	elementos del dominio cuyas imágenes pertenecen a $A$.
	
	\begin{figure}[H]
		\centering
		\begin{tikzpicture}[scale=1]
			% Centros
			\coordinate (x) at (0,0);
			\coordinate (y) at (5, 0);
			
			% Circunferencias
			\draw[] (x) circle (2);
			\draw[] (y) circle (2);
			\filldraw[pattern=north east lines] (x) circle (1);
			\filldraw[pattern=north west lines] (y) circle (1);
			
			\node at ($(x) + (-2,1.5)$) {$X$};
			\node at ($(y) + (-2,1.5)$) {$Y$};
			\node at ($(x) + (0,-1.5)$) {$f^{-1}(A)$};
			\node at ($(y) + (-1,-1)$) {$A$};
			
			% Flechas
			\draw[latex-, blue, thick] (x) to [bend left = 20] (y);
		\end{tikzpicture}
		\caption{Definición de preimagen}
		\label{fig:preimagen}
	\end{figure}
	\[ f^{-1}(A) = \{ x \in X / f(x) \in A \} \]
	Es claro que: $ x \in f^{-1}(A)$ si, y solo si $f(x) \in A $, en palabras, un elemento del dominio pertenece a la preimagen de $A$ si y sólo si su imagen pertenece a $A$.
\end{fmd-definition}

\begin{fmd-example}[Preimagen]
	Sea $f: \mathbb{R} \rightarrow \mathbb{R} / f(x) = x^2$. Determinamos las preimágenes
	de los siguientes subconjuntos del codominio
	\[ (-\infty, -1], (-1, 1], (-1, 1), [4,9] \]
	\begin{enumerate}[label=\roman*)]
		\item $f^{-1} (-\infty, -1] = \{ x \in \mathbb{R} / f(x) \in (-\infty, -1] \}$
		
		Ahora bien \[ f(x) \in (-\infty, -1) \iff x^2 \le -1 \iff x \in \varnothing \]
		Resulta
		\[ f^{-1} (-\infty, -1] = \varnothing \]
	\end{enumerate}
	
	
	\begin{enumerate}[label=\roman*)]
		\setcounter{enumi}{1}
		\item En el segundo caso:
		\[ x \in f^{-1}(-1, 1] \rightarrow f(x) \in (-1, 1] \rightarrow x^2 \in (-1, 1]
		\rightarrow -1 < x \le 1\]
		\[ \rightarrow x^2 \le 1 \rightarrow |x|^2 \le 1 \rightarrow -1 \le x \le 1
		\rightarrow x \in [-1, 1] \]
		Entonces $f^{-1}(-1, 1] = [-1, 1]$
		\item Se tiene \[ x \in f^{-1}(-1,1) \rightarrow f(x) \in (-1,1) \rightarrow
		x^2 \in (-1, 1) \rightarrow x^2 < 1 \]
		\[ \rightarrow |x| < 1  \rightarrow -1 < x < 1 \rightarrow x \in (-1, 1)\]
		Luego $f^{-1} (-1,1) = (-1,1)$
	\end{enumerate}
	
	\begin{enumerate}[label=\roman*)]
		\setcounter{enumi}{3}
		\item Finalmente
		\[ x \in f^{-1}[4,9] \rightarrow f(x) \in [4,9] \rightarrow x^2 \in [4,9]
		\rightarrow 4 \le x^2 \le 9 \]
		\[ \rightarrow x^2 \ge 4 \land x^2 \le 9 \rightarrow |x| \ge 2 \land |x|\le 3
		\]
		\[ \rightarrow x \in [-3, -2] \vee x \in [2, 3] \rightarrow x \in [-3,-2] 
		\cup [2,3] \]
		Entonces: \[ f^{-1}[4,9] = [-3, -2] \cup [2,3] \]
	\end{enumerate}
\end{fmd-example}

\subsubsection{Propiedades de la preimagen}
Sean $f:X \rightarrow Y$ y los subconjuntos $A \subset Y$, $B \subset Y$.
\begin{enumerate}[label=\textbf{\alph*)}]
	\item La preimagen de la unión es la unión de las preimágenes.
	\[ f^{-1} \left( A \cup B \right) = f^{-1}(A) \cup f^{-1}(B) \]
	\begin{figure}[H]
		\centering
		\begin{tikzpicture}[scale=1]
			% Centros
			\coordinate (x) at (0,0);
			\coordinate (a) at (-.5,.5);
			\coordinate (b) at (.5,-.5);
			
			\coordinate (y) at (5, 0);
			\coordinate (fa) at (4.5,.5);
			\coordinate (fb) at (5.5,-.5);
			
			% Circunferencias
			\draw[thick] (x) circle (2);
			\filldraw[pattern=north east lines] (a) circle (.8);
			\filldraw[pattern=north west lines] (b) circle (.8);
			\draw[thick] (y) circle (2);
			\filldraw[pattern=north east lines] (fa) circle (1);
			\filldraw[pattern=north west lines] (fb) circle (1);
			
			\node at ($(x) + (-2,1.5)$) {$X$};
			\node at ($(y) + (-2,1.5)$) {$Y$};
			\node at ($(a) + (1,1)$) {$f^{-1}(A)$};
			\node at ($(b) + (-1,-1)$) {$f^{-1}(B)$};
			\node at ($(fa) + (1,1)$) {$A$};
			\node at ($(fb) + (-1,-1)$) {$B$};
			
			% Flechas
			\draw[latex-, blue, thick] (a) to [bend left = 10] (fa);
			\draw[latex-, blue, thick] (b) to [bend right = 15] (fb);
		\end{tikzpicture}
		\caption{Preimagen de la unión $A \cup B$.}
		\label{fig:imagenes3}
	\end{figure}
	\[ x \in f^{-1}\left( A \cup B \right) \rightarrow f(x) \in A \cup B \rightarrow
	f(x) \in A \mbox{ o } f(x) \in B \]
	\[ \rightarrow x \in f^{-1}(A) \mbox{ o } x \in f^{-1}(B) \rightarrow x \in f^{-1}
	(A) \cup f^{-1} (B)\]
\end{enumerate}

\begin{enumerate}[label=\textbf{\alph*)}]
	\setcounter{enumi}{1}
	\item La preimagen de la intersección es igual a la intersección de las preimágenes.
	\[ f^{-1} \left( A \cap B \right) = f^{-1}(A) \cap f^{-1}(B) \]
	Se tiene:
	\[x \in f^{-1}\left( A \cap B \right) \mbox{ si y sólo si } f(x) \in A \cap B \mbox{ que es verdad si } f(x) \in A \mbox{ y } f(x) \in B \]
	\[ \mbox{ por tanto } x \in f^{-1}(A) \mbox{ y } x \in f^{-1}(B) \mbox{ si y solo si } x \in f^{-1}(A) \cap f^{-1}
	(B) \]
\end{enumerate}

\begin{enumerate}[label=\textbf{\alph*)}]
	\setcounter{enumi}{1}
	\item La imagen inversa del complemento de un subconjunto del codominio es 
	igual al complemento de su preimagen.
	\[ f^{-1}\left( A^c \right) = \left[ f^{-1}(A) \right]^c \]
	En efecto:
	\[ x \in f^{-1}\left( A^c \right) \mbox{ si y sólo si } f(x) \in A^c \mbox{ por tanto } x \not \in f^{-1}(A) \mbox{ de donde } x \in \left[ f^{-1}(A) \right]^c\]
\end{enumerate}

\begin{fmd-example}
	El conjunto $\Omega$ consiste en los posibles resultados que se obtienen al lanzar
	una moneda: $\Omega = \{ c, s \}$.
	
	Se define $f: \Omega \rightarrow \mathbb{R}$ mediante $f(c) = 1, \ f(s) = 0$.
	El diagrama es:
	
	\begin{figure}[H]
		\centering
		\begin{tikzpicture}[scale=1]
			% Circunferencia
			\draw[] (0,0) circle (2);
			\node at (-2, 1.5) {$\Omega$};
			\draw[-latex] (3, -.5) -- (10, -.5) node[above] {$\mathbb{R}$};
			
			% Puntos
			\coordinate (c) at (0,1);
			\coordinate (s) at (0,-1);
			
			\coordinate (R0) at (5, -.5);
			\coordinate (R1) at (7, -.5);
			
			\draw[fill] (c) circle (2pt) node[left] {$c$};
			\draw[fill] (s) circle (2pt) node[left] {$s$};
			
			\draw[fill] (R0) circle (2pt) node[below] {$0$};
			\draw[fill] (R1) circle (2pt) node[below] {$1$};
			
			% Flechas
			\draw[blue, thick, -latex] (c) to [bend left = 15] (R1);
			\draw[blue, thick, -latex] (s) to [bend right = 15] (R0);
		\end{tikzpicture}
	\end{figure}
	Determinar $f^{-1}(-\infty, x], \forall \, x \in \mathbb{R}$\vspace{2mm}
	
	Por definición de preimagen $f^{-1}(-\infty, x] = \{ w \in \Omega \ f(w) \le x \}$, entonces
	\[
	f^{-1}(-\infty, x] = \begin{cases}
		\varnothing & \mbox{si } x < 0\\
		\{s\} & \mbox{si } 0 \le x < 1\\
		\{c,s\} & \mbox{si } 1 \le x
	\end{cases}	
	\]
\end{fmd-example}

\subsection{Restricción y extensión de una función} \label{sec:restr}
\vspace{3mm} \index{función!restricción} \index{restricción!función}
\begin{fmd-definition}[Restricción de una función]
	Sean $f:X \rightarrow Y$, $A \subset X$ y la función $g: A \rightarrow Y \mid 
	g(x) = f(x) \, \forall \, x \in A$.
	
	Decimos que $g$ es la restricción de la aplicación $f$ al subconjunto $A$, y
	la denotamos $g:f \mid A$.
\end{fmd-definition}

\begin{fmd-definition}[Extensión de una función] \index{extensión!función} \index{función!extension}
	Si $g$ es la restricción de $f$ al subconjunto $A$, entonces $f: X \rightarrow
	Y$ es una extensión de la función $g$ sobre el conjunto $X$.
\end{fmd-definition}
Es claro que la restricción es única y la extensión no necesariamente, en efecto 
si $g: A \rightarrow Y$ y $A \subset X$ entonces podemos definir una extensión de
$g$ al conjunto $X$ de la siguiente manera:

Sea $y_0 \in Y$, definimos: 
\[ f: X \rightarrow Y \mbox{ mediante } f(x) = \begin{cases}
	g(x) & \mbox{si } x \in A\\
	y_0 & \mbox{si } x \in X - A
\end{cases} \]

\rule{\textwidth}{.5pt}

\subsection{Ejercicios}

\begin{enumerate}
	\item Sean $A$ y $B$ conjuntos. Demuestre que existe una biyección entre $\mathcal{P}(A \times B)$ y el conjunto de todas las relaciones de $A$ en $B$.
	
	\item Sean $A$ y $B$ conjuntos finitos y sea $f: A \to B$ una función. 
	\begin{enumerate}
		\item Demuestre que si $f$ es inyectiva, entonces $|A| \leq |B|$.
		\item Demuestre que si $f$ es sobreyectiva, entonces $|A| \geq |B|$.
		\item Demuestre que si $f$ es biyectiva, entonces $|A| = |B|$.
	\end{enumerate}
\end{enumerate}
	\chapter{Lógica matemática}

\section*{Introducción}

Este capítulo introduce los principios y reglas de la lógica matemática, una disciplina que permite analizar argumentos, deducir conclusiones y establecer conexiones entre proposiciones. Se estudian los conceptos de proposición, operadores lógicos (conjunción, disyunción, negación, implicación, bicondicional), tablas de verdad y equivalencias lógicas.
Se analizan las tautologías y contradicciones, proposiciones compuestas que siempre son verdaderas o siempre son falsas, respectivamente. Se introduce la lógica predicativa, una extensión de la lógica proposicional que permite expresar relaciones entre objetos y cuantificar sobre ellos utilizando cuantificadores universales y existenciales.
Finalmente, se presenta el sistema axiomático de Peano, un conjunto de axiomas que permite construir formalmente los números naturales y demostrar teoremas básicos de la aritmética, como el principio de inducción matemática.

\subsection*{Relevancia en ingeniería}

\begin{itemize}
	\item \textit{Diseño de circuitos digitales}: La lógica matemática es la base del diseño de circuitos digitales, donde se utilizan compuertas lógicas para implementar funciones booleanas.
	\item \textit{Verificación y validación de software}: Para demostrar la corrección de programas y sistemas.
	\item \textit{Inteligencia artificial}: La lógica matemática es una herramienta fundamental en la inteligencia artificial, especialmente en áreas como el razonamiento automático y la representación del conocimiento.
	\item \textit{Sistemas de control}: Se aplica en el diseño de sistemas de control para puentes y estructuras, garantizando respuestas adecuadas ante diferentes condiciones.
\end{itemize}

\section{Lógica proposicional}
\index{lógica!proposicional}
\subsection{Proposiciones}
\vspace{1em}
\begin{fmd-definition}[Proposición] \index{proposición}
	Una \gls{proposicion} es una oración que puede ser verdadera o falsa.
\end{fmd-definition}

\begin{fmd-example}
	\begin{itemize}
		\item Proposiciones verdaderas
		\begin{itemize}
			\item El sol sale por el este.
			\item 2 + 2 = 4
			\item La Tierra gira alrededor del Sol.
		\end{itemize}
		
		\item Proposiciones falsas
		\begin{itemize}
			\item Los unicornios existen
			\item 1 + 1 = 3
			\item La luna está hecha de queso verde
		\end{itemize}
	\end{itemize}
\end{fmd-example}

Ahora, veamos ejemplos de oraciones que \textbf{no son proposiciones}:

\begin{fmd-example}
	\begin{itemize}
		\item Preguntas:
		\begin{itemize}
			\item ?`Cómo estás hoy?
			\item ?`Qué hora es?
			\item ?`Dónde está mi libro?
		\end{itemize}
		
		\item Mandatos u Órdenes:
		\begin{itemize}
			\item !`Cerrá la puerta!
			\item Estudiá para el examen.
			\item Limpiá tu habitación.
		\end{itemize}
		
		\item Expresiones abiertas o incompletas:
		\begin{itemize}
			\item $x + 3 = 7$ (no es una proposición completa sin asignar un valor a $x$).
			\item ``Alguien ganará el premio'' (no especifica quién ganará el premio).
		\end{itemize}
	\end{itemize}
\end{fmd-example}

Las proposiciones se representan mediante letras (como $p$, $q$, $r$) y se combinan para formar argumentos más complejos.

\subsection{Operadores lógicos} \label{sec:operadores_logicos}
\index{operador!lógico}
A continuación presentamos varios \glspl{operadorlogico} y sus correspondientes \gls{tablav}. En una tabla de verdad se evalúan todos los posibles valores de verdad de las proposiciones componentes y se determina el valor de verdad resultante de la proposición completa. \index{tabla!verdad}

\subsubsection{Negación $\neg$} \index{negación}
La \gls{negacion} de una proposición $p$ se denota como $\neg p$ o $\sim p$. Representa la idea de que algo no es verdadero. Por ejemplo, si $p$ es ``llueve'', entonces $\neg p$ sería ``no llueve'', tabla \ref{tab:negacion}.

\begin{table}[h]
	\centering
	\begin{tabular}{|c|c|}
		\hline
		$p$ & $\neg p$\\ \hline
		$V$ & $F$ \\ 
		$F$ & $V$ \\ \hline
	\end{tabular}
	\caption{Tabla de verdad de la negación.}
	\label{tab:negacion}
\end{table}

La operación de negación ($\neg$) es análoga a la complementación ($^c$) de la teoría de conjuntos. El paralelismo se construye al considerar la pertenencia de un conjunto como una proposición $p$, si un elemento pertenece a $A$, $p$ es verdadero. El complemento $A^c$ contiene todos los elementos que no pertenecen a $A$, entonces $p$ es falso, por lo tanto $\neg p$ es verdadero, fig. \ref{fig:negacion}.

\begin{figure}[h]
	\centering
	\begin{tikzpicture}[scale=1]
		\filldraw[fill=lightgray] (0, 0) rectangle (5,3.2); % Rectángulo del universo
		\filldraw[fill=white] (2, 1.6) circle (1.2); % Dibuja y rellena el conjunto A
		
		\node at (2, 2.2) {$A$}; % Etiqueta el conjunto A
		\node at (1.85, 1.5) {$p$ es $V$};
		\node at (4.2, 2) {$A^c$}; % Etiqueta el complemento de A
		\node at (4, 1.5) {$\neg p$ es $V$};
		\node at (4.1, 1) {($p$ es $F$)};
	\end{tikzpicture}
	\caption{\(A^c\) es equivalente a $\neg p$}
	\label{fig:negacion}
\end{figure}

\subsubsection{Conjunción \glsentrysymbol{conjuncion}} \index{conjunción}
La \gls{conjuncion} de dos proposiciones $p$ y $q$ se denota como $p \land q$. Representa la idea de que ambas proposiciones son verdaderas. Por ejemplo, si $p$ es ``es lunes'' y $q$ es ``tengo una reunión'', entonces $p \land q$ sería ``es lunes y tengo una reunión''. Su tabla de verdad se muestra en la tabla \ref{tab:conjuncion}.

\begin{table}[H]
	\centering
	\begin{tabular}{|c|c|c|}
		\hline
		$p$ & $q$ & $p \land q$ \\
		\hline
		$V$ & $V$ & $V$ \\
		$V$ & $F$ & $F$ \\
		$F$ & $V$ & $F$ \\
		$F$ & $F$ & $F$ \\
		\hline
	\end{tabular}
	\caption{Tabla de verdad de la conjunción.}
	\label{tab:conjuncion}
\end{table}

La operación lógica de conjunción es análoga a la intersección de la teoría de conjuntos. La intersección $A \cap B$ contiene solo los elementos que pertenecen a $A$ y a $B$ ($p \land q$ es $V$), fig. \ref{fig:conjuncion}.

\begin{figure}[h]
	\centering
\begin{tikzpicture}
	% Dibujar el universal
	\draw (-3, -2) rectangle (3, 2);
	
	% Sombrear la intersección A ∩ B
	\begin{scope}
		\clip (-.75,0) circle (1.5cm);
		\filldraw[fill=lightgray, draw=black] (.75,0) circle (1.5cm);
	\end{scope}
	
	% Dibujar el conjunto A
	\draw (-.75,0) circle (1.5cm) node[left] {\(A \ (p) \)};
	
	% Dibujar el conjunto B
	\draw (.75,0) circle (1.5cm) node[right] {\(B \ (q)\)};
	
	% Etiqueta de la intersección
	\node at (0, .5) {\(A \cap B\)};
	\node at (0, -.5) {\((p \land q)\)};
\end{tikzpicture}
\caption{$A \cap B$ equivalente a $p \land q$}
\label{fig:conjuncion}
\end{figure}

\subsubsection{Disyunción inclusiva \glsentrysymbol{disyuncion}} \index{disyunción!inclusiva}
La \gls{disyuncion} de dos proposiciones $p$ y $q$ se denota como $p \lor q$. Representa la idea de que al menos una de las proposiciones es verdadera. Por ejemplo, si $p$ es ``estudiaré'' y $q$ es ``veré una película'', entonces $p \lor q$ sería ``estudiaré o veré una película'', tabla \ref{tab:disyuncion}.

\begin{table}[H]
	\centering
	\begin{tabular}{|c|c|c|} \hline
		 $p$ & $q$ & $p$ $\lor q$ \\ \hline
		 $V$ & $V$ & $V$ \\
		 $V$ & $F$ & $V$ \\
		 $F$ & $V$ & $V$ \\
		 $F$ & $F$ & $F$ \\ \hline
	\end{tabular}
	\caption{Tabla de verdad de la disyunción.}
	\label{tab:disyuncion}
\end{table}

La operación lógica de disyunción es análoga a la unión de la teoría de conjuntos. La unión $A \cup B$ contiene los elementos que pertenecen a $A$ o a $B$ o a ambos, en otras palabras, siempre y cuando un elemento pertenezca a $A$ o a $B$ o a ambos $p \lor q$ será $V$, fig. \ref{fig:disyuncion}.

\begin{figure}[H]
	\centering
	\begin{tikzpicture}
		% Dibujar el universal
		\draw (-3, -2) rectangle (3, 3);
		
		% Sombrear la unión A ∪ B
		\begin{scope}
			\fill[lightgray] (-.75,0) circle (1.5cm);
			\fill[lightgray] (.75,0) circle (1.5cm);
		\end{scope}
		
		% Dibujar el conjunto A
		\draw (-.75,0) circle (1.5cm);
		\node at (-1.25, .25) {\(A\)};
		\node at (-1.25, -.25) {$(p)$};
		
		% Dibujar el conjunto B
		\draw (.75,0) circle (1.5cm);
		\node at (1.25, .25) {\(B\)};
		\node at (1.25, -.25) {$(q)$};
		
		% Etiqueta de la unión
		\node at (0, 2.5) {\(A \cup B\)};
		\node at (0, 2) {\((p \lor q)\)};
	\end{tikzpicture}
	\caption{$A \cup B$ equivalente a $p \lor q$}
	\label{fig:disyuncion}
\end{figure}

\subsubsection{Diferencia simétrica o disyunción excluyente $\oplus$}
\index{diferencia!simétrica} \index{disyunción!excluyente}
La diferencia simétrica o \gls{disyuncione} es verdadera cuando \textit{exactamente una} de las dos proposiciones es verdadera, por ejemplo ``$p$ o $q$ pero no ambos''. Se denota por $p \oplus q$ o $p \nleftrightarrow q$ o $p \underline{\vee} q$, tabla \ref{tab:disyuncion_excluyente}.

\begin{table}[H]
	\centering
	\begin{tabular}{|c|c|c|} \hline
		$p$ & $q$ & $p \oplus q$ \\ \hline
		$V$ & $V$ & $F$ \\
		$V$ & $F$ & $V$ \\
		$F$ & $V$ & $V$ \\
		$F$ & $F$ & $F$ \\ \hline
	\end{tabular}
	\caption{Tabla de verdad de la disyunción excluyente.}
	\label{tab:disyuncion_excluyente}
\end{table}

\begin{fmd-example}[Disyunción excluyente]
	\begin{itemize}
		\item ``Podés elegir té $(p)$ o café $(q)$'' (implícitamente excluyente, ya que normalmente no se consumen ambas bebidas al mismo tiempo). 
		\item ``El interruptor está encendido $(p)$ o apagado $(q)$'' (excluyente, ya que un interruptor solo puede estar en uno de esos dos estados).
		\item ``Ganará la final el equipo local $(p)$ o el equipo visitante $(q)$'' (excluyente, ya que solo uno de los dos equipos puede ganar la final).
	\end{itemize}
\end{fmd-example}

Esta operación es análoga a la diferencia simétrica de la teoría de conjuntos. $A \triangle B$ contiene los elementos que pertenecen a $A$ o a $B$ pero no a ambos, fig. \ref{fig:disyuncion_excluyente}.

\begin{figure}[h]
	\centering
	\begin{tikzpicture}
		% Dibujar el universal
		\draw (-3, -2) rectangle (3, 3);
		
		% Sombrear la parte de A que no está en B
		\filldraw[fill=lightgray, draw=black] (-.75,0) circle (1.5cm);
		\begin{scope}
			\clip (.75,0) circle (1.5cm);
			\draw[fill=white] (-.75,0) circle (1.5cm);
		\end{scope}
		
		% Sombrear la parte de B que no está en A
		\filldraw[fill=lightgray, draw=black] (.75,0) circle (1.5cm);
		\begin{scope}
			\clip (-.75,0) circle (1.5cm);
			\draw[fill=white] (.75,0) circle (1.5cm);
		\end{scope}
		
		\draw (-.75, 0) circle (1.5cm);
		
		% Etiquetas de los conjuntos A y B
		\node at (-1.25, .25) {\(A\)};
		\node at (-1.25, -.25) {\((p)\)};
		\node at (1.25, .25) {\(B\)};
		\node at (1.25, -.25) {\((q)\)};
		
		% Etiqueta de la diferencia simétrica
		\node at (0, 2.5) {\(A \triangle B\)};
		\node at (0, 2) {\((p \oplus q)\)};
	\end{tikzpicture}
	\caption{$A \triangle B$ equivalente a $p \oplus q$}
	\label{fig:disyuncion_excluyente}
\end{figure}


\subsubsection{Implicación o condicional $\implies$} \index{implicación} \index{condicional}
Si se combinan dos proposiciones por medio de las palabras \textit{``si..., entonces...''} se obtiene un \gls{implicacion} o \textit{proposición condicional}. La implicación de una proposición $p$ a otra $q$ se denota como $p \implies q$. Representa la idea de que si $p$ es verdadero, entonces $q$ también debe serlo. Por ejemplo, si $p$ es ``estudiaré'', entonces $q$ es ``aprobaré el examen''. La tabla de verdad es:

\begin{table}[H]
	\centering
	\begin{tabular}{|c|c|c|} \hline
		$p$ & $q$ & $p \implies q$ \\ \hline
		$V$ & $V$ & $V$ \\
		$V$ & $F$ & $F$ \\
		$F$ & $V$ & $V$ \\
		$F$ & $F$ & $V$ \\ \hline
	\end{tabular}
	\caption{Tabla de verdad de la implicación}
	\label{tab:implicacion}
\end{table}

Se observa que $p \implies q$ es verdadero si y sólo si no se da el caso de que $p$ sea verdadero y $q$ falso.

En teoría de conjuntos la inclusión $A \subset B$ es verdadero si y sólo si no existe un elemento que esté en $A$ pero no en $B$. La conexión entre la implicación lógica y la inclusión de conjuntos se visualiza en la fig. \ref{fig:inclusion_logica}.

\begin{figure}[H]
	\centering
	\begin{tikzpicture}
		% Dibujar el universal
		\draw (-3, -2.5) rectangle (3, 3.2);
		
		% Dibujar el conjunto B
		\filldraw[fill=lightgray, draw=black] (0,0) circle (2cm);
		\node at (1, .25) {$B$};
		\node at (1, -.25) {$(q)$};
		
		% Dibujar el conjunto A dentro de B
		\filldraw[fill=white, draw=black] (-0.5,0) circle (1cm);
		\node at (-.5, .25) {$A$};
		\node at (-.5, -.25) {$(p)$};
		
		% Etiqueta de la implicación e inclusión
		\node at (0, 2.8) {\(A \subset B\)};
		\node at (0, 2.3) {\((p \implies q)\)};
	\end{tikzpicture}
	\caption{$A \subset B$ equivalente a $p \implies q$}
	\label{fig:inclusion_logica}
\end{figure}

Proposiciones. Para algún $x \in U$:
\begin{itemize}
	\item $p: x \in A$.
	\item $q: x \in B$.
	\item La implicación $p \implies q$ se interpreta como: Si $x$ está en $A$, entonces $x$ está en $B$.
\end{itemize}

Paralelismo con la tabla de verdad:
\begin{itemize}
	\item $p = V$, $q = V$ $\left(x \in A \land x \in B \right)$: Corresponde a elementos que están tanto en $A$ como en $B$, lo cual es consistente con $A \subset B$, la implicación es verdadera.
	\item $p=V$, $q = F$, $(x \in A \land x \not \in B)$: Este caso viola la inclusión $A \subset B$ ya que hay un elemento en $A$ que no está en $B$, la implicación es falsa.
	\item $p = F$, $q = V$ $(x \not \in A \land x \in B)$: Esto es consistente con $A \subset B$, ya que elementos fuera de $A$ pueden estar en $B$ sin violar la inclusión, la implicación es verdadera.
	\item $p = F$, $q = F$ $\left( x \not \in A \land x \not \in B \right)$: También es consistente con $A \subset B$, ya que elementos que no están en $A$ ni tampoco en $B$ no afectan la inclusión, la implicación es verdadera.
\end{itemize}


\subsubsection{Doble implicación o bicondicional $\iff$} \index{implicación!doble} \index{implicación!condicional} \index{bicondicional} \index{si y solo si} \label{sec:bicondicional}
 La doble implicación, o \gls{bicondicional}, o proposición ``si y solo si'' se denota como $p \iff q$. Representa la idea de que $p$ es verdadero si y solo si $q$ también lo es. Puede verse, si es verdadera, como cumplimiento en ambos sentidos las implicaciones $p \implies q$ y $q \implies b$. Tabla \ref{tab:bicondicional}.

\begin{table}[H]
	\centering
	\begin{tabular}{|c|c|c|} \hline
		$p$ & $q$ & $p \iff q$ \\ \hline
		$V$ & $V$ & $V$ \\
		$V$ & $F$ & $F$ \\
		$F$ & $V$ & $F$ \\
		$F$ & $F$ & $V$ \\ \hline
	\end{tabular}
	\caption{Tabla de verdad de la bicondicional}
	\label{tab:bicondicional}
\end{table}

\begin{fmd-example}
	Si $p$ es ``Es fin de semana'' y $q$ es ``No tengo clases''.
	\begin{itemize}
		\item La implicación $p \implies q$ sería: ``Es un fin de semana entonces no tengo clases''.
		\item La implicación recíproca $q \implies p$ sería: ``No tengo clases entonces es un fin de semana''
		\item La bicondicional $p \iff q$ sería: ``Es fin de semana si y solo si no tengo clases''
	\end{itemize}
\end{fmd-example}

En la teoría de conjuntos, su equivalente es la \textbf{igualdad de conjuntos}. Si definimos, igual que antes, $p$: pertenencia al conjunto $A$ y $q$: pertenencia al conjunto $B$, la doble implicación $p \iff q$ se interpreta como: un elemento está en A si y solo si está en B.

\subsection{Condiciones necesarias y suficientes}
\index{condición!necesaria} \index{condición!suficiente}
\subsubsection{Condición necesaria y condición suficiente}

Consideremos la tabla de verdad de la implicación, tabla \ref{tab:implicacion}, que, reescribimos a continuación:

\begin{table}[H]
	\centering
	\begin{tabular}{|c|c|c|} \hline
		$p$ & $q$ & $p \implies q$ \\ \hline
		$V$ & $V$ & $V$ \\
		$V$ & $F$ & $F$ \\
		$F$ & $V$ & $V$ \\
		$F$ & $F$ & $V$ \\ \hline
	\end{tabular}
\end{table}

Hay tres casos en que $p \implies q$ es $V$, y entre ellos hay uno en que $p$ es $V$, en el cual resulta $q$ verdadera. Es obvio que nos referimos al primer renglón de la tabla, y se tiene que si $p \implies q$ es $V$ y $p$ es $V$, entonces $q$ es $V$. Se dice entonces que en antecedente $p$ es \textit{condición suficiente} para el consecuente $q$.

En cambio, si $p$ es $F$, nada podemos decir de $q$, puesto que puede ser $V$ o $F$, por otra parte, cuando $p \implies q$ es $V$, si $q$ es $V$, entonces $p$ puede ser $V$ o $F$; mas, para que $p$ sea $V$ se necesita que $q$ lo sea. Se dice entonces que $q$ es \textit{condición necesaria} para $p$.

Resumiendo, si $p \implies q$ es $V$, entonces $p$ es condición suficiente para $q$ y $q$ es condición necesaria para $p$, suele expresarse así:

\begin{itemize}
	\item $q$ si $p$: condición suficiente;
	\item $p$ sólo si $q$: condición necesaria.
\end{itemize}

\begin{fmd-example}
	``Si $T$ es equilátero, entonces $T$ es isósceles''
	
	En este caso:
	\begin{itemize}
		\item $p$: $T$ es equilátero;
		\item $q$: $T$ es isósceles.
	\end{itemize}
	
	$p$ es condición suficiente para $q$, en este ejemplo, que un triángulo sea equilátero es suficiente para asegurar que sea isósceles. Por otra parte, $T$ es equilátero sólo si es isósceles; es decir, que un triángulo sea isósceles es necesario para que sea equilátero.
\end{fmd-example}

\subsubsection{Condición necesaria y suficiente}

Sea ahora la doble implicación $p \iff q$, esto significa $(p \implies q) \land (q \implies p)$. Si $p \iff q$ es $V$, entonces $p \implies q$ es $V$ y $q \implies p$, es $V$. Se tiene, atendiendo a la primera, que $p$ es condición suficiente para $q$; y, teniendo en cuenta la segunda implicación, ocurre que $p$ es condición necesaria para $q$.

Si $p \implies q$ es $V$, entonces el antecedente $p$ es condición necesaria y suficiente para el consecuente $q$.

Análogamente, en el caso de la doble implicación verdadera, el consecuente $q$ es también condición necesaria y suficiente para el antecedente $p$.

\begin{fmd-example}
	``$T$ es equilátero si y sólo si $T$ es equiángulo''
	
	Es la doble implicación de las proposiciones:
	\begin{itemize}
		\item $p$: $T$ es equilátero;
		\item $q$: $T$ es equiángulo.
	\end{itemize}
	Aquella es $V$, y cualquiera de las dos proposiciones es condición necesaria y suficiente de la otra.
\end{fmd-example}

\subsection{Tautología, contradicción y contingencia}

\subsubsection{Tautología} \index{tautología}
\vspace{1em}
\begin{fmd-definition}[Tautología]
	Cuando una proposición compuesta, como por ejemplo $(p \implies q) \land p \implies q$ es verdadera, independientemente de los valores de verdad de las proposiciones componentes, se dice que tal proposición es una \gls{tautologia} o \textit{ley lógica}, o sea, una tautología es una proposición lógica que siempre es verdadera.
\end{fmd-definition}

En el cálculo proposicional se utilizan las siguientes leyes o tautologías cuyas demostraciones se reduce a la confección de las correspondientes tablas de verdad.

\begin{enumerate}
	\item \textbf{Identidad}
	\[ p \implies p; \qquad p \iff p \]
	
	\[ p \land \mbox{verdadero} = p \quad \mbox{y} \quad p \lor \mbox{falso} = p \]
	
	\item \textbf{Ley del tercero excluido}
	\[ p \lor \neg p = V\]
	Una proposición es verdadera o su negación es verdadera.
	
	\item \textbf{No contradicción}
	\[ p \land \neg p = F \]
	
	\item \textbf{Involución o doble negación}
	\[ \neg (\neg p) \iff p \]
	La doble negación es equivalente a una afirmación: ``no, no $p$, equivale a $p$''
	\item Idempotencia
	\[ (p \land p) \iff p \quad y \quad (p \lor p) \iff p\]
	\item \textbf{Conmutatividad}
	\begin{enumerate}[label=\alph*)]
		\item De la conjunción: $p \land q \iff q \land p$. 		
		\item De la disyunción: $p \lor q \iff q \lor p$.
	\end{enumerate}
	\item \textbf{Asociatividad}
	\begin{enumerate}[label=\alph*)]
		\item De la conjunción: $(p \land q) \land r \iff p \land (q \land r)$,
		
		\item De la disyunción: $(p \lor q) \lor r \iff p \lor (q \lor r)$
	\end{enumerate}
	\item \textbf{Distributividad}
	\begin{enumerate}[label=\alph*)]
		\item De la conjunción respecto de la disyunción:
		\[ p \lor (q \land r) \iff (p \land r) \lor (q \land r) \]
		
		\item De la disyunción respecto de la conjunción:
		\[(p \land (q \lor r) \iff (p \lor r) \land (q \lor r) \]
		
		\item De la implicación respecto de la disyunción
		\[ \left( p \implies q \lor r \right) \iff \left( (p \implies q) \lor (p \implies r) \right) \]

		\item De la implicación respecto de la conjunción
		\[ \left( p \implies q \land r \right) \iff \left( (p \implies q) \land (p \implies r) \right) \]
	\end{enumerate}
	\item \textbf{Leyes de De Morgan}
	\begin{enumerate}[label=\alph*)]
		\item La negación de una disyunción es equivalente a la conjunción de las negaciones:
		\[ \neg (p \lor q) \iff \neg p \land \neg q\]
		\item La negación de  una conjunción es equivalente a la disyunción de las negaciones.
		\[ \neg (p \land q) \iff \neg p \lor \neg q \]
	\end{enumerate}
	
	\item \textbf{Leyes de absorción total}
	\begin{enumerate}
		\item Primera forma: \[ p \lor (p \land q) \iff p \]
		\item Segunda forma: \[ p \land (p \lor q) \iff p \]
	\end{enumerate}
	Si $p$ es verdadero (o falso) ambos miembros son verdaderos (o falsos) independientemente de $q$.
	
	\item \textbf{Leyes de absorción parcial}
	\begin{enumerate}
		\item \( p \lor \left(\neg p \land q\right) \iff p \lor q \)
		\item \( p \land \left(\neg p \lor q\right) \iff p \land q \)
	\end{enumerate}
\end{enumerate}

La mayoría de estas leyes lógicas tienen sus análogas en teoría de conjuntos, algunas de las cuales se listan en la tabla \ref{tab:logica_conjuntos}.

\begin{table}[H]
	\centering
	\small
	\begin{tabular}{|l|c|c|}
		\hline
		\textbf{Ley} & \textbf{Lógica} & \textbf{Conjuntos} \\ \hline
		Identidad & \makecell{ \( p \implies p \) \\ \( p \iff p \) \\ \( p \land \text{V} = p \) \\ \( p \lor \text{F} = p \)} & \makecell{ \(A = A\) \\ \\ \( A \cap U = A \) \\ \( A \cup \emptyset = A \)} \\ \hline
		Tercero excluido & \( p \lor \neg p = \text{V} \) & \( A \cup A^c = U \) \\ \hline
		No contradicción & \( p \land \neg p = \text{F} \) & \( A \cap A^c = \emptyset \) \\ \hline
		Involución & \( \neg(\neg p) \iff p \) & \( \left(A^c\right)^c = A \) \\ \hline
		Idempotencia & \makecell{ \( p \land p \iff p \) \\ \( p \lor p \iff p \)} & \makecell{ \( A \cap A = A \) \\ \( A \cup A = A \)} \\ \hline
		Conmutatividad & \makecell{\( p \land q \iff q \land p \) \\ \( p \lor q \iff q \lor p \)} & \makecell{\( A \cap B = B \cap A \) \\ \( A \cup B = B \cup A \)} \\ \hline
		Asociatividad & \makecell{ \( (p \land q) \land r \iff p \land (q \land r) \) \\ \( (p \lor q) \lor r \iff p \lor (q \lor r) \)} & \makecell{\( (A \cap B) \cap C = A \cap (B \cap C) \) \\ \( (A \cup B) \cup C = A \cup (B \cup C) \)} \\ \hline
		Distributividad & \makecell{\( p \land (q \lor r) \iff (p \land q) \lor (p \land r) \) \\ \( p \lor (q \land r) \iff (p \lor q) \land (p \lor r) \)} & \makecell{\( A \cap (B \cup C) = (A \cap B) \cup (A \cap C) \) \\ \( A \cup (B \cap C) = (A \cup B) \cap (A \cup C) \)} \\ \hline
		Leyes de De Morgan & \makecell{ \( \neg(p \lor q) \iff \neg p \land \neg q \) \\ \( \neg(p \land q) \iff \neg p \lor \neg q \)} & \makecell{\( (A \cup B)^c = A^c \cap B^c \) \\ \( (A \cap B)^c = A^c \cup B^c \)} \\ \hline
		Ley de absorción & \makecell{\( p \lor (p \land q) \iff p \) \\ \( p \land (p \lor q) \iff p \)} & \makecell{\( A \cup (A \cap B) = A \) \\ \( A \cap (A \cup B) = A \)} \\ \hline
	\end{tabular}
	\caption{Paralelismo entre las leyes lógicas y las de conjuntos.}
	\label{tab:logica_conjuntos}
\end{table}

\subsubsection{Contradicción}
\index{contradicción}
Una \gls{contradiccion} es una proposición lógica que siempre es falsa, sin importar los valores de verdad de sus componentes. Es una fórmula que es falsa en todas las interpretaciones posibles.

\begin{fmd-example}[Contradicción]
	\begin{itemize}
		\item Conjunción de una proposición y su negación:
		\[ p \land \neg p \]
		
		\item Negación de una tautología:
		\[ \neg \left( p \lor \neg p \right) \]
	\end{itemize}
\end{fmd-example}

\subsubsection{Contingencia} \index{contingencia}
Una \gls{contingencia} es una proposición que puede ser tanto verdadera como falsa, dependiendo de las circunstancias. Su valor de verdad depende de la situación o el contexto.

\begin{fmd-example}[Contingencia]
	\begin{itemize}
		\item \textbf{Afirmaciones sobre el futuro}: ``\textit{Lloverá mañana en Caaguazú}''. Esta afirmación es contingente porque depende de factores meteorológicos que aún no se han determinado.
		\item \textbf{Juicios de valor}: ``\textit{La pizza es más rica que la pasta}''. Esta afirmación es contingente porque depende de las preferencias personales de cada individuo.
		\item \textbf{Hipótesis científicas}: ``\textit{La materia oscura existe}''. Aunque hay evidencia que la apoya, aún no se ha probado de manera concluyente, por lo que es una afirmación contingente.
		\item \textbf{Eventos históricos}: ``\textit{La Segunda Guerra Mundial comenzó en 1939}''. Esta afirmación es contingente en el sentido de que pudo haber ocurrido de otra manera si diferentes factores históricos hubieran sido distintos.
	\end{itemize}
\end{fmd-example}

\textbf{?`Qué hace a una proposición contingente?}

\begin{enumerate}
	\item \textit{Dependencia del contexto}: Su valor de verdad depende completamente de las circunstancias específicas en las que se evalúa.
	\item \textit{Posibilidad y necesidad}: Una contingencia es posible (puede ser verdadera) y al mismo tiempo, es posible que no sea verdadera (puede ser falsa).
	\item \textit{No es una verdad necesaria}: A diferencia de las tautologías, las contingencias no son verdades universales que se mantienen en todos los mundos posibles.
	\item \textit{No es una falsedad necesaria}: Tampoco son falsedades absolutas como las contradicciones.
\end{enumerate}

\subsection{Funciones booleanas}
En lógica simbólica, las \textit{funciones booleanas} son aquellas que toman uno o más valores de verdad (verdadero o falso) como entrada y producen un único valor de verdad como salida. Estas funciones se construyen utilizando operadores lógicos homónimos, que son:

\begin{itemize}
	\item \textbf{Conjunción} (AND): Representada por $\land$ o $\cdot$. Es verdadera solo si ambas proposiciones de entrada son verdaderas.
	\item \textbf{Disyunción} (OR): Representada por $\lor$ o $+$. Es verdadera si al menos una de las proposiciones de entrada es verdadera.
	\item \textbf{Negación} (NOT): Representada por $\neg$ o una barra sobre la proposición. Invierte el valor de verdad de la proposición.
\end{itemize}

A partir de estos operadores básicos, podemos construir funciones booleanas más complejas. Algunas de las funciones booleanas más comunes son:

\begin{enumerate}
	\item \textbf{Función Identidad}:
	\begin{itemize}
		\item Definición: La salida es igual a la entrada.
		\item Expresión: $f(p) = p$
		\item Ejemplos:
		\begin{enumerate}
			\item Si $p$ es verdadero, entonces $f(p)$ es verdadero.
			\item Si $p$ es falso, entonces $f(p)$ es falso.
		\end{enumerate}
	\end{itemize}
	
	\item \textbf{Función Constante}:
	\begin{itemize}
		\item Definición: La salida siempre es el mismo valor de verdad, independientemente de la entrada.
		\item Expresión:
		\begin{itemize}
			\item $f(p) = 1$ (Función constante verdadera)
			\item $f(p) = 0$ (Función constante falsa)
		\end{itemize}
		\item Ejemplos:
		\begin{enumerate}
			\item Si $p$ es verdadero, $f(p)$ sigue siendo verdadero (para la función constante verdadera) o falso (para la función constante falsa).
			\item Si $p$ es falso, $f(p)$ sigue siendo verdadero (para la función constante verdadera) o falso (para la función constante falsa).
		\end{enumerate}
	\end{itemize}
	
	\item \textbf{Función Negación}:
	\begin{itemize}
		\item Definición: La salida es la negación de la entrada.
		\item Expresión: \( f(p) = \neg p \)
		\item Ejemplos:
		\begin{enumerate}
			\item Si $p$ es verdadero, entonces $f(p)$ es falso.
			\item Si $p$ es falso, entonces $f(p)$ es verdadero.
		\end{enumerate}
	\end{itemize}
	
	\item \textbf{Función Conjunción}:
	\begin{itemize}
		\item Definición: La salida es verdadera solo si todas las entradas son verdaderas.
		\item Expresión: \( f(p, q) = p \land q \)
		\item Ejemplos:
		\begin{enumerate}
			\item Si $p$ es verdadero y $q$ es verdadero, entonces $f(p, q)$ es verdadero.
			\item Si $p$ es verdadero y $q$ es falso, o viceversa, o ambos son falsos, entonces $f(p, q)$ es falso.
		\end{enumerate}
	\end{itemize}
	
	\item \textbf{Función Disyunción}:
	\begin{itemize}
		\item Definición: La salida es verdadera si al menos una de las entradas es verdadera.
		\item Expresión: \( f(p, q) = p \lor q \)
		\item Ejemplos:
		\begin{enumerate}
			\item Si $p$ es verdadero o $q$ es verdadero, o ambos son verdaderos, entonces $f(p, q)$ es verdadero.
			\item Si $p$ es falso y $q$ es falso, entonces $f(p, q)$ es falso.
		\end{enumerate}
	\end{itemize}
	
	\item \textbf{Función Implicación}:
	\begin{itemize}
		\item Definición: La salida es falsa solo si la primera entrada es verdadera y la segunda es falsa
		\item Expresión: \( f(p, q) = p \implies q  \)
		\item Ejemplos:
		\begin{enumerate}
			\item Si $p$ es verdadero y $q$ es verdadero, entonces $f(p, q)$ es verdadero.
			\item Si $p$ es verdadero y $q$ es falso, entonces $f(p, q)$ es falso.
			\item Si $p$ es falso, independientemente del valor de $q$, $f(p, q)$ es verdadero
		\end{enumerate}
	\end{itemize}
	
	\item \textbf{Función Equivalencia} (Bicondicional):
	\begin{itemize}
		\item Definición: La salida es verdadera si ambas entradas tienen el mismo valor de verdad
		\item Expresión: \( f(p, q) = p \iff q  \)
		\item Ejemplos:
		\begin{enumerate}
			\item Si $p$ es verdadero y $q$ es verdadero, o si $p$ es falso y $q$ es falso, entonces $f(p, q)$ es verdadero.
			\item Si $p$ es verdadero y $q$ es falso, o viceversa, entonces $f(p, q)$ es falso.
		\end{enumerate}
	\end{itemize}
\end{enumerate}

\subsection{Proposiciones equivalentes}
\index{proposición!equivalente}
Al unir dos proposiciones cualesquiera por medio de la frase \textit{``si y sólo si''} cuyo símbolo\footnote{Sección \ref{sec:bicondicional}, pág. \pageref{sec:bicondicional}.} es \( \iff \)) se obtiene una proposición compuesta que se llama \textit{equivalencia}. Las proposiciones conectadas de esta manera reciben los nombres de \textit{miembro izquierdo} y \textit{miembro derecho} de la equivalencia. Al afirmar la equivalencia de las proposiciones, se excluye la posibilidad de que una sea verdadera y la otra falsa; por lo tanto, una equivalencia es verdadera si sus miembros izquierdo y derecho son o bien ambos verdaderos o bien ambos falsos; en caso contrario la equivalencia es falsa.

Dos proposiciones son \textit{equivalentes} si y solo si tienen el mismo valor de verdad en todas las posibles combinaciones de valores de verdad de sus proposiciones componentes. En otras palabras, sus tablas de verdad son idénticas.

La equivalencia lógica captura la idea de que dos proposiciones, aunque expresadas de manera diferente, representan la misma información desde el punto de vista lógico. Si dos proposiciones son equivalentes, podemos sustituir una por la otra en cualquier contexto lógico sin alterar el significado o la validez del razonamiento.

\begin{fmd-example}[Proposiciones equivalentes]
	Dada la proposición compuesta:
	\begin{center}
		\emph{``$x$ es un número positivo si, y solo si, $3x$ es un número positivo''}
	\end{center}
	
	De \textit{$x$ es un número positivo} ($p$) se sigue que \textit{$3x$ es un número positivo} ($q$), (en símbolos $p \implies q$) y, recíprocamente, de \textit{$3x$ es un número positivo} se sigue que \textit{$x$ es un número positivo} ($q \implies p$). Las condiciones de que $x$ sea un número positivo y que $3x$ sea un número positivo son equivalentes entre sí ($p \iff q$).
	
	La condición de ser $x$ un número positivo es \textit{necesaria y suficiente} para que $3x$ sea un número positivo, recíprocamente, para que $x$ sea un número positivo es necesario y suficiente que $3x$ lo sea, significa que, cada una de las proposiciones representa una \textit{condición necesaria y suficiente} para la otra.
\end{fmd-example}

\subsubsection{Verificación de equivalencia}

La forma más directa de comprobar si dos proposiciones son equivalentes es construir sus tablas de verdad y compararlas. Si las columnas de resultados finales son idénticas, entonces las proposiciones son equivalentes. Utilizaremos el símbolo de equivalencia ``$\equiv$''

\begin{fmd-example}
	Probar la ley de De Morgan: \( \neg \left(p \land q \right) \equiv \neg p \lor \neg q \)
	\begin{table}[H]
		\centering
		\begin{tabular}{|c|c|c|c|c|c|c|}
			\hline
			$p$ & $q$ & $p \land q$ & $\neg (p \land q)$ & $\neg p$ & $\neg q$ & $\neg p \lor \neg q$ \\
			\hline
			V & V & V & F & F & F & F \\
			V & F & F & V & F & V & V \\
			F & V & F & V & V & F & V \\
			F & F & F & V & V & V & V \\
			\hline
		\end{tabular}
	\end{table}
	Se observa que la cuarta columna, de $\neg \left(p \land q \right)$, y la última columna, de \( \neg p \lor \neg q \) son idénticas, por lo tanto estas proposiciones son equivalentes.
\end{fmd-example}

\begin{fmd-example}[Implicación material]
	Probar que: \( p \implies q \equiv \neg p \lor q \)

\begin{table}[H]
	\centering
	\begin{tabular}{|c|c|c|c|c|}
		\hline
		$p$ & $q$ & $p \implies q$ & $\neg p$ & $\neg p$ $\lor q$ \\
		\hline
		V & V & V & F & V \\
		V & F & F & F & F \\
		F & V & V & V & V \\
		F & F & V & V & V \\
		\hline
	\end{tabular}
\end{table}
	Se observa que la tercera columna, de $p \implies q$, y la última columna, de \( \neg p \lor q \) son idénticas, por lo tanto estas proposiciones son equivalentes. Esta equivalencia se conoce como \textit{ley de implicación material}.
	\label{ex:implicacion}
\end{fmd-example}

El concepto de equivalencia lógica es fundamental en la simplificación de expresiones lógicas, en la demostración de teoremas y en la resolución de problemas lógicos. Nos permite manipular y transformar proposiciones manteniendo su significado original, lo que facilita el análisis y la comprensión de argumentos lógicos complejos.

\subsection{Álgebra de proposiciones}
\index{álgebra!proposiciones}
El álgebra de proposiciones nos proporciona un marco formal para combinar y manipular proposiciones utilizando operadores lógicos, lo que nos permite analizar y razonar sobre argumentos y demostraciones de manera rigurosa.

\begin{fmd-example}[Simplificación de expresiones lógicas]
	Simplifica la siguiente expresión lógica utilizando las leyes del álgebra de proposiciones:
	
	\[
	(\neg p \land q) \lor (p \land \neg q) \lor (p \land q)
	\]
	\textbf{Solución}
	\[ \begin{array}{ll}
		(\neg p \land q) \lor (p \land \neg q) \lor (p \land q) & \\
		\equiv (\neg p \land q) \lor [(p \land \neg q) \lor (p \land q)] & \mbox{agrupando términos}\\
		\equiv (\neg p \land q) \lor [p \land (\neg q \lor q)] & \mbox{distributividad}\\
		\equiv (\neg p \land q) \lor p & \mbox{ley del tercero excluido}\\
		\equiv p \lor (\neg p \land q) & \mbox{conmutatividad de la disyunción}\\
		\equiv p \lor q & \mbox{ley de absorción}
		\end{array} \]
\[ (\neg p \land q) \lor (p \land \neg q) \lor (p \land q) \equiv p \lor q \]
\end{fmd-example}

\begin{fmd-example}[Demostración de equivalencia]
	Demuestra que las siguientes expresiones son equivalentes:
	
	\[
	p \implies \left( q \lor r \right) \equiv \left( p \land \neg q \right) \implies r
	\]
	\textbf{Solución}
	\[ \begin{array}{ll}
	p \implies \left( q \lor r \right) & \\
		\equiv \neg p \lor \left(q \lor r\right) & \mbox{implicación material}\\
		\equiv \left(\neg p \lor q \right) \lor r& \mbox{asociatividad}\\
		\equiv \neg \left(p \land \neg q\right) \lor r & \mbox{De Morgan}\\
		\equiv \left(p \land \neg q\right) \implies r & \mbox{implicación material}
	\end{array} \]
\end{fmd-example}

\subsection{Proposiciones condicionales}
\index{proposiciones!condicionales}
Cuando tenemos una proposición en forma de implicación (como $p \implies q$,  ``si $p$, entonces $q$''), podemos derivar tres proposiciones adicionales, estas son la recíproca, la proposición contraria (o la negación de la proposición original) y la proposición contrarrecíproca.

\begin{enumerate}
	\item \textbf{Proposición recíproca} ($q \implies p$): Se forma al \textit{invertir la implicación} de una proposición dada. Ejemplo con el teorema de Pitágoras: \index{proposición!recíproca} \index{recíproca}
	\begin{itemize}
		\item Este teorema establece que: ``en un triángulo rectángulo, la suma de los cuadrados de los catetos es igual al cuadrado de la hipotenusa'' ($p \implies q$).
		
		
		\item La proposición recíproca afirmaría que: ``si la suma de los cuadrados de los catetos es igual al cuadrado de la hipotenusa, entonces el triángulo es rectángulo'' ($q \implies q$).
	\end{itemize}
	
	\item \textbf{Proposición Contraria} ($\neg p \implies \neg q$): \index{proposición!contraria} \index{contraria}
	Se obtiene al reemplazar tanto el antecedente como el consecuente de la proposición original por sus negaciones. Por ejemplo:
	
	\begin{itemize}
		\item Proposición original: ``Si llueve, entonces la calle estará mojada'' ($p \implies q$)
		\item Contraria:``Si no llueve, entonces la calle no estará mojada'' ($\neg p \implies \neg q$).
	\end{itemize}
	
	\item \textbf{Proposición Contrarrecíproca} ($\neg q \implies \neg p$): La contrarrecíproca es el resultado de intercambiar el antecedente y el consecuente en la proposición contraria. Por ejemplo: \index{proposición!contrarrecíproca} \index{contrarrecíproca}
	\begin{itemize}
		\item Proposición original: ``Si estudio, entonces saco buenas notas'' ($p \implies q$)
		\item Proposición contraria: ``Si no estudio, entonces no saco buenas notas'' ($\neg p \implies \neg q$).
		\item Contrarrecíproca: ``Si no saco buenas notas, entonces no estoy estudiando'' ($\neg q \implies \neg p$).
	\end{itemize}
\end{enumerate}

Estas tres proposiciones están relacionadas entre sí y nos ayudan a explorar diferentes aspectos de una implicación. Es importante entenderlas para razonar correctamente en lógica y matemáticas.

\textbf{Resumen}, en símbolos:

\begin{itemize}
	\item Implicación: $p \implies q$
	
	\item Recíproca: $q \implies p$
	
	\item Contraria: $\neg p \implies \neg q$
	
	\item Contrarrecíproca: $\neg q \implies \neg p$
\end{itemize}

Las cuatro implicaciones anteriores se llaman conjugadas, y cualquiera de ellas puede tomarse como directa. El siguiente esquema nos proporciona una relación que las vincula:
\begin{figure}[H]
	\centering
	\begin{tikzpicture}[scale=1]
		\draw (-1.5, .75) rectangle (7.5, -3.75);
		
		\coordinate (A) at (0, 0);
		\coordinate (B) at (6, 0);
		\coordinate (C) at (0, -3);
		\coordinate (D) at (6, -3);
		
		\node at (A) {$p \implies  q$};
		\node at (B) {$q \implies  p$};
		\node at (C) {$\neg p \implies \neg q$};
		\node at (D) {$\neg q \implies \neg p$};
		
		\node at ($(A)!0.5!(B)$) {recíprocos};
		\node at ($(C)!0.5!(D)$) {recíprocos};
		
		\node[rotate=90] at ($(A)!0.5!(C)$) {contrarios};
		\node[rotate=90] at ($(B)!0.5!(D)$) {contrarios};
		
		\node[rotate=-30] at ($(A)!0.55!(D)$) {contra \quad recíprocos};
		\node[rotate=30] at ($(C)!0.55!(B)$) {contra \quad recíprocos};
	\end{tikzpicture}
\end{figure}

Es fácil verificar que las implicaciones contrarrecíprocas son equivalentes, es decir, las siguientes bicondicionales son tautologías:
\[ (p \implies q) \iff (\neg q \implies \neg p) \]
\[ (q \implies p) \iff (\neg p \implies \neg q) \]

Si la implicación directa es $V$, también lo es la contrarrecíproca, y no podemos afirmar la verdad de la recíproca o de la contraria. Pero, si son verdaderos un condicional y su recíproco o contrario, entonces son verdaderos los cuatro, y las proposiciones antecedente y consecuente son equivalentes.

\subsubsection{Demostración matemática}
\index{demostración!matemática}

Consideremos, previamente, los siguientes conceptos:
\begin{itemize}
	\item Decimos que $q$ es un \textit{múltiplo} de $p$ si $\exists n \in \mathbb{N} \ / \ q = np$
	\item También decimos que $n$ y $p$ son \textit{factores} de $q$ o que \textit{dividen} a $q$.
	\item Un elemento $p\in \mathbb{N}$ distinto de 1 se llama \textit{primo} si sus únicos factores son el 1 y el propio $p$.
	\item Dos números enteros son \textit{coprimos} cuando no tienen ningún divisor común positivo aparte del 1.
\end{itemize}


Se presenta continuamente la necesidad de demostrar la verdad de
\[ \begin{array}{c}
	p\\
	\mbox{Hipótesis}
\end{array} \implies \begin{array}{c}
q\\
\mbox{Tesis}
\end{array} \]
y, de acuerdo con lo expuesto, se presentan dos métodos:

\begin{enumerate}[label=\roman*)]
	\item \textit{Directo}: Si $p = F$, nada hay que probar, pues en este caso $p \implies q$ es $V$. Si $p$ es $V$ hay que establecer que el valor de verdad de $q$ es $V$.
	
	\begin{fmd-example}[Demostración por el método directo] \label{ex:directo}
		\textit{Proposición} ($p \Rightarrow q$): Si el cuadrado de un número entero, $a^2$ es múltiplo de un número primo $k$, entonces el número $a$ debe ser también múltiplo de $k$.
		
		\begin{itemize}
			\item Hipótesis ($p$): el cuadrado de un número entero $a^2$ es múltiplo de un número primo $k$;
			\item Tesis ($q$): $a$ es también múltiplo de $k$.
		\end{itemize}
		
		\textit{Demostración}:
		
		Supongamos que $a^2$ es múltiplo de $k$, donde $k$ es un número primo. Entonces, existe un entero $m$ tal que:
		$$a^2 = km$$
		
		La factorización única en números primos\footnote{Teorema \ref{teo:aritmetica} fundamental de la aritmética, pág. \pageref{teo:aritmetica}} nos dice que todo número entero se puede expresar de forma única como un producto de números primos (despreciando el orden de los factores). 
		
		Si descomponemos $a$ en sus factores primos, tendremos:
		$$a = p_1^{e_1}p_2^{e_2}...p_n^{e_n}$$
		donde $p_1, p_2, ..., p_n$ son números primos y $e_1, e_2, ..., e_n$ son números enteros no negativos.
		
		Elevando al cuadrado ambos lados de la igualdad, obtenemos:
		$$a^2 = p_1^{2e_1}p_2^{2e_2}...p_n^{2e_n}$$
		
		Como $a^2 = km$, entonces la factorización prima de $a^2$ debe contener el factor primo $k$ con alguna multiplicidad. Pero como la factorización prima es única, esto implica que $k$ debe ser igual a uno de los primos $p_1, p_2, ..., p_n$.
		
		Por lo tanto, $k$ divide a $a$, es decir, $a$ es múltiplo de $k$.
	\end{fmd-example}
	
	
	
	\item \textit{Indirecto}:
	\begin{enumerate}[label=\alph*)]
		\item Se utiliza la \textit{contrarrecíproca}, esto es, demostrar la verdad de $p \implies q$ es equivalente a probar la verdad de $\neg q \implies \neg p$.
		
		\begin{fmd-example}[Demostración por contrarrecíproca]
	\textit{Proposición} ($ p \Rightarrow q$): Si un número entero es par, entonces su cuadrado también es par.
	
	\begin{itemize}
		\item Hipótesis ($p$): Un número entero es par;
		\item Tesis ($q$):  su cuadrado también es par
	\end{itemize}
	
	\textit{Demostración}:
	
			La contrarrecíproca ($\neg q \Rightarrow \neg p$) es: Si el cuadrado de un número entero no es par (es decir, es impar), entonces el número entero tampoco es par (es impar).
			
			Supongamos que $n$ es el número entero del que estamos hablando,  y que $n^2$ es impar ($\neg q$). Si esto es así, entonces $n^2 = 2k + 1$ para algún entero $k$.
			
			Supongamos ahora que $n$ es par ($p$). Entonces $n = 2m$ para algún entero $m$. 
			Elevando al cuadrado ambos lados, obtenemos:
			\[ n^2 = (2m)^2 = 4 m^2 = 2 (2m)^2 \]
			
			Esto implica que $n^2$ es par, lo cual contradice nuestra suposición inicial de que $n^2$ era impar.
			
			Por lo tanto, nuestra suposición de que $n$ es par debe ser falsa. Concluimos que si $n^2$ es impar ($\neg q$), entonces $n$ también debe ser impar ($\neg p$).
			
			Hemos demostrado la contrarecíproca del teorema original, lo que implica que el teorema original también es verdadero.
		\end{fmd-example}
		
		
		\item Por \textit{contradicción} o \textit{reducción al absurdo}:
		\begin{enumerate}
			\item Se supone que la tesis ($q$) es falsa, con lo cual, se parte del caso $p = V$ y $q = F$;
			\item Se derivan otras afirmaciones utilizando reglas lógicas;
			\item Si se llega a una contradicción (por ejemplo $q \land \neg q = V$), se concluye que $q = V$.
		\end{enumerate}
		\begin{fmd-example}[Demostración por reducción al absurdo]
			\textit{Proposición}: Existen infinitos números primos.
			
			\textit{Demostración de Euclides}:
			
			Supongamos lo contrario, es decir, que existe un número finito de números primos. Llamémoslos $p_1, p_2, ..., p_n$. Consideremos el número:
			$$q = p_1p_2...p_n + 1$$
			
			Este número $q$ no es divisible por ninguno de los números primos $p_1, p_2, ..., p_n$, ya que al dividir $q$ entre cualquiera de estos números siempre obtendremos un resto de 1. 
			
			Si $q$ es primo, entonces hemos encontrado un nuevo número primo que no estaba en nuestra lista original, lo que contradice nuestra suposición inicial. Si $q$ no es primo, entonces debe tener un factor primo. Pero este factor primo no puede ser ninguno de los $p_1, p_2, ..., p_n$, ya que hemos demostrado que $q$ no es divisible por ninguno de ellos. Nuevamente, esto contradice nuestra suposición inicial.
			
			En ambos casos llegamos a una contradicción, lo que demuestra que nuestra suposición inicial era falsa. Por lo tanto, debe haber infinitos números primos.
		\end{fmd-example}
	\end{enumerate}
\end{enumerate}

\begin{lgnote}
También se tiene el \textit{método de inducción} del que se hablará en la sección \ref{sec:induccion}, pág. \pageref{sec:induccion}.
\end{lgnote}

\begin{fmd-example}[Irracionalidad de $\sqrt{3}$ por reducción al absurdo]
\textit{Proposición}: $\sqrt{3}$ es un número irracional.

\textit{Demostración}:

Supongamos lo contrario, es decir, que $\sqrt{3}$ es racional. Entonces, existen dos números enteros coprimos $a$ y $b$ tales que:
$$\sqrt{3} = \frac{a}{b}$$
Elevando al cuadrado ambos lados de la igualdad, obtenemos:
$$3 = \frac{a^2}{b^2}$$
Lo cual implica que $a^2 = 3b^2$.

De esta última igualdad, podemos concluir que $a^2$ es un múltiplo de 3. Si el cuadrado de un número es múltiplo de 3, entonces el número también debe ser múltiplo de 3 (ejemplo \ref{ex:directo}). Por lo tanto, podemos escribir $a = 3k$, donde $k$ es un entero. Sustituyendo en la ecuación anterior:
$$(3k)^2 = 3b^2$$
$$9k^2 = 3b^2$$
$$3k^2 = b^2$$

Ahora vemos que $b^2$ también es múltiplo de 3, lo que implica que $b$ también es múltiplo de 3.

Hemos llegado a una contradicción: asumimos al principio que $a$ y $b$ eran coprimos (es decir, no tenían factores comunes excepto 1), pero acabamos de demostrar que ambos son múltiplos de 3, lo que significa que tienen un factor común de 3. 

Esta contradicción demuestra que nuestra suposición inicial era falsa. Por lo tanto, $\sqrt{3}$ no puede ser racional y debe ser irracional.
\end{fmd-example}

\subsubsection{Negación de una implicación} \index{negación!implicación}
Las proposiciones $p \implies q$ y $\neg (p \land \neg q)$ son equivalentes, por implicación material y ley de De Morgan:
\[ p \implies q \equiv \neg p \lor q \equiv \neg \left( p \land \neg q \right) \]

En consecuencia, la negación de la primera equivale a la negación de la segunda, es decir:
\[ \neg (p \implies q) \iff \neg [ \neg (p \land \neg q)] \]

y por la propiedad de \textit{involución} se tiene:
\[ \boxed{\neg (p \implies q) \iff (p \land \neg q)} \]

La negación de una implicación no es una implicación, sino la conjunción del antecedente con la negación del consecuente.

\begin{fmd-example}[Negación de una implicación]
	Sean las implicaciones:
	
	\begin{enumerate}[label=\roman*)]
		\item Si hoy es lunes ($p$), entonces mañana es miércoles ($q$).
		
		Cuya negación es: ``Hoy es lunes ($p$) y mañana no es miércoles ($\neg q$)''
		
		\item ``Si estudias para el examen ($p$), entonces obtendrás una buena calificación ($q$)''
		
		La negación de esta afirmación sería: ``Estudias para el examen ($p$), pero no obtienes una buena calificación ($\neg q$)''
		
		\item ``Si un número es divisible por 4, entonces también es divisible por 2'' ($p \implies q$).
		
		La negación es: ``Hay un número que es divisible por 4, pero no es divisible por 2'' ($ p \land \neg q$).
	\end{enumerate}
\end{fmd-example}

\subsection{Argumento y razonamiento deductivo válido}

\subsubsection{Argumento} \index{argumento}
\begin{fmd-definition}[Argumento]
	Un \gls{argumento} es una secuencia de proposiciones (llamadas premisas) que pretende respaldar o justificar una proposición final (llamada conclusión). Un argumento se presenta usualmente de la siguiente forma:
	\[ p_1, p_2, \dots, p_n \implies q \]
	
	\begin{itemize}[itemsep=-3pt]
		\item $p_1$: premisa 1
		\item $p_2$: premisa 2
		\item $\qquad \vdots$
		\item $p_n$: premisa $n$
		\item Por lo tanto, $q$: conclusión.
	\end{itemize}
\end{fmd-definition}

\subsubsection{Razonamiento deductivo válido}
\vspace{1em} \index{razonamiento!válido}
\begin{fmd-definition}
	Un argumento se considera un \gls{razonamiento} si, y solo si, es imposible que todas sus premisas sean verdaderas y su conclusión sea falsa. En otras palabras, la verdad de las premisas garantiza la verdad de la conclusión. La validez de un argumento depende únicamente de su forma lógica, no del contenido específico de las proposiciones.
\end{fmd-definition}

El razonamiento deductivo válido es un proceso lógico en el que, a partir de premisas o afirmaciones, se llega a una conclusión de manera irrefutable. Un razonamiento es válido si la conclusión sigue necesariamente de las premisas, siguiendo las reglas de la lógica.

Por ejemplo:
\[ \frac{\begin{array}{l}
		p \implies  q\\
		p
\end{array}}{q} \]

Si tenemos las premisas $(p \Rightarrow q)$ y $(p)$, entonces la conclusión $(q)$ es válida. Esta es la ley del \textit{modus!ponens} de la que se hablará más en la pág. \pageref{def:ponens}.

\subsubsection{Relación entre argumento y razonamiento deductivo válido}

Un argumento puede ser válido o inválido. Un argumento válido es aquel en el que la conclusión se sigue necesariamente de las premisas, de esta manera, representa un razonamiento deductivo válido. Por otro lado, un argumento inválido es aquel en el que la conclusión no se sigue lógicamente de las premisas, incluso si las premisas son verdaderas.


\begin{fmd-example}[Argumento válido] \index{argumento!válido}
	\[ \frac{\begin{array}{l}
			p \implies  q\\
			p
	\end{array}}{q} \]
	\begin{itemize}
		\item \textit{Premisa 1}: Si llueve, entonces la calle está mojada (\( p \implies q \)).
		\item \textit{Premisa 2}: Está lloviendo ($p$)
		\item \textit{Conclusión}: Por lo tanto, la calle está mojada ($q$).
	\end{itemize}
	Este argumento es válido porque si las dos premisas son verdaderas, la conclusión también debe ser verdadera.
\end{fmd-example}

\begin{fmd-example}[Argumento inválido] \index{argumento!inválido}
	\[ \frac{\begin{array}{l}
			p \implies  q\\
			q
	\end{array}}{p} \]
	\begin{itemize}
		\item \textit{Premisa 1}: Si llueve, entonces la calle está mojada ($p \implies q$)
		\item \textit{Premisa 2}: La calle está mojada ($q$).
		\item \textit{Conclusión}: Por lo tanto, está lloviendo ($p$).
	\end{itemize}
	Este argumento es inválido porque la calle podría estar mojada por otras razones además de la lluvia. Aunque las premisas sean verdaderas, la conclusión no se sigue necesariamente de ellas.
\end{fmd-example}

El razonamiento deductivo válido es esencial en la lógica matemática y en muchas otras disciplinas, ya que nos permite construir argumentos sólidos y confiables. Al utilizar argumentos válidos, podemos estar seguros de que si nuestras premisas son verdaderas, nuestra conclusión también lo será.

\subsection{Reglas de Inferencia}
\index{reglas!inferencia}
Las \gls{reglasinferencia} son patrones lógicos fundamentales que nos permiten deducir nuevas proposiciones a partir de un conjunto de proposiciones dadas (premisas). Son herramientas esenciales en la construcción de argumentos válidos y en la demostración de teoremas en lógica matemática.

Son como las reglas de un juego. El juego se juega con proposiciones o fórmulas lógicas. Se empieza con conjuntos de fórmulas que se denominan \textit{premisas}. El objeto del juego es utilizar las reglas de inferencia de manera que conduzcan a otras fórmulas que se denominan \textit{conclusiones}. El paso lógico de las premisas a la conclusión es una \textit{deducción}.\vspace{3mm}

La conclusión que se obtiene se dice que es una \textit{consecuencia lógica} de las premisas si cada paso que se da para llegar a la conclusión está permitido por una regla.\vspace{3mm}

La idea de inferencia es: \textit{de premisas verdaderas se obtienen sólo conclusiones verdaderas}.

\textbf{Características}:

\begin{itemize}
	\item \textit{Validez}: Una regla de inferencia es válida si, siempre que las premisas sean verdaderas, la conclusión deducida también es verdadera.
	\item \textit{Forma lógica}: Las reglas de inferencia se basan en la forma lógica de las proposiciones, no en su contenido específico. Esto significa que se pueden aplicar a cualquier conjunto de proposiciones que tengan la misma estructura lógica, independientemente de su significado.
	\item \textit{Deducción}: Las reglas de inferencia nos permiten realizar deducciones, esto es, pasar de proposiciones conocidas a nuevas proposiciones que se siguen lógicamente de ellas.
\end{itemize}


\begin{enumerate}[label=\alph*)]
	\item Ley del \textbf{modus ponens}: \label{def:ponens} \index{modus!ponens}
	\[ p \implies q, \ p \ \vdash \ q \]
	Donde:
	\begin{itemize}
		\item $\vdash$ denota la relación de inferencia lógica (``se deduce'').
	\end{itemize}
	
	Se utiliza para derivar una conclusión a partir de una afirmación condicional y su antecedente.
	
	Notación clásica:
	\[
	\begin{array}{l}
		p  \implies q\\
		p \\ \hline
		q
	\end{array}
	\]
	
	\begin{fmd-example}[Modus ponens]
		Supongamos que tenemos la afirmación:
		
		$p \implies q$: ``Si sigues comiendo de esa manera, entonces no lograrás tu peso ideal''
		
		\begin{itemize}			
			\item Si se verifica $p$: ``sigues comiendo de esa manera''
			\item entonces $q$: ``no lograrás a tu peso ideal''
		\end{itemize}
	\end{fmd-example}
	
	\item Ley del \textbf{modus tollens}: \index{modus!tollens}
	\[ (p \implies q), \neg q \ \vdash \ \neg p \]
	
	Es otra regla de inferencia que se utiliza para derivar una conclusión negando el consecuente de una afirmación condicional.
	
	Notación clásica:
	\[
	\begin{array}{l}
		p  \implies q\\
		\neg q \\ \hline
		\neg p
	\end{array}
	\]
	
	\begin{fmd-example}[Modus tollens]
		Supongamos que tenemos la afirmación:
		
		\begin{center}
			$p \implies q$: ``Si llueve, entonces la calle estará mojada''
		\end{center}
		
		\begin{itemize}
			\item Si observamos $\neg q$: ``la calle no está mojada''
			\item entonces podemos concluir $\neg p$ ``no está lloviendo''
		\end{itemize}
	\end{fmd-example}
	
	\item Ley del \textbf{silogismo hipotético}: \index{silogismo!hipotético}
	\[ (p \implies q), (q \implies r) \vdash \left( p \implies r \right)  \]
	
	Nos permite combinar dos afirmaciones condicionales para obtener una nueva afirmación condicional
	
	Notación clásica:
	\[
	\begin{array}{l}
		p  \implies q\\
		q \implies r \\ \hline
		p \implies r
	\end{array}
	\]
	\begin{fmd-example}[Silogismo hipotético]
		Supongamos que tenemos las afirmaciones:
		\begin{itemize}
			\item $p \implies q$: Si no me despierto a hora, entonces no voy a ir a trabajar;
			\item $q \implies r$: Si no voy a trabajar, entonces no me pagan mi sueldo.
			\item Por lo tanto, $p \implies r$: si no me despierto a hora, entonces no me van a pagar mi sueldo.
		\end{itemize}
	\end{fmd-example}
	
	\item Ley del \textbf{silogismo disyuntivo} (modus tollendo ponens)
	
	\[ p \lor q, \neg p \vdash q \]
	o también:
	\[ p \lor q, \neg q \vdash p \]
	
	o
	\[
	\begin{array}{l}
		p  \lor q\\
		\neg p \\ \hline
		q
	\end{array} \qquad \qquad \begin{array}{l}
	p  \lor q\\
	\neg q \\ \hline
	p
	\end{array}
	\]
	
	Esta ley de inferencia establece que si tenemos una disyunción de dos proposiciones y sabemos que una de ellas es falsa, entonces podemos concluir que la otra proposición debe ser verdadera.
	
	La ley del silogismo disyuntivo se basa en la idea de que si tenemos dos opciones ($p$ o $q$) y descartamos una de ellas ($\neg p$ o $\neg q$), la otra opción debe ser la correcta.
	
	\begin{fmd-example}[Silogismo disyuntivo] \index{silogismo!disyuntivo}
		\begin{itemize}
			\item \textit{Premisa 1}: ``Hoy iré al cine o me quedaré en casa estudiando'' ($p \lor q$)
			\item \textit{Premisa 2}: ``No iré al cine hoy'' ($\neg p$)
			\item \textit{Conclusión}: ``Por lo tanto, me quedaré en casa estudiando'' ($q$)
		\end{itemize}
	\end{fmd-example}
	
	\item Ley de \textbf{adición} o adición disyuntiva: \index{adición!disyuntiva}
	\[ p \vdash p \lor q \]
	Notación clásica:
	\[
	\begin{array}{l}
		p\\ \hline
		\therefore p \lor q
	\end{array}
	\]
	Donde:
	\begin{itemize}
		\item $\therefore$ significa ``entonces''.
	\end{itemize}
	
	Si sabemos que una proposición $p$ es verdadera, entonces añadir otra proposición $q$ a través de una disyunción no cambia el hecho de que al menos una de las dos es verdadera (en este caso, $p$). Por lo tanto, la disyunción completa $p \lor q$ también será verdadera.
	
	\begin{fmd-example}[Adición]
		\begin{itemize}
			\item \textit{Premisa}: ``Está lloviendo'' ($p$)
			\item \textit{Conclusión}: ``Por lo tanto, está lloviendo o estoy comiendo pizza'' ($p \lor q$), donde $q$ es ``estoy comiendo pizza''
		\end{itemize}
	\end{fmd-example}
	
	\item Ley de \textbf{adjunción} o adición conjuntiva: \index{adjunción} \index{adición!conjuntiva}
	\[ p, q \ \vdash \ p \land q \]
	Notación clásica:
	\[
	\begin{array}{l}
		p\\
		q\\ \hline
		\therefore p \land q
	\end{array}
	\]
	Si sabemos que dos proposiciones $p$ y $q$ son ambas verdaderas, entonces su combinación mediante una conjunción (AND) también debe ser verdadera, ya que la conjunción solo es verdadera cuando todas sus componentes lo son.
	
	\begin{fmd-example}[Adjunción]
		\begin{itemize}
			\item \textit{Premisa 1}: ``El cielo es azul'' ($p$)
			\item \textit{Premisa 2}: ``La hierva es verde'' ($q$)
			\item \textit{Conclusión}: ``Por lo tanto, el cielo es azul y la hierva es verde'' ($p \land q$)
		\end{itemize}
	\end{fmd-example}
\end{enumerate}

\textbf{Ejercicios:}
\begin{fmd-example}
	\begin{enumerate}
		\item Justificar la validez del razonamiento:
		\[
		\begin{array}{lc}
			\mbox{\textit{Premisa} 1:} & p \implies q\\
			\mbox{\textit{Premisa} 2:} & \neg r \implies \neg q \\
			\mbox{\textit{Premisa} 3:} & \neg \left( \neg p \land \neg t \right) \\
			\mbox{\textit{Premisa} 4:} & t \implies s \\
			\mbox{\textit{Premisa} 5:} & \neg r \\ \hline
			\mbox{\textit{Conclusión}:} & s
		\end{array}
		\]
		\item Justificar la validez del razonamiento cuyas premisas son:
		\begin{itemize}
			\item \textit{Premisa} 1: Hoy llueve o hace frío,
			\item \textit{Premisa} 2: Hoy llueve o no hace frío,
			\item \textit{Conclusión}: Hoy llueve.
		\end{itemize}
		
		Consideremos $p$: ``hoy llueve''; $q$: ``hace frío'', entonces:
		\[ 
		\begin{array}{l}
			p \lor q\\
			p \lor \neg q \\ \hline
			p
		\end{array}
		\]
	\end{enumerate}
\end{fmd-example}

\subsection{Falacias formales}

\subsubsection{Falacias lógicas} \index{falacias!lógicas}
Una \gls{falacia} es un razonamiento incorrecto pero con apariencia de correcto. \index{falacias}
\[ \mbox{\textbf{Falacias}}
\begin{cases}
	\mbox{Formales}&\\
	\mbox{Informales} & \begin{cases}
		\mbox{Ambigüedad} & \begin{cases}
			\mbox{Por equívoco}&\\
			\mbox{Anfibología} & 
		\end{cases}\\
		\mbox{Materiales} & \begin{cases}
			\mbox{Datos insuficientes} & \\
			\mbox{Pertinencia}
		\end{cases}
	\end{cases}
\end{cases}	
\]
\vspace{-1mm}
\[ \mbox{Datos insuf.} \begin{cases}
	\mbox{Generalización}\\
	\mbox{inadecuada}\\
	\mbox{Falsa prueba}\\
	\mbox{Falsa causa}
\end{cases} \quad \mbox{Pertinencia} \begin{cases}
	\mbox{Ad hominen}\\
	\mbox{Ad baculum}\\
	\mbox{Ad populum}\\
	\mbox{Ad verecundiam}\\
	\mbox{Ad ignorantiam}\\
	\mbox{Tu quoque}\\
	\mbox{Hombre de paja}
\end{cases}\]

Algunos ejemplos:
\begin{itemize}
	\item \textbf{Anfibología}: ambigüedad estructural. \textit{``Todo hombre ama a una mujer, Romeo ama a Julieta, luego todo hombre ama a Julieta''}
	\item \textbf{Datos insuficientes}: \textit{``Todos los hombres son iguales''}.
	\item \textbf{Falsa causa y prueba}: \textit{``El fumar es malo para la salud. Me duele un pie. Eso es por el tabaco''}.
	\item \textbf{Ad hominem}: se ataca a la persona y no al argumento.
	\item \textbf{Ad baculum}: Falsas autoridad. \textit{``Porque lo digo yo!''}
	\item \textbf{Ad populum}: \textit{``Todo un pueblo no puede equivocarse''}
	\item \textbf{Ad verecundiam}: \textit{``La raíz de 2 es irracional, porque lo dijo Euclides''}
	\item \textbf{Ad ignorantiam}: \textit{``Nadie ha demostrado que hay vida en otros planetas, luego no la hay''}
	\item \textbf{Tu quoque}: \textit{``Tu también''} o \textit{``tu más''}.
	\item \textbf{Hombre de paja}, del espantapájaros o del monigote:
	\begin{itemize}
		\item A: \textit{``Creo que es malo que los adolescentes vayan solos de vacaciones''}
		\item B (refutación falaz): \textit{``Obligar a nuestros hijos a quedarse encerrados en casa es perjudicial para su desarrollo emocional''}
	\end{itemize}
\end{itemize}

\subsubsection{Sesgos cognitivos}

Los sesgos cognitivos son patrones de pensamiento sistemáticos que se desvían de la norma o la racionalidad, lo que puede llevar a juicios o decisiones inexactas.

\begin{enumerate}
	\item \textbf{Sesgo de Confirmación}
	
	Tendencia a buscar, interpretar, favorecer y recordar información que confirma nuestras creencias o hipótesis previas, mientras que se da menos importancia a la información que las contradice.
	
	\begin{example}
		Una persona que cree en la astrología tenderá a recordar las veces que las predicciones astrológicas se cumplieron, ignorando las veces que no lo hicieron.
	\end{example}
	
	\item \textbf{Sesgo de Anclaje}
	
	Tendencia a depender demasiado de la primera pieza de información que se encuentra (el ``ancla'') al tomar decisiones.
	
	\begin{example} En una negociación, el primer precio mencionado tiende a influir en las ofertas posteriores, incluso si ese precio no es relevante para el valor real del artículo.
	\end{example}
	
	\item \textbf{Efecto de primacía}:
	
	Este sesgo describe nuestra tendencia a recordar y dar mayor importancia a la información que recibimos primero.
	
	\begin{example}
		Dos estudiantes de primaria, Ana y Juan, tienen una discusión en el patio. Ana, molesta, corre inmediatamente a la oficina de la profesora para contarle su versión de los hechos, acusando a Juan de haberla empujado. Minutos después, Juan también llega a la oficina, visiblemente alterado, y le explica a la profesora que Ana fue quien inició la pelea insultándolo. A pesar de escuchar ambas versiones, la profesora tiende a creer más en la historia de Ana, ya que fue la primera en contársela. Cuando el segundo estudiante presenta su versión, la profesora ya tiene una opinión preconcebida.
	\end{example}
	
	\item \textbf{Heurística de Disponibilidad}
	
	Tendencia a juzgar la probabilidad de un evento en función de la facilidad con la que ejemplos o instancias vienen a la mente.
	
	\begin{example} Después de ver noticias sobre accidentes aéreos, las personas pueden sobreestimar la probabilidad de que ocurra un accidente aéreo, incluso si estadísticamente es muy poco probable.
	\end{example}
	
	\item \textbf{Efecto de Halo}
	
	Tendencia a que una impresión general sobre una persona, empresa o producto influya en los sentimientos y pensamientos sobre su carácter o propiedades.
	
	\begin{example}
		Si alguien es percibido como atractivo, también se le puede atribuir otras cualidades positivas, como ser inteligente o amable.
	\end{example}
	
	\item \textbf{Sesgo de Autoservicio}
	
	Tendencia a atribuir los éxitos a factores internos (habilidad, esfuerzo) y los fracasos a factores externos (mala suerte, circunstancias).
	
	\begin{example}
		Un estudiante que obtiene una buena calificación en un examen puede atribuirlo a su inteligencia, mientras que si obtiene una mala calificación, puede culpar al profesor o a la dificultad del examen.
	\end{example}
	
	\item \textbf{Falacia del Jugador}
	
	Creencia errónea de que los eventos pasados ​​pueden influir en eventos futuros en juegos de azar o situaciones similares.
	
	\begin{example}Un jugador que ha perdido varias veces seguidas en una máquina tragamonedas puede creer que tiene más probabilidades de ganar en la siguiente ronda, aunque cada ronda es independiente.
	\end{example}
	
	\item \textbf{Sesgo de retrospectiva} (o sesgo ``ya lo sabía''): Es la tendencia a creer, después de que un evento ha ocurrido, que se podría haber predicho o anticipado. Esto lleva a sobreestimar nuestra capacidad de predecir eventos pasados y puede resultar en una falsa sensación de confianza en nuestras habilidades predictivas.
	
	\begin{example}
		Después de un partido de fútbol, un aficionado podría decir "Sabía que ese equipo iba a ganar", aunque antes del partido no estuviera tan seguro.
	\end{example}
	
	\item \textbf{Sesgo de punto ciego}: Es la tendencia a no reconocer nuestros propios sesgos cognitivos, mientras que somos capaces de identificarlos fácilmente en los demás. Esencialmente, es un sesgo sobre nuestros sesgos.
	
	\begin{example}
		Una persona puede criticar a un amigo por ser demasiado confiado en sus habilidades, pero no reconocer su propia tendencia a sobreestimar sus capacidades.
	\end{example}
	
	\item \textbf{Aversión a la pérdida}: Es la tendencia a preferir evitar pérdidas a adquirir ganancias equivalentes. En otras palabras, el dolor de perder algo se siente más intensamente que la alegría de ganar algo del mismo valor.
	
	\begin{example}
		Un inversor puede ser reacio a vender una acción que ha perdido valor, incluso si es la decisión financieramente más sensata, porque la pérdida se siente más dolorosa que la ganancia potencial de invertir en otra cosa.
	\end{example}
\end{enumerate}

\begin{lgnote}
	Esta es solo una breve lista de algunos de los sesgos cognitivos más comunes. Existen muchos otros sesgos que pueden influir en nuestro pensamiento y toma de decisiones. Es importante ser conscientes de estos sesgos para poder tomar decisiones más informadas y racionales.
\end{lgnote}

\subsubsection{Falacias formales}
\vspace{1em} \index{falacias!formales}
\begin{fmd-definition}[Falacias formales]
Las \glspl{falaciaformal} son errores de razonamiento que ocurren debido a una estructura lógica incorrecta en un argumento, independientemente de la verdad o falsedad de las proposiciones involucradas. Por lo tanto, el problema radica en la forma del argumento, no en su contenido.
\end{fmd-definition}

A pesar de parecer convincentes a primera vista, las falacias formales violan las reglas de inferencia válidas, lo que las hace inválidas desde el punto de vista lógico. Identificar y evitar estas falacias es crucial para construir argumentos sólidos y confiables.

Veamos algunos ejemplos comunes de falacias formales:

\begin{enumerate}
	\item \textbf{Afirmación del consecuente} \index{falacia!afirm. consec}
	
	\begin{itemize}
		\item Estructura:
		\[ \begin{array}{l}
			p \implies q\\
			q \\ \hline
			p
		\end{array} \]
		\item Ejemplo:
		\begin{itemize}
			\item Si llueve, entonces la calle está mojada.
			\item La calle está mojada.
			\item Por lo tanto, está lloviendo. (Incorrecto, la calle podría estar mojada por otras razones)
		\end{itemize}
	\end{itemize}
	
	\item \textbf{Negación del antecedente} \index{falacia!neg. antec}
	\begin{itemize}
		\item Estructura:
		\[ \begin{array}{l}
			p \implies q\\
			\neg p \\ \hline
			\neg q
		\end{array} \]
		\item Ejemplo:
		\begin{itemize}[itemsep=-5pt]
			\item Si estudio mucho, aprobaré el examen
			\item No estudié mucho
			\item Por lo tanto, no aprobaré el examen (Incorrecto, podría aprobar por otras razones).
		\end{itemize}
	\end{itemize}
	
	\item \textbf{Falacia del término medio no distribuido} \index{falacia!término medio}
	\begin{itemize}
		\item Estructura:
		\begin{itemize}[itemsep=-5pt]
			\item Premisa 1: Todo $A$ es $B$
			\item Premisa 2: Todo $C$ es $B$
			\item Conclusión: Por lo tanto, todo $A$ es $C$
		\end{itemize}
		\item Ejemplo:
		\begin{itemize}[itemsep=-5pt]
			\item Todos los perros son mamíferos
			\item Todos los gatos son mamíferos
			\item Por lo tanto, todos los perros son gatos (Incorrecto)
		\end{itemize}
	\end{itemize}
	
	\item \textbf{Falacia de la división} \index{falacia!división}
	\begin{itemize}
		\item Estructura:
		\begin{itemize}[itemsep=-5pt]
			\item Premisa 1: El todo tiene la propiedad $x$
			\item Conclusión: Por lo tanto, cada parte del todo tiene la propiedad $x$
		\end{itemize}
		\item Ejemplo
		\begin{itemize}[itemsep=-5pt]
			\item El equipo de fútbol es el mejor del mundo
			\item Por lo tanto, cada jugador es el mejor en su posición (Incorrecto)
		\end{itemize}
	\end{itemize}
\end{enumerate}

Reconocer las falacias formales es esencial para evaluar críticamente los argumentos y evitar ser engañado por razonamientos que parecen válidos pero que en realidad son lógicamente defectuosos. Al estudiar y comprender estas falacias, podemos fortalecer nuestras habilidades de razonamiento y construir argumentos más sólidos y persuasivos.

\rule{\textwidth}{.5pt}

\subsection{Ejercicios}

\begin{enumerate}[label=\textbf{\arabic*}.]
	\item Escribe en símbolos:
	\begin{enumerate}[itemsep=-2pt, label=\alph*)]
		\item ``Me quedo en casa si y solo si llueve y hace frío, o si no llueve pero hace frío.''
		\item ``Si el aumento de la inflación implica la disminución de la balanza de pagos, entonces, si no disminuye la balanza de pagos no aumenta la inflación''.
		\item ``Si $x = 1$ e $y = 2$, entonces $z = 3$. Si, si $y = 2$, $z = 3$ entonces $w = 0$. $x = 1$. Por consiguiente, $w = 0$''.
	\end{enumerate}
	
	\item Formalice los argumentos, utilizando variables proposicionales, y determine si son válidos.
	\begin{enumerate}[label=\alph*)]
		\item \begin{itemize}[itemsep=-3pt]
			\item Si estudio mucho, entonces aprobaré el examen.
			\item Si apruebo el examen, entonces pasaré el curso.
			\item Conclusión Si estudio mucho, entonces pasaré el curso.
		\end{itemize}
		\item \begin{itemize}[itemsep=-3pt]
			\item Si el sistema es seguro, entonces los datos están encriptados.
			\item Los datos no están encriptados.
			\item Por lo tanto, el sistema no es seguro.
		\end{itemize}
	\end{enumerate}
	
	\item Exprese en lenguaje natural:
	\begin{enumerate}[label=\alph*)]
		\item \( (p \land \neg q) \lor (r \land (p \lor q))  \)
		
		Siendo:
		\begin{itemize}[itemsep=-3pt]
			\item $p$: ``El número es par''
			\item $q$: ``El número es divisible por 3''
			\item $r$: ``El número es mayor que 10''.
		\end{itemize}
		
		\item $\neg [(p \implies q) \iff (r \land \neg s)] \lor (\neg p \land t)$
		
		Siendo:
		\begin{itemize}[itemsep=-3pt]
			\item $p$: ``El proyecto se completa a tiempo''
			\item $q$: ``El cliente está satisfecho''
			\item $r$: ``Se cumplen todos los requisitos técnicos''
			\item $s$: ``Se excede el presupuesto''
			\item $t$: ``Se implementan nuevas tecnologías''
		\end{itemize}
	\end{enumerate}
	\item Construya las tablas de verdad para las siguientes proposiciones y determine si son tautologías, contradicciones o contingencias:
	\begin{enumerate}[label=\alph*)]
		\item $p \oplus q = \neg (p \land q) \land (p \lor q)$;
		\item $\left[(p \implies q) \land (q \implies r)\right] \implies (p \implies r)$
	\end{enumerate}
	
	\item Usando leyes lógicas, determine si las siguientes proposiciones son tautologías, contradicciones o contingencias:
	\begin{enumerate}
		\item \( (p \land q) \land (\neg p \lor \neg q) \)
		\item $\left[ (p \land q) \implies r \right] \iff \left[p \implies (q \implies r)\right]$
	\end{enumerate}
	
	\item Demuestre las siguientes equivalencias lógicas:
	\begin{enumerate}[label=\alph*)]
		\item $ \neg(p \lor (q \land r)) \equiv (\neg p \land \neg q) \lor (\neg p \land \neg r)$
		\item $(p \implies q) \land (r \implies s) \equiv (\neg q \lor s) \lor (\neg p \lor r)$
	\end{enumerate}
	
	\item Simplifique las siguientes expresiones:
	\begin{enumerate}[label=\alph*)]
		\item \( (p \land q) \lor (p \land \neg q) \lor (\neg p \land q) \lor (\neg p \land \neg q) \)
		\item $\neg [(p \lor q) \land (\neg p \lor r)] \lor (q \land r)$
	\end{enumerate}
	
	\item Utilizando el método de reducción al absurdo, demuestre que la siguiente proposición es una tautología:
	\[ \left[(p \implies q) \land p\right] \implies q \]
	
	\item Demuestre por contradicción que si $A \subset B$ y $B \cap C=\emptyset$, entonces $A \cap C= \emptyset$.
	
	\item Un ingeniero de software está diseñando un sistema de control de acceso que requiere que se cumplan dos condiciones para permitir el acceso:
	
	\begin{itemize}[itemsep=-2pt]
		\item La persona debe tener una tarjeta de acceso válida;
		\item La persona debe ingresar el código de acceso correcto.
	\end{itemize}
	Escriba una expresión lógica que represente la condición necesaria y suficiente para permitir el acceso al sistema.
	
	\item Un circuito lógico tiene tres entradas, $A$, $B$ y $C$, y una salida, $S$. La salida $S$ es verdadera si y solo si al menos dos de las entradas son verdaderas. Diseñe una expresión lógica que represente el comportamiento de este circuito.
	
	\item Considere la siguiente proposición: ``Si un número es divisible por 6, entonces es divisible por 2 y por 3''. Escriba la contrarrecíproca de esta proposición.
	
	\item Un ingeniero civil está diseñando un sistema de drenaje para una carretera. El sistema debe activarse si se cumplen las siguientes condiciones:
	\begin{itemize}[itemsep=-3pt]
		\item El nivel de agua en el sensor 1 supera un umbral crítico ($p$).
		\item El nivel de agua en el sensor 2 supera un umbral crítico ($q$).
		\item No hay un corte de energía en el sistema ($r$).
	\end{itemize}
	
	Escriba una expresión lógica que represente la condición necesaria y suficiente para que el sistema de drenaje se active.
	
	% Ingeniería Electromecánica
	\item Un sistema electromecánico tiene tres componentes principales: un motor ($m$), un sensor de temperatura ($t$) y un interruptor de emergencia ($e$). El sistema funcionará correctamente solo si se cumplen las siguientes condiciones:
	
	\begin{itemize}[itemsep=-3pt]
		\item El motor está encendido y el sensor de temperatura no detecta un sobrecalentamiento, o
		\item El interruptor de emergencia está apagado.
	\end{itemize}
	
	Escriba una expresión lógica que represente el funcionamiento correcto del sistema.
	
	% Ingeniería Eléctrica
	\item Un circuito lógico tiene cuatro entradas, $A$, $B$, $C$ y $D$, y una salida, $Y$. La salida $Y$ es verdadera si y solo si se cumplen las siguientes condiciones:
	
	\begin{itemize}[itemsep=-3pt]
		\item $A$ y $B$ son verdaderas, o
		\item $C$ es verdadera y $D$ es falsa.
	\end{itemize}
	
	Diseñe una expresión lógica que represente el comportamiento de este circuito y simplifique la expresión resultante tanto como sea posible.
	
	% Ingeniería Industrial
	\item Una empresa de logística tiene tres almacenes: $A$, $B$ y $C$. Un pedido puede ser enviado desde cualquiera de estos almacenes, pero se deben cumplir ciertas condiciones para seleccionar el almacén adecuado:
	
	\begin{itemize}[itemsep=-3pt]
		\item Si el pedido es urgente y el almacén $A$ tiene el producto en stock, entonces se envía desde $A$.
		\item Si el pedido no es urgente y el almacén $B$ tiene el producto en stock y el costo de envío desde $B$ es menor que desde $C$, entonces se envía desde $B$.
		\item En cualquier otro caso, se envía desde $C$.
	\end{itemize}
	Escriba una expresión lógica que represente la selección del almacén para un pedido dado.
	
	% Ingeniería Civil
	\item Un puente levadizo se abrirá solo si se cumplen todas las siguientes condiciones:
	
	\begin{itemize}[itemsep=-3pt]
		\item Hay un barco que necesita pasar ($b$).
		\item El puente no está en mantenimiento ($m$).
		\item Se ha emitido una señal de autorización desde el centro de control ($s$).
	\end{itemize}	
	Además, el puente se cerrará inmediatamente si se detecta un vehículo en el puente ($v$), independientemente de cualquier otra condición.
	
	Escriba expresiones lógicas que representen las condiciones para abrir y cerrar el puente.
	
	% Ingeniería Electromecánica
	\item Un sistema de control de temperatura en un edificio utiliza dos termostatos ($T_1$ y $T_2$) y un sistema de calefacción ($C$). El sistema de calefacción se encenderá si se cumple alguna de las siguientes condiciones:
	\begin{itemize}[itemsep=-3pt]
		\item La temperatura medida por $T_1$ es inferior a la temperatura deseada ($d_1$) y la temperatura medida por $T_2$ también es inferior a la temperatura deseada ($d_2$).
		\item La temperatura medida por $T_1$ es inferior a un umbral crítico ($c_1$), independientemente de la temperatura medida por $T_2$.
	\end{itemize}	
	Escriba una expresión lógica que represente el encendido del sistema de calefacción.
	
	% Ingeniería Eléctrica
	\item Un sistema de alarma contra incendios se activará si se detecta humo ($h$) o si se detecta un aumento rápido de temperatura ($t$). Sin embargo, la alarma no se activará si el sistema está en modo de prueba ($p$).
	
	Escriba una expresión lógica que represente la activación de la alarma contra incendios.
	
	% Ingeniería Industrial
	\item Una línea de producción tiene tres estaciones de trabajo: $E_1$, $E_2$ y $E_3$. Un producto debe pasar por las tres estaciones en orden. Sin embargo, si la estación $E_2$ está fuera de servicio ($f$), el producto puede pasar directamente de $E_1$ a $E_3$.
	
	Escriba una expresión lógica que represente el flujo de un producto a través de la línea de producción.
	
	% Desafío - Combinación de campos
	\item Un sistema de control de tráfico en una intersección utiliza sensores para detectar la presencia de vehículos en cada carril ($v_1$, $v_2$, $v_3$, $v_4$) y semáforos para controlar el flujo de tráfico ($s_1$, $s_2$). El sistema debe cumplir las siguientes condiciones:
	\begin{itemize}[itemsep=-3pt]
		\item Si hay un vehículo en el carril 1 y no hay vehículos en el carril 3, entonces el semáforo 1 debe estar en verde y el semáforo 2 en rojo.
		\item Si hay un vehículo en el carril 2 o en el carril 4, entonces el semáforo 2 debe estar en verde y el semáforo 1 en rojo.
		\item Si no hay vehículos en ningún carril, ambos semáforos deben estar en rojo.
	\end{itemize}
	
	Escriba expresiones lógicas que representen el control de los semáforos en esta intersección.
	
	% Desafío - Pensamiento crítico
	\item Un ingeniero está diseñando un sistema de seguridad para una bóveda bancaria. El sistema requiere que se cumplan tres condiciones para abrir la bóveda:
	
	\begin{itemize}[itemsep=-3pt]
		\item Se debe ingresar la contraseña correcta ($c$).
		\item Se debe escanear una huella dactilar autorizada ($h$).
		\item Se debe ingresar un código de seguridad de un solo uso generado por un token ($t$).
	\end{itemize}	
	Sin embargo, si se detecta un intento de intrusión ($i$), la bóveda se bloqueará inmediatamente, incluso si se cumplen las tres condiciones anteriores.
	
	Escriba una expresión lógica que represente la apertura de la bóveda.
\end{enumerate}

\rule{\textwidth}{.5pt}

\section{Lógica predicativa}
\index{lógica!predicativa}
La lógica predicativa, también conocida como lógica de primer orden o cálculo de predicados, es una extensión de la lógica proposicional que nos permite expresar relaciones y propiedades de objetos, así como cuantificar sobre ellos. Mientras que la lógica proposicional se limita a tratar proposiciones completas como unidades indivisibles, la lógica predicativa nos permite descomponer las proposiciones en sujetos y predicados, lo que nos da un mayor poder expresivo para formalizar el lenguaje natural y razonar sobre él.


\subsubsection{Conceptos clave}
\vspace{1em}
\begin{fmd-definition}[Función proposicional $P(x)$] \index{función!proposicional}
	Función proposicional en una variable o indeterminada $x$ es toda oración en la que figura $x$ como sujeto u objeto directo, la cual se convierte en proposición para cada especificación de $x$.
\end{fmd-definition}

\begin{fmd-example}
	\[ P(x, y): x \mid y \]
	
	\begin{itemize}
		\item $x \mid y$ se lee: $x$ es divisor de $y$
	\end{itemize}
	
	$P(x, y)$ no es proposición ya que no podemos afirmar la verdad o falsedad del enunciado. Mas para cada particularización de valores se tiene un proposición:
	\[ 
	P(-2, 6): -2 \mid 6 \quad (V)
	\]
	\[ P(12, 6): 12 \mid 6 \quad (F) \]
\end{fmd-example}

\begin{itemize}
	\item \textbf{Predicados}: Los predicados son funciones proposicionales que toman uno o más argumentos (objetos) y devuelven un valor de verdad (verdadero o falso). Representan propiedades o relaciones entre objetos. Por ejemplo, ``ser rojo'' o ``ser mayor que'' son predicados. \index{predicados}
	\item \textbf{Constantes}: Las constantes representan objetos específicos en el dominio del discurso. Por ejemplo, ``Juan'' o ``5'' son constantes.
	\item \textbf{Variables}: Las variables representan objetos genéricos o desconocidos en el dominio del discurso. Se utilizan para cuantificar sobre objetos. Por ejemplo, $x$ o $y$ son variables.
\end{itemize}

Los cuantificadores nos permiten expresar afirmaciones sobre la cantidad de objetos que cumplen una determinada propiedad o relación.

\subsection{Cuantificador existencial} \index{cuantificador!existencial} \index{existencial}
El \gls{cuantificadore} se utiliza para afirmar que existe al menos un elemento en un conjunto o dominio que cumple una determinada propiedad o relación.
\begin{itemize}
	\item Símbolo: \glsentrysymbol{cuantificadore} (se lee ``existe'' o ``hay'')
	\item Estructura: $\exists \, x \ P(x)$, donde:
	\begin{itemize}
		\item $\exists \, x$: Indica que existe al menos un elemento x en el dominio que satisface la afirmación.
		\item $P(x)$: Es un predicado que expresa una propiedad o relación sobre el elemento $x$.
	\end{itemize}
\end{itemize}

\begin{fmd-example}[Cuantificador existencial]
	``Algunos estudiantes aprobaron el examen'' se formalizaría como: \[\exists \, x \left( E(x) \land A(x) \right)\] donde $E(x)$ significa ``$x$ es un estudiante'' y $A(x)$ significa ``$x$ aprobó el examen''.
\end{fmd-example}

\subsection{Cuantificador universal}
\index{cuantificador!universal} \index{universal}
El \gls{cuantificadoru} se utiliza para afirmar que una propiedad o relación se cumple para todos los elementos de un conjunto o dominio específico.

\begin{itemize}
	\item Símbolo: \glsentrysymbol{cuantificadoru} (se lee ``para todo'' o ``para cada'')
	\item Estructura: $\forall x \ P(x)$, donde:
	\begin{itemize}
		\item $\forall x$: Indica que la afirmación se aplica a todos los elementos $x$ en el dominio.
		\item $P(x)$: Es un predicado que expresa una propiedad o relación sobre el elemento $x$.
	\end{itemize}
\end{itemize}

\begin{fmd-example}[Cuantificador universal]
	``Todos los perros ladran'' se formalizaría como: $\forall x \ \left(P(x) \implies L(x)\right)$ donde $P(x)$ significa ``$x$ es un perro'' y $L(x)$ significa ``$x$ ladra''.
\end{fmd-example}

\subsection{Negación de cuantificadores}
La negación de una proposición universal es una proposición existencial, y viceversa.
\index{negación!cuantificadores}
\begin{enumerate}[label=\roman*)]
	\item $\neg \left( \forall x \ P(x) \right) \equiv \exists \ x \ \neg P(x)$
	
	Esto es:
	\begin{itemize}
		\item Proposición original: $\forall x \ P(x)$, ``Para todo $x$, $P(x)$ es verdadero
		\item Negación: $\exists \, x \ \neg P(x)$, ``Existe al menos una $x$ para la cual $P(x)$ es falso''
	\end{itemize}
	
	\begin{fmd-example}[Negación del universal] \index{negación!universasl}
		\begin{itemize}
			\item Original: ``Todos los gatos son negros'', ($\forall x \ P(x)$, donde $x$ es un gato, $P(x)$ ``$x$ es negro'')
			\item Negación: ``Existe al menos un gato que no es negro'', ($\exists \ x \ \neg P(x)$).
		\end{itemize}
	\end{fmd-example}
	
	\item $\neg \left( \exists \, x \ P(x) \right) \equiv \forall x \ \neg P(x)$
	
	Esto es:
	\begin{itemize}
		\item Proposición original: $\exists \, x \ P(x)$, ``Existe al menos una $x$ para la cual $P(x)$ es verdadero
		\item Negación: $\forall x \ \neg P(x)$, ``Para toda $x$, $P(x)$ es falso'' 
	\end{itemize}
	
	\begin{fmd-example}[Negación del existencial] \index{negación!existencial}
		\begin{itemize}
		\item Original: ``Algunos estudiantes aprobaron el examen'' o ``Al menos un estudiante aprobó el examen'' ($\exists \, x \ P(x)$, donde $x$ es un estudiante, $P(x)$ ``$x$ aprobó el examen'')
		
		\item Negación: ``Ningún estudiante aprobó el examen'' o, dicho de otra manera, ``todos los estudiantes no aprobaron el examen'' ($\forall x \ \neg P(x)$).
		\end{itemize}
	\end{fmd-example}
\end{enumerate}

\rule{\textwidth}{.5pt}

\subsection{Ejercicios}

\begin{enumerate}[label=\textbf{\arabic*}.]
	
	% Ingeniería Civil
	\item Exprese la siguientes afirmaciones en lenguaje simbólico utilizando cuantificadores y predicados:
	\begin{enumerate}
		% Ingeniería Civil
		\item ``Todos los edificios construidos con hormigón armado son resistentes a los terremotos, pero algunos edificios no son resistentes a los terremotos''.
		
		% Ingeniería Electromecánica
		\item ``Existe al menos un motor eléctrico que es eficiente y silencioso, pero no todos los motores eléctricos son eficientes''.
		
		% Ingeniería Eléctrica
		\item Niegue la siguiente proposición: ``Para todo circuito eléctrico, si el circuito está cerrado y hay una fuente de voltaje, entonces fluirá corriente''.
		
		% Ingeniería Industrial
		\item ``Existe un proceso de producción que minimiza el costo y maximiza la calidad, pero no todos los procesos de producción maximizan la calidad''.
		
		% Ingeniería Genérica
		\item ``Para todo ingeniero, si el ingeniero tiene experiencia y está motivado, entonces será exitoso en su carrera''.
	\end{enumerate}
	
	\item Considere los siguientes casos. Utilice lógica predicativa para justificar su respuesta.
	
	\begin{enumerate}
		% Ingeniería Civil
		\item Un ingeniero civil está analizando la resistencia de diferentes materiales de construcción. Se sabe que:
		\begin{enumerate}[itemsep=-3pt, label=\roman*)]
			\item Todos los materiales compuestos son ligeros.
			\item Algunos materiales compuestos son resistentes a la corrosión.
			\item Todos los materiales metálicos son resistentes a la corrosión.
			\item Ningún material metálico es ligero.
		\end{enumerate}
		
		?`Se puede concluir que todos los materiales resistentes a la corrosión son ligeros?
		
		% Ingeniería Electromecánica
		\item Un ingeniero electromecánico está diseñando un sistema de control para un robot. El sistema debe cumplir las siguientes especificaciones:
		\begin{enumerate}[itemsep=-3pt, label=\roman*)]
			\item El robot debe detenerse si detecta un obstáculo o si recibe una señal de parada de emergencia.
			\item El robot debe moverse hacia adelante solo si no detecta un obstáculo y no recibe una señal de parada de emergencia.
		\end{enumerate}
		Formalice estas especificaciones.
		
		% Ingeniería Eléctrica
		\item Un ingeniero eléctrico está analizando un circuito digital complejo. Se sabe que:
		\begin{enumerate}[itemsep=-3pt, label=\roman*)]
			\item Si la señal A es alta y la señal B es baja, entonces la salida Y es alta.
			\item Si la señal C es alta o la señal D es baja, entonces la salida Y es baja.
		\end{enumerate}
		?`Es posible que la salida Y sea alta si la señal A es baja?
		
		% Ingeniería Industrial
		\item Una empresa de manufactura produce dos tipos de productos: A y B. Se sabe que:
		
		\begin{enumerate}[itemsep=-3pt, label=\roman*)]
			\item Todos los productos A son rentables.
			\item Algunos productos B son rentables.
			\item Todos los productos rentables tienen alta demanda.
		\end{enumerate}
		?`Se puede concluir que todos los productos B tienen alta demanda?
		
		% Ingeniería Genérica
		\item Un equipo de ingenieros está trabajando en un proyecto de investigación y desarrollo. Se sabe que:
		\begin{enumerate}[itemsep=-3pt, label=\roman*)]
			\item Si el proyecto es innovador y tiene potencial comercial, entonces recibirá financiación.
			\item El proyecto es innovador.
			\item El proyecto no ha recibido financiación.
		\end{enumerate}
		?`Qué se puede concluir sobre el potencial comercial del proyecto?
	\end{enumerate}
	
	\item Considere la siguiente afirmación:
	
	``Para todos los números reales $x$ e $y$, si $x$ es positivo e $y$ es negativo, entonces $xy$ es negativo.''
	
	Exprese esta afirmación utilizando cuantificadores y predicados, y luego demuestre su validez utilizando lógica simbólica.
	
	\item Sea $f: \mathbb{R} \rightarrow \mathbb{R}$ una función. Utilice lógica simbólica para demostrar la equivalencia de las siguientes afirmaciones:
	\begin{enumerate}[itemsep=-3pt]
		\item $f$ es inyectiva.
		\item Para todos los números reales $x$ e $y$, si $f(x) = f(y)$, entonces $x = y$.
	\end{enumerate}
\end{enumerate}

\rule{\textwidth}{.5pt}

\section{Sistema axiomático de Peano} \index{Peano}

\subsection{Axiomas de Peano}
\textbf{Sistema axiomático}:

\begin{itemize}
	\item \textit{Términos primitivos}: elementos, conjuntos o relaciones, cuya naturaleza no queda especificada de antemano.
	\item \textit{Axiomas}: propiedades que debe satisfacer los términos primitivos.
	\item \textit{Definiciones}: de los términos no primitivos.
	\item \textit{Teoremas}: propiedades que se deducen de los axiomas.
\end{itemize}

\textbf{Axiomas de Peano}\footnote{Giuseppe Peano, 1858-1932. Matemático, lógico y filósofo italiano.} \label{sec:peano} \index{axiomas!Peano}
\begin{itemize}
	\item[N1]: Existe una función inyectiva, la función ``siguiente'', $s: \mathbb{N} \rightarrow \mathbb{N} \setminus \{1\}$;
	\item[N2]: Existe un único elemento en $\mathbb{N}$, denotado por 1, tal que: $\forall n \in \mathbb{N}, 1 \ne s(n)$;
	\item[N3]: Dado cualquier subconjunto $X \subseteq \mathbb{N}$, si $1 \in X$ y $n \in X \implies s(n) \in X$, entonces $X = \mathbb{N}$.
\end{itemize}
Dado $n \in \mathbb{N}$, su imagen $s(n)$ se llama su \textbf{sucesor}.
\begin{itemize}
	\item N1 establece que todo número tiene un sucesor y que números naturales diferentes tienen sucesores diferentes (inyectiva)
	\item N2 establece que existe un único número natural que no tiene sucesor, el 1.
	\item N3 es una formulación del \textbf{principio de inducción matemática}. La condición $n \in X \implies s(n) \in X$ es equivalente a $s(X) \subseteq X$.
\end{itemize}

\subsection{Principio de inducción} \label{sec:induccion} \index{principio!inducción} \index{inducción}
La idea básica es: $P$ es un propiedad válida de 1 hasta $n \in \mathbb{N}$, se argumenta que su sucesor $s(n)$ también satisface la propiedad $P$, entonces $P$ es válido $\forall n \in \mathbb{N}$. En otras palabras:

\begin{itemize}
	\item $P(1)$ es V (base)
	\item $P(h)$ es V $\implies P(s(h))$ es V, entonces $P(n)$ es V para todo $n \in \mathbb{N}$.
\end{itemize}

El principio de inducción matemática es una técnica de demostración muy poderosa que nos permite establecer la veracidad de una afirmación para todos los números naturales. Es como si tuviéramos una escalera infinita y quisiéramos demostrar que podemos subir a cualquier peldaño. El principio de inducción nos dice que si podemos subir al primer peldaño (caso base) y si desde cualquier peldaño podemos subir al siguiente (paso inductivo), entonces podemos llegar a cualquier peldaño de la escalera.

\begin{fmd-example}[Suma de los primeros $n$ números naturales]
	La suma de los primeros n números naturales es igual a \( \dfrac{n(n+1)}{2} \)
	
	Demostración por inducción:
	
	\begin{itemize}
		\item \textit{Caso base} $(n=1)$: La suma del primer número natural es 1, y \( \dfrac{n(n+1)}{2} \) para $n = 1$ también es 1, por lo tanto $P(1)$ es verdadera.
		
		\item \textit{Paso inductivo}: Supongamos que $P(k)$ es verdadera:
		\[ 1 + 2 + \dots + k = \dfrac{k(k+1)}{2} \]
		Queremos demostrar que $P(k+1)$ también es verdadera.
		\[ \begin{split}
			1 + 2 + \dots + k + (k+1) &= \dfrac{k(k+1)}{2}  + (k+1) = (k+1) \left(\dfrac{k}{2} + 1\right) \\ &= \dfrac{(k+1)(k+2)}{2} = \dfrac{(k+1)\left[ (k+1) + 1 \right]}{2}
		\end{split} \]
		Lo cual es exactamente $P(k+1)$.
		
		Por lo tanto, por el principio de inducción, la proposición es verdadera para todo número natural $n$.
	\end{itemize}
\end{fmd-example}

\begin{fmd-example}[Desigualdad de Bernoulli]
	Para todo número real $x > -1$ y todo número natural $n \geq 0$, se cumple:
	\[ (1+x)^n \geq 1 + nx  \]
	Demostración por inducción:
	
	\begin{itemize}
		\item \textit{Caso base} $(n=0)$: $(1+x)^0 = 1 = 1 + 0x$. Por lo tanto, $P(0)$ es verdadera.
		
		\item \textit{Paso inductivo}: Supongamos que $P(k)$ es verdadera:
		\[ (1+x)^k \geq 1 + kx \]
		Queremos demostrar que $P(k+1)$ también es verdadera.
		\[ \begin{split}
			(1 + x)^{(k+1)} &= (1 + x)^k (1 + x) \\
			& \geq (1+kx)(1+x) \quad \text{(por la hipótesis inductiva)} \\
			& = 1 + (k+1)x + kx^2 \\
			& \geq 1 + (k+1)x
		\end{split} \]
		La última desigualdad se cumple porque $kx^2 \geq 0 \ \forall x$
		
		Por lo tanto, por el principio de inducción, la proposición es verdadera para todo número natural $n$.
	\end{itemize}
\end{fmd-example}



Basados en los axiomas de Peano, podemos definir:

\begin{fmd-definition}[Adición en $\N$]
	\begin{enumerate}[label=\alph*)]
		\item $a + 1 = s(a)$ cualquiera sea $a \in \mathbb{N}$
		\item $a + s(b) = s(a + b)$ cualesquiera que sean $a, b \in \mathbb{N}$
	\end{enumerate}
\end{fmd-definition}

\begin{fmd-definition}[Multiplicación en $\N$]
	\begin{enumerate}[label=\alph*)]
		\item $a \cdot 1 = a$, $\forall a \in \mathbb{N}$
		\item $a \cdot s(b) = a \cdot b + a$, cualesquiera sean $a, b \in \mathbb{N}$.
	\end{enumerate}
\end{fmd-definition}

Estas definen la ley de composición interna en $\mathbb{N}$ y se cumplen $\forall a, b, c \in \mathbb{N}$:
\begin{enumerate}
	\item \textit{Asociatividad}: $(a+b)+c = a + (b+c)$ y $(ab)c = a(bc)$
	\item \textit{Distributividad}: $a(b+c) = ab + ac$
	\item \textit{Conmutatividad}: $a + b = b + a$ y $ab = ba$
	\item \textit{Ley de corte o cancelación}: $a+b = a + c \implies b=c$ y $ab=ac \implies b=c$
\end{enumerate}

Se define la \textbf{potenciación} en $\mathbb{N}$, mediante: \index{potenciación}

\begin{fmd-definition}[Potenciación en $\N$]
	\begin{enumerate}[label=\alph*)]
		\item $a^1 = a$
		\item $a^{s(b)} = a^ba$
	\end{enumerate}
\end{fmd-definition}


\begin{fmd-proposition}
	La potenciación es distributiva respecto al producto: $(ab)^n = a^n b^n$
\end{fmd-proposition}

\begin{fmd-proof}
	 Por inducción
	\begin{enumerate}[label=\roman*)]
		\item $n=1 \implies (ab)^1 = ab = a^1 b^1$ (caso base)
		\item $(ab)^h = a^h b^h \implies (ab)^{s(h)} = a^{s(h)}b^{s(h)}$
		\[ (ab)^{s(h)} = (ab)^h (ab) = a^hb^hab = a^ha b^hb = a^{s(h)}b^{s(h)} \]
	\end{enumerate}
\end{fmd-proof}


\subsection{Relación de orden}
\vspace{3mm}
\begin{fmd-definition}[Relación de orden]
	La relación $\le$ es una \textit{relación de orden}, es decir, satisface $\forall a, b, c \in \mathbb{N}$:
	\begin{enumerate}
		\item Reflexividad: $a \le a $
		\item Antisimetría: $a \le b \land b \le a \implies a=b$
		\item Transitividad: $a \le b \land b \le c \implies a \le c$
	\end{enumerate}
\end{fmd-definition}

\begin{fmd-theorem}[Principio del buen orden] \index{principio!buen orden}
	Si $A \subseteq \mathbb{N}$ es un conjunto no vacío, entonces admite un elemento mínimo $a_0 \in A$, es decir, $\exists \, a_0 \in A / a_0 \le a, \forall a \in A$.
\end{fmd-theorem}

\begin{fmd-proof}
	Supongamos que se verifica el principio de inducción.
	
	Tomemos un subconjunto $A \subseteq \N$ no vacío y supongamos que no tiene un mínimo. Consideremos ahora el conjunto $S$ de todos los números menores que todos los elementos de $A$. Naturalmente $1 \ni A$, que es menor que cualquier otro número, luego $1 \in S$. Además, para cada $n \in S \implies n + 1 \in S$, de lo contrario $n+1$ sería el mínimo de $A$. Por tanto, por el principio de inducción $S = \N$ y $A$ es el conjunto vacío.
	
	Sin embargo, esto contradice nuestra hipótesis inicial, luego $A$ debe tener un elemento mínimo.
\end{fmd-proof}

\subsection{Teorema fundamental de la aritmética} \label{sec:arit}
\index{teorema!aritmética}
\vspace{1em}
\begin{fmd-theorem}[Teorema Fundamental de la Aritmética]\label{teo:aritmetica}
	Todo número entero mayor que 1 puede ser expresado de forma única (salvo por el orden de los factores) como un producto de números primos.
\end{fmd-theorem}
\begin{fmd-proof}
	
\begin{itemize}
	\item Hipótesis: Sea $n \in \mathbb{N}$ y $n > 1$.
	\item Tesis: Existen primos $p_1, p_2, \ldots, p_k$ (no necesariamente distintos) tales que:
	\[n = p_1 \cdot p_2 \cdot \ldots \cdot p_k\]
	Además, esta factorización es única salvo el orden de los factores.
\end{itemize}

\textit{Demostración}:

La demostración se realiza en dos partes: existencia y unicidad.

\begin{enumerate}
	\item \textbf{Existencia}:
	
	Procedemos por inducción sobre $n$.
	
	\begin{itemize}
		\item \textit{Base}: Para $n = 2$, la afirmación es trivial ya que 2 es primo.
		
		\item \textit{Paso inductivo}: Supongamos que la afirmación es cierta para todo número natural menor que $n$. Consideremos $n$.
		
		Si $n$ es primo, la factorización es trivial.
		
		Si $n$ no es primo, entonces existen $a, b \in \mathbb{N}$ tales que $1 < a, b < n$ y $n = a \cdot b$. Por hipótesis de inducción, $a$ y $b$ tienen factorizaciones en primos:
		
		\[a = p_1 \cdot p_2 \cdot \ldots \cdot p_r\]
		\[b = q_1 \cdot q_2 \cdot \ldots \cdot q_s\]
		
		Por lo tanto,
		\[n = a \cdot b = (p_1 \cdot p_2 \cdot \ldots \cdot p_r) \cdot (q_1 \cdot q_2 \cdot \ldots \cdot q_s)\]
		
		Lo que demuestra la existencia de la factorización para $n$.
	\end{itemize}
	
	\item \textbf{Unicidad}:
	
	Supongamos que $n$ tiene dos factorizaciones:
	\[n = p_1 \cdot p_2 \cdot \ldots \cdot p_r = q_1 \cdot q_2 \cdot \ldots \cdot q_s\]
	donde $p_i$ y $q_j$ son primos.
	
	Como $p_1$ divide al producto $q_1 \cdot q_2 \cdot \ldots \cdot q_s$, por el lema de Euclides, $p_1$ debe dividir a algún $q_j$. Sin pérdida de generalidad, supongamos que $p_1 | q_1$.
	
	Como $q_1$ es primo, debe ser que $p_1 = q_1$. Dividiendo ambos lados por $p_1$, obtenemos:
	
	\[p_2 \cdot \ldots \cdot p_r = q_2 \cdot \ldots \cdot q_s\]
	
	Repitiendo este proceso, concluimos que $r = s$ y que cada $p_i$ es igual a algún $q_j$.
	
	Por lo tanto, las dos factorizaciones son iguales salvo el orden de los factores.
\end{enumerate}
\end{fmd-proof}


\subsection{Conjuntos finitos e infinitos} \label{sec:finitos} \index{conjuntos!finitos} \index{conjuntos!infinitos}
\vspace{1em}
\begin{fmd-definition}[Conjuntos finitos e infinitos]
	Dado un natural $n \in \mathbb{N}$, escribiremos:
	
	\[ I_n \coloneqq \{ m \in \mathbb{N}; 1 \le m \le n\} \]
	Dado un conjunto $X$ diremos que es \textbf{finito}, si es vacío o existe $n \in \mathbb{N}$ y una biyección: $f: I_n \rightarrow X$.
	
	Un conjunto que no es finito es \textit{infinito}.
\end{fmd-definition}

\textbf{Notas}:
\begin{itemize}
	\item $\forall n \in \mathbb{N} $, el propio $I_n$ es finito, con $X = I_n$;
	\item Si $X = \emptyset$ decimos que tiene 0 elementos;
	\item Si $f: I_n \rightarrow X$ es una biyección, decimos que $X$ tiene $n$ elementos;
	\item Llamamos a $n$ la \textbf{cardinalidad} de $X$, se suele escribir: $|X| = n$.
	\item Si $g: X \rightarrow Y$ es una biyección, entonces $f:I_n \rightarrow X$ será una biyección si y sólo si, $g \circ f:I_n \rightarrow Y$ es una biyección. En otras palabras, $X$ será finito si $Y$ es finito y tendrán la misma cardinalidad $n$.
\end{itemize}

\subsubsection{Numerabilidad}
\index{numerabilidad}
No todos los conjuntos infinitos tienen la misma cardinalidad. Un conjunto es \textit{numerable} si y solo si es biyectable con $\N$. Un conjunto es \textit{contable} si y sólo si es finito o numerable.

\begin{fmd-theorem}[Numerabilidad de $\Z$]
	El conjunto de los enteros $\Z$ es numerable.
\end{fmd-theorem}

\begin{fmd-proof}
	$\Z$ es numerable si existe una función biyectiva $f: \N \rightarrow \Z$
	
	Observamos lo siguiente:
	
	\[ \begin{array}{lcccccccc}
		\N: & 1 & 2 & 3 & 4 & 5 & 6 & 7 & \cdots\\
		f & \downarrow & \downarrow & \downarrow & \downarrow & \downarrow & \downarrow & \downarrow & \cdots\\
		\Z: & 0 & -1 & 1 & -2 & 2 & -3 & 3 & \cdots
	\end{array} \]
	De donde:
	\[ f(n) = \begin{cases}
		\dfrac{n-1}{2} & \text{ si } n \text{ es impar}\\[3mm]
		- \dfrac{n}{2} & \text{ si } n \text{ es par}
	\end{cases} \]
	
	Probemos que esta función es biyectiva:
	
	\begin{itemize}
		\item Inyectividad
		\[ n_1, n_2 \in \N, \quad f(n_1) = f(n_2) \implies n_1 = n_2\]
		\begin{enumerate}[label=\arabic*)]
			\item $n_1$, $n_2$ impares:
			\[ f(n_1) = f(n_2) \rightarrow \dfrac{n_1 - 1}{2} = \dfrac{n_2 - 1}{2} \implies n_1 = n_2 \]
			\item $n_1$, $n_2$ pares:
			\[ f(n_1) = f(n_2) \rightarrow -\dfrac{n_1}{2} = -\dfrac{n_2}{2} \implies n_1 = n_2 \]
		\end{enumerate}
		$f(n)$ es inyectiva.
		
		\item Sobreyectividad
		
		Debemos probar que: \( \forall z \in \Z \ \exists \, n \in \N : f(n) = z \)
		\begin{enumerate}[label=\arabic*)]
			\item $z = 0$:
			\[ z = f(n) = \dfrac{n - 1}{2} = 0 \implies n = 1 \]
			\item $z > 0$:
			\[ z = f(n) = \dfrac{n - 1}{2} \implies n = 2z + 1 \]
			\item $z < 0$:
			\[ z = f(n) = -\dfrac{n}{2} \implies n = -2z \]
		\end{enumerate}
		$f(n)$ es sobreyectiva.
	\end{itemize}
	
	$f(n)$ es biyectiva, por tanto, por definición de numerabilidad, $\Z$ es numerable.
\end{fmd-proof}

\begin{theorem}[Numerabilidad de $\Q$]
	El conjunto de los números racionales $\Q$ es numerable.
\end{theorem}

\begin{theorem}[$\R$ no es numerable]
	$\R$, el conjunto de los números reales no es numerable.
\end{theorem}

\begin{fmd-theorem}[Teorema de Cantor] \index{Cantor} \index{teorema!Cantor}
	El conjunto potencia $\mathcal{P}(A)$ de cualquier conjunto $A$ tiene una cardinalidad estrictamente mayor que la cardinalidad de $A$:
	\[ |\mathcal{P}(A)| >  |A| \]
	
	\begin{proof}
		Procedemos por contradicción como lo hizo Cantor\footnote{Georg Cantor, 1845-1918, matemático ruso.}. Supongamos que existe una función $f: A \rightarrow \mathcal{P}(A)$ que es sobreyectiva, es decir, que para todo elemento $B$ en $\mathcal{P}(A)$, existe un elemento $a \in A$ tal que $f(a) = B$.
		
		Considere el siguiente conjunto: $ C = \{ a \in A \mid a \not \in f(a) \} $
		
		$C$ es el conjunto de todos los elementos $a \in A$ tales que $a$ no pertenece al conjunto $f(a)$.
		Como $f(a)$ es un elemento de $\mathcal{P}(A)$, $f(a) \subset A$ de donde $C$ es un subconjunto de $A$, por lo tanto, $C \in \mathcal{P}(A)$.
		
		Como $f$ es sobreyectiva, debe existir un elemento $c \in A$ tal que $f(c) = C$.
		
		Ahora analicemos los casos:
		
		\begin{itemize}
			\item Caso 1:
			
			 Si $c \in C$, entonces $c \not \in f(c)$ por definición de $C$, pero $f(c) = C$, por lo que $c \not \in C$ que es una contradicción.
			\item Caso 2:
			
			Si $c \not \in C$, entonces $c \in f(c)$, por definición de $C$, pero $f(c) = C$, por lo que $c \in C$, que es también una contradicción.
		\end{itemize}
		En ambos casos llegamos a una contradicción. Por lo tanto, nuestra suposición inicial de que existe una función sobreyectiva $f: A \rightarrow \mathcal{P}(A)$ debe ser falsa, lo que implica que la cardinalidad de $\mathcal{P}(A)$ es estrictamente mayor que la cardinalidad de $A$.
	\end{proof}
\end{fmd-theorem}

\rule{\textwidth}{.5pt}

\subsection{Ejercicios}

\begin{enumerate}
% Axiomas de Peano y Principio de Inducción
\item Demuestre por inducción:
\begin{enumerate}
	\item Para todo número natural $n$, la suma de los primeros $n$ números impares es igual a $n^2$.
	\item Para todo número natural $n \geq 1$, \( 1^3 + 2^3 + \dots + n^3 = \left[ \dfrac{n(n+1)}{2} \right]^2 \)
	\item Para la sucesión definida por recurrencia: $a_1 = 2$, $a_{n+1} = 3a_n - 1$; $a_n = \frac{3^n + 1}{2}$ para todo $n \in \mathbb{N}$.
\end{enumerate}

\item Demuestre que todo número natural $n > 1$ tiene un divisor primo. (Sugerencia: use el principio del buen orden)

% Teorema Fundamental de la Aritmética
\item Demuestre que $\sqrt{6}$ es irracional utilizando el Teorema Fundamental de la Aritmética.

\item Dé un ejemplo de dos conjuntos infinitos $A$ y $B$ tales que $A \cap B$ sea finito y no vacío.

% Conjuntos Finitos e Infinitos, Numerabilidad
\item Demuestre que el conjunto de todas las sucesiones finitas de 0's y 1's es numerable.

\item Sea $A$ un conjunto contable. Demuestre que el conjunto de todas las funciones de $A$ en $\{0, 1\}$ es no numerable.

% Aplicaciones a la Ingeniería
\item Un circuito digital tiene $n$ entradas. ?`Cuántas funciones booleanas diferentes se pueden construir con estas $n$ entradas?

\item Un ingeniero civil está diseñando un puente colgante con $n$ cables principales numerados de 1 a $n$. La tensión en cada cable se determina según las siguientes condiciones:

\begin{enumerate}[itemsep=-3pt, label=\roman*)]
	\item La tensión en el primer cable ($T_1$) es de 1000 kN.
	\item Para cada cable subsiguiente, la tensión aumenta en un 10\% respecto al cable anterior y se le suma una carga constante de 50 kN.
\end{enumerate}

\begin{enumerate}
	\item Formule una relación de recurrencia que describa la tensión en el k-ésimo cable.
	\item Utilizando el principio de inducción matemática, demuestre que la fórmula general para la tensión en el k-ésimo cable es:
	
	\[ T_k = 1000 \cdot 1.1^{k-1} + 500 \cdot (1.1^{k-1} - 1), \quad \text{para } k \geq 1 \]
	
	\item Calcule la tensión en el décimo cable ($T_{10}$) utilizando la fórmula demostrada.
\end{enumerate}

% Conexiones con otras áreas de la matemática / Problemas históricos
\item Demuestre que hay infinitos números primos de la forma $4k + 3$, donde $k$ es un entero no negativo. (Sugerencia: Adapte la demostración de Euclides de la infinitud de los primos)

\end{enumerate}
	\chapter{Estructuras algebraicas}

Una estructura algebraica es una par $(G, *)$, en donde $G$ es un conjunto no vacío y $*$ es una operación aplicable a los elementos dicho conjunto. Podrían haber más operaciones aplicables, por ejemplo, sin son aplicables las operaciones $*$ y $\bdot$, la estructura algebraica sería $(G, *, \bdot)$.

En general, una estructura algebraica es una $n-tupla: (a_1, a_2, \dots, a_n)$, donde $a_1$ es un conjunto no vacío dado, y $\{ a_2, \dots, a_n \}$ es un conjunto de operaciones aplicables a los elementos del conjunto $a_1$.

Las estructuras algebraicas nos permiten estudiar y clasificar objetos matemáticos en función de sus propiedades algebraicas. Algunos ejemplos de estructuras algebraicas incluyen grupos, anillos y campos.

\section{Grupos}

\subsection{Axiomas de grupo} \label{sec:axiomas-grupo}
\vspace{2mm}
\begin{fmd-definition}[Grupo]
	Dado un conjunto $G$, no vacío, una operación binaria o función $*: G \times G \rightarrow G$, define en $G$ una \textit{estructura de grupo} $(G, *)$ si se cumplen las siguientes propiedades:
	\begin{enumerate}
		\item[\textbf{G1}:] \textbf{Asociatividad}: $\forall x, y, z \in G \implies (x*y)*z = x * (y*z)$
		\item[\textbf{G2}:] \textbf{Existencia del neutro}: $\exists \, e \in G \mid \forall x \in G \implies x*e = e*x = x$
		\item[\textbf{G3}:] \textbf{Existencia de inversos}: $\forall x \in G, \exists x' \in G \mid x * x' = x'*x = e$
	\end{enumerate}
\end{fmd-definition}

Por lo tanto, un grupo es una estructura algebraica que consta de:
\begin{enumerate}[label=\alph*)]
	\item Un conjunto no vacío;
	\item Una operación binaria (una función con dos argumentos);
	\item Tres propiedades.
\end{enumerate}

\begin{fmd-example}
	Conjunto: $\Z$; Operación binaria: suma $(+)$; Propiedades:
	\begin{enumerate}
		\item Asociatividad: $(3 + 4) + 2 = 3 + (4 + 2)$
		\item Neutro: el cero, $0 + 3 = 3 + 0 = 3$
		\item Inversos: $4 + (-4) = (-4) + 4 = 0$
	\end{enumerate}
	En general: $\forall x, y, z \in \Z$
	\begin{enumerate}
		\item Asociatividad: $(x + y) + z = x + (y + z)$
		\item Neutro: $\exists \, 0 \in \Z / 0 + x = x + 0 = x$
		\item Inversos: $\forall x \in \Z , \exists \, (-x) \in \Z / x + (-x) = (-x) + x = 0$
	\end{enumerate}
	$(\Z, +)$ es un grupo.
\end{fmd-example}


\begin{fmd-definition}[Grupo abeliano]
	En caso de que $(G, *)$ cumpla además la propiedad conmutativa,
	
	\begin{enumerate}
		\item[\textbf{G4}:] \textbf{Conmutatividad}: $\forall x, y \in G \implies x*y = y*x$
	\end{enumerate}
	se llama grupo \textit{abeliano}\footnote{En homenaje al matemático noruego Niels Henrik Abel (1802-1829)} o \textit{conmutativo}.
\end{fmd-definition}


\textbf{Notas}:
\begin{enumerate}[label=\roman*)]
	\item Se dice que la operación $(*)$ es \textbf{cerrada} en el conjunto $G$, si $*: G \times G \rightarrow G$, en este caso, se dice que $(G, *)$ tiene una regla o \textbf{ley de composición interna}.
	
	\item Cuando $(G, *)$ es abeliana, es común llamar \textit{suma} o \textit{adición} a $*$, se dice que $*$ es aditiva, suele usarse el signo $+$, su inverso es el \textit{opuesto} y se indica $a' = -a$. Es común llamar \textit{cero} al neutro y denotarlo por $0$;
	\item Cuando $(G, *)$ es \textit{no abeliana} (o no lo sabemos), decimos que $*$ es multiplicativa, se usa ``$\bdot$'', su inverso se dice \textit{recíproco} y se utiliza $a' = a^{-1}$. Es usual llamar \textit{unidad} al neutro y denotarlo $1$.
	\item Cuando la cardinalidad\footnote{Ver sección \ref{sec:finitos}} $|G|$ es finita, decimos que el grupo es \textit{finito}.
	\item Dado un grupo $(G, *)$, $a\in G$ y $m \in \mathbb{Z}$, por convención $a^m = a*a* \cdots *a$, con $m$ factores si $m>0$, y $a^m = (a^{|m|})^{-1} =$ $\left(a*a* \cdots *a\right)' = a'*a'* \cdots *a'$ con $|m|$ factores $a'$ si $m<0$.
\end{enumerate}


\begin{fmd-definition}[Monoide]
	El par $(M, *)$, donde $M \ne \emptyset$, y $*$ es una función, es un monoide\footnote{Corresponde a la definición dada en \cite{rojoAlgebra8vaEd}.} si y solo si $*$ es una \textit{ley de composición interna} en $M$.
\end{fmd-definition}

\begin{definition}[Idempotencia] \label{def:idempotencia}
	Si $(G, *)$ es un monoide, es decir, una estructura algebraica con una ley de composición interna, se dice que $g \in G$ es idempotente si $g * g = g$
	
	La idempotencia es la propiedad de realizar una operación determinada varias veces y aún así obtener siempre el mismo resultado que se obtendría si se realizase una sola vez.
\end{definition}

\begin{fmd-definition}[Semigrupo]
	El par $(A, *)$, donde $A \ne \emptyset$, y $*$ es una función, es un semigrupo si y solo si $*$ es una ley interna y asociativa en $A$.
\end{fmd-definition}

\begin{proposition}[Unicidad del neutro]
	En cualquier grupo el neutro es único.
\end{proposition}
\begin{proof}
	Sae $(G, *)$ un grupo cualquiera. Supongamos $e, e' \in G$ dos elementos que cumplen la propiedad del neutro:
	\[ \begin{cases}
		\forall x \in G:& \quad e*x = x*e = x\\
		\forall x \in G:& \quad e'*x = x*e' = x
	\end{cases} \]
	Entonces:
	\[ \begin{cases}
		x = e':& \quad e*e' = e'*e = e'\\
		x = e:& \quad e'*e = e*e' = e
	\end{cases} \]
	por tanto:
	\[ e = e*e' = e' \]
\end{proof}


\begin{fmd-proposition}[Ley de corte o de cancelación]
	Sea $(G, *)$ un grupo. $\forall a, b, c \in G$ se cumple:
	
	\begin{enumerate}[label=\alph*)]
		\item Ley de corte por la izquierda: $a * b = a * c \implies b = c$
		\item Ley de corte por la derecha: $b*a = c*a \implies b = c$
	\end{enumerate}
	\begin{proof}
		Para demostrar a) basta con premultiplicar por el inverso $a'$ y en b) posmultiplicar.
	\end{proof}
\end{fmd-proposition}

Se dice que los elementos son \textbf{regulares} si se cumple la ley de corte.

\begin{proposition}[Unicidad del inverso]
	Para cualquier elemento $x \in G$ existe un inverso y es único.
	
	\begin{proof}
		Sea $(G, *)$ un grupo cualquiera. Supongamos $x', x'' \in G$ dos elementos que cumplen la propiedad del inverso:
		\[ \begin{cases}
			\forall x \in G:& \quad x*x' = x'*x = e\\
			\forall x \in G:& \quad x*x'' = x''*x = e
		\end{cases} \]
		Entonces:
		\[ x * x' = x * x'' \]
		
		por ley de cancelación: \[ x' = x'' \]
	\end{proof}
\end{proposition}

\begin{fmd-definition}[Orden de un grupo]
	Sea $(G, *)$ grupo. Llamamos orden de $G$ al número de elementos de $G$ (si $G$ es finito). Lo denotamos por $o(G) \coloneqq |G| = \mbox{card} (G) = \# (G)$
\end{fmd-definition}

Decimos que $G$ es de \textbf{orden par} u \textbf{orden impar} si $o(G)$ es par o impar respectivamente.

El grupo de orden 1, llamado \textbf{grupo trivial}, es el que consta de un solo elemento, el cual debe ser el neutro para cumplir con las propiedades de grupo. Es común utilizar tablas para representar las operaciones, denominadas \textbf{tablas de Cayley}\footnote{En homenaje al matemático británico Arthur Cayley (1821-1895)}.

Los conjuntos correspondientes a los grupos de orden 1, 2 y 3 son $\{ e \}$, $\{ e, a \}$, $\{e, a, b\}$. Sus respectivas tablas de Cayley se muestran en la tabla \ref{tab:cayley}:

\begin{table}[H]
	\centering
	\[ \begin{array}{c|c}
		* & e \\ \hline
		e & e
	\end{array} \qquad \quad \begin{array}{c|cc}
		* & e & a \\ \hline
		e & e & a \\
		a & a & e
	\end{array} \qquad \quad \begin{array}{c|ccc}
		* & e & a & b \\ \hline
		e & e & a & b \\
		a & a & b & e \\
		b & b & e & a
	\end{array}\]
	\caption{Tablas de Cayley para grupos de orden 1, 2 y 3.}
	\label{tab:cayley}
\end{table}
\begin{itemize}
	\item Notar que cada fila y columna contiene todos los elementos del grupo. No existen duplicados en ninguna fila ni columna.
	
	\item Estos grupos son abelianos ya que las tablas son simétricas respecto de la diagonal principal.
	\item Notar la similitud con las operaciones entre números enteros, si convertimos $* \rightarrow +, e \rightarrow 0, a \rightarrow 1, b \rightarrow 2$, que se muestran en la tabla \ref{tab:cayley2}, en este caso se dice que el grupo $(G, *)$ es \textbf{isomorfo} con su correspondiente $(\Z_n, +)$. Más adelante se verán estos conceptos con más detalle.
\end{itemize}

\begin{table}[H]
	\centering
	\[ \begin{array}{c|c}
		+ & 0 \\ \hline
		0 & 0
	\end{array} \qquad \quad \begin{array}{c|cc}
		+ & 0 & 1 \\ \hline
		0 & 0 & 1 \\
		1 & 1 & 0
	\end{array} \qquad \quad \begin{array}{c|ccc}
		+ & 0 & 1 & 2 \\ \hline
		0 & 0 & 1 & 2 \\
		1 & 1 & 2 & 0 \\
		2 & 2 & 0 & 1
	\end{array}\]
	\caption{Tablas de Cayley para enteros módulo 1, 2 y 3.}
	\label{tab:cayley2}
\end{table}

\textbf{Tarea}: Realizar las tablas de Cayley para grupos de orden 4.

\begin{fmd-example}[Grupo de Klein]
	El grupo de Klein\footnote{En honor al matemático alemán Felix Klein (1849-1925).} es abeliano con 4 elementos, en el cual cada elemento es su propio inverso (\textit{propiedad involutiva}), y componiendo cualesquiera dos de ellos, a excepción del neutro, produce el tercero, es decir:
	\[ V = \{ e, a, b, c \}; \quad a^2 = b^2 = (ab)^2 = c^2 = e \]
	\vspace{-15mm}
	\begin{table}[H]
		\centering
		\[ \begin{array}{c|cccc}
			\ast & e & a & b & c \\ \hline
			e & e & a & b & c \\
			a & a & e & c & b \\
			b & b & c & e & a\\
			c & c & b & a & e
		\end{array}\]
		\caption{Tabla de Cayley para el grupo de Klein.}
		\label{tab:Klein}
	\end{table}
\end{fmd-example}

\subsection{Grupos de números}

\textbf{Sistemas numéricos}.

Veamos el concepto de grupo para los \textit{sistemas numéricos} $\N, \Z, \Q, \R, \C$ y las operaciones usuales de suma $(+)$ y multiplicación $(\bdot)$.

En la fig. \ref{fig:sisnum} se representan gráficamente. Se observa que: $\N \subset \Z \subset \Q \subset \R \subset \C$.

\begin{figure}[h]
	\centering
	\includestandalone[scale=1.2]{resources/ltx/rectaN}
	\caption{Representación gráfica de los sistemas numéricos.}
	\label{fig:sisnum}
\end{figure}

\begin{table}[h]
	\centering
	\begin{tabular}{|l|c|c|c|c|c|}
		\hline
		& $(\N, +)$ & $(\Z, +)$ & $(\Q, +)$ & $(\R, +)$ & $(\C, +)$ \\
		& $(\N, \cdot)$ & $(\Z, \cdot)$ & $(\Q^{*}, \cdot)$ & $(\R^{*}, \cdot)$ & $(\C^{*}, \cdot)$ \\ \hline
		Asociativa & \cmark & \cmark & \cmark  & \cmark &  \cmark \\
		& \cmark & \cmark & \cmark  & \cmark  &  \cmark \\ \hline
		Neutro & 0 & 0 & 0  & 0 & 0 \\
		& 1 & 1 & 1  & 1  &  1 \\ \hline
		Inversos & \xmark & \cmark & \cmark  & \cmark & \cmark \\
		& \xmark & \xmark & \cmark  & \cmark & \cmark \\ \hline
	\end{tabular}
	\caption{Comprobación de las propiedades de Grupo para sistemas numéricos. Los conjuntos con $*$ indican que se excluye el cero. Los 7 grupos que comprueban las tres propiedades, además, son \textit{abelianos}.}
\end{table}
Notar que, a partir de $(\Z, \bdot)$ se deduce que restringiendo al par $\left(\{ -1, 1 \}, \bdot \right)$, este cumple las propiedades de grupo.


\subsection{Grupos de matrices} \label{sec:matrices}

Consideremos $\mathbf{M}_{m\times n}(\R)$, el conjunto de todas las matrices $m \times n$ con $m, n \in \N$ y elementos reales, el conjunto también puede ser representado por $\R^{m \times n}$.

\begin{equation}
	\mathbf{M}_{m \times n}(\R) = \left\{ \begin{bmatrix}
		a_{11} & \cdots & a_{1n}\\
		\vdots & \ddots & \vdots\\
		a_{m1} & \cdots & a_{mn}
	\end{bmatrix} | \, a_{ij} \in \R \right\}
	\label{eq:matrices}
\end{equation}

\subsubsection{Suma de matrices}

Dadas las matrices $A = (a_{ij})$ y $B = (b_{ij})$, se define la suma: $S = A + B = (s_{ij})$ de modo que $s_{ij} = a_{ij} + b_{ij}$, es decir, la matriz $S$ tiene como elementos la suma de los elementos correspondientes de $A$ y $B$.

Ejemplo:
\[ \begin{bmatrix}
	1 & 2\\
	3 & 4
\end{bmatrix} + \begin{bmatrix}
	5 & 6\\
	7 & 8
\end{bmatrix} = \begin{bmatrix}
	6 & 8\\
	10 & 12
\end{bmatrix} \]

Notar que la suma elemento a elemento ocurre en $\R$, conforme a la definición dada en \eqref{eq:matrices}. Por lo tanto, probar que $(\mathbf{M}_{m\times n}(\R), +)$ es un grupo se reduce a probar que $(\R, +)$ es un grupo, lo cual ya fue analizado junto con los demás sistemas numéricos, y se vió que $(\R, +)$ es un grupo abeliano, con lo cual,

las matrices cuadradas con la operación suma, cumplen:
\begin{itemize}
	\item[G1] Asociatividad: $(A + B) + C = A + (B + C)$ \quad \cmark 
	\item[G2] Neutro: $e_{2\times 3} = \begin{bmatrix} 0 & 0 & 0\\0 & 0 & 0 \end{bmatrix}$;
	$e_{m \times n} = \begin{bmatrix} 0 & \cdots & 0\\ \vdots & \ddots & \vdots\\ 0 & \cdots & 0 \end{bmatrix}$; $a_{ij} = 0$ \quad \cmark 
	\item[G3] Inversos (opuestos):  $(a_{ij})' = (-a_{ij})$ \quad \cmark 
	\item[G4] Conmutatividad: $A + B = B + A$ \quad \cmark 
\end{itemize}
por tanto:

\begin{center}
	$\left( \mathbf{M}_{m \times n}(\R), + \right)$ es un grupo abeliano.
\end{center}

El producto indicado $m \times n$ se conoce como \textbf{orden de la matriz}, así se dice por ejemplo de $A_{2 \times 3}$ es una matriz de orden $2 \times 3$.

\subsubsection{Grupos de vectores}

Un caso especial de matrices son los vectores. Consideremos, el conjunto de vectores en $\R^n$:
\begin{equation}
	\mathbf{M}_{1 \times n}(\R) = \left\{ \begin{bmatrix}
		a_1 \\ \vdots \\ a_n \end{bmatrix} | \, a_{i} \in \R \right\}
	\label{eq:vectores} = \R^n = \underbrace{\R \times \cdots \times \R}_{n \; veces}
\end{equation}

con la operación usual de suma.

Vectores con la operación suma. $\forall u, v, w \in \R^n$ se cumple:
\begin{itemize}
	\item[G1] Asociatividad: $(u + v) + w = u + (v + w)$ \quad \cmark 
	\item[G1] Neutro: $e_2 = \begin{bmatrix} 0 \\ 0 \end{bmatrix}$; $e_n = \begin{bmatrix} 0 \\ \vdots \\ 0 \end{bmatrix}$, $a_i = 0$ \quad \cmark 
	\item[G1] Inversos (opuestos): $u' = -u$ \quad \cmark 
	\item[G1] Conmutatividad: $u + v = v + u$ \quad \cmark 
\end{itemize}

\begin{center}
	$(\R^n, +)$ es un grupo abeliano.
\end{center}

Los vectores en $\R^2$ se representan en la fig. \ref{fig:grupoR2}:

\begin{figure}[h]
	\centering
	\includestandalone[]{resources/ltx/grupoR2}
	\caption{$(\R^2, +)$ es un grupo abeliano}
	\label{fig:grupoR2}
\end{figure}

\begin{enumerate}[label=\roman*)]
	\item Neutro: $(0,0)$
	\item Inverso de $u = (u_1, u_2)$ es $-u = (-u_1, -u_2)$.
\end{enumerate}

\subsubsection{Producto de matrices}

Dadas las matrices $A = (a_{ik}) \in \mathbf{M}_{m\times n}$ y $B = (b_{kj}) \in \mathbf{M}_{n\times r}$,

\begin{equation}
	A \cdot B = 
	\begin{bmatrix}
		a_{11} & \cdots & a_{1n}\\
		\vdots & \ddots & \vdots\\
		a_{m1} & \cdots & a_{mn}
	\end{bmatrix} \cdot \begin{bmatrix}
		b_{11} & \cdots & b_{1r}\\
		\vdots & \ddots & \vdots\\
		b_{n1} & \cdots & b_{nr}
	\end{bmatrix} = \begin{bmatrix}
		c_{11} & \cdots & c_{1r}\\
		\vdots & \ddots & \vdots\\
		c_{m1} & \cdots & c_{mr}
	\end{bmatrix} = C
\end{equation}
donde:

\begin{equation}
	c_{ij} = \sum_{k=1}^{n} a_{ik} b_{kj}
	\label{eq:cij}
\end{equation}
es, por definición, el producto de matrices.

Ejemplo:

\[ \begin{bmatrix}
	1 & 2 & 3\\
	-1 & 0 & 1
\end{bmatrix} \cdot \begin{bmatrix}
	4 \\ 5 \\ 1
\end{bmatrix} = \begin{bmatrix}
	17 \\ -3
\end{bmatrix} \]

\textbf{Notas}:

\begin{enumerate}[label=\roman*)]
	\item La operación es $\bdot : \textbf{M}_{m \times n} \times \mathbf{M}_{n \times r} \rightarrow \mathbf{M}_{m \times r}$, es decir, $(A, B) \rightarrow C$. El número de columnas de $A$ debe ser igual al número de filas de $B$ para que se realizable el producto. La matriz $C$ es de orden igual al número de filas de $A$ por el número de columnas de $B$.
	
	\item Las matrices de orden $m\times n$ con $m \ne n$ son llamadas \textbf{matrices rectangulares} y cuando $m = n$ se dice que son \textbf{cuadradas}.
	
	\item El producto de matrices es asociativo, es decir: \[(A_{m\times n} \cdot B_{n \times r}) \cdot C_{r \times s} = A_{m \times n} \cdot (B_{n \times r} \cdot C_{r \times s})\] Como puede comprobarse de la definición.
	
	\item Las matrices rectangulares con la operación de multiplicación definida en \eqref{eq:cij} no forman un grupo, pues la operación no es \textbf{cerrada}, es decir, en general las matrices $A, B, C$, con $C = AB$ no son del mismo orden, en otras palabras, no son del mismo conjunto, de donde no se cumple la condición $\bdot: G \times G \rightarrow G$, dada en la definición de grupo.
\end{enumerate}

Consideremos solo matrices cuadradas $\mathbf{M}_{n \times n}(\R) = \mathbf{M}_n(\R)$, en este caso, la operación de multiplicación $(\cdot)$ es cerrada, esto es:

\[ \bdot: \mathbf{M}_n \times \mathbf{M}_n \rightarrow \mathbf{M}_n\]

\begin{itemize}
	\item[G1:] Asociatividad: $(AB)C = A(BC)$ \quad \cmark
	\item[G2:] Elemento neutro: \quad \cmark
	\[I_n = \begin{bmatrix}
		1 & \cdots & 0\\
		\vdots & \ddots & \vdots\\
		0 & \cdots & 1 
	\end{bmatrix}; a_{ij} = \delta_{ij} = \begin{cases} 1 & \mbox{ si } i = j;\\ 0 & \mbox{ si } i \ne j \end{cases} \] $I_n$ es la matriz identidad de orden $n$; $\delta_{ij}$ se conoce como delta de Kronecker\footnote{En homenaje al matemático alemán Leopold Kronecker (1823-1891).}.
	\item[G3:] Inversos \quad \xmark
	
	No toda matriz $A \in \mathbf{M}_n$ tiene inversa, como se verá más adelante en el curso.
\end{itemize}

de donde, en general las matrices cuadradas de orden $n$ con la operación de multiplicación no son un grupo, se cumple que:

\begin{center}
	$(\mathbf{M}_n(\R), \bdot)$ es un semigrupo con elemento neutro.
\end{center}

Se llaman \textbf{matrices invertibles} aquellas que admiten inversa. La inversa de $A$ se denota como $A^{-1}$. Esto nos lleva a la definición siguiente:

\begin{fmd-definition}[Grupo General Lineal]
	
	Dado el conjunto de matrices cuadradas invertibles con entradas en $\R$:
	\[ GL_n(\R) = \left\{ A \in \mathbf{M}_n(\R) \mid \exists A^{-1} \right\} \subset \mathbf{M}_n(\R) \]
	y la operación de multiplicación de matrices $(\bdot)$.
	
	$(GL_n(\R), \bdot)$ se llama \textit{grupo general lineal}.
\end{fmd-definition}

Notar que $(GL_n(\R), \bdot)$ es efectivamente un grupo.

Se verá más adelante en el curso que el determinante de matrices invertibles es distinto de cero. Incorporando este concepto de determinante definimos otro grupo más pequeño que el anterior.

\begin{fmd-definition}[Grupo Especial Lineal]
	
	El conjunto de matrices cuadradas invertibles con determinantes iguales a uno:
	\[ SL_n(\R) = \left\{ A \in \mathbf{M}_n(\R) \mid \det A = 1 \right\} \]
	
	con la operación de multiplicación matricial, $(SL_n(\R), \bdot)$ se llama \textit{grupo especial lineal}.
\end{fmd-definition}
Notar que:
\[ SL_n(\R) \subset GL_n(\R) \subset \mathbf{M}_n(\R)\]

\subsection{Homomorfismo de grupos} \label{sec:homogrupo}
\vspace{1em}
\begin{fmd-definition}[Homomorfismo de grupos]
	Dados los grupos $(G, \ast)$ y $(H, \bdot)$, un homomorfismo de $G$ a $H$ es una función $\phi: G \rightarrow H \, / \,  \forall a, b \in G \implies \phi(a*b) = \phi(a) \bdot \phi(b)$.
\end{fmd-definition}
\begin{figure}[H]
	\centering
	\begin{tikzpicture}[scale=1]
		% Coordenadas de puntos
		\coordinate (A0) at (0.5,1.5);
		\coordinate (Am1) at (.5,.5);
		\coordinate (A1) at (.5,-.5);
		\coordinate (A2) at (.5,-1.5);
		
		\coordinate (B0) at (5.5,1.5);
		\coordinate (B1) at (5,.5);
		\coordinate (B2) at (6,0);
		\coordinate (B3) at (6.5,-.75);
		\coordinate (B4) at (6,-1.5);
		
		% Circunferencias
		\node at (-1.5, 2) {$(G, *)$};
		\node at (7.5, 2) {$(H, \bdot)$};
		\draw[very thick] (0,0) circle (2);
		\draw[very thick] (6,0) circle (2);
		\node at (0, -2.5) {Dominio};
		\node at (6, -2.5) {Codominio};
		
		% Puntos
		\draw[fill] (A0) circle (1mm) node[left] {$a$};
		\draw[fill] (Am1) circle (1mm) node[left] {$b$};
		%			\draw[fill] (A1) circle (1mm) node[left] {$a*b$};
		\draw[fill] (A2) circle (1mm) node[left] {$a*b$};
		
		\draw[fill] (B0) circle (1mm) node[right] {$\phi(a)$};
		\draw[fill] (B1) circle (1mm) node[right] {$\phi(b)$};
		%			\draw[fill] (B2) circle (1mm) node[right] {2};
		%			\draw[fill] (B3) circle (1mm) node[right] {3};
		\draw[fill] (B4) circle (1mm) node[right, xshift=15mm] {$\phi(a*b) = \phi(a) \bdot \phi(b)$};
		
		% flechas
		\draw[red, thick, -latex] (A0) to [bend left = 25] (B0);
		\draw[blue, thick, -latex] (Am1) to [bend left = 25] (B1);
		%			\draw[blue, thick, -latex] (A1) to [bend left = -25] (B1);
		\draw[red, thick, -latex] (A2) to [bend left = -25] (B4);
	\end{tikzpicture}
	\caption{}
\end{figure}

\subsubsection{Propiedades básicas}
\begin{enumerate}
	\item La imagen del neutro de $G$ es el neutro de $H$: $\phi(e_G) = e_H$.
	\[ \phi(a) = \phi(a*e_G) = \phi(a) \bdot \phi(e_G) = \phi(a) \bdot e_H \implies \phi(e_G) = e_H\]
	por elemento neutro en $(G, *)$, homomorfismo, elemento neutro en $(H, \bdot)$ y ley de cancelación.
	
	\item La imagen del inverso de todo elemento en $G$ es el inverso de su imagen en $H$: $\phi(a'_G) = a'_H$ o, en otra notación: $f(x^{-1}) = \left[f(x)\right]^{-1}$.
	
	\[ \forall x \in G: x * x^{-1} = e \implies f(x * x^{-1}) = f(e)\] por homomorfismo y propiedad 1:
	
	\[ f(x) \cdot f(x^{-1}) = e_H \implies f(x^{-1}) = \left[f(x)\right]^{-1} \]
	
	En un diagrama:
	\begin{figure}[H]
		\centering
		\begin{tikzpicture}[scale=1]
			% Coordenadas de puntos
			\coordinate (A0) at (0.5,1.5);
			\coordinate (Am1) at (.5,.5);
			\coordinate (A1) at (.5,-.5);
			\coordinate (A2) at (.5,-1.5);
			
			\coordinate (B0) at (5.5,1.5);
			\coordinate (B1) at (5,.5);
			\coordinate (B2) at (6,0);
			\coordinate (B3) at (6.5,-.75);
			\coordinate (B4) at (6,-1.5);
			
			% Circunferencias
			\node at (-1.5, 2) {$(G, *)$};
			\node at (7.5, 2) {$(H, \bdot)$};
			\draw[very thick] (0,0) circle (2);
			\draw[very thick] (6,0) circle (2);
			\node at (0, -2.5) {Dominio};
			\node at (6, -2.5) {Codominio};
			
			% Puntos
			\draw[fill] (A0) circle (1mm) node[left] {$x$};
			\draw[fill] (Am1) circle (1mm) node[left] {$e$};
			%			\draw[fill] (A1) circle (1mm) node[left] {$a*b$};
			\draw[fill] (A2) circle (1mm) node[left] {$x^{-1}$};
			
			\draw[fill] (B0) circle (1mm) node[right] {$f(x)$};
			\draw[fill] (B1) circle (1mm) node[right] {$f(e) = e'$};
			%	\draw[fill] (B2) circle (1mm) node[right] {2};
			%	\draw[fill] (B3) circle (1mm) node[right] {3};
			\draw[fill] (B4) circle (1mm) node[right, xshift=15mm] {$f(x^{-1}) = \left[f(x)\right]^{-1}$};
			
			% flechas
			\draw[red, thick, -latex] (A0) to [bend left = 25] (B0);
			\draw[blue, thick, -latex] (Am1) to [bend left = 25] (B1);
			%	\draw[blue, thick, -latex] (A1) to [bend left = -25] (B1);
			\draw[red, thick, -latex] (A2) to [bend left = -25] (B4);
		\end{tikzpicture}
	\end{figure}
\end{enumerate}

\textbf{Se dice que}:
\begin{enumerate}[label=\roman*)]
	\item $\phi$ es un \textit{mono}morfismo si es \textit{inyectiva};
	\item $\phi$ es un \textit{epi}morfismo si es \textit{sobreyectiva};
	\item $\phi$ es un \textit{iso}morfismo si es \textit{biyectiva};
	\item $\phi$ es un \textit{endo}morfismo si $G = H$;
	\item $\phi$ es un \textit{auto}morfismo si es un endomorfismo biyectivo.
\end{enumerate}
A veces deja de escribirse $*$ y $\bdot$ de modo que $\phi(ab) = \phi(a)\phi(b)$, sin embargo no debe perderse de vista que $ab$ ocurre en $G$ y $\phi(a)\phi(b)$ en $H$.

\begin{fmd-example}
	Dados $(\R, +)$ grupo de números reales para la adición y $(\R^+, \bdot)$ grupo de números reales positivos para la multiplicación. La aplicación $f: \R \rightarrow \R^+$ definida por $f(x) = 2^x$ es un homomorfismo ya que:
	\[ f(x + y) = 2^{x + y} = 2^x 2^y = f(x)f(y)\]
	En particular, $f$ es biyectiva, luego $(\R, +)$ y $(\R^+, \bdot)$ son isomorfos.
\end{fmd-example}

\begin{fmd-example}
	Para cualquier grupo $G$
	
	\begin{enumerate}
		\item El mapa nulo $\phi: G \rightarrow G$, dado por $\phi(x) = e_G, \forall x \in G$ y el mapa identidad $id: G \rightarrow G$, dado por $id(x) = x, \forall x \in G$, son endomorfismos.
		\item Si $a\in G$, el mapa $C_a: G \rightarrow G / C_a(b) = a^{-1}ba$ es un automorfismo. En efecto: \vspace{-1mm}
		\[ C_a(bc) = a^{-1}bca = a^{-1}b (aa^{-1}) ca = (a^{-1}ba)(a^{-1} ca) = C_a(b)C_a(c)\]
		es un homomorfismo, falta verificar que es invertible, su inversa es $C_{a^{-1}}: G \rightarrow G$ (verificar).
	\end{enumerate}
\end{fmd-example}

\begin{fmd-example}
	Dados $(\R^{+}, \times)$ grupo de números reales positivos con la operación de multiplicación y $(\R, +)$ grupo de números reales con la adición. Probar que la aplicación $f: \R^{+} \rightarrow \R$ definida por $f(x) = \ln x$ es un isomorfismo.
	
	Es un homomorfismo ya que $\ln(xy) = \ln x + \ln y$. Para probar que es un isomorfismo debemos mostrar que $f$ es biyectiva.
	
	\begin{enumerate}[label=\alph*)]
		\item $f$ es 1-1 si $f(x) = f(y) \implies x = y$
		\[ \ln x = \ln y \implies e^{\ln x} = e^{\ln y} \implies x = y \]
		$f$ es inyectiva.
		
		\item $f$ es sobreyectiva pues $\mbox{Im}(f) = \R$
		
		\begin{figure}[H]
			\centering
			\includestandalone[]{resources/ltx/log}
		\end{figure}
	\end{enumerate}
	De donde $f$ es biyectiva, por lo tanto los grupos mencionados son isomorfos bajo la función logaritmo natural.
\end{fmd-example}

\subsection{Grupos de simetrías}

\subsubsection{Grupos de funciones}
?`Cómo construimos grupos cuyos elementos sean funciones? ?`Cuál debe ser la operación?

Una función asigna a un elemento de $A$ un único elemento de $B$, $f: A \rightarrow B$, es una regla de correspondencia, por lo que la opción natural es adoptar la operación de composición, ahora bien, la composición, en general ``cambia de conjunto''  \[A \xrightarrow{f} B \xrightarrow{g} C = A \xrightarrow{g \circ f} C\]

La operación debe ser cerrada:
\[A \xrightarrow{f} A \xrightarrow{g} A = A \xrightarrow{g \circ f} A\]

Adoptamos la notación $B^A \coloneqq \{ f: A \rightarrow B \}$, de donde: $A^A = \{ f: A \rightarrow A \}$.

Como la composición es asociativa y tiene elemento neutro, hasta aquí podemos decir que:
\[ (A^A, \circ) \mbox{ es un semigrupo con neutro } \mathbf{1}_A: A \rightarrow A /  \mathbf{1}_A(x) = x \]

No es un grupo porque no toda función tiene inversa, en este sentido es similar al caso de las matrices visto en \ref{sec:matrices}, para que una función admita inversa debe ser biyectiva (ver \ref{sec:finversas}). 

\subsubsection{Grupos de permutaciones}

\begin{fmd-definition}[Permutación]
	Dado un conjunto $A$, una \textbf{permutación} sobre $A$ es una función biyectiva $f: A \rightarrow A$
\end{fmd-definition}

Notación $S_A \coloneqq \{ f: A \rightarrow A | \, f \mbox{ es biyectiva}\}$.

Conforme a todo lo dicho:
\[ (S_A, \circ ) \mbox{ es un grupo.}\]

En particular, consideremos un conjunto finito $A = \{ a_1, \dots, a_n \}$, de $n$ elementos, es decir, $|A| = n$. En este caso se escribe $S_A = S_n$, el cual recibe el nombre de \textbf{grupo de simetrías} de $A$.

Por ejemplo: para $n = 3$ el conjunto sería $A = \{a, b, c\}$, y el de simetrías denotamos por $S_3$.

?`Cuáles serían los elementos de $S_3$?

La cantidad de asociaciones posibles, es decir, la cardinalidad de $A^A$ sean las funciones biyectivas o no, es $3^3 = 27$, si restringimos solo a las biyectivas, tenemos $|S_3| = 3! = 6$.

El conjunto $S_3$ con sus 6 elementos es:
\[ \begin{array}{rcccccc}
	S_3 = & \left\{ \begin{pmatrix}
		a & b & c\\
		a & b & c
	\end{pmatrix}, \right. & \begin{pmatrix}
		a & b & c\\
		a & c & b
	\end{pmatrix}, & \begin{pmatrix}
		a & b & c\\
		b & a & c
	\end{pmatrix}, & \begin{pmatrix}
		a & b & c\\
		b & c & a
	\end{pmatrix}, & \begin{pmatrix}
		a & b & c\\
		c & a & b
	\end{pmatrix}, & \left. \begin{pmatrix}
		a & b & c\\
		c & b & a
	\end{pmatrix} \right\} \\[5mm]
	= & \left\{  \varepsilon \right. & \sigma_1 & \sigma_3 & \phi_1 & \phi_2 & \left. \sigma_2 \right\}
\end{array} \]

En donde los elementos no deben entenderse como matrices, aunque se denomina \textbf{notación matricial}, en cuyas primeras filas se muestran los elementos de $A$ y en segundas filas sus asignaciones correspondientes, $\begin{pmatrix}
	a & b & c \\ i & j & k
\end{pmatrix}$ es la permutación que aplica $a \mapsto i, b \mapsto j, c \mapsto k $. Como ejemplo se muestran los diagramas de la fig. \ref{fig:permut}.

\begin{figure}[h]
	\centering
	\begin{tikzpicture}[scale=1]
		% Coordenadas de puntos
		\coordinate (A0) at (0,1.5);
		\coordinate (A1) at (0,0);
		\coordinate (A2) at (0,-1.5);
		
		\coordinate (B0) at (4,1.5);
		\coordinate (B1) at (4,0);
		\coordinate (B2) at (4,-1.5);
		
		% Elipses
		\node at (-1, 2) {$A$};
		\node at (5, 2) {$A$};
		\draw[thick] (0,0) ellipse (1 and 2);
		\draw[thick] (4,0) ellipse (1 and 2);
		
		% Puntos
		\draw[fill] (A0) circle (1mm) node[left] {$a$};
		\draw[fill] (A1) circle (1mm) node[left] {$b$};
		\draw[fill] (A2) circle (1mm) node[left] {$c$};
		
		\draw[fill] (B0) circle (1mm) node[right] {$b$};
		\draw[fill] (B1) circle (1mm) node[right] {$a$};
		\draw[fill] (B2) circle (1mm) node[right] {$c$};
		
		% flechas
		\draw[red, thick, -latex] (A0) -- (B0) node[above, blue, midway]{$\sigma_3$};
		\draw[red, thick, -latex] (A1) -- (B1);
		\draw[red, thick, -latex] (A2) -- (B2);
	\end{tikzpicture}
	\caption{}
	\label{fig:permut}
\end{figure}

Si colocamos los elementos de $A$ en un triángulo equilátero, trazando bisectrices en cada ángulo, como se muestra en la fig. \ref{fig:gtriang}, cada elemento de $S_3$ puede entenderse como:
\begin{figure}[h]
	\centering
	\includestandalone[scale=2]{resources/ltx/grupoS3}
	\caption{}
	\label{fig:gtriang}
\end{figure}

\begin{itemize}
	\item Identidad: $i_A = \varepsilon$
	\item Reflexiones $(\sigma)$: respecto a cada bisectriz.
	\begin{itemize}
		\item Respecto de la bisectriz que pasa por $a$: $\sigma_1$ ($a$ es un \textbf{punto fijo})
		\item Respecto de $b$: $\sigma_2$ ($b$ es un punto fijo)
		\item Respecto de $c$: $\sigma_3$ ($c$ es un punto fijo)
	\end{itemize}
	\item Rotaciones $(\phi)$: los elementos \textit{giran}\footnote{El giro es de $60 \unit{\degree}$ (en general $360/n$), en sentido antihorario de acuerdo a cómo fueron colocados $a, b, c$ en la fig. \ref{fig:gtriang}} y se reubican en nuevos vértices.
	\begin{itemize}
		\item $\phi_1$: $a \mapsto b, b \mapsto c, c \mapsto a $, es decir:  $(a, b, c) \mapsto (b, c, a)$
		\item $\phi_2$: $a \mapsto c, c \mapsto b, b \mapsto a $, es decir: $(a, c, b) \mapsto (c, b, a)$
	\end{itemize}
\end{itemize}

\[ S_3 = \{ \varepsilon, \sigma_1, \sigma_2, \sigma_3, \phi_1, \phi_2\} \]

El conjunto $S_3$ se conoce como \textbf{simetrías del triángulo}, similarmente $S_4$ contiene las \textbf{simetrías del cuadrado}. En general las simetrías son transformaciones de reflexiones y rotaciones que mantienen las formas. Estas simetrías forman un grupo llamado \textbf{grupo de simetrías} $S_n$. Cuando la forma es un polígono regular el grupo de simetrías es llamado \textbf{grupo diédrico} $D_n$.

Utilizando números $A = \{1,2,3\}$ en sustitución de $\{a, b, c\}$ se tiene la tabla \ref{tab:S3}.
\begin{table}[h]
	\centering
	\begin{tabular}{|c|c|c|c|c|c|c|} \hline
		$A$ & $e$& $r$ & $r^2$ & $f$ & $rf$ & $r^2f$ \\ \hline
		1 & 1 & 3 & 2 & 1 & 2 & 3 \\ \hline
		2 & 2 & 1 & 3 & 3 & 1 & 2\\ \hline
		3 & 3 & 2 & 1 & 2 & 3 & 1\\ \hline
	\end{tabular}
	\caption{}
	\label{tab:S3}
\end{table}

En donde hemos llamado $r$ a una rotación y $f$ a una reflexión:
\[ \begin{array}{ccc}
	\varepsilon = e; & r = \phi_1; & r^2 = \phi_2;\\
	 f = \sigma_1; & rf = \sigma_3; & r^2f = \sigma_2
\end{array} \]

Los grupos de simetrías para diferentes triángulos son:

\begin{itemize}
	\item Equilátero: $\{ e, r, r^2, f, rf, r^2f \}$;
	\item Isósceles: $\{ e, f \}$;
	\item Escaleno: $\{ e \}$
\end{itemize}

\textbf{Tarea:}
Escribe el grupo de simetrías del cuadrado y del rectángulo.

\textbf{Composición de funciones en $S_3$}

Se muestra en la fig. \ref{fig:compos} un ejemplo de composición de funciones y en la tabla \ref{fig:compos} la composición de todos los pares de funciones, la operación es:

\[ \sigma_3 \circ \sigma_1 = \begin{pmatrix}
	1 & 2 & 3\\
	2 & 1 & 3
\end{pmatrix} \circ \begin{pmatrix}
	1 & 2 & 3\\
	1 & 3 & 2
\end{pmatrix} = \begin{pmatrix}
	1 & 2 & 3\\
	3 & 1 & 2
\end{pmatrix} = \phi_2\]

\begin{figure}[H]
	\centering
	\begin{tikzpicture}[scale=1]
		% Coordenadas de puntos
		\coordinate (A0) at (0,1.5);
		\coordinate (A1) at (0,0);
		\coordinate (A2) at (0,-1.5);
		
		\coordinate (B0) at (4,1.5);
		\coordinate (B1) at (4,0);
		\coordinate (B2) at (4,-1.5);
		
		\coordinate (C0) at (8,1.5);
		\coordinate (C1) at (8,0);
		\coordinate (C2) at (8,-1.5);
		
		% Elipses
		\draw[thick] (0,0) ellipse (1 and 2);
		\draw[thick] (4,0) ellipse (1 and 2);
		\draw[thick] (8,0) ellipse (1 and 2);
		
		% Puntos
		\draw[fill] (A0) circle (1mm) node[left] {$1$};
		\draw[fill] (A1) circle (1mm) node[left] {$2$};
		\draw[fill] (A2) circle (1mm) node[left] {$3$};
		
		\draw[fill] (B0) circle (1mm) node[right] {$2$};
		\draw[fill] (B1) circle (1mm) node[right] {$1$};
		\draw[fill] (B2) circle (1mm) node[right] {$3$};
		
		\draw[fill] (C0) circle (1mm) node[right] {$3$};
		\draw[fill] (C1) circle (1mm) node[right] {$1$};
		\draw[fill] (C2) circle (1mm) node[right] {$2$};
		
		% flechas
		\draw[red, thick, -latex] ($(A0)+(.4,0)$) -- ($(B0)+(-.4,0)$) node[above, blue, midway]{$\sigma_1$};
		\draw[red, thick, -latex] ($(A1)+(.4,0)$) -- ($(B1)+(-.4,0)$);
		\draw[red, thick, -latex] ($(A2)+(.4,0)$) -- ($(B2)+(-.4,0)$);
		
		\draw[red, thick, -latex] ($(B0)+(.5,0)$) -- ($(C0)+(-.4,0)$) node[above, blue, midway]{$\sigma_3$};
		\draw[red, thick, -latex] ($(B1)+(.5,0)$) -- ($(C1)+(-.4,0)$);
		\draw[red, thick, -latex] ($(B2)+(.5,0)$) -- ($(C2)+(-.4,0)$);
		
		\draw[blue, very thick, -latex] ($(A2)+(.4,-.3)$) to[bend right=20] ($(C2)+(-.4, -.3)$) node[below, xshift=-60, yshift=-20] {$\phi_2=\sigma_3 \circ \sigma_1$};
	\end{tikzpicture}
	\caption{}
	\label{fig:compos}
\end{figure}

\begin{table}[H]
	\centering
	\begin{tabular}{c|cccccc}
		$\circ$ & $\varepsilon$ & $\sigma_1$ & $\sigma_2$ & $\sigma_3$ & $\phi_1$ & $\phi_2$ \\ \hline
		$\varepsilon$ & $\varepsilon$ & $\sigma_1$ & $\sigma_2$ & $\sigma_3$ & $\phi_1$ & $\phi_2$ \\
		$\sigma_1$ & $\sigma_1$ & $\varepsilon$ & $\phi_1$ & $\phi_2$ & $\sigma_2$ & $\sigma_3$ \\
		$\sigma_2$ & $\sigma_2$ & $\phi_2$ & $\varepsilon$ & $\phi_1$ & $\sigma_3$ & $\sigma_1$\\
		$\sigma_3$ & $\sigma_3$ & $\phi_1$ & $\phi_2$ & $\varepsilon$ & $\sigma_1$ & $\sigma_2$\\
		$\phi_1$ & $\phi_1$ & $\sigma_3$ & $\sigma_1$ & $\sigma_2$ & $\phi_2$ & $\varepsilon$\\
		$\phi_2$ & $\phi_2$ & $\sigma_2$ & $\sigma_3$ & $\sigma_1$ & $\varepsilon$ & $\phi_1$\\
	\end{tabular}
	\caption{Grupo $(S_3, \circ)$.}
	\label{tab:compos}
\end{table}

\begin{itemize}
	\item[G1:] Asociatividad \cmark
	\item[G2:] Elemento neutro: $\varepsilon$ \cmark
	\item[G3:] Inversos: \cmark
	\begin{itemize}
		\item Todos son sus propios inversos a excepción de $\phi_1$ y $\phi_2$ que son inversos entre si.
	\end{itemize}
	\item[G4:] Conmutatividad: \xmark
\end{itemize}

\subsection{Subgrupos} \label{sec:subgrupo}
\vspace{3mm}	
	\begin{fmd-definition}[Subgrupo]
		El subconjunto no vacío $H$ de $G$, es un subgrupo de $(G, *)$ si y solo si $(H, *)$ es un grupo.
	\end{fmd-definition}
	Notar que como el neutro es único y $H \subseteq G \implies e_G = e_H$
	
	\textbf{Subgrupos}:
	\begin{enumerate}
		\item $(\mathbb{Z}, +) \le (\mathbb{Q}, +) \le (\mathbb{R}, +) \le (\mathbb{C}, +)$
		\item $\left( \{1, -1 \}, \bdot \right) \le \left( \mathbb{Q} \setminus \{0\}, \bdot \right) \le \left( \mathbb{R} \setminus \{0\}, \bdot \right) \le \left( \mathbb{C} \setminus \{0\}, \bdot \right)$
	\end{enumerate}
	
	\begin{fmd-theorem}[Condición suficiente para existencia de subgrupo] \label{teo:subgrupo}
		Dado el grupo $(G, *)$, si $H \subseteq G$, con $H \ne \emptyset$, que verifica:
		\[ a \in H \land b \in H \implies a * b' \in H\]
		entonces $(H, *)$ es un subgrupo de $(G, *)$.
	\end{fmd-theorem}

	\begin{fmd-proof}
		\begin{itemize}
			\item Asociatividad: se verifica por ser $H \subset G$.
			\item El neutro pertenece a $H$. En efecto: $H \ne \emptyset \implies \exists a \in H$
			\vspace{-3mm}
			\[a \in H \land a \in H \implies a*a'\in H \implies e \in H\]
			\item Todo elemento de $H$ admite inverso en $H$.
			\vspace{-3mm}
			\[ e \in H \land a \in H \implies e * a' \in H \implies a'\in H \]
			\item $H$ es cerrado para $*$
			\vspace{-3mm}
			\[ a \in H \land b \in H \implies a \in H \land b' \in H \implies a*(b')' \in H \implies \]
			\[ a*b \in H\]
		\end{itemize}
	\end{fmd-proof}
	Es fácil verificar que la condición también es necesaria.
	
Observemos que los enteros pares forman subgrupo de $(\Z, +)$ pero los impares no ?`por qué?

\begin{definition}[Subgrupo trivial]
	Sea $(G, *)$ un grupo no vacío, consideremos el conjunto $H = \{ e \}$, claramente $H \subset G$, se cumple que $(H, *)$ es un subgrupo de $(G, *)$, pues:
	\[ e \in H \land e \in H \implies e * e' = e \in H \]
	
	Al subgrupo $(H, *)$ definido de esta manera se le llama \textit{subgrupo trivial}.
\end{definition}

\begin{fmd-definition}[Subgrupo cíclico]
	Sea $(G, *)$ un grupo y $a \in G$, consideremos el conjunto $H = \{ a^n: n \in \mathbb{Z}\}$, $(H, *)$ es un subgrupo de $(G, *)$.
	
	\begin{proof}
		Sean $g, h \in H$ entonces $ \exists m, n \in \mathbb{Z} / g = a^m \land h = a^n$, en consecuencia $g*h^{-1} = a^m * a^{-n} = a^{m-n} \in H $
		así tenemos que $(H, *)$ es un subgrupo de $(G,*)$
	\end{proof}
	Notemos que si $a=1_G$ entonces $H=\{a^n:n \in \mathbb{Z}\} = \{1_G\}$, pero si $a\neq 1_G$ en $H$ tenemos al menos dos elementos:
	\[ a = a^1 \mbox{ y } 1_G = a^0 \]
	Al subgrupo $(H, *)$ se le llama \textit{subgrupo cíclico} generado por $a$ y se denota por
	\[ H = \langle a \rangle \]
\end{fmd-definition}
	
	Sea $G$ un grupo y $a \in G$:
	\begin{itemize}
		\item \textbf{Notación multiplicativa}:
		\[ a^1 = a, \; a^0 = 1, \; a^n = \underbrace{a \cdots a}_{n \, veces}, \forall n \ge 2; \quad a^{-n} = (a^{-1})^n = \underbrace{a^{-1} \cdots a^{-1}}_{n \, veces}\] \vspace{-5mm}
		
		\item \textbf{Notación aditiva}:
		\[ 1a = a, \; 0a = 0, \; na = \underbrace{a + \cdots + a}_{n \, veces}, \forall n \ge 2; \quad -na = \underbrace{-a - \cdots -a}_{n \; veces}\]
	\end{itemize}
	
	\begin{fmd-example}
		El grupo $(\mathbb{Z}, +)$ es cíclico, pues cualquier elemento $p \in \mathbb{Z}$ es de la forma: $p = 1^p = \underbrace{1+1+\cdots+1}_{p \, veces} = 1 \cdot p$
	\end{fmd-example}
	
	$\mathbb{Q}, \mathbb{R}$ y $\mathbb{C}$ no son cíclicos. El grupo de números pares $\langle 2 \rangle$ es cíclico. El menor grupo cíclico es el trivial.
	
	Utilizando la notación multiplicativa los subgrupos cíclicos con generador $x$ deben ser de la forma:
	\[ \langle x \rangle = \{ \dots, x^{-3}, x^{-2}, x^{-1}, 1, x, x^2, x^3,  \dots \} \]
	
	En la notación aditiva tenemos:
	\[ \langle y \rangle = \{ \dots, -3y, -2y, -y, 0, y, 2y, 3y,  \dots \} \]


	\begin{fmd-definition}[Orden de un elemento]
		Sean $(G, *)$ un grupo y $a \in G$. Consideremos al conjunto $\{ n \in \mathbb{Z}^{+}: a^n = 1_G \}$. Se define el \textit{orden} de $a$ como:
		\[ o(a) = |a| = \min \{ n \in \mathbb{Z}^{+}: a^n = 1_G\} \in \mathbb{Z}^{+} \]
		en este caso diremos que $a$ es de orden finito.
		
		Si $\{ n \in \mathbb{Z}^{+}: a^n = 1_G\} = \emptyset$ definimos el orden de $a$ como $o(a) = \infty$.
	\end{fmd-definition}
	\begin{itemize}
		\item Notar que $o(1_G) = 1$ y es único, i.e., $\exists! \, a \in G$ con $o(a) = 1$ ($a = 1_G$).
		\item El orden del generador coincide con el del subgrupo cíclico que genera:
		\[ o(a) = |\langle a \rangle| \]
	\end{itemize}
	
	\begin{fmd-example}
		Sea $G = \{ 1, i, -1, -i \}$ siendo $i = \sqrt{-1}$ la unidad imaginaria. Su tabla de Cayley para la operación de multiplicación es:
		\[ \begin{array}{r|rrrr}
			\times & 1 & i & -1 & -i\\ \hline
			1 & 1 & i & -1 & -i\\
			i & i & -1 & -i & 1\\
			-1 & -1 & -i & 1 & i\\
			-i & -i & 1 & i & -1 
		\end{array} \]
		Notar que el grupo es abeliano y cíclico, generado por $i$, es decir $G = \langle i \rangle$. El elemento neutro es el 1.
		
		El orden del elemento $-1$ es 2, porque $(-1)^2 = 1$, el orden del elemento generador $i$ es 4, porque $i^4 = 1$, que es igual al orden $|G|$ del subgrupo generado.
	\end{fmd-example}
	
	\begin{fmd-example}
		Considerando el grupo de simetrías del triángulo equilátero $D_3 = \{ e, r, r^2, f, rf, r^2f \}$, en donde $r$ es una rotación y $f$ una reflexión.
		
		Como $r^3 = e$ el orden de $r$ es $|r| = 3$, análogamente, como $f^2 = e$ el orden de $f$ es $|f| = 2$.
	\end{fmd-example}
	
	\textbf{Tarea}: Probar que:
	\begin{enumerate}
		\item Si un grupo es cíclico, entonces es abeliano.
		\item Cualquier subgrupo de un  grupo cíclico, también es cíclico.
		\item Cualquier grupo $(G, \bdot)$ de orden 1, 2, 3 o 4 es abeliano
	\end{enumerate}

	Sean $(G, *)$ un grupo y $\{G_i\}_{i \in I}$ una familia de subgrupos de $(G, *)$
	\begin{fmd-theorem}
		La intersección de toda familia no vacía de subgrupos de $(G, *)$ es un subgrupo.
	\end{fmd-theorem}
	
	\begin{fmd-proof}
		\begin{itemize}
			\item[H)] $(G, *)$ es grupo. $\{G_i\}$ es tal que $(G, *)$ es un subgrupo de $G, \forall i \in I$.
			\item[T)] $\left( \bigcap_{i \in I} G_i, * \right)$ es subgrupo de $(G, *)$
		\end{itemize}
		\begin{enumerate}[label=\roman*)]
			\item $\forall i: e \in G_i$, pues $(G_i, *)$ es un grupo: $e \in \bigcap_{i \in I} G_i \implies \bigcap_{i \in I} G_i \ne \emptyset$
			\item $\bigcap_{i \in I} G_i \subset G$ por definición de inclusión
			\item Sean $a, b \in \bigcap_{i \in I}G_i \implies a \in G_i \land b \in G_i, \forall i \implies$
			
			$a * b'\in G_i, \forall i \implies a * b' \in \bigcap_{i \in I}G_i$
		\end{enumerate}
	\end{fmd-proof}
	Esta propiedad no se verifica en el caso de la unión.

	\begin{fmd-definition}[Núcleo de homomorfismo]
		Núcleo o \textit{kernel} del homomorfismo $f: G \rightarrow G'$ es la totalidad de los elementos de $G$, cuyas imágenes por $f$ se identifican con el neutro de $G'$.
		\[ N(f) = \ker f = \{ x \in G / f(x) = e' \} \]
	\end{fmd-definition}
	Es claro que el núcleo de $f$ es la preimagen de $e'$.
	
\begin{figure}[H]
	\centering
	\includestandalone[scale=1.1]{resources/ltx/kernel}
	\caption{}
	\label{fig:kernel}
\end{figure}

	\begin{proposition}
		El núcleo de todo homomorfismo de grupos es un subgrupo del primero.
	\end{proposition}
	\begin{proof}
		En efecto:
		\begin{itemize}
			\item $\forall a, b \in \ker f \implies f(ab) = f(a)f(b) = e_{G'} e_{G'} = e_{G'} \implies ab \in \ker f$.
			\item $a \in \ker f \implies f(a^{-1}) = \left[ f(a) \right]^{-1}=e_{G'}^{-1} = e_{G'} \implies a^{-1} \in \ker f$
		\end{itemize}
	\end{proof}
	
	\begin{proposition}
		El homomorfismo $f: G \rightarrow G'$ es inyectivo, es decir, un monomorfismo si y solo si el núcleo es unitario: $\ker(f) = \{ e\}$.
	\end{proposition}
	\begin{proof}:
		\begin{itemize}
			\item Si $f$ es inyectivo, entonces $\forall a \in G \land a \ne e_G$ debemos tener $f(a) \ne f(e_G) = e_G$, es decir, $\ker f = \{e_G\}$;
			\item Recíprocamente, si $\ker f = \{ e_G\}$, entones: $f(a) = f(b) \implies f(a) \left[f(b)\right]^{-1} = e_H \implies f(ab^{-1}) = e_H \implies ab^{-1}=e_G \implies a=b$
		\end{itemize}
	\end{proof}


	\begin{fmd-definition}[Imagen de un homomorfismo]
		Es la totalidad de las imágenes de los elementos del primer grupo.
	\end{fmd-definition}
	
	\begin{proposition}
		La imagen de todo homomorfismo de grupos es un subgrupo del segundo: $Im(f) \subseteq G'$.
	\end{proposition}


	\begin{definition}[Subgrupo distinguido]
		El subgrupo $(H, *)$ de $(G, *)$ es \textbf{distinguido} si y sólo si existe un grupo $(G', *')$ y un homomorfismo $f: G \rightarrow G'$, cuyo núcleo es $H$.
		
		En símbolos:
		
		$H \subset G$ es distinguido $\iff \exists G'$ grupo, y $f:G \rightarrow G'$ homomorfismo $/ N(f) = H$
	\end{definition}
	
	Subgrupos distinguidos de todo grupo $(G, *)$ son el mismo $G$ y $\{e\}$ \cite{rojoAlgebra8vaEd} p249.

\subsection{Subgrupo normal} \label{sec:subgrupo_normal}

	\subsubsection{Relación de equivalencia y clases}
	
	Sea $(H, *)$ un subgrupo de $(G, *)$. Definimos en $G$ la relación $\sim$ mediante: $a \sim b \iff a' * b \in H$, es decir, dos elementos está relacionados si y sólo si la composición del inverso del primero con el segundo pertenece a $H$.
	
	La relación es de equivalencia pues verifica la reflexividad $a \sim a$, simetría $a\sim b \implies b \sim a$ y transitividad $a\sim b \land b \sim c \implies a \sim c$, en efecto
	
	\begin{fmd-proof} Propiedades de equivalencia:
		\begin{enumerate}[label=\roman*)]
			\item Reflexividad: $a \equiv_{H} a \iff aa^{-1} = e \in H$;
			\item Simetría: Si $a \equiv_{H} b \implies ab^{-1} \in H$ luego $(ab^{-1})^{-1} \in H$, así $ba^{-1} \in H$. De modo que $b \equiv_{H} a$
			\item Transitividad: Sean $a \equiv_{H} b$ y $b \equiv_{H} c$ luego $ab^{-1}, bc^{-1} \in H$ multiplicando ambos elementos $\left(ab^{-1}\right) \left(bc^{-1}\right) = aec^{-1} = ac^{-1} \in H$. Se tiene que $a \equiv_{H} c$.
		\end{enumerate}
		Concluimos que $a \equiv_{H} b$ es una relación de equivalencia.
	\end{fmd-proof}
	
	\textbf{Algunas definiciones} $\forall a \in G \land H \subseteq G$:
	\begin{itemize}
		\item \textbf{Conjugado}: Para $a,b, g \in G$, se dice que $b$ es el \textbf{conjugado} de $a$ por $g$ si existe este elemento $g$ tal que $b = g^{-1}ag$.
		\item \textbf{Clase\footnote{Algunos autores utilizan los términos: \textit{coclase} o \textit{cogrupo} en vez de \textit{clase}.} lateral izquierda de H}: es el subgrupo $aH \coloneqq \{ah; h \in H\} $
		\item \textbf{Clase lateral derecha de H}: es el subgrupo $Ha \coloneqq \{ha; h \in H\} $
		\item \textbf{Grupo conjugado\footnote{Otros llaman \textit{coclases} a las clases laterales y \textit{clases} a los grupos conjugados.} de H}: es el subgrupo $a^{-1}Ha \coloneqq \{a^{-1}ha; h \in H\} $
		\item \textbf{Centro de G}: $Z(G) \coloneqq \{z\in G / zx = xz \forall x \in G\}$. Subgrupo que contiene todos los elementos que conmutan con cualquier elemento de $G$.
	\end{itemize}
	
	\begin{fmd-definition}[Subgrupo normal o invariante] \label{def:normal}
		El subgrupo $(H, *)$ de $(G, *)$ es normal o invariante, si y solo si se verifica: $ x \in G \land y \in H \implies x * y * x^{-1} \in H $
	\end{fmd-definition}
	
	En otras palabras, en un grupo normal, el conjugado de un elemento de $H$ pertenece a $H$.
	
	\textbf{Notas}:
	\begin{enumerate}
		\item Notación: $H \triangleleft G$ se lee \textit{$H$ es un subgrupo normal} (o invariante) de $G$;
		\item En toda $G$ siempre existen por lo menos dos subgrupos normales:
		\begin{itemize}
			\item El subgrupo trivial: $\{e\}$;
			\item El grupo entero: $G$
		\end{itemize}
		\item Un subgrupo es distinguido si y sólo si es invariante.
	\end{enumerate}
	
	\begin{proposition}[Subgrupos de grupos abelianos] \label{prop:subgrupo}
		Todo subgrupo de un grupo abeliano es normal, es decir: $\forall x, y: x\in G \land y \in H \implies xyx^{-1} = xx^{-1}y = ey = y \in H$;
	\end{proposition}
	
	\begin{definition}[Grupo simple]
		Un \textbf{grupo simple} es un grupo no trivial con exactamente dos subgrupos normales, el subgrupo trivial y él mismo.
	\end{definition}
	
	\begin{proposition}
		Si $f: G \rightarrow H$ es un homomorfismo, entonces $\ker f \subseteq G$ es un subgrupo normal. \[ \ker f \trianglelefteq G \]
	\end{proposition}
	
	\begin{proof}
		En efecto: $\forall a \in G \land b \in \ker f$
		\[f(a^{-1}ba) = f(a^{-1}) f(b) f(a) = f^{-1}(a) e_H f(a) = f^{-1}(a)f(a) = e_H\]
		así:
		\[a^{-1} (\ker f) a \in \ker f\]
	\end{proof}
	
	La definición \ref{def:normal} es equivalente a decir que si  $H$ es un subgrupo normal de $G$, entonces las clases laterales izquierda y derecha de $H$ son iguales, es decir:
	\begin{equation} \label{eq:normal}
		H \triangleleft G \implies gH = Hg
	\end{equation}
	La expresión $Hg$ no es otra cosa que operar $h * g = hg \, \forall \, h \in H$ y $g \in G$.
	
	\begin{fmd-example}
		Recordemos el grupo $(S_3, \circ)$ con la tabla \ref{tab:compos}, que reescribimos aquí:
		\begin{table}[H]
			\centering
			\begin{tabular}{c|cccccc}
				$\circ$ & $\varepsilon$ & $\sigma_1$ & $\sigma_2$ & $\sigma_3$ & $\phi_1$ & $\phi_2$ \\ \hline
				$\varepsilon$ & $\varepsilon$ & $\sigma_1$ & $\sigma_2$ & $\sigma_3$ & $\phi_1$ & $\phi_2$ \\
				$\sigma_1$ & $\sigma_1$ & $\varepsilon$ & $\phi_1$ & $\phi_2$ & $\sigma_2$ & $\sigma_3$ \\
				$\sigma_2$ & $\sigma_2$ & $\phi_2$ & $\varepsilon$ & $\phi_1$ & $\sigma_3$ & $\sigma_1$\\
				$\sigma_3$ & $\sigma_3$ & $\phi_1$ & $\phi_2$ & $\varepsilon$ & $\sigma_1$ & $\sigma_2$\\
				$\phi_1$ & $\phi_1$ & $\sigma_3$ & $\sigma_1$ & $\sigma_2$ & $\phi_2$ & $\varepsilon$\\
				$\phi_2$ & $\phi_2$ & $\sigma_2$ & $\sigma_3$ & $\sigma_1$ & $\varepsilon$ & $\phi_1$\\
			\end{tabular}
		\end{table}
		
		El conjunto $H = \{\varepsilon, \sigma_1\}$ es un subgrupo de $S_3$, sus clases a derecha e izquierda son:
		
		\begin{table}[H]
			\centering
			\begin{tabular}{c|c}
				Clases a derecha & Clases a izquierda\\ \hline
				$H = \{\varepsilon, \sigma_1\}$ & $H = \{ \varepsilon, \sigma_1 \}$\\
				$H\phi_1 = \{ \phi_1, \sigma_2 \}$ & $\phi_1H = \{ \phi_1, \sigma_3 \}$\\
				$H\phi_2 = \{ \phi_2, \sigma_3 \}$ & $\phi_2H = \{ \phi_2, \sigma_2 \}$
			\end{tabular}
		\end{table}
		Se observa que las clases a derecha e izquierda no son iguales, por tanto, $H$ no es un subgrupo normal de $S_3$.
	\end{fmd-example}
	
	\begin{fmd-definition}[Producto de clases laterales]
		Sea $H \triangleleft G$ y sean $Ha$ y $Hb$ clases laterales de $H$ definimos el producto de clases laterales derechas como $ (Ha) (Hb) = Hab $.
	\end{fmd-definition}
	Veamos si este producto está bien definido.
	\begin{fmd-proof}
		Sean $a_1, a_2 \in Ha$ y $b_1, b_2 \in Hb$ se debe probar que $a_1b_1 \in Hab$ y $a_2b_2 \in Hab$ o, lo que es lo mismo, $a_1b_1 \equiv_{H} a_2b_2$.
		
		Se tiene que, si
		\[ a_1 \in Ha \, \exists \, h_1 \in H \, | \, a_1 = h_1a_2, \quad b_1 \in Hb \, \exists \, h_2 \in H \, | \, b_1 = h_2b_2 \]
		luego:
		\[ a_1b_1 = (h_1a_2)(h_2b_2) = h_1 (a_2 h_2) b_2 \]
		pero por ser $H$ un subgrupo normal $Hg = gH \, \forall g \in G$, así
		\[ a_1b_1 = h_1(a_2h_2)b_2 = h_1(h_3a_2)b_2 = (h_1h_3)a_2b_2 \implies (a_1b_1)(a_2b_2)^{-1} = h_1h_3 \in H \] 
	\end{fmd-proof}

\subsection{Grupo cociente}
\vspace{1em}
\begin{fmd-theorem}[Grupo cociente] \label{teo:cociente}
	Si $H \triangleleft G$ y sea el conjunto $G/H = \{ \bar{g} / g \in G \} = \{ Hg / g \in G \}$, entonces $G/H$ es un grupo bajo la operación $(Ha)(Hb) = Hab, \forall a, b \in G$.
\end{fmd-theorem}

\begin{fmd-definition}[Grupo cociente]
	El grupo $(G/H, *)$ a que se refiere el teorema \ref{teo:cociente} se llama grupo cociente de $G$ por la relación de equivalencia compatible con $*$.
\end{fmd-definition}

\textbf{Notas}
\begin{itemize}
	\item $G/\{e\} = G$;
	\item $G / G = \{ e \}$
\end{itemize}

\begin{fmd-example} \label{ex:cociente}
	$(\Z, +)$ es un grupo abeliano. Sea $H$ el conjunto de los múltiplos de 5, esto es, $H = \{ \dots, -10, -5, 0, 5, 10, \dots \}$. Las clases laterales izquierda de $H$ en $\Z$ son:
	\[ \begin{array}{l}
		\bar{0} = 0 + H = \{ \dots, -10, -5, 0, 5, 10, \dots \}\\
		\bar{1} = 1 + H = \{ \dots, -9, -4, 1, 6, 11, \dots \}\\
		\bar{2} = 2 + H = \{ \dots, -8, -3, 2, 7, 12, \dots \}\\
		\bar{3} = 3 + H = \{ \dots, -7, -2, 3, 8, 13, \dots \}\\
		\bar{4} = 4 + H = \{ \dots, -6, -1, 4, 9, 14, \dots \}
	\end{array} \]
	
	Notar que las clases laterales derecha $\bar{x} = H + x$ serán iguales a las izquierdas, por tanto, $(H, +)$ es un subgrupo normal, $H \triangleleft \Z$.
	
	Para cualquier otro entero $n \in \Z$, $\bar{n} = n + H$ coincide con una de las clases anteriores. Luego, el grupo cociente $\Z / H = \{ \bar{0}, \bar{1}, \bar{2}, \bar{3}, \bar{4} \}$ forma un grupo para la adición según el teorema \ref{teo:cociente}. La siguiente es la tabla de adición:
	\begin{table}[H]
		\centering
		\[
		\begin{array}{c|ccccc}
			+ & \bar{0} & \bar{1} & \bar{2} & \bar{3} & \bar{4}\\ \hline
			\bar{0} & \bar{0} & \bar{1} & \bar{2} & \bar{3} & \bar{4}\\
			\bar{1} & \bar{1} & \bar{2} & \bar{3} & \bar{4} & \bar{0}\\
			\bar{2} & \bar{2} & \bar{3} & \bar{4} & \bar{0} & \bar{1}\\
			\bar{3} & \bar{3} & \bar{4} & \bar{0} & \bar{1} & \bar{2}\\
			\bar{4} & \bar{4} & \bar{0} & \bar{1} & \bar{2} & \bar{3}\\
		\end{array}
		\]
	\end{table}
\end{fmd-example}

\subsubsection{Grupos finitos}

Se G un grupo. Por definición $G$ es finito si y sólo si $|G| = n$. Orden de un grupo finito es el número cardinal del mismo.

\begin{fmd-definition}[Índice de un subgrupo]
	Sea $H$ un subgrupo del grupo finito $G$. El grupo cociente $G/H$, de las clases a izquierda de $H$, es finito y su cardinal se llama \textbf{índice del subgrupo} $H$ en $G$.
\end{fmd-definition}

El índice de $H$ en el ejemplo \ref{ex:cociente} es igual a $|\Z/H| = 5$.

\begin{fmd-theorem} \label{teo:finito}
Si $H$ es un subgrupo de orden $k$ del grupo finito $G$, entonces toda clase izquierda de $H$ tiene $k$ elementos.

\begin{itemize}
	\item[H)] $(G, *)$ es grupo finito; $|H| = k$.
	\item[T)] $|gH| = k \; \forall g \in G$.
\end{itemize}
\end{fmd-theorem}
\begin{fmd-proof}
	Debemos probar que $H$ y $gH$ son coordinables, y para ello difinimos:
	\[ f: H \rightarrow gH \mbox{ mediante } f(a) = g * a \]
	\begin{enumerate}[label=\roman*)]
		\item $f$ es la restricción de la \textit{translación a izquierda} $f_g: G \rightarrow G$ al subconjunto $H$ y en consecuencia es inyectiva.
		\item $f$ es sobreyectiva, pues para todo $y \in gH$ existe $x = g'*y$ tal que:
		\[ f(x) = f(g' * y) = g * g' * x = x \]
	\end{enumerate}
	En consecuencia, $f$ es biyectiva y $|gH| = |H| = k$
\end{fmd-proof}

\begin{fmd-theorem}[Lagrange] \label{teo:Lagrange}
	El orden de todo subgrupo de un grupo finito es divisor del orden del grupo.
\end{fmd-theorem}

\begin{fmd-proof} En efecto:
	
	Si $H$ es un subgrupo  de $G$ y $o(H) = k$ por el teorema \ref{teo:finito} el cardinal de toda coclase a izquierda de $H$ es $k$, y como estas son disjuntas, resulta:
	\[ o(G) = mk = m o(H) \]
	es decir:
	\[ o(H) \mid o(G) \]
\end{fmd-proof}



\subsection{Aritmética modular} \label{sec:aritmod}

La aritmética modular es un sistema aritmético para clases de equivalencia de números enteros llamadas \textbf{clases de congruencia}. Es uno de los aportes más significativos de Gauss a la Teoría de Números, en su famoso Disquisitiones Arithmeticae (Investigaciones sobre aritmética) de 1801.

Consideremos el ``reloj'' de la fig \ref{fig:relojMod3}. Se conoce como \textit{enteros módulo} 4 al conjunto $\Z_4 = \{ 0, 1, 2, 3 \}$.
\begin{figure}[h]
	\centering
	\includestandalone[scale=1.5]{resources/ltx/relojMod3}
	\caption{Enteros módulo 4}
	\label{fig:relojMod3}
\end{figure}

Veamos las operaciones de suma y multiplicación en $\Z_4$.

\subsubsection{Suma}

Para la operación suma se tienen los valores de la tabla \ref{tab:mod4}.
\begin{table}[h]
	\centering
	\begin{tabular}{c|cccc}
		$+$ & 0 & 1 & 2 & 3\\
		\hline
		0 & 0 & 1 & 2 & 3\\
		1 & 1 & 2 & 3 & 0\\
		2 & 2 & 3 & 0 & 1\\
		3 & 3 & 0 & 1 & 2\\
	\end{tabular}
	\caption{Enteros módulo 4 con la operación suma. Es un grupo abeliano.}
	\label{tab:mod4}
\end{table}

La operación es cerrada ya que:
\[ +: \Z_4 \times \Z_4 \rightarrow \Z_4  \]
\[ \mbox{Ejm: observando la tabla \ref{tab:mod4}} (1,3) \rightarrow 0 \]

\begin{itemize}
	\item[G1:] Es asociativa \cmark
	\item[G2:] Elemento neutro: 0 \cmark
	\item[G3:] Inversos: \cmark  \begin{itemize}
		\item 0 es su propio inverso;
		\item 2 es su propio inverso;
		\item 1 y 3 son inversos.
	\end{itemize}
	\item[G4:] Es conmutativa \cmark
\end{itemize}

\begin{center}
	$(\Z_4, +)$ es un grupo abeliano y finito.
\end{center}

En general, para los enteros múdulo $n$ con la operación suma:
\begin{center}
	$(\Z_n, +)$ es un grupo abeliano finito.
\end{center}

\subsubsection{Producto}

\begin{itemize}
	\item $(\Z_4, \bdot)$
	
	Para la operación producto se tienen los valores de la tabla \ref{tab:mod4p}.
	\begin{table}[H]
		\centering
		\begin{tabular}{c|cccc}
			$\bdot$ & 0 & 1 & 2 & 3\\
			\hline
			0 & 0 & 0 & 0 & 0\\
			1 & 0 & 1 & 2 & 3\\
			2 & 0 & 2 & 0 & 2\\
			3 & 0 & 3 & 2 & 1\\
		\end{tabular}
		\caption{Conjunto de enteros módulo 4 con la operación producto.}
		\label{tab:mod4p}
	\end{table}
	
	\begin{itemize}
		\item[G1:] Es asociativa \cmark
		\item[G2:] Elemento neutro: 1 \cmark
		\item[G3:] Inversos: \xmark
	\end{itemize}
	
	\begin{center}
		$(\Z_4, \bdot)$ no es un grupo, es un semigrupo con neutro.
	\end{center}
	
	\item $(\Z^{*}_4, \bdot)$
	
	Si se excluye el cero, se tiene la tabla \ref{tab:mod4pa}.
	
	\begin{table}[H]
		\centering
		\begin{tabular}{c|cccc}
			$\bdot$ & 1 & 2 & 3\\
			\hline
			1 & 1 & 2 & 3\\
			2 & 2 & 1 & 3\\
			3 & 3 & 3 & 1
		\end{tabular}
		\caption{Enteros módulo 4, excluyendo el cero $(\Z^{*}_4, \cdot)$.}
		\label{tab:mod4pa}
	\end{table}
	
	\begin{itemize}
		\item[G1:] Es asociativa \cmark
		\item[G2:] Elemento neutro \xmark
	\end{itemize}
	
	\begin{center}
		$(\Z^{*}_4, \bdot)$ no es un grupo, es un semigrupo.
	\end{center}
	
	\item $(\Z^{*}_3, \bdot)$
	
	Si se excluye el cero, se tiene la tabla \ref{tab:mod3pa}.
	
	\begin{table}[H]
		\centering
		\begin{tabular}{c|cccc}
			$\bdot$ & 1 & 2\\
			\hline
			1 & 1 & 2 \\
			2 & 2 & 1\\
		\end{tabular}
		\caption{Enteros módulo 3, excluyendo el cero $(\Z^{*}_3, \bdot)$.}
		\label{tab:mod3pa}
	\end{table}
	
	\begin{itemize}
		\item[G1:] Es asociativa \cmark
		\item[G2:] Elemento neutro: 1 \cmark
		\item[G3:] Elemento inverso: \cmark
		\begin{itemize}
			\item 1 es su propio inverso;
			\item 2 es su propio inverso
		\end{itemize}
		\item[G4:] Es conmutativa: \cmark
	\end{itemize}
	
	\begin{center}
		$(\Z^{*}_3, \bdot)$ es un grupo abeliano.
	\end{center}
\end{itemize}

\begin{proposition} \label{prop:primo}
	$(\Z_p^{*}, \bdot)$ es un grupo multiplicativo abeliano si y sólo si $p$ es primo.
\end{proposition}

Observemos ahora la fig. \ref{fig:relojconrecta} imaginando el giro sin deslizar de la rueda, vemos que los números enteros tienen relación con $0, 1, 2$ o $3$ dependiendo de su posición, esta relación es de congruencia.

\begin{figure}[H]
	\centering
	\includestandalone[scale=.9]{resources/ltx/reloj-con-recta}
	\caption{}
	\label{fig:relojconrecta}
\end{figure}

Notemos que a cada $x \in \Z$ le corresponderá el elemento de $\Z_4$ que resulta de tomar el residuo de la división de $x$ entre 4. Los posibles residuos de la división de un entero entre 4 son 0, 1, 2, 3, en general:

Por lo tanto, podemos decir que todos los números de $\Z$ con el mismo residuo de la división entre 4 serán equivalentes (se puede verificar fácilmente que cumple con las propiedades de equivalencia), se dice que son congruentes módulo 4.

Por ejemplo:
\[ 4753 \equiv 1 \, (\mbox{mod } 4); \qquad 1234 \equiv 2 \, (\mbox{mod } 4) \]

\emph{Dos enteros $a, b \in \Z$ son congruentes módulo $n$ si al dividir entre $n$ dan el mismo residuo}.

\begin{fmd-proposition}
	La congruencia módulo $n$, con $n$ entero positivo fijo, es una relación de equivalencia.
\end{fmd-proposition}

\emph{Los enteros módulo $n$ pueden considerarse como el conjunto de clases que se forman con la relación de congruencia.}

Por la definición de división sabemos que el dividendo es igual al cociente por el divisor más el residuo, por lo tanto, para dos números $x, y$ congruentes módulo $n$ ($x, y$ tienen el mismo residuo), se tiene

\[ \begin{array}{c}
	x = c_1n + r \\
	y = c_2n + r \\ \hline
	x - y = (c_1 - c_2)n
\end{array} \]

en consecuencia, $x, y$ son congruentes módulo $n$ si y sólo si $n$ divide a $x - y$.
\[ x \equiv_{n} y \iff n \mid x - y \]

El conjunto de números con el mismo residuo $r$ son las \textbf{clases} de $r$, en el caso de $\Z_3$, estas son:
\[ \begin{split}
	\mbox{clase del 0}: \quad [0] = \bar{0} =& \{\dots, -6, -3, 0, 3, 6, \dots \}\\
	\mbox{clase del 1}: \quad [1] = \bar{1} =& \{\dots, -5, -2, 1, 4, 7, \dots \}\\
	\mbox{clase del 2}: \quad [2] = \bar{2} =& \{\dots, -4, -1, 2, 5, 8, \dots \}\\
\end{split} \]

Se observa que hablar de, por ejemplo, la clase $\bar{4}$ es lo mismo que hablar de la clase $\bar{1}$, ya que están el mismo conjunto. Además, llamamos a 1 \textbf{representante} de la clase $\bar{1}$, 2 representante de la clase $\bar{2}$, etc.

Llamemos $3\Z$ al conjunto formado por los múltiplos de 3, en general $n\Z$ al conjunto formado por los múltiplos de $n$.

$\bar{0}$ es el conjunto de todos los múltiplos de 3 $(3\Z)$, o lo que es lo mismo, el conjunto de todos los números que al dividir entre 3 dan resto nulo. $\bar{1}$ son todos los números enteros tales que al dividir entre 3 dan resto 1, análogamente para $\bar{2}$.

Se observa que $\Z/3\Z$ tiene 3 clases, $\Z/4\Z$ tendrá 4 clases, en general $\Z/n\Z$ tiene $n$ clases de conformidad con el teorema \ref{teo:finito}.


Este es precisamente el conjunto cociente, es decir:
\[ \frac{\Z}{\sim} = \Z / 3\Z = \{ \bar{0}, \bar{1}, \bar{2} \}\]
En general
\[ \Z / n \Z = \{ \bar{0}, \bar{1}, \dots, \overline{n-1} \} \]

Las operaciones de suma entre las clases de congruencia de $\Z/n\Z$ se ven en las tabla \ref{tab:sclase}. Esto es, cualquier elemento, por ejemplo, de la clase del 0 (múltiplos de 3) $+$ cualquier elemento de la clase del 1 (números enteros tales que al dividir entre 3 dan resto 1), es igual un número de la clase del 1.

\begin{table}[H]
	\centering
	\begin{tabular}{c|cccc}
		$+$ & $\bar{0}$ & $\bar{1}$ & $\bar{2}$\\
		\hline
		$\bar{0}$ & $\bar{0}$ & $\bar{1}$ & $\bar{2}$\\
		$\bar{1}$ & $\bar{1}$ & $\bar{2}$ & $\bar{0}$\\
		$\bar{2}$ & $\bar{2}$ & $\bar{0}$ & $\bar{1}$
	\end{tabular}
	\caption{ Grupo cociente $\left( \Z/3\Z, + \right)$.}
	\label{tab:sclase}
\end{table}

\begin{fmd-example} 
	
	Sea $\left( \Z_5 \setminus \{0\}, \bdot \right) = \left( \{ [1], [2], [3], [4] \}, \bdot \right)$.
	
	El orden del grupo es $o(\Z_5 \setminus \{0\}) = 4$.
	
	Este es un grupo multiplicativo, ya que 5 es primo\footnote{Ver proposición \ref{prop:primo}} y sus congruencias o clases todas tienen inverso.
	
	\begin{table}[H]
		\centering
		\[
		\begin{array}{c|cccc}
			\bdot & \bar{1} & \bar{2} & \bar{3} & \bar{4}\\ \hline
			\bar{1} & \bar{1} & \bar{2} & \bar{3} & \bar{4}\\
			\bar{2} & \bar{2} & \bar{4} & \bar{1} & \bar{3}\\
			\bar{3} & \bar{3} & \bar{1} & \bar{4} & \bar{2}\\
			\bar{4} & \bar{4} & \bar{3} & \bar{2} & \bar{1}\\
		\end{array}
		\]
	\end{table}
	\begin{itemize}
		\item $[1]$ es el neutro, tiene orden 1 claramente;
		\item $[2]$ tiene orden 4, ya que $[2^4] = [16] = [1]$ y $[2^2] = [4]$ y $[2^3] = [3]$;
		
		Luego, el grupo es cíclico y $[2]$ es un generador del grupo.
		
		\item $[3]$ tiene orden 4, ya que $[3^4] = [81] = [1]$ y $[3^2] = [9] = [4]$ y $[3^3] = [2]$;
		
		Luego $[3]$ también es un generador del grupo.
		\item $[4]$ tiene orden 2, ya que $[4^2] = [16] = [1]$
	\end{itemize}
	Notar que, en todos los casos, el orden de los elementos (que resultaron: $1, 4, 4, 2$ para $[1], [2], [3], [4]$ respectivamente), divide al orden del grupo (que es 4), como tiene que ocurrir según el teorema de Lagrange\footnote{Teorema \ref{teo:Lagrange}}.
\end{fmd-example}

\begin{fmd-example}
	Se proporciona la tabla de adición para $\Z_4$, y la tabla de multiplicación para $S=\{ 1, 3, 7, 9 \}$ en $\Z_{10}$, observe que $S$ es un conjunto reducido de residuos para los enteros múdulo 10. Sea $f: \Z_4 \rightarrow S$ definida por:
	\[ f(0) = 1, \quad f(1) = 3, \quad f(2) = 9, \quad f(3) = 7 \]
		\[ \begin{array}{c|cccc}
			+ & 0 & 1 & 2 & 3 \\ \hline
			0 & 0 & 1 & 2 & 3\\
			1 & 1 & 2 & 3 & 0\\
			2 & 2 & 3 & 0 & 1\\
			3 & 3 & 0 & 1 & 2\\
		\end{array} \qquad 	\begin{array}{c|cccc}
		\times & 1 & 3 & 7 & 9 \\ \hline
		1 & 1 & 3 & 7 & 9\\
		3 & 3 & 9 & 1 & 7\\
		7 & 7 & 1 & 9 & 3\\
		9 & 9 & 7 & 3 & 1\\
		\end{array} \]
	Muestre que $f$ es un isomorfismo entre los grupos $(\Z_4, +)$ y $(S, \times)$.
	
	Para probar que es un isomorfismo, debemos probar que es un homomorfismo biyectivo.
	
	\begin{itemize}
		\item \textbf{?`Es un homomorfismo?}
		Por definición de homomorfismo, se debe cumplir:
		\[ \forall a, b \in \Z_4; f(a), f(b) \in S \implies f(a + b) = f(a) \times f(b) \]
		Por ejemplo, notemos que:
		\begin{itemize}
			\item $ f(1 + 2) = f(3) = 7 $ y $f(1) \times (2) = 3 \times 9 \equiv 7$
			\item $f(2 + 3) = f(1) = 3$ y $f(2) \times f(3) = 9 \times 7 \equiv 3$
		\end{itemize}
		y así se puede verificar para cualquier para de elementos, por lo que $f$ es un homomorfismo.
		
		\item \textbf{?`Es inyectiva?}
		La función es inyectiva, si:
		\[ a \ne b \implies f(a) \ne f(b) \]
		de la definción de $f$ puede observarse que es así, es decir, $f$ es inyectiva.
		
		\item \textbf{?`Es sobreyectiva?}
		Se observa que el codominio coincide con el rango, por lo que $f$ es sobreyectiva.		
	\end{itemize}
	Conclusión $f$ es un isomorfismo entre los grupos $(\Z_4, +)$ y $(S, \times)$.
\end{fmd-example}

\section{Anillos}

Un anillo es un conjunto equipado con dos operaciones binarias, una aditiva y otra multiplicativa, que satisfacen ciertas propiedades.

Como la sustracción se puede definir en términos de la adición, vagamente hablando en un anillo se puede sumar, restar y multiplicar.

\subsection{Definición y propiedades básicas}
\vspace{1em}
\begin{fmd-definition}[Anillo] \label{def:anillo}
	La terna $(A, *, \bdot)$ es un anillo si y sólo si:
	\begin{enumerate}
		\item $(A, *)$ es un grupo abeliano;
		\item $(A, \bdot)$ es un semigrupo;
		\item La segunda operación $(\bdot)$ es \textit{distributiva} a izquierda y derecha respecto de la primera $(*)$.
	\end{enumerate}
\end{fmd-definition}

La definición \ref{def:anillo} se traduce en los siguientes axiomas:

\begin{itemize}
	\item[A1]: La adición es ley de composición interna en $A$: $ \forall a, b \in A \implies a + b \in A $
	\item[A2]: La adición es asociativa en $A$: $ \forall a, b, c \in A: (a + b) + c = a + (b + c) $
	\item[A3]: Existe neutro $0 \in A$ respecto a la adición: $\exists 0 \in A / \forall a \in A: a + 0 = 0 + a = a$
	\item[A4]: Todo elemento de $A$ admite inverso aditivo: \[ \forall a \in A, \exists -a \in A / a + (-a) = (-a) + a = 0 \]
	\item[A5]: La adición es conmutativa: $\forall a, b \in A: a + b = b + a$
	\item[A6]: El producto es ley de composición interna en $A$. $\forall a, b \in A \implies ab \in A$
	\item[A7]: El producto es asociativo. $\forall a, b c \in A: (ab)c = a(bc)$
	\item[A8]: El producto es doblemente distributivo respecto de la suma.
	\[ \forall a, b, c \in A: \begin{cases}
		a (b + c) = ab + ac\\
		(b + c) a = ba + ca
	\end{cases} \]
\end{itemize}

Si además, ocurre que la segunda ley de composición es conmutativa, diremos que $(A, +, \bdot)$ es un \textbf{anillo conmutativo}. Si existe elemento neutro o identidad respecto del producto, que denotamos con 1, entonces se llamará \textbf{anillo con identidad o con unidad}.

\begin{example}
	Los números pares $\langle 2 \rangle = 2\Z$ forman un anillo conmutativo sin unidad ya que $1 \not \in 2\Z$.
\end{example}

Un anillo con identidad cuyos elementos no nulos son invertibles se llama \textbf{anillo de división}.

Se define la división por:
\[ x \div y \coloneqq x \times y^{-1} \]

\begin{fmd-example} Tres importantes anillos:
	\begin{enumerate}
		\item \textbf{Anillo de los números enteros ($\boldsymbol{\Z}$)}: Este es un ejemplo fundamental de un anillo conmutativo con unidad. El conjunto de números enteros, $\Z$, es un anillo con las operaciones usuales de suma y multiplicación.
		
		\item \textbf{Anillo de Polinomios $\boldsymbol{(A[x])}$ }: El conjunto de polinomios con coeficientes en un anillo, forma un anillo conmutativo con unidad. En este anillo, la suma y la multiplicación de polinomios satisfacen todas las propiedades de un anillo. Por ejemplo, el anillo de polinomios reales $\R[x]$ o el anillo de polinomios enteros $\Z[x]$ son ejemplos de anillos de polinomios. Nos extenderemos más sobre esto en la sección \ref{sec:polis}.
		
		\item \textbf{Anillo de Matrices} $\boldsymbol{(\mathbf{M}_n(\K))}$ (con $\K = \Z, \Q, \R, \C$)lo $A$, de matrices cuadradas $n \times n$ con coeficientes en $\K$, es un anillo unitario no conmutativo. Ver sección \ref{sec:matrices2}.
	\end{enumerate}
\end{fmd-example}

\begin{proposition}
	El producto de cualquier elemento de un anillo por el neutro de la primera ley es igual a éste.
	\begin{itemize}
		\item[H)] $(A, +, \bdot)$ es anillo.
		\item[T)] $a \cdot 0 = 0 \cdot a = 0$
	\end{itemize}
\end{proposition}
\begin{proof}
	Por ser anillo: $\forall x \in A: x + 0 = x$
	
	Premultiplicando por $a$ y aplicando distributiva: $a(x + 0) = ax + a0 = ax$, de donde
	\[ a0 = 0\]
	Análogamente se prueba $0a = 0$.
\end{proof}

\begin{proposition} \label{prop:anillo}
	En todo anillo, el producto del opuesto de un elemento, por otro, es igual al opuesto de su producto.
	\begin{itemize}
		\item[H)] $(A, +, \bdot)$ es anillo.
		\item[T)] $\forall a, b \in A: (-a) \cdot b = -(ab)$
	\end{itemize}
\end{proposition}
\begin{proof}
	Por distributividad y producto por 0: $ (-a)b + ab = (-a + a)b = 0b = 0 $;
	
	De donde, debe ser: $(-a)b = -(ab)$
	
	De manera similar se prueba que $a(-b) = -(ab)$
\end{proof}

\begin{proposition}
	En todo anillo, el producto de los opuestos de dos elementos es igual al producto de los mismos.
	\begin{itemize}
		\item[H)] $(A, +, \bdot)$ es anillo.
		\item[T)] $\forall a, b \in A: (-a) (-b) = ab$
	\end{itemize}
\end{proposition}
\begin{proof}
	Aplicando dos veces la proposición $\ref{prop:anillo}$ y por opuesto del opuesto, resulta:
	\[ (-a)(-b) = -[a(-b)] = -[-(ab)] = ab \]
\end{proof}

\textbf{Tarea}. Probar que:
\begin{enumerate}[label=\alph*)]
	\item $a(b - c) = ab - ac \quad \land \quad (b - c)a = ba - ca$
	\item Si el anillo es unitario:
	\begin{enumerate}[label=\roman*)]
		\item $(-1)a = a$
		\item $(1)(-1) = 1$
	\end{enumerate}
\end{enumerate}

\begin{fmd-theorem}[Unicidad de la unidad y los inversos]
	Si el anillo tiene unidad, esta es única. En un anillo de división, cada elemento tiene un inverso único.
\end{fmd-theorem}

\begin{fmd-definition}[Subanillo]
	Un subconjunto no vacío $S$ de $A$ se llama \textbf{subanillo} de $A$ si $S$ mismo forma un anillo para las operaciones de $A$.
	
	Observamos que $S$ es un subanillo de $A$ si y sólo si
	\[ a, b \in S \implies a + b \in S \land ab \in S\] 
\end{fmd-definition}

\textbf{Notas}:
\begin{itemize}
	\item $\{0\}$ y $A$ son subanillos de cualquier anillo $A$. $\{0\}$ es llamado \textbf{anillo trivial};
	\item Para todo entero positivo $n$, el conjunto:
	\[n\Z = \{0, \pm n, \pm 2n, \pm 3n, \dots\}\]
	es un subanillo de $\Z$.
	\item $\{0, 2, 4\}$ es un subanillo de $\Z_6$. Note que, aunque 1 es la unidad en $\Z_6$, 4 es la unidad en $\{0, 2, 4\}$.
	\item El conjunto de las matrices diagonales es un subanillo de las matrices $n-$cuadradas sobre $\R$.
\end{itemize}

\subsubsection{Anillos sin divisores de cero}
Los anillos sin divisores de cero son una clase especial de anillos. Un anillo se considera ``sin divisores de cero'' cuando no contiene elementos distintos de cero que, cuando se multiplican, dan como resultado cero sin ser ellos mismos cero. En otras palabras, en un anillo sin divisores de cero, si $a$ y $b$ son elementos del anillo y su producto $a b = 0$, entonces $a$ y/o $b$ deben ser cero.

\begin{fmd-definition}[Anillos sin divisores de cero]
	El anillo $(A, +, \bdot)$ no tiene divisores de cero si y sólo si elementos no nulos dan producto no nulo. En símbolos:
	\[ (A, +, \bdot) \mbox{ carece de divisores de cero } \iff \forall x, y \in A: x \ne 0 \land y \ne 0 \implies x \cdot y \ne 0 \]
	
	Equivalentemente, por medio de implicación contrarrecíproca, se tiene:
	\[ (A, +, \bdot) \mbox{ carece de divisores de cero } \iff \forall x, y \in A: x \cdot y = 0 \implies x = 0 \lor y = 0 \]
\end{fmd-definition}

\begin{fmd-definition}[Coprimos]
	Dos elementos $a$ y $b$ de $A$ son \textbf{coprimos} (o primos entre sí) si todo común divisor de $a$ y $b$ es inversible.
\end{fmd-definition}

En $\Z$, los enteros 2 y 3 admiten a $-1$ y 1 como divisores comunes. Estos son los únicos elementos inversibles en $\Z$, y se dice que 2 y 3 son coprimos o primos entre sí.

\begin{fmd-theorem} \label{teo:primo}
	El anillo $(\Z_n, +, \bdot)$ no tiene divisores de cero si y sólo si $n$ es primo.
\end{fmd-theorem}
\begin{fmd-proof}
	
	\begin{itemize}
		\item \textbf{Parte 1}: Si $n$ es primo, entonces el anillo no tiene divisores de cero
		
		Supongamos que $n$ es un número primo. Queremos demostrar que en el anillo $(\mathbb{Z}_n, +, \bdot)$, no existen divisores de cero, es decir, no hay elementos $a, b \in \mathbb{Z}_n$ tales que $a \cdot b \equiv 0 \mod n$ con $a \not\equiv 0 \mod n$ y $b \not\equiv 0 \mod n$.
		
		Dado que $n$ es primo, los únicos elementos no nulos en $\mathbb{Z}_n$ son aquellos que son coprimos con $n$. Esto significa que si $a$ y $b$ no son congruentes con $0$ módulo $n$, entonces $a$ y $b$ son coprimos con $n$. En otras palabras, el máximo común divisor de $a$ y $n$ es $1$, y el máximo común divisor de $b$ y $n$ también es $1$.
		
		Ahora, si $a \cdot b \equiv 0 \mod n$, esto implicaría que el producto de $a$ y $b$ es divisible por $n$. Sin embargo, debido a que $n$ es primo y $a$ y $b$ son coprimos con $n$, el producto $a \cdot b$ no puede ser divisible por $n$, a menos que uno de los factores sea divisible por $n$, lo cual es una contradicción. Por lo tanto, no existen divisores de cero en $\mathbb{Z}_n$ cuando $n$ es primo.
		
		\item \textbf{Parte 2}: Si $n$ no es primo, entonces el anillo tiene divisores de cero.
		
		Supongamos que $n$ no es un número primo. Queremos encontrar elementos $a$ y $b$ en $\mathbb{Z}_n$ tales que $a \cdot b \equiv 0 \mod n$ y $a \not\equiv 0 \mod n$ y $b \not\equiv 0 \mod n$.
		
		Si $n$ no es primo, entonces podemos escribir $n$ como el producto de dos enteros positivos $n = p \cdot q$, donde $p$ y $q$ son enteros mayores que $1$.
		
		Ahora, consideremos los elementos $a$ y $b$ en $\mathbb{Z}_n$ de la siguiente manera:
		\[a = [p]_n \mbox{ y } b = [q]_n\]
		
		Aquí, $[p]_n$ y $[q]_n$ representan las clases de congruencia módulo $n$. Estas clases son distintas de la clase $[0]_n$ ya que $p$ y $q$ no son congruentes con $0$ módulo $n$.
		
		Ahora veamos lo que sucede cuando multiplicamos $a$ y $b$ en $\mathbb{Z}_n$:
		\[a \cdot b = [p]_n \cdot [q]_n\]
		
		Dado que $p$ y $q$ no son congruentes con $0$ módulo $n$, esto significa que $p \cdot q \equiv 0 \mod n$. En otras palabras, $a \cdot b$ pertenece a la clase $[0]_n$, lo que significa que $a \cdot b \equiv 0 \mod n$.
		
		Además, dado que $p \not\equiv 0 \mod n$ y $q \not\equiv 0 \mod n$, esto implica que $a \not\equiv 0 \mod n$ y $b \not\equiv 0 \mod n$.
		
		Por lo tanto, hemos encontrado elementos $a$ y $b$ en $\mathbb{Z}_n$ tales que $a \cdot b \equiv 0 \mod n$ y $a \not\equiv 0 \mod n$ y $b \not\equiv 0 \mod n$, lo que demuestra que el anillo $\mathbb{Z}_n$ tiene divisores de cero cuando $n$ no es primo.
		
	\end{itemize}
	
	En resumen, hemos demostrado que si $n$ es primo, entonces el anillo $\mathbb{Z}_n$ no tiene divisores de cero, y si $n$ no es primo, entonces el anillo $\mathbb{Z}_n$ tiene divisores de cero. Esto establece la afirmación ``si y sólo si'' que estábamos buscando.
	
\end{fmd-proof}

\begin{fmd-definition}[Dominio de integridad]
	Todo anillo conmutativo, con unidad y sin divisores de cero, se llama \textit{dominio de integridad}.
\end{fmd-definition}

Las ternas $(\Z, +, \bdot)$, $(\Q, +, \bdot)$, $(\R, +, \bdot)$, $(\C, +, \bdot)$ son dominios de integridad.

El anillo $\Z_p$ de enteros módulo $p$ siendo este un número primo, es un dominio de integridad.

EL anillo $\Z_n$ de enteros módulo $n$ si $n$ no es primo, no es un dominio de integridad.

Por ejemplo, en $\Z_6$, los números 2 y 3 son distintos de cero, pero su producto es congruente con cero módulo 6, esto es: $2 \cdot 3 \equiv 0 \mod{6}$. Esto muestra que $\Z_6$ no es un dominio de integridad, ya que tiene elementos no nulos (2 y 3) cuyo producto es cero sin que ninguno de los factores sea cero.

\begin{proposition}[Cancelación]
	Sean $a, b$ y $c$ elementos de un dominio de integridad. Si $a\ne 0$ y $ab = ac$, entonces $b=c$.
\end{proposition}

\begin{proof}
	De $ab = bc$ se tiene $a(b-c)=0$. Como $a\ne 0$, tiene que ser $b-c=0$.
\end{proof}

\subsubsection{Unidades de un anillo}

Las unidades de un anillo son elementos especiales que tienen inversos multiplicativos dentro del anillo. El conjunto de todas las unidades de un anillo $A$ se denota a menudo como $U(A)$. Los elementos que no tienen inversos multiplicativos se llaman elementos no invertibles o no unidades.

\begin{fmd-definition}[Unidades de un anillo]
	Sea $A$ un anillo, $x \in A$ es una \textit{unidad} si $xy = yx = 1$ para algún $y \in A$.
\end{fmd-definition}

\begin{example}
	Un ejemplo común de unidades es el conjunto de números racionales $U(\Q^*)$ como unidades en el anillo de los números racionales ($\mathbb{Q}$). Cada número no nulo en $\mathbb{Q}$ tiene un inverso multiplicativo en $\mathbb{Q}$, lo que significa que \(\Q^* = \Q \setminus \{0\}\).
\end{example}
	
\begin{example}
	En el anillo de los números enteros (\(\mathbb{Z}\)), las unidades son \(\{1, -1\}\), ya que solo estos dos números tienen inversos multiplicativos en \(\mathbb{Z}\). Otros números enteros no tienen inversos multiplicativos en \(\mathbb{Z}\), por ejemplo, no hay ningún número entero cuyo producto con \(2\) sea igual a \(1\), el elemento neutro multiplicativo.
\end{example}

\begin{fmd-example}
Sea el anillo de los enteros módulo 10 $(Z_{10}, +, \times)$. Por el teorema \ref{teo:primo}, este anillo tiene divisores de cero, los cuales están resaltados en la tabla \ref{tab:mod10x}, por ejemplo: $2 \times 5 \equiv 0 \mod{10}$; $5 \times 6 \equiv 0 \mod{10}$.

Sus unidades son $U(10) = \{ 1, 3, 7, 9 \}$, pues por ejemplo $3 \times 7 = 1 \mod{10}$; $9 \times 9 = 1 \mod{10}$;

Las no unidades o no invertibles son: $\{ 0, 2, 4, 5, 6, 8\}$
	\begin{table}[H]
		\centering
		\begin{tabular}{c|cccccccccc}
			$\times$ & $0$ & $1$ & $2$ & $3$ & $4$ & $5$ & $6$ & $7$ & $8$ & $9$ \\ \hline
			$0$ & $0$ & $0$ & $0$ & $0$ & $0$ & $0$ & $0$ & $0$ & $0$ & $0$ \\
			$1$ & $0$ & \cellcolor{yellow!40} $\!\!1$ & $2$ & $3$ & $4$ & $5$ & $6$ & $7$ & $8$ & $9$ \\
			$2$ & $0$ & $2$ & $4$ & $6$ & $8$ & \cellcolor{blue!10} $\!\!0$ & $2$ & $4$ & $6$ & $8$ \\
			$3$ & $0$ & $3$ & $6$ & $9$ & $2$ & $5$ & $8$ & \cellcolor{yellow!40} $\!\!1$ & $4$ & $7$ \\
			$4$ & $0$ & $4$ & $8$ & $2$ & $6$ & \cellcolor{blue!10} $\!\!0$ & $4$ & $8$ & $2$ & $6$ \\
			$5$ & $0$ & $5$ & \cellcolor{blue!10} $\!\!0$ & $5$ & \cellcolor{blue!10} $\!\!0$ & $5$ & \cellcolor{blue!10} $\!\!0$ & $5$ & \cellcolor{blue!10} $\!\!0$ & $5$ \\
			$6$ & $0$ & $6$ & $2$ & $8$ & $4$ & \cellcolor{blue!10} $\!\!0$ & $6$ & $2$ & $8$ & $4$ \\
			$7$ & $0$ & $7$ & $4$ &\cellcolor{yellow!40} $\!\!1$ & $8$ & $5$ & $2$ & $9$ & $6$ & $3$ \\
			$8$ & $0$ & $8$ & $6$ & $4$ & $2$ & \cellcolor{blue!10} $\!\!0$ & $8$ & $6$ & $4$ & $2$ \\
			$9$ & $0$ & $9$ & $8$ & $7$ & $6$ & $5$ & $4$ & $3$ & $2$ & \cellcolor{yellow!40} $\!\!1$
		\end{tabular}
		\caption{Tabla de Cayley de $(\Z_{10}, \times)$.}
		\label{tab:mod10x}
	\end{table}
\end{fmd-example}

\begin{fmd-example}
	La existencia de divisores de cero en un anillo causa resultados inusuales cuando se buscan los ceros de polinomios con coeficientes en el anillo.
	
	Consideremos por ejemplo, la ecuación $x^2 - 4x + 3 = 0$. En el conjunto de los enteros, podemos encontrar las soluciones factorizando:
	\[ x^2 - 4x + 3 = 0 = (x - 3)(x-1) = 0\]
	tiene dos soluciones en los enteros: $x = 3$ y $x = 1$. Sin embargo, en $\Z_{12}$, la ecuación tiene cuatro soluciones: $x = 1$, $x = 3$, $x = 7$, y $x = 9$.
	
	Esto se debe a que en $\Z_{12}$, $2 \cdot 6 = 0$, $3 \cdot 4 = 0$, $4 \cdot 6 = 0$, $6 \cdot 8 = 0$, y así sucesivamente. Estos productos son 0 porque $2$, $3$, $4$, y $6$ son divisores de cero en $\Z_{12}$.
	
	En los dominios de integridad, como $\Z$, $\Z_{11}$ y $\Z_{13}$ ($\Z_p$, $p$ primo), no hay divisores de cero. Por lo tanto, podemos encontrar todos los ceros de un polinomio con coeficientes en un dominio de integridad simplemente factorizando el polinomio y luego igualando cada factor a 0.
\end{fmd-example}

\subsection{Homomorfismo de anillos}

Un homomorfismo de anillos es una función que preserva la estructura algebraica entre dos anillos.

\begin{fmd-definition}[Homomorfismo de anillos]
	Dados dos anillos $(A, +, \bdot)$ y $(B, +, \bdot)$. La aplicación $f: A \rightarrow B$ es un homomorfismo de anillos si se cumplen las siguientes condiciones:
	\begin{enumerate}
		\item $f(a + b) = f(a) + f(b), \forall a, b \in A$;
		\item $f(a\cdot b) = f(a) \cdot f(b), \forall a, b \in A$
	\end{enumerate}
\end{fmd-definition}

En el contexto de los homomorfismos de anillos, se utilizan términos similares a los que se emplean en los homomorfismos de grupos. De esta manera, un homomorfismo que conserva la inyectividad se llama \textbf{monomorfismo}, y un homomorfismo que es reversible o invertible se denomina \textbf{isomorfismo}. Un homomorfismo que mapea un anillo $A$ en sí mismo se conoce como \textbf{endomorfismo}, y un isomorfismo que relaciona un anillo $A$ consigo mismo se llama \textbf{automorfismo}.

\begin{fmd-example}
Consideremos el anillo de los polinomios en una variable $x$ con coeficientes reales, $\mathbb{R}[x]$. La función $\phi: \mathbb{R}[x] \to \mathbb{R}$ que asigna a cada polinomio su valor en $x = 0$ es un homomorfismo de anillos.

Para ver esto, note que para cualquier para de polinomios $p(x)$ y $q(x)$, tenemos:
\[
\phi \left(p(x) + q(x)\right) = p(0) + q(0) = \phi(p(x)) + \phi(q(x))
\]
y:
\[
\phi \left(p(x) \cdot q(x)\right) = p(0) \cdot q(0) = \phi \left(p(x)\right) \cdot \left(q(x)\right)
\]
Por lo tanto, $\phi$ cumple las condiciones de un homomorfismo de anillos.

En particular, $\phi(x^2 + 2x + 1) = (0)^2 + 2(0) + 1 = 1$.
\end{fmd-example}

\subsection{Anillo de matrices} \label{sec:matrices2}
Se vió en \ref{sec:matrices}, que el conjunto de matrices con coeficientes reales y la operación suma $(\mathbf{M}_{m \times n}(\R), +)$ es un grupo abeliano, en particular, para matrices cuadradas $m=n$ $(\mathbf{M}_{n}(\R), +)$ es un grupo abeliano y $(\mathbf{M}_{n}(\R), \bdot)$ es un semigrupo con elemento neutro.

\textbf{Matrices cuadradas}
\begin{itemize}
	\item[A1] Cerrada bajo la suma: $A, B \in \R^{n \times n} \implies A + B \in \R^{n\times n}$  \cmark
	\item[A2] Asociatividad con la suma: $(A + B) + C = A + (B + C)$ \cmark
	\item[A3] Existencia del opuesto: $ \exists N \in \R^{n \times n} / \forall A \in \R^{n \times n}: A + N = N + A = A $  \cmark
	\item[A4] Inversos aditivos: $ \forall A \in \R^{n \times n}, \exists (-A) \in \R^{n \times n} / A + (-A) = (-A)  + A = N$ \cmark
	\item[A5] Conmutatividad de la adición: $A + B = B + A$ \cmark
	\item[A6] Cerrado bajo el producto: $A \in \R^{n \times n} \land B \in \R^{n \times n} \implies AB \in \R^{n \times n}$  \cmark
	\item[A7] Asociatividad con el producto: $(AB)C = A(BC)$ \cmark
	\item[A8] Propiedad distributiva: $A(B+C) = AB + AC \land (B + C)A = BA + CA$ \cmark 
	\item[A9] Identidad: $\exists I \in \R^{n \times n} / \forall A \in \R^{n \times n}: A I = I A = A $ \cmark
	\item[A10] Conmutatividad: $AB \ne BA$  \xmark
\end{itemize}
Existen divisores de cero, es decir, matrices no nulas pueden dar producto no nulo. Por ejemplo, en el caso $(\R^{2 \times 2}, +, \bdot)$:

\[ A = \begin{bmatrix}
	1 & 1\\
	1 & 1
\end{bmatrix} \ne N \quad \mbox{ y } \quad B = \begin{custommatrix}{rr}
1 & -1\\
-1 & 1
\end{custommatrix} \ne N \]
y, sin embargo:
\[ AB = \begin{bmatrix}
	0 & 0\\
	0 & 0
\end{bmatrix} = N\]

Se trata de un anillo no conmutativo, con identidad y divisores de cero.

\begin{fmd-example}
	Determinar si la aplicación entre dos anillos $(\Z, +_Z, \bdot_Z)$ y $(\mathbf{M}_2(\R), +_M, \bdot_M)$ dado por $f: \Z \rightarrow \mathbf{M}_2(\R)$
	\[ f(n) = \begin{custommatrix}{rr}
		n & 0 \\ 0 & -n
	\end{custommatrix} \]
	es un homomorfismo de anillos.
	
	Es un homomorfismo de anillos, si cumple:
	\begin{enumerate}[label=\roman*)]
		\item $f(n +_Z m) = f(n) +_M f(m)$ \quad \cmark
		\[ \begin{split}
			f(n +_Z m) =& \begin{bmatrix}
				n +_Z m & 0 \\
				0 & -(n +_Z m)
			\end{bmatrix} = \begin{bmatrix}
				n +_Z m & 0 \\
				0 & -n +_Z m
			\end{bmatrix} \\[2mm] =&
			\begin{bmatrix}
				n & 0 \\
				0 & -n
			\end{bmatrix} +_M \begin{bmatrix}
				m & 0 \\
				0 & -m
			\end{bmatrix} = f(n) +_M f(m)
		\end{split} \]
		\item $f(n \bdot_Z m) = f(n) \bdot_M f(m)$ \quad \xmark
		\[ \begin{split}
			f(n) \bdot_M f(m) &= \begin{bmatrix}
				n & 0 \\
				0 & -n
			\end{bmatrix} \begin{bmatrix}
				m & 0 \\
				0 & -m
			\end{bmatrix} = \begin{bmatrix}
				n \bdot_Z m & 0 \\
				0 & n \bdot_Z m
			\end{bmatrix} \\
			& \ne \begin{bmatrix}
				n \bdot_Z m & 0 \\
				0 & - (n \bdot_Z m)
			\end{bmatrix} = f(n \bdot_Z m)
		\end{split} \]
	\end{enumerate}
	No es un homomorfismo de anillos.
\end{fmd-example}

\subsection{Anillo de polinomios} \label{sec:polis}

En general, en los libros de álgebra, los polinomios se presentan como una expresión del tipo:
\begin{equation} \label{eq:poli}
	a_0 + a_1 x + a_2 x^2 + \dots + a_nx^n
\end{equation}
en donde cada $a_i, i = 0, 1, 2, \dots, n$ es un número real o complejo. Ahora bien, la naturaleza de los ``objetos'' $x, x^2, \dots, x^n$ está en discusión. Por ejemplo:
\[ 1 - 3 \pi + 4\pi^2 - 7\pi^4 \]
donde los coeficientes $(1, -3, 4, 0, -7)$ de las potencias de $\pi$ están en $\Z$ pero, obviamente $\pi$ y sus potencias no lo están, por lo que la expresión del polinomio no tiene sentido en $\Z$.

\eqref{eq:poli} sugiere la existencia de un conjunto ``mayor'' que contenga los coeficientes $a_i$, y un objeto especial, a saber $x$, no perteneciente al conjunto de los coeficientes.

Consideraciones importantes:
\begin{itemize}
	\item Haremos la distinción entre polinomios, como \eqref{eq:poli} y funciones polinomiales como $P(x) = 2 + 3x -5x^2$ que representa un valor numérico de esa expresión, para algún valor de la variable $x$.
	\item Se entenderá a los polinomios como \textit{secuencias finitas} del tipo $(a_0, a_1, \dots, a_n, 0, 0, \dots)$
	\item Reinterpretamos a $x^i$, no como la potencia de un número, sino como la posición del coeficiente $a_i$ en la secuencia.
\end{itemize}

Ahora, hagamos un repaso de la suma y multiplicación de polinomios.

\begin{fmd-example}[Repaso de suma y multiplicación de polinomios]
	Consideremos los polinomios $p = a_0 + a_1x + a_2x^2$ y $q = b_0 + b_1x + b_2x^2$
	
	\begin{itemize}
		\item Suma
		\[ p + q = c_0 + c_1x + c_2x^2 = (a_0 + b_0) + (a_1 + b_1)x + (a_2 + b_2)x^2\]
		en general: \[ c_i = a_i + b_i \]
		\item Multiplicación
		\[ \begin{split}
			pq =& (a_0 + a_1x + a_2x^2)(b_0 + b_1x + b_2x^2)\\
			=& a_0b_0 + (a_0b_1 + a_1b_0)x + (a_0b_2 + a_1b_1 + a_2b_0)x^2 + (a_1b_2 + a_2b_1)x^3 + a_2b_2x^4\\
			=& c_0 + c_1x + c_2x^2 + c_3x^3 + c_4x^4
			\end{split} \]
			Se observa que la suma de los subíndices de los coeficientes, siempre es igual al exponente de $x$, es decir, los términos entre paréntesis son de la forma $a_jb_{i-j}x^{i}$ y se puede escribir, por ejemplo:
			\[ c_3 = a_1b_2 + a_2b_1 = {\color{red} a_0b_3} + a_1b_2 + a_2b_1 + {\color{red} a_3b_0} \]
			considerando que $a_3 = b_3 = 0$ para en general poner:
			\[ c_i = \sum_{j=0}^{i} a_jb_{i-j}x^{i}\]
			Los subíndices de $a$ \textit{ascienden} y los de $b$ \textit{descienden}.
	\end{itemize}
\end{fmd-example}

Ahora estamos listos para analizar anillos de polinomios.

\subsubsection{Sucesión de elementos de un anillo}

Consideremos un anillo conmutativo $A$, y una sucesión de elementos de $A$, $(a_i)_{i \in \N}$, que cumplen $a_{n+1} = 0$ desde un cierto $n$ en adelante, es decir:
\[ (a_i)_{i \in \N} = (a_0, a_1, a_2, \dots, a_n, 0, 0, \dots)\]

\subsubsection{Operaciones internas}
Consideremos dado el anillo conmutativo $A$, y dos sucesiones de elementos de $A$, $(a_i)_{i \in \N}$ y $(b_i)_{i \in \N}$. Las operaciones internas están dadas por:
\begin{itemize}
	\item Suma: $(a_i)_{i \in \N} + (b_i)_{i \in \N} = (a_0 + b_0, a_1 + b_1, \dots)$
	\item Multiplicación: $(a_i)_{i \in \N} \cdot (b_i)_{i \in \N} = (a_0b_0, \, a_0b_1 + a_1b_0, \, \dots, \, a_0b_k + a_1 b_{k-1} + \dots + a_k b_0, \dots)$
\end{itemize}

\begin{fmd-theorem}
	Sea $A^*$ el conjunto formado por todas las sucesiones $(a_i)_{i \in \N}$ definidas anteriormente, entonces la terna $(A^*, +, \bdot)$, con las operaciones internas definidas anteriormente, tiene estructura de anillo conmutativo.
\end{fmd-theorem}

\subsubsection{El elemento $\mathbf{x}$}

Designemos mediante $x$ al elemento del anillo $(A^*, +, \bdot)$ dado por:
\[ x = (0, 1, 0, 0, \dots) \]
Entonces:
\[ x^n = (0, 0, \dots, 0, \overbrace{1}^{n+1}, 0, \dots) \]

\subsubsection{El anillo $A[x]$}
A la terna $(A^*, +, \bdot)$ la denominamos anillos de los polinomios de variable $x$ y la denotamos por $A[x]$
\[ (a_i)_{i\in \N} = a_0 + a_1 x + a_2 x^2 + \dots + a_b x^n = \sum_{i=0}^{n} a_i x^i \]

\subsubsection{El anillo $\Z[x]$}
El anillo $\Z[x]$ lo constituyen los polinomios de coeficientes enteros.

\begin{example}
	Son polinomios de coeficientes enteros:
	\begin{itemize}
		\item $2 + 3x - 4x^3$
		\item $x^6 + x^7$
		\item $-6 + 4x^3 - 25x^8 + x^{35}$
	\end{itemize}
\end{example}

\subsubsection{Notación con sumatorias}

$ \displaystyle A[x] =  (a_i)_{i \in \N} = \sum_{i=0}^{n} a_i x^i$

Las operaciones en $A[x]$ se pueden poner como:
\begin{itemize}
	\item $ \displaystyle (a_i)_{i \in \N} + (b_i)_{i \in \N} = \sum_{i=0}^{n} a_i x^i  + \sum_{i=0}^{m} b_i x^i  = \sum_{i=0}^{\max(n,m)} a_i x^i $
	\item $ \displaystyle a(x) \cdot b(x) = \sum_{i=0}^{n + m} \left( \sum_{j=0}^{i} a_j b_{i-j} x^i \right)$
\end{itemize}

\begin{fmd-example}
	Consideremos los polinomios $f(x) = 2 + 2x + x^2 + 2x^3$, $g(x)=1 + 2x + 2x^2$ en $\Z_3[x]$. Luego, en la notación precedente:
	\begin{itemize}
		\item $f(x) = (a_i)_{i\in \N} = (2, 2, 1, 2, 0, \dots)$
		\item $g(x) = (b_i)_{i \in \N} = (1, 2, 2, 0, 0, \dots)$
	\end{itemize}
	
	El grado de $f(x)$ es $n=3$ y el grado de $g(x)$ es $m=2$, por lo tanto, la suma va hasta el término de grado $\max(3, 2) = 3$ y la multiplicación hasta el grado $3 + 2 = 5$.
	
	Observemos también que, por lo dicho precedentemente, $a_i = 0$ $\forall i > 3$ y $b_i = 0$ $\forall i > 2$.
	
	La suma y la multiplicación de estos polinomios, en $\Z_3[x]$, son respectivamente:
	\begin{itemize}
		\item Suma: $f(x) + g(x) = (0, 1, 0, 2, 0, \dots) = x + 2x^3$
		\item Multiplicación: $c_i = \sum_{j=0}^{i} a_j b_{i-j}$
		\begin{align*}
			c_0 =& a_0b_0 = 2 \times 1 \equiv 2\\
			c_1 =& a_0b_1 + a_1b_0 = 2\times 1 + 2 \times 2 \equiv 0\\
			c_2 =& a_0b_2 + a_1b_1 + a_2b_0 = 1 \times 1 + 2 \times 2 + 2 \times 2 \equiv 0\\
			c_3 =& a_1b_2 + a_2b_1 + a_3b_0 = 2 \times 1 + 1 \times 2 + 2 \times 2 \equiv 2\\
			c_4 =& a_2b_2 + a_3b_1 = 2 \times 2 + 1 \times 2 \equiv 0\\
			c_5 =& a_3b_2 = 2 \times 2 \equiv 1
		\end{align*}
		\[f(x)g(x) = (2, 0, 0, 2, 0, 1, 0, \dots) = 2 + 2x^3 + x^5 \]
	\end{itemize}
	
\end{fmd-example}

\begin{fmd-definition}[Anillos de polinomios con varias indeterminadas]
	Definimos el anillo de polinomios $A[x_1, \dots, x_n]$ con $n$ indeterminadas con coeficientes en el anillo $A$ inductivamente por
	\[ \begin{array}{c}
		A[x_1, x_2] \equiv \left(A[x_1] \right)[x_2]\\
		A[x_1, x_2, x_3] \equiv \left(A[x_1, x_2] \right)[x_3]\\
		\vdots \\
		A[x_1, \dots, x_n] \equiv \left( A[x_1, \dots x_{n-1}] \right)[x_n]
	\end{array} \]
	es decir:
	\[ A[x_1, \dots, x_n] \equiv A[x_1][x_2]\cdots [x_{n-1}][x_n] \]
	y los polinomios de $A[x_1, \dots, x_n]$ se expresan como suma de monomios
	\[ p(x_1, \dots, x_n) = \sum a_{i_1,\dots, i_n}x_1^{i_1} \cdots x_n^{i_n} \]
	donde $a_{i_1,\dots, i_n} \in A$ son los coeficientes.
\end{fmd-definition}

\textbf{Nota}: De acuerdo con la definición, es claro que dos polinomios son iguales si y sólo si, lo son todos los coeficientes de ambos.

\begin{fmd-example}
	Sea un anillo $A$ conmutativo. Consideremos el anillo $A[x, y]$ de polinomios en dos indeterminadas y con coeficientes en $A$. Cualquier elemento de $A[x, y]$ es de la forma \[ p(x, y) = \sum_{i,j} a_{ij} x^i y^j\]
	que puede verse como
	\[ A[x, y] = p(x,y) = \sum_{i,j} \left(a_{ij} x^i\right) y^j = \left( (a_i)_j \right)_{i, j \in \N} = A[x][y] \]
\end{fmd-example}

\subsection{Ideales} \label{sec:ideales}

Los ideales son a los anillos lo que los subgrupos normales a los grupos.

Supongamos un anillo $A$ y un subconjunto $I \subseteq A$, la pregunta es:

?`Qué propiedades debe tener $I$ de manera el conjunto de clases de $I$ sea un anillo?

Como $(A, +)$ es un grupo abeliano $\implies$ $(I, +) \trianglelefteq (A, +)$, es decir por \eqref{eq:normal},  $(I, +)$ debe ser un subgrupo normal de $(A, +)$.

En general, como $(A, +)$ es abeliano todo subgrupo es normal\footnote{Ver proposición \ref{prop:subgrupo}.}, por lo tanto el primer requerimiento para $(I, +)$ es que sea un subgrupo de $(A, +)$.
\[ (I, +) \le (A, +) \]

Ahora bien, para construir el grupo cociente $A/I$ con la operación de adición (por ejemplo la clase de $x \in A$ será $\bar{x} = x + I \in A/I$), como $A$ es un anillo, debemos buscar la posibilidad de multiplicar entre sí los elementos de $A/I$ tal que el resultado esté en $A/I$, es decir:
\[ \forall x, y \in A: (x + I)(y + I) = x y + I \]
Veamos qué debe pasar para que esto se cumpla. Tomemos cualesquiera $i_1, i_2 \in I$
\[ (x + i_1) (y + i_2) = xy + i_1y + xi_2 + i_1i_2 \implies i_1y + xi_2 + i_1i_2 = i_3 \in I\]
si queremos que $i_3$ pertenezca a $I$, cada término de la última igualdad debe pertenecer a $I$, y por lo tanto le pertenecerá la suma, pues ya hemos impuesto que $(I, +)$ sea un subgrupo.

Si $i_1i_2 \in I$, debemos suponer que $I$ es un subanillo de $A$, luego para que la multiplicación sea cerrada, se debe cumplir además $i_1y, xi_2 \in I$, en otras palabras $I$ debe \textit{absorber} la multiplicación con los elementos de $A$, por derecha y por izquierda.

\begin{fmd-definition}[Ideales] \label{def:ideales}
	Dado un anillo $A$, un subanillo $I \le A$ se denomina \textbf{ideal por derecha} si:
	\[ a \in I \implies xa \in I, \forall x \in A \]
	e \textbf{ideal por izquierda} si
	\[ a \in I \implies ax \in I, \forall x \in A \]
	Se dice simplemente \textbf{ideal}, si lo es por derecha e izquierda.
\end{fmd-definition}

Claramente, si el anillo es conmutativo, la distinción de ideales laterales es inconsecuente.

\begin{fmd-theorem}[Test de ideales]
	Un subconjunto no vacío $I$ de un anillo $A$ es un ideal si:
	\begin{itemize}
		\item $a - b \in I,\, \forall a, b \in I$ (ver teorema \ref{teo:subgrupo});
		\item $xa, ax \in I$, $\forall a \in I \land \forall x \in A$ (por definición \ref{def:ideales} de ideales).
	\end{itemize}
\end{fmd-theorem}

\begin{example}
	Para todo anillo $A$, $\{0\}$ y $A$ son ideales de $A$. El $\{0\}$ es llamado ideal \textit{trivial}.
\end{example}

\begin{example} \label{ex:ideal}
	Para cualquier entero positivo $n$, el conjunto $n\Z = \{0, \pm n, \pm 2n, \pm 3n, \dots\}$ es un ideal de $\Z$.
\end{example}

\subsection{Anillo cociente}
\vspace{3mm}
\begin{fmd-theorem}[Existencia del anillo cociente]
	Sea $R$ un anillo y $A$ un subanillo de $R$. El conjunto de clases $\{ r + A \, | \, r \in R \}$ es un anillo bajo las operaciones $(s + A) + (t + A) = s + t + A$ y $(s + A)(t + A) = st + A$ si y sólo si $A$ es un ideal de $R$.
\end{fmd-theorem}

\begin{proof}
	Se demuestra con la discusión dada en la introducción a ideales, sección \ref{sec:ideales}.
\end{proof}

\begin{fmd-example} \label{ex:cociente1}
	Sea $\Z / 4 \Z = \{ \bar{0}, \bar{1}, \bar{2}, \bar{3} \} = \{ 0 + 4\Z, 1 + 4\Z, 2 + 4\Z, 3 + 4\Z \}$.
	
	Por el ejemplo \ref{ex:ideal}, $4\Z$ es un ideal de $\Z$ por tanto $\Z/4\Z$ es efectivamente un anillo cociente.
	
	Para ver cómo sumar y multiplicar, consideremos $\bar{2}$ y $\bar{3}$:
	\[ \bar{2} + \bar{3} = (2 + 4\Z) + (3 + 4\Z) = 5 + 4\Z = 1 + 4 + 4\Z = 1 + 4\Z = \bar{1} \]
	\[ \bar{2} \times \bar{3} = (2 + 4\Z)(3 + 4\Z) = 6 + 4\Z = 2 + 4 + 4\Z = 2 + 4\Z = \bar{2}\]
	se concluye que las operaciones son básicamente las mismas que en aritmética módulo 4.
\end{fmd-example}

\begin{fmd-example}
	Sea $2\Z / 6 \Z = \{ \bar{0}, \bar{2}, \bar{4} \} = \{ 0 + 6\Z, 2 + 6\Z, 4 + 6\Z\}$.
	
	Por el ejemplo \ref{ex:cociente1}, aquí efectuamos las operaciones como en aritmética módulo 6.
	
	Por ejemplo:
	\[ \bar{4} + \bar{4} = (4 + 6\Z) + (4 + 6\Z) = 2 + 6\Z = \bar{2}\]
	\[ \bar{4} \times \bar{4} = (4 + 6\Z)(4 + 6\Z) = 4 + 6\Z = \bar{4}\]
\end{fmd-example}


\section{Cuerpos}
También llamados \textit{campos}. Los cuerpos son estructuras algebraicas más específicas y más ricas en propiedades que los anillos y los grupos. En la fig. \ref{fig:gak} se ve un esquema que relaciona estas estructuras.

\begin{figure}[H]
	\centering
	\includestandalone[scale=1.2]{resources/ltx/gak}
	\caption{Diagrama de Venn de estructuras algebraicas.}
	\label{fig:gak}
\end{figure}

?`Qué conjuntos nos permiten sumar, restar, multiplicar y dividir? Veamos la tabla \ref{tab:cuerpo}:

\begin{table}[H]
	\centering
	\begin{tabular}{|l||c|c|c|c|c||c|c|c|c|c|}
		\hline
		& \multicolumn{5}{c||}{\textbf{Adición}} & \multicolumn{5}{c|}{\textbf{Multiplicación}}\\ \cline{2-11}
		& Cerrado & Asoc. & 0 &  Op. & Conm. & Cerrado & Asoc.& 1 & Conm. & Inv. \\ \hline \hline
		$\Z$ & \cmark & \cmark& \cmark & \cmark & \cmark & \cmark & \cmark& \cmark & \cmark & \xmark\\
		$\R^{2\times 3}$ & \cmark & \cmark & \cmark & \cmark & \cmark & \xmark & \xmark & \xmark & \xmark & \xmark\\
		$\R^{2 \times 2}$ & \cmark & \cmark & \cmark & \cmark & \cmark & \cmark & \cmark & \cmark & \xmark& \xmark\\
		\rowcolor{lightgray!40}%
		$\Q$ & \cmark & \cmark & \cmark& \cmark & \cmark & \cmark & \cmark & \cmark & \cmark& \cmark\\
		\rowcolor{lightgray!40}%
		$\Z_5$ & \cmark & \cmark & \cmark & \cmark & \cmark & \cmark & \cmark & \cmark & \cmark & \cmark\\
		$\Z_6$ & \cmark & \cmark & \cmark & \cmark & \cmark & \cmark & \cmark & \cmark & \cmark & \xmark\\ \hline
	\end{tabular}
	\caption{}
	\label{tab:cuerpo}
\end{table}
Los conjuntos $\Q$: números racionales y $\Z_5$: enteros múdulo 5, tienen en común las siguientes propiedades:
\begin{itemize}
	\item Forman grupos conmutativos para la adición;
	\item Son cerrados para la operación de multiplicación, i.e. son anillos;
	\item $ab = ba \implies$ son anillos conmutativos;
	\item Los elementos distintos de 0, tienen inversos multiplicativos (permiten la división).
\end{itemize}

\subsection{Definición y propiedades básicas}
\vspace{3mm}
\begin{fmd-definition}[Cuerpo] \label{def:cuerpo}
	Sea $(\K, +, \bdot)$ un anillo unitario conmutativo, y denotemos mediante $1 \in \K$ el elemento unidad. Diremos que $(\K, +, \bdot)$ es un \textbf{cuerpo} si para todo elemento $x \in \K$ distinto del neutro aditivo (el cual denotaremos con 0) existe un elemento $x^{-1} \in \K$ tal que $xx^{-1} = 1$.
\end{fmd-definition}

\subsubsection{Axiomas de cuerpo}

La definición \ref{def:cuerpo} se traduce en axiomas. Sea $\K$ un conjunto no vacío dotado de dos operaciones internas, que llamaremos suma ($+$) y producto ($\bdot$), que cumplen los siguientes axiomas:

\begin{enumerate}
	\item \textbf{Axiomas de la Suma}
	
	\begin{itemize}
		\item[(A1)] \textit{Asociatividad de la suma}.
		
		$\forall a, b, c \in \K$:
		\[(a+b)+c=a+(b+c) \]
		\item[(A2)] \textit{Conmutatividad de la suma}:
		
		$\forall a, b \in \K$:
		\[ a + b = b + a \]
		\item[(A3)] \textit{Existencia del elemento neutro de la suma}:
		
		Existe un elemento en $\K$, denotado por $0$, tal que $\forall a \in \K$:
		\[ a + 0 = a \]
		\item[(A4)] \textit{Existencia de elemento inverso de la suma}:
		
		$\forall a \in \K$, existe un elemento en $\K$, denotado por $-a$, tal que:
		\[ a + (-a) = 0 \]
	\end{itemize}
	
	\item \textbf{Axiomas del Producto}
	
	\begin{itemize}
		\item[(M1)] \textit{Asociatividad del producto}:
		
		\( \forall a, b, c \in \K \):
		\[ (a \cdot b) \cdot c = a \cdot (b \cdot c) \]
		\item[(M2)] \textit{Conmutatividad del producto}:
		
		\( \forall a, b \in \K \):
		\[a \cdot b = b \cdot a\]
		\item[(M3)] \textit{Existencia de elemento neutro del producto}:
		
		Existe un elemento en $\K$, denotado por $1$, distinto de $0$, tal que $\forall a \in \K$: \[a \cdot 1 = a\]
		\item[(M4)] \textit{Existencia de elemento inverso del producto}:
		
		$\forall a \in \K$,  con $a \ne 0$, existe un elemento en $\K$, denotado por $a^{-1}$, tal que:
		\[a \cdot a^{-1} = 1\]
	\end{itemize}
	
	\item \textbf{Axioma de Distributividad}
	\begin{itemize}
		\item[(D)] \textit{Distributividad del producto respecto de la suma}:
		
		$\forall a, b, c \in \K$:
		\[a \bdot (b+c) = (a \cdot b) + (a \bdot c)\]
	\end{itemize}
\end{enumerate}

	
\subsection{Propiedades generales}

Sea $(\K, +, \cdot)$ un cuerpo. Algunas de sus propiedades generales son las siguientes:

\begin{enumerate}
	\item \textbf{Unicidad de los elementos neutros}: El elemento neutro de la suma (0) y el elemento neutro del producto (1) son únicos.
	
	\item \textbf{Unicidad de los elementos inversos}: Para cada elemento $a \in K$, su inverso aditivo $-a$ y, si $a \neq 0$, su inverso multiplicativo $a^{-1}$ son únicos.
	
	\item \textbf{Ley de cancelación para la suma}: $\forall a, b, c \in K$, si $a + c = b + c$, entonces $a = b$.
	
	\item \textbf{Ley de cancelación para el producto}: $\forall a, b, c \in K$ con $c \neq 0$, si $a \cdot c = b \cdot c$, entonces $a = b$.
	
	\item \textbf{Producto por cero}: $ \forall a \in K$, se cumple que $a \cdot 0 = 0$.
	
	\item \textbf{Solubilidad única de ecuaciones lineales}: Si $a\ne 0$, entonces la ecuación $ax = b$ admite solución y es única en $\K$;
	
	\item \textbf{Regla de los signos}:
	 $\forall a, b \in K$:
	 \begin{itemize}
	 	\item $(-a) \cdot b = a \cdot (-b) = -(a \cdot b)$
	 	\item $(-a) \cdot (-b) = a \cdot b$
	 \end{itemize}
	
	\item \textbf{No existen divisores de cero}: Si $a, b \in K$ y $a \cdot b = 0$, entonces $a = 0$ o $b = 0$.
	
	\item \textbf{Elemento opuesto del opuesto}: $\forall a \in K$, se cumple que $-(-a) = a$.
	
	\item \textbf{Inverso del inverso}: $\forall a \in K$ con $a \neq 0$, se cumple que $(a^{-1})^{-1} = a$
\end{enumerate}

\subsubsection{Subcuerpo}
\vspace{1em}
\begin{fmd-definition}[Subcuerpo]
	Dado un cuerpo $\K$, un subconjunto $\mathbb{F} \subset \K$ es un \textbf{subcuerpo} si es en sí mismo un cuerpo con las operaciones heredadas de $\K$.
\end{fmd-definition}

\subsection{Homomorfismo de cuerpos}
\vspace{1em}
\begin{fmd-definition}[Homomorfismo de cuerpos]
	Sean $(\K, +, \bdot)$, $(\mathbb{L}, +, \bdot)$ cuerpos. Una aplicación $f: \K \rightarrow \mathbb{L}$ se dice que es un homomorfismo entre cuerpos, si y sólo si $f$ es un homomorfismo entre los anillos $(\K, +, \bdot)$ y $(\mathbb{L}, +, \bdot)$ con imagen no trivial, es decir $Im(f) \ne \{ 0 \}$
\end{fmd-definition}

\textbf{Observaciones}

De acuerdo con la definición anterior.
\begin{itemize}
	\item $f$ es homomorfismo entre los grupos $(\K, +)$ y $(\mathbb{L}, +)$;
	\item $f$ es homomorfismo entre los grupos $(\K^* =  \K \setminus \{0\}, \bdot)$ y $(\mathbb{L}^* = \mathbb{L} \setminus \{0\}, \bdot)$.
\end{itemize}

\begin{fmd-theorem}
	Sea $f: \K \rightarrow \mathbb{L}$ un homomorfismo de cuerpos, entonces:
	\begin{enumerate}[label=\alph*)]
		\item $f(0) = 0$ y $f(-a) = -f(a)$, $\forall a \in \K$;
		\item $f(1) = 1$ y $f(a^{-1}) = f(a)^{-1}$, $\forall a \in \K, a \ne 0$;
		\item $Im(f)$ es un \textit{subcuerpo} de $\mathbb{L}$;
		\item $f$ es inyectivo.
 	\end{enumerate}
\end{fmd-theorem}

\begin{fmd-proof}
\begin{enumerate}[label=\alph*)]
	\item Se deduce del hecho de ser $f$ un homomorfismo entre los grupos aditivos de $\K$ y $\mathbb{L}$;
	\item Se deduce del hecho de ser $f$ un homomorfismo entre los grupos multiplicativos de $\K^*$ y $\mathbb{L}^*$;
	\item $f$ es un homomorfismo entre los anillos $\K$ y $\mathbb{L}$, por tanto $\mbox{Im}(f)$ es un subanillo de $\mathbb{L}$ y al ser $\mathbb{L}$ anillo conmutativo, también lo es $\mbox{Im}(f)$. Además $1 = f(1) \in \mbox{Im}(f)$, luego $\mbox{Im}(f)$ es también unitario.
	
	Falta demostrar que para todo $b \in \mbox{Im}(f)$ con $b\ne 0$ su inverso $b^{-1}$ pertenece a $\mbox{Im}(f)$. En efecto, si $b \in \mbox{Im}(f)$ con $b \ne 0$, entonces $b = f(a)$ para algún $a \ne 0$ en $\K$ (si fuera $a=0$, $f(a)$ sería 0). Por tanto, $b^{-1} = f(a)^{-1} = f(a^{-1}) \in \mbox{Im}(f)$;
	
	\item $f$ es un homomorfismo entre los anillos $\K$ y $\mathbb{L}$, y por tanto $\ker f$ es un ideal de $\K$. Al ser $\K$ un cuerpo, sus únicos ideales son $\{0\}$ y $\K$. Si fuera $\ker f = \K$ entonces $f$ sería homomorfismo nulo, o sea $\mbox{Im}(f) = \{0\}$, en contradicción con la definición de homomorfismo entre cuerpos. Ha de ser por tanto $\ker f = \{0\}$, lo cual implica que $f$ es inyectivo.
\end{enumerate}
\end{fmd-proof}

\subsection{Cuerpos ordenados}

Un \textit{cuerpo ordenado} es un cuerpo $(\K, +, \bdot)$ en el que además se define una relación de orden total, denotada por $\leq$, que es compatible con las operaciones del cuerpo. Es decir, satisface los siguientes axiomas adicionales:

\begin{enumerate}
	\item \textbf{Axiomas de Orden Total}
	
	\begin{enumerate}[label=\roman*)]
		\item \textit{Reflexividad:} $\forall a \in K$, $a \leq a$.
		\item \textit{Antisimetría:} $\forall a, b \in K$, si $a \leq b$ y $b \leq a$, entonces $a = b$.
		\item \textit{Transitividad:} $\forall a, b, c \in K$, si $a \leq b$ y $b \leq c$, entonces $a \leq c$.
		\item \textit{Totalidad:} $\forall a, b \in K$, se cumple que $a \leq b$ o $b \leq a$.
	\end{enumerate}
	
	\item \textbf{Compatibilidad con las Operaciones}
	
	\begin{itemize}
		\item \textit{Compatibilidad con la suma:} $\forall a, b, c \in K$, si $a \leq b$, entonces $a + c \leq b + c$.
		\item \textit{Compatibilidad con el producto:} $\forall a, b, c \in K$, si $a \leq b$ y $0 \leq c$, entonces $a \cdot c \leq b \cdot c$.
	\end{itemize}
\end{enumerate}

\begin{fmd-example}[Cuerpos ordenados]
\begin{itemize}
	\item Los números racionales $\mathbb{Q}$ con el orden usual.
	\item Los números reales $\mathbb{R}$ con el orden usual.
	\item \textit{No es un cuerpo ordenado:} El cuerpo de los números complejos $\mathbb{C}$, ya que no se puede definir un orden total en $\mathbb{C}$ que sea compatible con sus operaciones.
\end{itemize}
\end{fmd-example}

\begin{lgremark}
	La existencia de un orden en un cuerpo permite introducir conceptos como positivo, negativo, valor absoluto, y desigualdades, enriqueciendo el análisis y la geometría en estos cuerpos.
\end{lgremark}

\subsubsection{Cotas}

Sea $(\K, +, \bdot, \leq)$ un cuerpo ordenado.

\begin{itemize}
	\item \textbf{Cota superior}: Un elemento $M \in \K$ es una \textit{cota superior} de un conjunto $S \subseteq \K$ si $\forall s \in S$, se cumple que $s \leq M$.
	\item \textbf{Cota inferior}: Un elemento $m \in K$ es una \textit{cota inferior} de un conjunto $S \subseteq \K$ si $\forall s \in S$, se cumple que $m \leq s$.
	\item \textbf{Conjunto acotado}: Un conjunto $S \subset K$ es \textit{acotado superiormente} si tiene una cota superior, y es \textit{acotado inferiormente} si tiene una cota inferior. Si $S$ es acotado superior e inferiormente, se dice simplemente que es \textbf{acotado}.
\end{itemize}

\begin{fmd-example}[Cotas]
	Ejemplos en $\R$
	\begin{itemize}
		\item El conjunto $S = \{x \in \mathbb{R} : 0 < x < 1\}$ es acotado.
		\item Cotas superiores: Cualquier número real mayor o igual a 1.
		\item Cotas inferiores: Cualquier número real menor o igual a 0.
		\item El conjunto $\mathbb{N}$ es acotado inferiormente (cota inferior: 0), pero no superiormente.
	\end{itemize}
\end{fmd-example}

\subsubsection{Valor Absoluto}

El \textbf{valor absoluto} de un elemento $a \in \K$ se define como:
\[ |a| =
\begin{cases}
	a, & \text{si } a \geq 0 \\
	-a, & \text{si } a < 0
\end{cases} \]

\begin{figure}[H]
	\centering
	\begin{tikzpicture}
		\begin{axis}[axis lines=middle, xlabel={$\K$}, ylabel={$\K^{+}$},
			xticklabels={}, yticklabels={}, ymin=-1, ymax=2.5, scale mode=scale uniformly,
			scale=.6, xmin=-2.5, xmax=2.5]
			\addplot[thick, domain=0:2] {x};
			\addplot[thick, domain=-2:0] {-x};
		\end{axis}
	\end{tikzpicture}
	\caption{Valor absoluto}
\end{figure}

\textbf{Propiedades del Valor Absoluto en un Cuerpo Ordenado}

\begin{enumerate}
	\item $|a| \geq 0, \forall a \in \K$.
	\item $|a| = 0 \iff a = 0$.
	\item $|a \cdot b| = |a| \cdot |b|, \quad \forall a, b \in \K$.
	\item $|a + b| \leq |a| + |b|, \quad \forall a, b \in \K$ (desigualdad triangular).
\end{enumerate}

\subsubsection{Intervalos}

Sean $a, b \in \K$ con $a < b$. Se definen los siguientes intervalos:

\begin{itemize}
	\item \textit{Intervalo abierto}: $(a, b) = \{x \in \K \mid a < x < b\}$
	\begin{figure}[H]
		\centering
		\begin{tikzpicture}
			\draw[latex-latex] (-3,0) -- (3,0) node[right] {$\K$} ; % Eje horizontal
			\draw[very thick] (-1,0) -- (2,0); % Línea entre a y b
			\draw[thick, fill=white] (-1,0) circle (2.5pt) node[below=2pt] {$a$}; % Punto abierto a
			\draw[thick, fill=white] (2,0) circle (2.5pt) node[below=2pt] {$b$}; % Punto abierto b
		\end{tikzpicture}
	\end{figure}
	\item \textit{Intervalo cerrado}: $[a, b] = \{x \in \K \mid a \leq x \leq b\}$
	\begin{figure}[H]
		\centering
		\begin{tikzpicture}
			\draw[latex-latex] (-3,0) -- (3,0) node[right] {$\K$} ; % Eje horizontal
			\draw[very thick] (-1,0) -- (2,0); % Línea entre a y b
			\draw[thick, fill=black] (-1,0) circle (2.5pt) node[below=2pt] {$a$}; % Punto abierto a
			\draw[thick, fill=black] (2,0) circle (2.5pt) node[below=2pt] {$b$}; % Punto abierto b
		\end{tikzpicture}
	\end{figure}
	\item \textit{Intervalos semiabiertos}:
	\begin{itemize}
		\item $(a, b] = \{x \in \K \mid a < x \leq b\}$
			\begin{figure}[H]
			\centering
			\begin{tikzpicture}
				\draw[latex-latex] (-3,0) -- (3,0) node[right] {$\K$} ; % Eje horizontal
				\draw[very thick] (-1,0) -- (2,0); % Línea entre a y b
				\draw[thick, fill=white] (-1,0) circle (2.5pt) node[below=2pt] {$a$}; % Punto abierto a
				\draw[thick, fill=black] (2,0) circle (2.5pt) node[below=2pt] {$b$}; % Punto abierto b
			\end{tikzpicture}
		\end{figure}
		\item $[a, b) = \{x \in \K \mid a \leq x < b\}$
			\begin{figure}[H]
			\centering
			\begin{tikzpicture}
				\draw[latex-latex] (-3,0) -- (3,0) node[right] {$\K$} ; % Eje horizontal
				\draw[very thick] (-1,0) -- (2,0); % Línea entre a y b
				\draw[thick, fill=black] (-1,0) circle (2.5pt) node[below=2pt] {$a$}; % Punto abierto a
				\draw[thick, fill=white] (2,0) circle (2.5pt) node[below=2pt] {$b$}; % Punto abierto b
			\end{tikzpicture}
		\end{figure}
	\end{itemize}
\end{itemize}

\begin{fmd-example}[Intervalos]
	En $\R$:
	\begin{itemize}
		\item $(-2, 3)$ es un intervalo abierto por ambos lados, que representa todos los números reales estrictamente entre -2 y 3.
		\item $[0, 1]$ es un intervalo cerrado por ambos lados, representa todos los números reales entre 0 y 1, incluyendo 0 y 1.
	\end{itemize}
\end{fmd-example}




    
    \newpage
    \printglossary[type=main, title={Glosario}]
    \printsymbols[title=Lista de símbolos]
    \newpage

    \nocite{*}
    \printbibliography

    \newpage

    \appendix
\end{document}
