\newglossaryentry{conjunto}{name=conjunto,
	description={Colección de objetos o elementos que comparten una característica común},
	text=conjunto,
	plural=conjuntos}

\newglossaryentry{elemento}{name=elemento,
	description={Objeto individual dentro de un conjunto},
	text=elemento,
	plural=elementos
	}

\newglossaryentry{pertenencia}{name=pertenencia (\ensuremath{\in}), symbol={\ensuremath{\in}},
	text=pertenencia, description={Relación que indica si un elemento está o no en un conjunto.}}

\newglossaryentry{inclusion}{name=inclusión (\ensuremath{\subset}), symbol={\ensuremath{\subset}}, text=inclusión,
	description={Relación entre conjuntos donde un conjunto (subconjunto) está completamente contenido en otro}}

\newglossaryentry{conjuntovacio}{name=conjunto vacío (\ensuremath{\emptyset}) , symbol={\ensuremath{\emptyset}},
	description={Conjunto que no contiene ningún elemento},
	text=conjunto vacío,
	plural=conjuntos vacíos}

\newglossaryentry{conjuntounitario}{name=conjunto unitario,
	description={Conjunto que contiene un solo elemento},
	text=conjunto unitario,
	plural=conjuntos unitarios}

\newglossaryentry{conjuntouniversal}{name=conjunto universal (\ensuremath{U}), symbol={\ensuremath{U}},
	description={Conjunto que contiene todos los elementos considerados en un contexto particular},
	text=conjunto universal,
	plural=conjuntos universales}

\newglossaryentry{union}{name=unión (\ensuremath{\cup}), symbol={\ensuremath{\cup}},
	description={La unión de dos conjuntos $A$ y $B$ consiste en todos los elementos que pertenecen tanto a $A$, a $B$ o a ambos},
	text=unión,
	plural=uniones}
	
\newglossaryentry{interseccion}{name=intersección (\ensuremath{\cap}), symbol={\ensuremath{\cap}},
		description={La intersección de dos conjuntos $A$ y $B$ contiene los elementos que pertenecen tanto a $A$ como a $B$},
		text=intersección,
		plural=intersecciones}
		
\newglossaryentry{complemento}{name=complemento \ensuremath{A^c},
	description={Contiene todos los elementos que no están en $A$},
	text=complemento,
	plural=complementos}
	
\newglossaryentry{venn}{name=diagrama de Venn,
	description={Representación visual de conjuntos},
	plural=diagramas de Venn}
	
\newglossaryentry{difsimetrica}{name=diferencia simétrica (\ensuremath{\triangle}),
text=diferencia simétrica, symbol={\ensuremath{\triangle}}, description={Es el conjunto que contiene los elementos que pertenecen al sustraendo y no al minuendo}}

\newglossaryentry{conjuntopotencia}{name=conjunto potencia (\ensuremath{\mathcal{P}(A)}),
	text=conjunto potencia,
	description={Conjunto de todos los subconjuntos de un conjunto dado \ensuremath{A}},
	plural=conjuntos potencia}

\newglossaryentry{productocartesiano}{name=producto cartesiano \ensuremath{(A \times B)},
	text={producto cartesiano},
	description={Conjunto de todos los pares ordenados posibles tomando el primer elemento de $A$ y el segundo de $B$}}
	
\newglossaryentry{uniondisjunta}{
	name=unión disjunta,
	description={Cuando los conjuntos son mutuamente excluyentes},
	plural=uniones disjuntas
}

\newglossaryentry{relacion}{
	name=relación (\ensuremath{\mathcal{R}}),
	text=relación,
	symbol={\ensuremath{\mathcal{R}}},
	description={Alguna regla de correspondencia entre elementos de dos conjuntos},
	plural=relaciones
}

\newglossaryentry{clasequiv}{
	name=clase de equivalencia (\ensuremath{[a]}),
	text=clase de equivalencia,
	description={El conjunto de elementos que tienen relación de equivalencia con $a$},
	plural=clases de equivalencia
}

\newglossaryentry{conjuntocociente}{
	name=conjunto cociente,
	description={Conjunto formado por las clases de equivalencia},
	plural=conjuntos cociente
}

\newglossaryentry{particion}{
	name=partición,
	description={Partes no vacías de un conjunto que no se intersectan},
	plural=particiones
}

\newglossaryentry{relequivalencia}{
	name=relación de equivalencia (\ensuremath{\equiv}),
	text=relación de equivalencia,
	symbol={\ensuremath{\equiv}},
	description={Toda relación reflexiva, simétrica y transitiva},
	plural=relaciones de equivalencia
}

\newglossaryentry{funcion}{name=función,
	description={Regla que asigna a cada elemento de un conjunto (dominio) un único elemento de otro conjunto (codominio)},
	plural=funciones}

\newglossaryentry{dominio}{name=dominio,
	description={Conjunto de partida en una función},
	plural=dominios}

\newglossaryentry{codominio}{name=codominio,
	description={Conjunto de llegada en una función},
	plural=codominios}

\newglossaryentry{rango}{name=rango, 
	description={Conjunto de todos los valores que toma una función (imagen del dominio)},
	plural=rangos}
	
\newglossaryentry{preimagen}{name=preimagen, 
		description={Elemento del dominio que al ser evaluado por una función produce un valor en el codominio},
		plural=rangos}

\newglossaryentry{funcioninyectiva}{name=función inyectiva,
	description={Función donde cada elemento del dominio se mapea a un elemento distinto del codominio},
	plural=funciones inyectivas}

\newglossaryentry{funcionsobreyectiva}{name=función sobreyectiva,
	description={Función donde cada elemento del codominio tiene al menos un elemento del dominio que se mapea a él},
	plural=funciones sobreyectivas}

\newglossaryentry{funcionbiyectiva}{name=función biyectiva,
	description={Función que es a la vez inyectiva y sobreyectiva},
	plural=funciones biyectivas}

\newglossaryentry{funcioninversa}{name=función inversa,
	description={Función que "deshace" la acción de otra función},
	plural=funciones inversas}

\newglossaryentry{composicion}{name=composición de funciones \ensuremath{(g \circ f)},
	text={composición de funciones},
	description={Aplicación sucesiva de dos funciones, donde la salida de la primera función ($f$) se convierte en la entrada de la segunda función ($g$)}}

\newglossaryentry{proposicion}{name=proposición,
	description={Enunciado que puede ser verdadero o falso},
	plural=proposiciones}

\newglossaryentry{operadorlogico}{name=operador lógico,
	description={Símbolos que conectan proposiciones para formar proposiciones compuestas},
	plural=operadores lógicos}

\newglossaryentry{negacion}{name=negación (\ensuremath{\neg}), text=negación, symbol={\ensuremath{\neg}},
	description={Niega una proposición}}

\newglossaryentry{conjuncion}{name=conjunción (\ensuremath{\land}),
	text=conjunción,
	symbol={\ensuremath{\land}},
	description={Verdadera si ambas proposiciones son verdaderas}}

\newglossaryentry{disyuncion}{name=disyunción (\ensuremath{\lor}),
	text=disyunción,
	symbol={\ensuremath{\lor}},
	description={Verdadera si al menos una de las proposiciones es verdadera}}

\newglossaryentry{disyuncione}{name=disyunción excluyente (\ensuremath{\oplus}),
	text=disyunción excluyente,
	symbol={\ensuremath{\oplus}},
	description={Verdadera si solo una de las proposiciones es verdadera}}

\newglossaryentry{implicacion}{name=implicación (\ensuremath{\implies}),
	text=implicación,
	symbol={\ensuremath{\implies}},
	description={Verdadera a menos que el antecedente sea verdadero y el consecuente falso}}

\newglossaryentry{bicondicional}{name=bicondicional (\ensuremath{\iff}),
	text=bicondicional,
	symbol={\ensuremath{\iff}}, description={Verdadera si ambas proposiciones tienen el mismo valor de verdad}}

\newglossaryentry{tablav}{name=tabla de verdad,
	description={Herramienta para determinar el valor de verdad de una proposición compuesta para todas las combinaciones posibles de valores de verdad de sus componentes},
	plural=tablas de verdad}

\newglossaryentry{tautologia}{name=tautología,
	description={Proposición compuesta que siempre es verdadera},
	plural=tautologías}

\newglossaryentry{contradiccion}{name=contradicción,
	description={Proposición compuesta que siempre es falsa},
	plural=contradicciones}
	
\newglossaryentry{argumento}{name=argumento,
	description={Secuencia de premisas que pretende justificar una conclusión},
	plural=argumentos}

\newglossaryentry{cuantificadore}{name=cuantificador existencial (\ensuremath{\exists}),
	text=cuantificador existencial,
	symbol={\ensuremath{\exists}},
	description={Indica que existe al menos un elemento que cumple una propiedad},
	plural=cuantificadores existenciales}

\newglossaryentry{cuantificadoru}{name=cuantificador universal (\ensuremath{\forall}),
	text=cuantificador universal,
	symbol={\ensuremath{\forall}}, description={Indica que todos los elementos cumplen una propiedad}}

\newglossaryentry{reglasinferencia}{name=reglas de inferencia,
	description={Patrones lógicos válidos que permiten deducir nuevas proposiciones a partir de premisas dadas}, plural=reglas de inferencia}
	
\newglossaryentry{falacia}{name=falacia,
		description={Razonamiento incorrecto con apariencia de correcto},
		plural=falacias}

\newglossaryentry{falaciaformal}{name=falacia formal,
	description={Argumento que parece válido pero que viola las reglas de la lógica},
	plural=falacias formales}

\newglossaryentry{grupo}{name=grupo,
	description={Conjunto no vacío con una operación binaria que cumple las propiedades de cierre, asociatividad, elemento neutro e inverso},
	plural=grupos}

\newglossaryentry{grupoabeliano}{name=grupo abeliano,
	description={Grupo en el que la operación es conmutativa},
	plural=grupos abelianos}

\newglossaryentry{subgrupo}{name=subgrupo,
	description={Subconjunto de un grupo que también es un grupo bajo la misma operación},
	plural=subgrupos}

\newglossaryentry{subgrupon}{name=subgrupo normal,
	description={Subgrupo que cumple la propiedad de que para cualquier elemento del grupo, el conjunto de productos a izquierda y a derecha con elementos del subgrupo son iguales}, plural=subgrupos normales}

\newglossaryentry{anillo}{name=anillo,
	description={Conjunto no vacío con dos operaciones binarias (suma y multiplicación) que cumple las propiedades de grupo abeliano bajo la suma, semigrupo bajo la multiplicación, y distributividad de la multiplicación sobre la suma}, plural=anillos}

\newglossaryentry{anilloconmutativo}{name=anillo conmutativo,
	description={Anillo en el que la multiplicación es conmutativa}, plural=anillos conmutativos}

\newglossaryentry{anillou}{name=anillo con unidad,
	description={Anillo que tiene un elemento neutro para la multiplicación}, plural=anillos con unidad}

\newglossaryentry{dominioi}{name=dominio de integridad,
	description={Anillo conmutativo con unidad que no tiene divisores de cero (elementos no nulos cuyo producto es cero).}, plural=dominios de integridad}

\newglossaryentry{cuerpo}{name=cuerpo,
	description={Anillo conmutativo con unidad donde todo elemento no nulo tiene un inverso multiplicativo}, plural=cuerpos}

\newglossaryentry{subcuerpo}{name=subcuerpo,
	description={Subconjunto de un cuerpo que también es un cuerpo bajo las mismas operaciones}, plural=subcuerpos}
