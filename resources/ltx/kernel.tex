\documentclass[tikz]{standalone}
\usepackage[spanish, mexico-com]{babel}
\usepackage{amssymb}
\usepackage{tikz}
\usetikzlibrary{arrows, calc}

\begin{document}
	\begin{tikzpicture}[scale=1]
		% Coordenadas de puntos
		\coordinate (a) at (0,1.5);
		\coordinate (b) at (0,.5);
		\coordinate (c) at (0,-.5);
		\coordinate (d) at (0,-1.5);
		
		\coordinate (B1) at (6,1.5);
		\coordinate (B2) at (6,.5);
		\coordinate (B3) at (6,-.5);
		\coordinate (B4) at (6,-1.5);
		
		% Circunferencias
		\node at (-1.5, 2) {$G$};
		\node at (7.5, 2) {$G'$};
		\draw[very thick] (0,0) circle (2);
		\draw[very thick] (6,0) circle (2);
		\draw[fill=black!20] (0, -1) ellipse (.6 and 1) node[xshift=-30] {$\ker$};
		\node at (0, -2.5) {Dominio};
		\node at (6, -2.5) {Codominio};
		
		% Puntos
		\draw[fill] (a) circle (1mm);
		\draw[fill] (b) circle (1mm);
		\draw[fill] (c) circle (1mm) node[below right] {$e$};
		\draw[fill] (d) circle (1mm);
		
		\draw[fill] (B1) circle (1mm);
		\draw[fill] (B2) circle (1mm);
		\draw[fill] (B3) circle (1mm) node[right, xshift=5] {$e'$};
		\draw[fill] (B4) circle (1mm);
		
		% flechas
		\draw[blue, thick, -latex] (c) to [bend left = 10] (B3);
		\draw[blue, thick, -latex] (d) to [bend left = -10] (B3);
		
		\node at (3, .1) {$f$};
	\end{tikzpicture}
\end{document}